\subsection{\secState{R}UAS Movement Automaton}\label{s:movementAutomatonDefinition}

\paragraph{Motivation:} An \emph{UAS Nonlinear Model} (eq. \ref{eq:UASNonlinearModelSimple}) can be modeled by \emph{Movement Automaton} (def. \ref{def:movementAutomaton}). 

\paragraph{Movement Primitives} by (def. \ref{def:MovementPrimitive})  are given as (eq. \ref{eq:movementPrimitive}). To define primitives the \emph{minimal time} is $1 s$. The \emph{maximal duration} is also $1s$. 

\begin{assumption}\label{ass:transitionTime}
    Let assume that \emph{transition time} of \emph{roll, pitch, yaw, linear velocity} is $0 s$.
\end{assumption}

Under the assumption (as. \ref{ass:transitionTime}) the \emph{movement transitions} (def. \ref{def:movementTransition}) have $0$ duration.

\begin{note}
    The assumption (as. \ref{ass:transitionTime}) can be relaxed under condition that \emph{path tracking controller exists}.
\end{note}

\paragraph{Movements} (def. \ref{def:Movement}) for \emph{fixed step} $k$ we start with discretization of the input variables.

\noindent The \emph{linear velocity} in text step is given:
\begin{equation}\label{eq:applyMovement}
    v(k+1) = v(k) +\delta v(k)
\end{equation}

\noindent The \emph{roll, pitch, yaw} for next step are given 

\begin{equation}\label{eq:applyMovement1}
    \begin{aligned}
        roll(k+1)  &= roll(k) + \delta roll(k)\\
        pitch(k+1) & = pitch(k) + \delta pitch(k)\\
        yaw(k+1) & = yaw(k) + \delta yaw(k)\\
    \end{aligned}    
\end{equation}

\noindent The $\delta v(k)$ is \emph{velocity change}, $\delta roll(k)$, $\delta pitch(k)$, $\delta yaw(k)$, are \emph{orientation changes} for current discrete step $k$. If the duration of \emph{transition} is $0 s$ (as. \ref{ass:transitionTime}) then 3D trajectory evolution in discrete time is given as: 

\begin{equation}\label{eq:applyMovement2}
    \begin{aligned}
        x(k+1)&= x(k) + v(k+1) \cos(pitch(k+1)) \cos(yaw(k+1)) & = \delta x(k)\\
        y(k+1)&= y(k) + v(k+1) \cos(pitch(k+1)) \sin(yaw(k+1)) & = \delta y(k)\\
        z(k+1)&= z(k) - v(k+1) \sin(pitch(k+1))                & = \delta z(k)\\
        time(k+1)& = time(k)+1                                & = \delta time(k)
    \end{aligned}    
\end{equation}

\noindent The $\delta x(k)$, $\delta y(k)$, $\delta z(k)$ are positional differences depending on \emph{input vector} for given discrete time $k$:
\begin{equation}\label{eq:ourImput}
    input(k) = \left[
                    \begin{gathered}
                    \delta x(k), \delta y(k), \delta z(k), \delta v (k),\\
                    \delta roll (k), \delta pitch(k), \delta yaw(k),\delta time (k)
                    \end{gathered} 
                \right]^T
\end{equation}

\noindent The \emph{state vector} for discrete time is given:
\begin{equation}\label{eq:ourState}
    state(k) = \left[
                    \begin{gathered}
                     x(k),  y(k),  z(k),  v (k),\\
                     roll (k),  pitch(k),  yaw(k), time (k)
                    \end{gathered} 
                \right]^T
\end{equation}

\noindent The nonlinear model (eq. \ref{eq:UASNonlinearModelSimple}) is then reduced to \emph{linear discrete model} (eq. \ref{eq:uasLinearDiscreteModel}) given by \emph{apply movements} function (eq. \ref{eq:applyMovement}, \ref{eq:applyMovement1}, \ref{eq:applyMovement2}).

\begin{equation}\label{eq:uasLinearDiscreteModel}
    state(k+1) = applyMovement(state(k), input(k)) 
\end{equation}

\paragraph{Movement Set} for linear discrete model (eq. \ref{eq:uasLinearDiscreteModel}) is defined as set of extreme unitary movements on main axes (tab. \ref{tab:movements1}) and diagonal axes (tab. \ref{tab:movements2}).

\begin{table}[H]
    \centering
    \begin{tabular}{r||r|r|r|r|r}
    
        $input(movement)$           &    Straight  & Down & Up & Left  & Right   \\\hline\hline
        $\delta     x(k)[m]$           &    1.00	  & 0.98  & 0.98  & 0.98 & 0.98  \\\hline
        $\delta     y(k)[m]$           &    0	      & 0	  & 0	  & 0.13 & -0.13 \\\hline
        $\delta     z(k)[m]$           &    0	      & -0.13 & 0.13  &	0	 & 0     \\\hline
        $\delta  roll(k) [^\circ]$	   &    0	      & 0	  & 0	  & 0    & 0     \\\hline
        $\delta pitch(k) [^\circ]$     &    0	      & $15^\circ$  & -$15^\circ$ & 0	 & 0     \\\hline
        $\delta   yaw(k) [^\circ]$     &    0	      & 0	  & 0	  & $15^\circ$ & -$15^\circ$ \\
    \end{tabular}
    \caption{Input values for main axes movements.}
    \label{tab:movements1}
\end{table}
\begin{table}[H]
    \centering
    \begin{tabular}{r||r|r|r|r}
        $input(movement)$             & Down-Left & Down-Right & Up-Left  & Up-Right   \\\hline\hline
        $\delta     x(k)[m]$           & 0.76  & 0.76  & 0.76 & 0.76  \\\hline
        $\delta     y(k)[m]$           & -0.13	& 0.13	& 0.13 & -0.13 \\\hline
        $\delta     z(k)[m]$           & -0.13 & -0.13 & 0.13 & 0.13  \\\hline
        $\delta  roll(k) [^\circ]$	& 0	    & 0	    & 0    & 0     \\\hline
        $\delta pitch(k) [^\circ]$     & -$15^\circ$ & -$15^\circ$ & $15^\circ$ & $15^\circ$     \\\hline
        $\delta   yaw(k) [^\circ]$    & $15^\circ$	& -$15^\circ$	& $15^\circ$ & -$15^\circ$ \\
    \end{tabular}
    \caption{Input values for diagonal axes movements.}
    \label{tab:movements2}
\end{table}

\begin{note}
    \emph{Movement set} in shorten form is given as
    \begin{equation}\label{eq:OurMovementSet}
        Movement Set= \left\{
        \begin{gathered}
            Straight, Left,Right, Up, Down,\\
            Down Left, Down Right,  Up Left,   Up Right
        \end{gathered}
        \right\}
    \end{equation}
\end{note}

\paragraph{Trajectory} by (def. \ref{def:MovementAutomatonTrajectory}) for initial time $time = 0$ , initial state $state(0)$ and \emph{Movement Buffer} (from def. \ref{def:MovementBuffer}):
\begin{equation}\label{eq:ourBuffer}
    Buffer \in Movement Set^* (eq. \ref{eq:OurMovementSet}), \quad  |Buffer| \in \N
\end{equation}

\noindent Trajectory (eq. \ref{eq:ourTrajectoryImplementation}) is then given as the time-series of discrete states:
\begin{equation}\label{eq:ourTrajectoryImplementation}
    Trajectory(state(0),Buffer)= \left\{\begin{gathered}state(0)+\sum_{j=0}^{i-1} input(movement(j)):\\i \in\left\{1\dots |Buffer|+1\right\}, \\movement(\cdot) \in Buffer\end{gathered}\right\}
\end{equation}

\noindent Trajectory (eq. \ref{eq:ourTrajectoryImplementation}) is ordered set of states bounded to discrete time $0\dots n$ , where $n$ is member count of \emph{Buffer}. Trajectory set has $n+1$ members:

\begin{equation}
    \begin{aligned}
    T&rajectory(state(0),Buffer)=\\
        &\left\{
        \begin{aligned}
            state(0) &= state(0) + \{\}\\
            state(1) &= state(0) + input(movement(1))\\
            state(2) &= state(0) + input(movement(1)) +input(movement(2))\\
             \vdots  &= \vdots\\
            state(n) &= state(0) + input(movement(1))+\dots+input(movement(n))\\
        \end{aligned}
        \right\}
    \end{aligned}
\end{equation}

\paragraph{State Projection} (eq. \ref{eq:ourStateProjection}) for the \emph{Trajectory} (eq. \ref{eq:ourTrajectoryImplementation}) is given as follow:
\begin{equation}\label{eq:ourStateProjection}
    StateProjection(Trajectory,time) = Trajectory.getMemberByIndex(time+1)
\end{equation}

\begin{note}
    \emph{Movement Automaton} for system (eq. \ref{eq:UASNonlinearModelSimple}) with given (as. \ref{ass:transitionTime}) is established with all related properties (sec. \ref{def:movementAutomaton}).
\end{note}