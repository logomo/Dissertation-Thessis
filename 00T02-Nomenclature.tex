\section*{(C)Nomenclature}
\noindent
This chapter summarize used symbols (tab. \ref{tab:symbols}), acronyms (tab. \ref{tab:acronym}), terminology (tab. \ref{tab:TerminologyExplanation}) and, organizations (tab. \ref{tab:organizations}) mentioned in work. 

\begin{tabularx}{\textwidth}{l|X} 
    Acronym & Meaning\\ \hline\hline
    UAV & Unmanned Aerial Vehicle\\ 
    UAS & Unmanned Autonomous System(including naval vehicles)\\ 
    RPAS & Remotely Piloted Aerial System(lesser degree of autonomy)\\ 
    OPA & Optionally Piloted Aircraft\\
    RPV & Remotely Piloted Vehicle\\
    PIC & Pilot-in-Command\\\hline
    LOS & Line Of Sight\\ 
    VLOS & Visual Line Of Sight\\ 
    BLOS & Behind Line Of Sight\\ \hline
    SAA & Sense And Avoid\\ 
    DAA & Detect And Avoid \\ 
    MAC & Mid-Air Collision \\
    OAC & Off-Air Collision \\
    ABSAA & Airborne Sense and Avoid\\
    GBSAA & Ground Based Sense and Avoid\\
    POA & Preemptive Obstacle Avoidance\\
    ROA & Reactive Obstacle Avoidance \\\hline
    TCAS &Traffic Alert and Collision Avoidance System\\
    ACAS X & Airborne Collision Avoidance System X\\
    ACAS XU & Airborne Collision Avoidance System X for UAS\\
    CD\&R & Collision Detection and Resolution\\ \hline
    RTK & Real-Time Kinematik\\ 
    GPS & Global Positioning System\\ 
    IMU & Internal Measurement Unit\\ 
    LiDAR &  Light Detection and Ranging \\ 
    ADS-B & Automatic Dependent Surveillance – Broadcast\\ 
    GSE & Ground Support Equipment\\\hline
    ATC & Air Traffic Control \\
    ATO & Air Traffic Organization\\
    C2 & Control and Communications\\\hline
    MASPS & Minimum Aviation System Performance Standard\\
    MOPS & Minimum Operational Performance Standard\\
    RVSM & Reduced Vertical Separation Minimum\\
    RHSM & Reduced Horizontal Separation Minimum\\
    SM & Safety Margin \\
    \caption{List of Acronyms}
    \label{tab:acronym}
\end{tabularx}
\newpage
\begin{tabularx}{\textwidth}{l|X}
    Acronym & Organization name \\ \hline\hline
    ICAO & International Civil Aviation Organization (UN)\\
    ITU & International Telecommunication Union (UN)\\\hline
    EASA & European Aviation Safety Agency (EU)\\ 
    JARUS&  Joint Authorities for Regulation of Unmanned Systems (EU)\\ \hline
    FAA & Federal Aviation Administration (USA)\\
    FCC & Federal Communications Commission (USA)\\\hline
    LSTS & Laboratório de Sistemas e Tecnologia Subaquática (PT)\\ 
    FEUP &Faculdade de Engenharia da Universidade do Porto (PT)\\ 
    ITK & Institutt for teknisk kybernetikk NTNU (NO)\\ 
    NTNU& Norges teknisk-naturvitenskapelige universitet (NO)\\ 
    IST & Instituto Superior Técnico - Universidade de Lisboa (PT)\\ 
    HWI & Honeywell International (CZ/USA)\\ 
    \caption{List of Organizations}
    \label{tab:organizations}
\end{tabularx} 
    



\begin{tabularx}{\textwidth}{l|X}  
    Symbol & Explanation \\ \hline\hline
    $A,B,C,D,\dots$ & Capital letters are used for matrices\\
    $A(\dots),B(\dots),\dots$ & Functional matrices, $(\dots)$ denotes parameters\\\hline
    $f(\dots),g(\dots),\dots$ & Vector or scalar functions $(\dots)$ denotes parameters\\
    $\vec{f}(\dots),\vec{g}(\dots),\dots$ & Explicit vector functions, when equation contains both types of scalar and vector functions\\\hline
    $t,x,y,z,\dots$ & Vectors or scalar coefficients \\
    $\vec{x},\vec{o},\vec{g},\dots$ & Explicit vectors, when function contains both types of scalar and vector parameters.\\\hline
    $\theta,\varphi$ & Greek letters denoting angles in radians\\
    \caption{List of symbols}
    \label{tab:symbols}
\end{tabularx} 



\begin{tabularx}{\textwidth}{S{0.22}|X} 
    \toprule
     Terminology &Definition  \\\hline
    \midrule
    \endhead
     Air Traffic Control & A service operated by appropriate authority to promote the safe, orderly, and expeditious flow of air traffic\\\hline
     Aircraft & A device that is used or intended to be used for flight in the air\\\hline
     Airspace & Any portion of the atmosphere sustaining aircraft flight and which has defined boundaries and specified dimensions. Airspace may be classified as to the specific types of flight allowed, rules of operation, and restrictions in accordance with International Civil Aviation Organization standards or State regulation\\\hline
     Civil Aircraft & Aircraft other than public aircraft. \\\hline
     Collision \mbox{Avoidance} & The Sense and Avoid system function where the UAS takes appropriate action to prevent an intruder from penetrating the collision volume. Action is expected to be initiated within a relatively short time horizon before closest point of approach. The collision avoidance function engages when all other modes of separation fail.\\\hline
     Communication Link & The voice or data relay of instructions or information between the UAS pilot and the air traffic controller and other NAS users.\\\hline
     Control Station & The equipment used to maintain control, communicate with, guide, or otherwise pilot an unmanned aircraft.\\\hline
     Crewmember (UAS) & In addition to the crewmembers identified in 14 CFR Part 1, a UAS flightcrew member includes pilots, sensor/payload operators, and visual observers, but may include other persons as appropriate or required to ensure safe operation of the aircraft.\\\hline
     Data Link & A ground-to-air communications system which transmits information via digital coded pulses.\\\hline
     Detect and Avoid & Term used instead of Sense and Avoid in the Terms of Reference for RTCA Special Committee 228. This new term has not been defined by RTCA and may be considered to have the same definition as Sense and Avoid when used in this document.\\\hline
     ICAO & International Civil Aviation Organization is a specialized agency of the United Nations whose objective is to develop the principles and techniques of international air navigation and to foster planning and development of international civil air transport.\\\hline
     Manned Aircraft & Aircraft piloted by a human onboard.\\\hline
     Model Aircraft & An unmanned aircraft that is capable of sustained flight in the atmosphere; flown within visual line-of-sight of the person operating the aircraft and flown for hobby or recreational purposes.\\\hline
     Optionally Piloted Aircraft & An aircraft that is integrated with UAS technology and still retains the capability of being flown by an onboard pilot using conventional control methods.\\\hline
     Pilot-in-Command & Pilot-in-command means the person who:\\
     &1.~ has final authority and responsibility for the operation and safety of the flight;\\
     &2.~has been designated as pilot-in-command before or during the flight; and\\
     &3.~ holds the appropriate category, class, and type rating, if appropriate, for the conduct of the flight.\\\hline
     Public Aircraft & An aircraft operated by a governmental entity (including federal, state, or local governments, and the U.S. Department of Defense and its military branches) for certain purposes\\\hline 
     RTCA & RTCA, Inc. is a private, not-for-profit corporation that develops consensus-based recommendations regarding communications, navigation, surveillance, and air traffic management system issues. RTCA functions as a Federal Advisory Committee. Its recommendations are used by the FAA as the basis for policy, program, and regulatory decisions and by the private sector as the basis for development, investment and other business decisions  (\url{www.rtca.org})\\\hline
     See and Avoid & When weather conditions permit, pilots operating instrument flight rules or visual flight rules are required to observe and maneuver to avoid another aircraft. \\\hline
     Self-Separation & Sense and Avoid system function where the UAS maneuvers within a sufficient time-frame to remain well clear of other airborne traffic.\\\hline 
     Sense and Avoid & The capability of a UAS to remain well clear from and avoid collisions with other airborne traffic. Sense and Avoid provides the functions of self-separation and collision avoidance to establish an analogous capability to “see and avoid” required by manned aircraft.\\\hline
     Unmanned Aircraft & 1.~ A device used or intended to be used for flight in the air that has no onboard pilot. This devise excludes missiles, weapons, or exploding warheads, but includes all classes of airplanes, helicopters, airships, and powered-lift aircraft without an onboard pilot.\\
     &2.~An aircraft that is operated without the possibility of direct human intervention from within or on the aircraft.\\\hline
     Unmanned Aircraft System & An unmanned aircraft and its associated elements related to safe operations, which may include control stations (ground, ship, or air-based), control links, support equipment, payloads, flight termination systems, and launch/recovery equipment.\\
     &~An unmanned aircraft and associated elements (including communications links and the components that control the unmanned aircraft) that are required for the pilot-in-command to operate safely and efficiently in the national airspace system. \\\hline
     Visual Line of Sight & Unaided (corrective lenses and/or sunglasses exempted) visual contact between a pilot-incommand or a visual observer and a UAS sufficient to maintain safe operational control of the aircraft, know its location, and be able to scan the airspace in which it is operating to see and avoid other air traffic or objects aloft or on the ground.\\
     \caption{Terminology}
     \label{tab:TerminologyExplanation}
\end{tabularx}

\begin{note}
\emph{Acronyms} (tab. \ref{tab:acronym}) and \emph{Terminology} (tab. \ref{tab:TerminologyExplanation}) are in compliance with \emph{ICAO}, \emph{FAA}, and, \emph{EASA} definitions, refer to  \cite{huerta2013integration} for more detailed information.
\end{note}