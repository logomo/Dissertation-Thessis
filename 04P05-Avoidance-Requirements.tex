\section{Avoidance Requirements}\label{s:AvoidanceRequirements}

\paragraph{SAA systems} have the following conflicting performance criteria:
\begin{enumerate}
    \item \emph{Energy efficiency} - minimize energy consumption and flight time.
    \item \emph{Trajectory tracking} - stick to the proclaimed trajectory in a mission plan.
    \item \emph{Safety} - avoid harm sources during the mission execution.
\end{enumerate}

\begin{note}
    \emph{Energy efficiency} and \emph{Trajectory Tracking} an optional criteria, while \emph{the safety} is mandatory.
\end{note}

\subsection{Energy Efficiency}\label{s:EnergyEfficiency}
\paragraph{Energy efficiency} can be measured by \emph{cost function} (eq. \ref{eq:consFunctionMeta}), consisting from the \emph{cost of flown trajectory} (eq. \ref{eq:costFunctionExecuted}) and the \emph{expected reach cost} (eq. \ref{eq:costFunctionReachable}) portions. There are optimalizaiton techniques based on \emph{Reach sets} \cite{kurzhanski2001dynamic}. The inputs for the \emph{cost function} are:
\begin{enumerate}
    \item \emph{Time} - current mission time.
    \item \emph{Initial state} - UAS state at the beginning of a \emph{mission}.
    \item \emph{Applied movements} - list of already executed movements.
    \item  \emph{Future movements} - list of movements to be applied in future.
    \item \emph{Current state} - UAS state at \emph{Time}, with current position and orientation included.
    \item \emph{Waypoint} - current goal waypoint.
\end{enumerate}


\begin{equation}\label{eq:consFunctionMeta}
    Cost (t,\dots)= costTrajectoryFlown(t,\dots) + expectedReachCost(t,waypoint,\dots)
\end{equation}


\paragraph{Cost of flown trajectory} (eq. \ref{eq:costFunctionExecuted}) from the \emph{initial state} to the \emph{current state} is calculated as a sum of energy consumed for each movement with the following components:
\begin{enumerate}
    \item \emph{Direct cost} - a cost of consumed energy to execute the movement.
    
    \item \emph{Horizontal cost} - a portion of the direct cost which was used for horizontal steering multiplied by \emph{horizontal penalization}.
    
    \item \emph{Vertical cost} - a portion of the direct cost which was used for ascending/descending multiplied by \emph{vertical penalization}.
\end{enumerate}


\begin{equation}\label{eq:costFunctionExecuted}
   \scriptsize \begin{gathered}costTrajectoryFlown \end{gathered} \left (\begin{gathered}time,\\ initialState,\\ appliedMovements \end{gathered}\right) = \sum_{\begin{gathered}  movement\in\\ appliedMovements\end{gathered}\normalsize} \left(\begin{gathered}
        duration.directCost+\\
        horizontal(cost,penalization)+\\
        vertical(cost,penalization)
   \end{gathered}
   \right) 
   \normalsize
\end{equation}
 
\paragraph{Expected reach cost} (eq. \ref{eq:costFunctionExecuted}) is calculated for a \emph{planned trajectory} portion and a \emph{direct waypoint distance} to the \emph{latest future UAS position}. \emph{Cost of the planned trajectory} is calculated by the same formula as a \emph{cost of flown trajectory} (eq. \ref{eq:costFunctionExecuted}), the initial state is replaced with a  \emph{current state}, and \emph{executed movements} are replaced with \emph{planned movements}.
    
\begin{equation}\label{eq:costFunctionReachable}
    \scriptsize
    \begin{gathered} expectedReachCost\end{gathered} \left(\begin{gathered}time,\\currentState,\\futureMovements,\\waypoint\end{gathered}\right)
    =
    \left (\begin{gathered}distance(futureState,waypoint)+\\ costTrajectoryFlown\left(\begin{gathered}currectState,\\futureMovements\end{gathered}\right) \end{gathered}\right)
    \normalsize
\end{equation}

\begin{note}
    The tuning parameters of cost function are \emph{Horizontal penalization} $\in [0,\infty]$ and \emph{Vertical penalization} $\in [0,\infty]$. Which are used to enhance the outcome of the \emph{cost function}.
    
    Following setup of tuning parameters are used in our simulations:
    \begin{enumerate}
        \item $horizontalPenalization \le verticalPenalization < \infty$ - in  \emph{uncontrolled airspace}, all kind of maneuvers are allowed. Horizontal maneuvers are cheaper for a plane UAS.
        \item $horizontalPenalizaiton < vetricalPenalization = \infty$ - in  \emph{controlled airspace} any kind of horizontal maneuvering must be allowed by UTM.
    \end{enumerate}
    
    The tuning parameters are set up $verticalPenalizaiton \le horizontalPenalizaiton < \infty$ for \emph{copter} UAS in an \emph{uncontrolled airspace}
\end{note}

\subsection{Trajectory Tracking}\label{s:trajectoryTracking}
\paragraph{Trajectory Tracking} is a crucial parameter for \emph{controlled airspace} and is expected to be important in \emph{upcoming UTM systems}. There is a \emph{mission plan} which is compared with \emph{real-time airspace situation} obtained from UAS \emph{Position notifications}. The optimalization based on \emph{Reach Set} is given in \cite{varaiya2000reach}.

\paragraph{Motivation:} \emph{Situation awareness} for modern DAA systems depends on planned trajectory tracking. The main conflict is between \emph{navigation precision} and \emph{situation evaluation}. If the planned trajectory is defined for the \emph{continuous domain}, it takes much effort to calculate collision points. 

The discrete domain of \emph{Movement Automaton} (def. \ref{def:movementAutomaton}) can be used as a \emph{tool for situation awareness}. The main idea is to use \emph{Movement automaton as a predictor for trajectory intersection} \cite{frazzoli2000trajectory,frazzoli2001robust}.

\paragraph{Movement Automaton trajectory tracking:} There is a \emph{reference trajectory} which is used as comparison by \emph{aviation authority} (ATM,UTM) given as:

\begin{multline}\label{eq:MissionReferencetrajectory}
    Reference Trajectory = \{(point_1,time_1), (point_2,time_2),\dots,\\,\dots(point_n,time_n)\} \quad n\in\N^+, point_k\in \R^3
\end{multline}

\emph{Reference Trajectory} (eq. \ref{eq:MissionReferencetrajectory}) is given as set of \emph{points} in \emph{Global Coordinate System} for given \emph{UAS}, \emph{operational time-frame} and other authority depending properties.

The \emph{movement} automaton is executing \emph{Trajectory(initialState,buffer)} where buffer is set of \emph{Movements}. The buffer is changing according to following pattern during \emph{mission} time frame:
\begin{equation}\label{eq:BufferEvolutionMission}
    \begin{aligned}
    \texttt{M}&\texttt{ission Start:}\\
    &buffer=\{planned Movements\},planned= m\\
    \texttt{D}&\texttt{uring Mission:}\\
    &buffer=\{executedMovements,plannedMovements\}, \\
    &\quad executed = n, planned =o\\
    \texttt{M}&\texttt{ission End:}\\
    &buffer=\{executedMovements\},executed = p;\\
    \texttt{M}&\texttt{ovement count constraints:}\\
    &m,n,o,p \in \N^+, n+o =m, m\le p
    \end{aligned}
\end{equation}

At the beginning of the mission (eq. \ref{eq:BufferEvolutionMission}) the buffer is filled with $m$ movements, the \emph{Trajectory} generated from this buffer and \emph{initial state} is \emph{predicted}.

During the \emph{mission execution phase}, the buffer contains \emph{executed movements} and \emph{planned movements}. Trajectory created from the \emph{initial state} and this buffer can be split into:

\begin{enumerate}
    \item\emph{Executed part} - trajectory portion generated and executed from \emph{executed movements}
    
    \item\emph{Predicted part} - trajectory portion generated as a future reference from \emph{planned movements} 
\end{enumerate}

After \emph{mission} execution, there is only \emph{executed movements} The trajectory generated from the \emph{initial state} and buffer is \emph{Executed Trajectory}

\begin{note}
    The part of the trajectory bounded to the past, the part of the trajectory lies in the future. The strong point of \emph{Movement Automaton} is its ability to work as \emph{predictor} and \emph{trajectory memory} at the same time.
    
    By selecting proper time series $t_1\dots t_n$ one can compare future or past segments of trajectory (eq. \ref{eq:BufferEvolutionMission}) with reference (eq. \ref{eq:MissionReferencetrajectory})
\end{note}

\paragraph{Reference Trajectory Deviation} for reference trajectory given by (eq. \ref{eq:MissionReferencetrajectory}) and \emph{Trajectory segment} (Executed/Predicted) (\ref{eq:BufferEvolutionMission}) with existing \emph{State projection function} (eq. \ref{eq:stateprojection}) for \emph{time series} is given as:

\begin{equation}\label{eq:ReferenceTrajectoryDeviation}
    Deviation\left(\begin{aligned}&timeSeries,\\ &Trajectory,\\ &Reference\end{aligned}\right) = \sum_{\scriptsize\begin{gathered}time_i \in\\ timeSeries\end{gathered}\normalsize} \left(\begin{gathered}StateProjection(Trajectory,time_i)\\-\\Reference(time_i)\end{gathered}\right)^2
\end{equation}

\emph{Reference Trajectory Deviation} (eq. \ref{eq:ReferenceTrajectoryDeviation}) is designed as discrete \emph{Mean Square Error} function, where the \emph{timeSeries} is set of \emph{times} from \emph{reference trajectory} $(point_i,time_i)$ pair. The \emph{state projection}.

\paragraph{Trajectory tracking} is defined as \emph{dual minimization problem} where the \emph{primary objective} is depending on the \emph{airspace type}:

\begin{enumerate}
    \item \emph{Reference trajectory deviation} (eq. \ref{eq:ReferenceTrajectoryDeviation}) in \emph{Controlled Airspace}.
    
    \item \emph{Cost of Flown Trajectory} (eq. \ref{eq:consFunctionMeta}) in \emph{Non-controlled Airspace}.
\end{enumerate}

\noindent\emph{Trajectory tracking} can be defined as an optimization problem (eq. \ref{eq:trajectoryTrackingOptimalizaitonProblem}). 

\begin{equation}\label{eq:trajectoryTrackingOptimalizaitonProblem}
    \begin{aligned}
         &\texttt{Minimize:} && costOfTrajectoryFlown &(\ref{eq:consFunctionMeta})\\
         &\texttt{Minimize:} && reference Trajectory Deviation
         &(\ref{eq:ReferenceTrajectoryDeviation})\\
         &\texttt{Subject to:} &&&\\
         &&&UAS\text{ }Dynamics&(\ref{eq:madSystemdefinition})\\
         &&&MovementAutomatonControl&\begin{gathered}(\ref{eq:madInitialState})\\\vdots\\(\ref{eq:madStateProjection})\end{gathered}\\
         &&&Mission&(\ref{eq:missionAbstractSet})\\
         &&&KnownWorld(t)&(\ref{eq:knownWorldSet})\\
         &&&SafetyMargin(t)&(\ref{eq:safetyMarginAbstract})\\
         &&&HardFlightConstraints&(\ref{eq:HFlightConstraints})\\
         &&&SoftFlightConstraints&(\ref{eq:SFlightConstraints})\\
         &&&HardSpaceConstraints&(\ref{eq:hardSpaceConstraints})\\
         &&&SoftSpaceConstraints&(\ref{eq:softSpaceConstraints})\\
    \end{aligned}
\end{equation}

The \emph{reference trajectory} is given by \emph{mission} set of \emph{waypoints}. The \emph{UAS} dynamics with specific \emph{Movement Automaton} goal is to fly in \emph{Known World} to keep \emph{Safety Margin} form  \emph{Obstacle Space}. The \emph{Obstacle space} is a result of \emph{Data fusion} procedure (sec. \ref{s:dataFusionDefinition}) combining the \emph{sensor reading}, information sources, and constraints.

\paragraph{Feasible Trajectory} for \emph{tracking problem} (eq. \ref{eq:trajectoryTrackingOptimalizaitonProblem}) is a trajectory which in addition to \emph{basic obstacle problem} (sec. \ref{s:BasicProblemDefinition}) keeps deviation from the \emph{reference trajectory} under certain threshold:
\begin{equation}\label{feasibleTrajectoryCondition}
    Deviation(timeSeries, Trajectory, Reference) \le performanceMargin \in \R^+
\end{equation}

Feasible trajectory condition (eq. \ref{feasibleTrajectoryCondition}) is used as \emph{margin} for airworthiness, and \emph{Deviation} is used as a performance indicator further in this work.


\subsection{Safety}\label{s:Safety}
\noindent \emph{Safety} is very broad term there are following incidents which can occur and will be discussed in (app. \ref{s:safetyMarginCalculation}) \emph{Safety margin} is a broad term describing minimal distance to the center of intruder/adversary, a surface of the obstacle, a boundary of the protected area.

\paragraph{Controlled airspace safety}
\noindent  Safety for \emph{controlled airspace} in given \emph{flight level} is given a list of incidents:
\begin{enumerate}
    \item \emph{Soft constrained zone breach} - UAS fly to \emph{soft constraint body} or \emph{hard constraint protection zone}, these incidents can happen, and have least avoidance priority. 
    \item \emph{Hard constrained zone breach} - UAS fly to \emph{hard constraint protection zone}, typical geo-fencing, restricted airspace breaches. 
    \item \emph{Well-clear breach} - UAS fly to \emph{well-clear barrel} without impacting other aircraft, via the wake turbulence or other induced physical phenomenon and vice-versa. This type of breaches are allowed in case of inevitable \emph{near miss situations} or \emph{Clash incidents} 
    \item \emph{Near miss situation}- UAS fly to \emph{near miss} cone/barrel, inducing wake turbulence, or other kinds of flight disturbance. These incidents are allowed at very low rate (near $1:10^6$).
    \item \emph{Clash incident} - UAS body impacts another aircraft hull/propulsion/steering systems and components. This kind of incidents are very severe, and they should never happen.
\end{enumerate}
\begin{note}
    It is assumed that flight level in controlled airspace is free of terrain, static ground obstacles, the climb/descent maneuvers are not covered in this work, and they are topic for multiple dissertation theses. For more information refer to ICAO document 4444.
\end{note}

\newpage
The \emph{relation} for breach of \emph{safety margin} and \emph{body margin} for each object is given in  (tab. \ref{tab:controlledAirspaceViolations}):

\begin{tabularx}{\textwidth}{S{0.20}||S{0.35}|S{0.35}}
    Violation of: & Safety Margin & Body Margin\\\hline\hline
    Soft constraint & none & Soft constraint zone breach \\\hline 
    Hard constraint & Soft constrained zone breach & Hard constrained zone breach\\\hline 
    Intruder  & \begin{minipage}{0.30\textwidth}Well clear breach,\\ Near Miss situation\end{minipage} & Clash incident\\
    \caption{Controlled airspace margins violations incidents.}
    \label{tab:controlledAirspaceViolations}
\end{tabularx}

\paragraph{Uncontrolled airspace safety}
\noindent Safety for \emph{uncontrolled airspace} is applied in $F/G$ class of airspace, which is given as airspace between the \emph{ground} level and \emph{other airspace prevalence}.
\begin{note}{Clarification of controlled/uncontrolled airspace:}
    \begin{itemize}
        \item[1.] \emph{Class F} airspace is given as space between the ground level or water surface, and it is constrained up to the 500 feet above ground level.
        \item[2.] \emph{Class C} airspace or \emph{Controlled airspace} is considered starting at first flight level, which is given by Air traffic control zone starting at least at 300 feet from highest ATC zone ground point. It is measured based on Above Sea Level altitude. \emph{This is not a problem in Portugal, because of terrain diversity, but its a huge problem in the Netherlands}.
        \item[3.] \emph{Class A} airspace starts at ground level and covers the majority of airport infrastructure - this is not a problem, because it`s modeled as hard constraint, which is unbreakable in non controlled airspace.
    \end{itemize}
\end{note}

\noindent Safety for \emph{uncontrolled airspace} is given a list of incidents:
\begin{enumerate}
    \item \emph{Soft constrained zone breach} - UAS fly into the \emph{soft constrained zone} or \emph{hard constraint protection} zone, it is allowed to happen on very a low rate.
    
    \item \emph{Hard constrained zone breach} - UAS fly into the \emph{hard constrained zone}, only airports and critical infrastructure are considered as hard constraints; it is not allowed to happen.  
    
    \item \emph{Intruder near miss} - UAS fly into \emph{other aircraft near miss zone}, it is allowed to happen on a very low rate in case of other intermediate threats with higher priority.
    
    \item \emph{Intruder clash} - UAS has contact with other man-made aircraft, it is not allowed to happen.
    
    \item \emph{Adversary clash} - UAS has contact with another flying object which did not intentionally avoided UAS. (\emph{Bird strike, Differential games,etc..} is out of the scope of this thesis).
    
    \item \emph{Structure harm} - UAS fly close to structure, and its propulsion can damage/harm structure.
    
    \item \emph{Structure crash} - UAS fly into a natural/man-made ground structure (building, tree, human).
    
    \item \emph{Ground harm} - UAS fly close to the ground, and its propulsion reflection can impact Ground or UAS.
    
    \item \emph{Ground collision} - UAS collides with the ground.
\end{enumerate} 

The \emph{relation} for breach of  \emph{safety margin} and \emph{body margin} for each object is given in (tab. \ref{tab:uncontrolledAirspaceViolations}):

\begin{tabularx}{\textwidth}{S{0.20}||S{0.35}|S{0.35}}
    Violation of: & Safety Margin & Body Margin\\\hline\hline
    Soft constraint & none & Soft constraint zone breach \\\hline 
    Hard constraint & Soft constrained zone breach & Hard constrained zone breach\\\hline 
    Intruder  & Intruder near miss & Intruder clash \\\hline
    Adversary & none & Adversary clash \\\hline
    Structure & Structure harm & Structure crash \\\hline
    Ground    & Ground harm    & Ground crash \\
    \caption{Non-controlled airspace margins violations incidents.}
    \label{tab:uncontrolledAirspaceViolations}
\end{tabularx}