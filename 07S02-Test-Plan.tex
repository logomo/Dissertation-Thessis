\section{Test Plan} \label{s:testPlan}

\noindent The \emph{Avoidance requirements} are given in (sec. \ref{s:AvoidanceRequirements}), namely:

\begin{enumerate}
    \item\emph{Safety Margin Enforcement} (sec. \ref{s:Safety}) - keep UAS safe depending on situation.
    
    \item\emph{Path Tracking} (sec. \ref{s:trajectoryTracking}) - track mission is given by a set of \emph{waypoints} in the manner of \emph{Energy Efficiency} (sec. \ref{s:EnergyEfficiency}).
\end{enumerate}

These are given as nominal behavior (sec. \ref{s:aviudabceGridRun}), further enhanced by rule-based behavior (sec. \ref{sec:ruleImplementation}).

The \emph{Navigation requirements}, out of this scope, are given in (sec. \ref{s:navigationRequirements}). These are satisfied by \emph{Mission Control Run} (sec. \ref{s:missionControlRun}).


\subsection{Testing approach}\label{s:testingApproach}

\noindent The purpose of this section is to show complex scenarios, not unit testing of framework functionality. The focus is on \emph{borderline} cases for typical situations in an \emph{expected environment}. The \emph{mode switch} between \emph{Navigation} and \emph{Emergency Avoidance}.

\noindent The \emph{Tests} are designed to focus on particular functionality in specific \emph{operational environment} with main \emph{obstacle/weather/intruder feature} with environment induced \emph{constraints}. There is also \emph{UTM} factor and \emph{Navigation penalty}.

\paragraph{Operational Environment} is classified according to:

\begin{enumerate}
    \item \emph{Operation space} - important for \emph{Low Altitude Operations}, the difficulty of \emph{Avoidance Maneuvers} is proportionally increasing with \emph{Obstacle density}. There are following main categories
    \begin{enumerate}[a.]
        \item \emph{Rural environment} - the relief and man-made structures are sparsely spread around the \emph{operation space};  the UAS is operating on \emph{very low altitude} ($\le 50$ feet).
        
        \item \emph{Urban environment} - the concentration of the man-made structures are much higher, and they are more incorporated info land relief pattern, the UAS is operating on \emph{very low altitude}.
        
        \item \emph{Open air} - the concentration of ground structures is very low, the concentration of \emph{cooperative} and \emph{non-cooperative intruders} is increased, the UAS is operating in altitude ranging from \emph{50 feet} to \emph{space border}. This brings us to:
    \end{enumerate}
    
    \item \emph{Airspace category} -  when \emph{Operation Space} pattern is categorized as \emph{Open air} and depending on \emph{altitude above mean sea level}. The UTM  is \emph{designed authority} for controlled airspace in current \emph{F/G class airspace}.
    \begin{enumerate}[a.]
        \item \emph{Controlled} - Open air where authority is present. The cases when \emph{Authority} is not enforced due to the UTM malfunction, $C2$ link loss or other cause are not considered.
        
        \item \emph{Non-Controlled} - Open air operation space where is no central arbiter to determine or enforce traffic attendants behavior.
        
    \end{enumerate}
\end{enumerate}

\paragraph{Static obstacles:}  Static obstacles with various features detectable by main \emph{LiDAR} sensor. The main purpose is to show avoidance capabilities combined with heavy restrictions imposed by \emph{soft} and \emph{hard} constraints. The original purpose of our approach was to provide robust framework for static obstacle avoidance. Three tests with increasing obstacle density and navigation complexity are delivered.

\paragraph{Operational Space Constraints} depends mainly on the  \emph{operational environment}.  The standard set of constraints were taken into account for our test cases:
\begin{enumerate}
    \item \emph{Rural, Urban environment (low altitude)} are geo-fencing zones, ground (hard constraints), non-controlled airspace altitudes (soft constraints).
    
    \item \emph{Non-controlled airspace constraints (open air)} are  geo-fencing zones (hard constraints), restricted airspace (hard constraint), weather (soft/hard constraint), controlled airspace (hard constraint), very low altitude border (soft constraints).
    
    \item \emph{Controlled airspace constraints (open air)} are  restricted airspace (hard constraint), weather (soft/hard constraint), non-controlled airspace boundary (hard constraints), UTM Directives (hard constraints).
\end{enumerate}

\paragraph{Air Traffic Attendants:} 
\begin{enumerate}
    \item \emph{Non-cooperative UAS} (Intruder) -  there are some intruders with some degree of authority, size and \emph{severity}. There were three test cases for non-cooperative intrudes. Non-cooperative Intruders can be categorized as following based on behavior:
    \begin{enumerate}[a.]
        \item\emph{Chaotic} intruders usually tend to behave unpredictable, for example, bird or \emph{UAS in distress}, for this type of intruders \emph{Maneuver Uncertainty  Intersection Model} is used (app. \ref{s:uncertaintyIntersection}).
        
        \item\emph{Harmonic} intruder usually follows long straight paths, for example, UAS converging to waypoint, for this type of intruder \emph{Body Volume Intersection Model} is used. (app. \ref{s:bodyvolumeIntersection}).
    \end{enumerate}

    \emph{Cooperative UAS} (Intruder) -  there are cooperative intruders who are obeying authority (UTM) or follow \emph{common consensus}. The work focus on \emph{UTM} authority implementation in four test cases. These test cases are reflecting the traffic management situations essential for successful UTM collision management
\end{enumerate}
    
\paragraph{Weather} impose  \emph{soft} and \emph{hard} space constraint, which can be moving or static. The \emph{soft constraint avoidance} is covered by \emph{hard constraint avoidance}. The \emph{static constrained area} is covered by \emph{static obstacle avoidance} capability due to the \emph{data fusion procedure} \cite{gomola2017probabilistic}. The only case which is not covered is \emph{Moving constrained area}; small constraints can be covered by intruder models. The ideal candidate is a \emph{storm}, because it covers quite a large area, the clouds are constantly moving, and severity is changing with time.



\paragraph{UTM:} The \emph{UAS Traffic Management} service should be implemented in \emph{controlled airspace} by 2035. It is necessary to study impact of UTM services on the \emph{Detect and Avoid} systems like ours. 

The most basic service is \emph{Identity provider} which should be implemented by 2020. 

Then there are \emph{location services}, which are necessary for coordinated collision avoidance, these were implemented in our solution up to necessary level for \emph{Rules Of the Air} implementation.

\noindent \emph{Mission tracking} is service tracking deviations from \emph{declared mission plan} and \emph{actual execution}. These statistics were used in all tests to track deviations from the reference trajectory.

\emph{Directives} for \emph{Traffic management} and \emph{Collision prevention} are implemented  as the functional life cycle of  \emph{Position notification} (sec. \ref{sec:positionNotification}), \emph{Collision Case} (sec. \ref{sec:collisionCase}) for UTM. The directive handling is implemented as \emph{Rule engine} (sec. \ref{sec:ruleImplementation}) on UAS side.


\paragraph{Navigation:} Navigation algorithm is depending on \emph{Navigation mode}. UAS is usually in \emph{Navigation mode} most of the time, despite this fact, UAS was forced into \emph{Emergency Avoidance Mode} most of the time in test cases. The navigation complexity has been divined into following categories:

\begin{enumerate}
    \item \emph{Open space} - UAS has visibility to goal waypoint most of the time; there are no traps.
    
    \item \emph{Hidden waypoint} - UAS does not have visibility to goal waypoint, most of the time; there are irregular traps sometimes.
    
    \item \emph{Maze solving} - UAS line of sight for goal waypoint is hindered by multiple obstacles, there are irregular traps often.
    
    \item \emph{Rule following} - UAS navigation capabilities are constrained by rule enforcement.
\end{enumerate}

\subsection{Test Cases Summary}\label{s:testCaseSummary}

\noindent \emph{Test cases} are summarized in (tab. \ref{tab:testCasesSummary}).

\begin{table}[H]
\scriptsize
\centering
\begin{tabular}{c||c|c|c|c|c|c}
\textit{\begin{tabular}[c]{@{}c@{}}Test Case\\ Name\end{tabular}}                                               & \textit{\begin{tabular}[c]{@{}c@{}}Operational\\ Environment\end{tabular}}                                               & \textit{\begin{tabular}[c]{@{}c@{}}Air Traffic\\Attendants\end{tabular}} & \textit{Weather}                                           & \textit{\begin{tabular}[c]{@{}c@{}}UTM\end{tabular}} & \textit{Navigation}                                        & \textit{Scenario}                                                                     \\\hline\hline
\begin{tabular}[c]{@{}c@{}}Building \\ Avoidance\end{tabular}                                                              & \begin{tabular}[c]{@{}c@{}}Non-controlled \\ (Rural) \\ $\begin{gathered}4 \times buildings \end{gathered}$\end{tabular} & \begin{tabular}[c]{@{}c@{}}-\end{tabular}                      & -                                                        & -                                                                & \begin{tabular}[c]{@{}c@{}}Open \\ space\end{tabular}      & \begin{tabular}[c]{@{}c@{}}Fly mission around \\ four buildings\end{tabular}          \\\hline
Slalom                                                                                                                      & \begin{tabular}[c]{@{}c@{}}Non-controlled\\(Rural) \\ $\begin{gathered}14 \times buildings\end{gathered}$\end{tabular}                                                             & \begin{tabular}[c]{@{}c@{}}-\end{tabular}                     & -                                                        & -                                                                & \begin{tabular}[c]{@{}c@{}}Hidden \\ waypoint\end{tabular} & \begin{tabular}[c]{@{}c@{}}Navigate to hidden\\ waypoint\end{tabular}                 \\\hline
Maze                                                                                                                       & \begin{tabular}[c]{@{}c@{}}Non-controlled\\(Urban) \\ $\begin{gathered}30 \times buildings\end{gathered}$\end{tabular}         & \begin{tabular}[c]{@{}c@{}}-\end{tabular}                      & -                                                        & -                                                                & \begin{tabular}[c]{@{}c@{}}Maze \\ structure\end{tabular}  & \begin{tabular}[c]{@{}c@{}}Solve maze with\\ multiple curves\end{tabular}             \\\hline
Storm                                                                                                                       & \begin{tabular}[c]{@{}c@{}}Non-controlled\\(Rural)\\ $0 \times buildings$\end{tabular} & -                                                                              & Storm                                                        & -                                                                & \begin{tabular}[c]{@{}c@{}}Open\\ Space\end{tabular}       & \begin{tabular}[c]{@{}c@{}}Avoid approaching\\ storm\end{tabular}                     \\\hline
\begin{tabular}[c]{@{}c@{}}Emergency\\ Converging\end{tabular}                                                              & \begin{tabular}[c]{@{}c@{}}Non-controlled\\ (Open air)\end{tabular} & \begin{tabular}[c]{@{}c@{}}Non-cooperative\\ UAS (1x)\end{tabular}                                & \begin{tabular}[c]{@{}c@{}}-\end{tabular} & -                                                       & \begin{tabular}[c]{@{}c@{}}Open\\ Space\end{tabular}       & \begin{tabular}[c]{@{}c@{}}Converging situation\\ resolution w. o. UTM\end{tabular} \\\hline
\begin{tabular}[c]{@{}c@{}}Emergency\\ Head on\end{tabular}                                                               & \begin{tabular}[c]{@{}c@{}}Non-controlled\\ (Open air)\end{tabular} & \begin{tabular}[c]{@{}c@{}}Non-cooperative\\ UAS (1x)\end{tabular}                                & \begin{tabular}[c]{@{}c@{}}-\end{tabular} & -                                                       & \begin{tabular}[c]{@{}c@{}}Open\\ Space\end{tabular}       & \begin{tabular}[c]{@{}c@{}}Head on situation\\ resolution w. o.  UTM\end{tabular}    \\\hline
\begin{tabular}[c]{@{}c@{}}Emergency\\ Multiple\end{tabular}                                                              & \begin{tabular}[c]{@{}c@{}}Non-controlled\\ (Open air)\end{tabular} & \begin{tabular}[c]{@{}c@{}}Non-cooperative\\ UAS (3x)\end{tabular}                          & \begin{tabular}[c]{@{}c@{}}-\end{tabular} & -                                                       & \begin{tabular}[c]{@{}c@{}}Open\\ Space\end{tabular}       & \begin{tabular}[c]{@{}c@{}}Multi-collision case\\ resolution w. o.  UTM\end{tabular} \\\hline
\begin{tabular}[c]{@{}c@{}}Rule-based\\ Converging\end{tabular}                                                             & \begin{tabular}[c]{@{}c@{}}Controlled \\ (Open air)\end{tabular}    & \begin{tabular}[c]{@{}c@{}}Cooperative\\ UAS(1x)\end{tabular}                                & \begin{tabular}[c]{@{}c@{}}-\end{tabular}        & Full                                                              & \begin{tabular}[c]{@{}c@{}}Follow\\ Rules\end{tabular}     & \begin{tabular}[c]{@{}c@{}}Converging situation\\ resolution with UTM\end{tabular}    \\\hline
\begin{tabular}[c]{@{}c@{}}Rule-based\\ Head on\end{tabular}                                                                & \begin{tabular}[c]{@{}c@{}}Controlled \\ (Open air)\end{tabular}    & \begin{tabular}[c]{@{}c@{}}Cooperative\\UAS(1x)\end{tabular}                                & \begin{tabular}[c]{@{}c@{}}-\end{tabular}        & Full                                                              & \begin{tabular}[c]{@{}c@{}}Follow\\ Rules\end{tabular}     & \begin{tabular}[c]{@{}c@{}}Head on situation\\ resolution with UTM\end{tabular}       \\\hline
\begin{tabular}[c]{@{}c@{}}Rule-based\\ Multiple\end{tabular}                                                              & \begin{tabular}[c]{@{}c@{}}Controlled \\ (Open air)\end{tabular}    & \begin{tabular}[c]{@{}c@{}}Cooperative\\UAS(3x)\end{tabular}                                & \begin{tabular}[c]{@{}c@{}}-\end{tabular}        & Full                                                              & \begin{tabular}[c]{@{}c@{}}Follow\\ Rules\end{tabular}     & \begin{tabular}[c]{@{}c@{}}Multi-collision case \\ resolution with UTM\end{tabular}   \\\hline
\begin{tabular}[c]{@{}c@{}}Rule-based\\ Overtake\end{tabular}                                                              & \begin{tabular}[c]{@{}c@{}}Controlled \\ (Open air)\end{tabular}    & \begin{tabular}[c]{@{}c@{}}Cooperative\\ UAS (1x)\end{tabular}                                & \begin{tabular}[c]{@{}c@{}}-\end{tabular}        & Full                                                              & \begin{tabular}[c]{@{}c@{}}Follow\\ Rules\end{tabular}     & \begin{tabular}[c]{@{}c@{}}Overtake by UAS\\ different speed ratio\end{tabular}      
\end{tabular}
\normalsize
\caption{Test Cases Summary.}
\label{tab:testCasesSummary}
\end{table}