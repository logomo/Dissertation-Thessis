\subsection{\secState{R}Reach Set Performance Criteria}\label{s:ReachSetPerformanceCriteria}

\paragraph{Motivation:} The need to Make \emph{Reach Set} scalable approach. This may be a problem due the \emph{Expansion rate}. \emph{Reach set} represented as a \emph{Trajectory Tree} (eq. \ref{eq:trajectoryTree}) for Avoidance Grid with $layerCount$ and Movement automaton with $movementCount$, the \emph{Node count} is given as:

\begin{equation}\label{eq:fullReachSetNodeCount}
    1+ \left(\sum_{i\in\{1\dots layerCount\}} (movementCount)^i\right)
\end{equation}

\emph{This scaling} is not feasible for \emph{Avoidance Grid} with many layers ($< 10$) or \emph{Movement Set} with many movements ($< 9$). There is need for \emph{Reduced Reach set calculation}.

\paragraph{Core Performance Criteria:} The scaling factor (eq. \ref{eq:fullReachSetNodeCount}) shows that there are going to be many trajectories. The main point is that not every trajectory in \emph{Reach Set} are giving us \emph{maneuverability advantage}. Our expectations lies in following \emph{Performance Requirements}:

\begin{enumerate}
    \item \emph{Reach set} must \emph{Cover} maximum of the \emph{possible unique maneuvers} in  \emph{Avoidance Grid}.
    \item \emph{Trajectories} in \emph{Reach Set} should be smoothest possible to prevent cargo damage / UAS wear.
\end{enumerate}

\paragraph{Trajectory footprint:} Discrete space of \emph{Avoidance Grid} is organized in cells. \emph{Cell} is minimal space portion accessible by \emph{property binding}. There is need to know if two trajectories contribution to \emph{Maneuverability} in this environment. 

Each trajectory passes trough space in \emph{Avoidance Grid}. If there exists a method to extract unique identifier for each \emph{trajectory passed cells}, we can compare two trajectories \emph{Coverage} in \emph{Avoidance Grid}.


\begin{definition}[Trajectory footprint] \label{def:trajectoryFootprint}
    For \emph{Trajectory} from \emph{Reach set} (def. \ref{def:ContainedReducedReachSet}) defined for \emph{Avoidance Grid} has membership function. \emph{Membership Function} returns \emph{ordered set of passing cells}:
    
    \begin{equation}\label{eq:setOfPassedCells}
        footprint
        \left(\begin{gathered}
            Trajectory,\\
            AvoidanceGrid    
        \end{gathered}\right) 
        = 
        \left\{
            \begin{aligned}
                cell \in& AvoidanceGrid:\\ &isMember(trajectory,cell)
            \end{aligned}
        \right\}
    \end{equation}
    
    \noindent Then we can define equality function for $Trajectory_1$ and $Trajectory_2$, as comparison of their footprints in common \emph{Avoidance Grid} as follow:
    
    \begin{equation}\label{eq:TrajectoryEquality}
        isEqual\left(\begin{gathered}
        Trajectory_1,\\Trajectory_2,\\AvoidanceGrid  
        \end{gathered}\right):
        \begin{cases}
            \left(\begin{gathered}
                footprint(Trajectory_1,\dots)\\
                = \\
                footprint(Trajectory_2,\dots)
            \end{gathered}\right)
            &:true\\
            Otherwise&: false\\
        \end{cases}
    \end{equation}
\end{definition}

\begin{note}
    Depending on \emph{Movement Automaton`s} movement set and \emph{Avoidance Grid} parameters, there can be multiple \emph{trajectories} which are equal.
\end{note}

\paragraph{Coverage set:} Now it is possible to create set of unique \emph{trajectory footprints} due to \emph{footprint function} (eq. \ref{eq:setOfPassedCells}). Similarly there is a possibility to create \emph{Reach set skeleton} containing unique trajectories, by using 
\emph{equality function} (eq. \ref{eq:TrajectoryEquality}).
\emph{Coverage set} is sufficient for now.

\begin{definition}[Coverage Set]\label{def:CoverageSet}
    \emph{Coverage set} (\ref{eq:CoverageSet}) is defined for \emph{Avoidance Grid} and \emph{Reach Set} pair as set of unique \emph{Trajectory footprints}:
    
    \begin{equation}\label{eq:CoverageSet}
        CoverageSet\left(\begin{gathered}AvoidanceGrid,\\ReachSet\end{gathered}\right)= \left\{
            footprint\left(\begin{gathered}     
                Trajectory,\\AvoidanceGrid
            \end{gathered}\right):
            \begin{aligned}
                \forall &Trajectory\\ &\in Reach Set
            \end{aligned}\right\}
    \end{equation}
\end{definition}

\paragraph{Coverage set properties:} Trajectory footprint (eq. \ref{eq:setOfPassedCells})is not \emph{bijection}, neither \emph{injection} for $ReachSet \to CoverageSet$. This implies following properties:

\begin{enumerate}
    \item Equal \emph{Reach Sets} in same \emph{Avoidance Grid} have equal \emph{Coverage Sets}.
    
    \item Equal \emph{Coverage Sets} does not imply \emph{Reach Set} equality.
    
    \item For two {Coverage Sets} there is a possibility to compare their member count to create coverage ratio.
\end{enumerate}

The second \emph{Property} gives us a preposition that there is a possibility of \emph{Reach Set Reduction} without loosing 
\emph{Coverage}.  

\begin{definition}[Coverage Ratio]\label{def:coverageRatio}
    \emph{Coverage Ratio} is a ratio of \emph{Coverage Set Member Count} between two \emph{Reach Sets}. Reach set with \emph{lesser count of unique Trajectories} is considered as \emph{Reduced Reach Set}.  Reach set with \emph{greater Count of unique Trajectories} is considered as \emph{Reference Reach Set}.
    
    \begin{equation}\label{eq:CoverageRatio} 
        \begin{gathered}
            referenceCoverage=|CoverageSet(ReferenceReachSet,AvoidanceGrid)|\\
            reducedCoverage=|CoverageSet(ReducedReachSet,AvoidanceGrid)|\\
            CoverageRatio = \frac{reducedCoverage}{referenceCoverage}\in[0,1]
        \end{gathered}
    \end{equation}
\end{definition}

\begin{note}
    \emph{Reference Reach Set} is usually \emph{Full Reach Set} containing all possible trajectories in space contained by \emph{Avoidance Grid}. 
    
    In case \emph{Full Reach Set} can not be computed, Avoidance Grid is too large, most complex \emph{Reach Set} is used as \emph{Reference Reach Set}.
\end{note}

\paragraph{Trajectory smoothness:} Trajectory other than straight line have some changes in \emph{UAS} heading. 

The goal is to minimize \emph{Maneuvering} of UAS, because:
\begin{enumerate}
    \item \emph{Every Heading Change} needs to be reported to \emph{UTM}.
    \item \emph{Sharp Maneuvering} can damage cargo/wear UAS.
    \item \emph{Often course changes} makes \emph{Intruder prediction} harder for other Civil General Aviation.
\end{enumerate}

For this purpose \emph{Smoothness Metric} needs to be applied for \emph{Reach Set} or \emph{Trajectory}.  In case of \emph{Movement Automaton Control} two distinguish \emph{Movement Sets} can be introduced:\emph{Smooth} nad \emph{Chaotic} movements set with following properties:

\begin{equation}\label{eq:ChaoticSmoothMovementSetDefinition}
    \begin{gathered}
        MovementSet=SmoothMovements\cup ChaoticMovements\\
        SmoothMovements\cap ChaoticMovements = \varnothing\\
        |SmoothMovements| > 0, \quad |ChaoticMovements|>0
    \end{gathered}
\end{equation}

Then \emph{Smoothnes clasificator} for $Trajectory(initialState,buffer)$ can be defined as $isSmooth$ and \emph{Smooth Movement Counter} function as $smoothCount$ like follow:

\begin{equation}\label{eq:SmoothnesClasificatorCounter}
    \begin{gathered}
        isSmooth(movement)=
        \begin{cases}
            movement \in SmoothMovements &:1\\
            movement \in ChaoticMovements &:0
        \end{cases}\\
        smoothCount(Trajectory(\dots,buffer))= 
        \begin{aligned}
            \sum isS&mooth(movement),\\& \forall movement \in Buffer
        \end{aligned}
    \end{gathered}
\end{equation}


\begin{definition}[Smoothness Rating for Trajectory]\label{def:SmoothnessRatingForTrajectory} 
    \emph{Smoothness} for trajectory generated by \emph{Movement Automaton} for some \emph{Initial State} with some \emph{Movement Buffer}, under assumption of \emph{Smooth and Chaotic Movement Set} split (eq. \ref{eq:ChaoticSmoothMovementSetDefinition}), with existing \emph{classification} and \emph{counter} functionals (eq. \ref{eq:SmoothnesClasificatorCounter}) is given as follows:

    \begin{equation}
        Smoothness(Trajectory(\dots,buffer)) = \frac{isSmooth(Trajectory)}{movementCount(Trajectory)} \in [0,1] 
    \end{equation}
    
    \noindent For \emph{Trajectory} with $buffer=\varnothing$ \emph{Smoothness} is given as 1.
\end{definition}
