%% fcup-thesis.tex -- document template for PhD theses at FCUP
%%
%% Copyright (c) 2015 João Faria <joao.faria@astro.up.pt>
%%
%% This work may be distributed and/or modified under the conditions of
%% the LaTeX Project Public License, either version 1.3c of this license
%% or (at your option) any later version.
%% The latest version of this license is in
%%     http://www.latex-project.org/lppl.txt
%% and version 1.3c or later is part of all distributions of LaTeX
%% version 2005/12/01 or later.
%%
%% This work has the LPPL maintenance status "maintained".
%%
%% The Current Maintainer of this work is
%% João Faria <joao.faria@astro.up.pt>.
%%
%% This work consists of the files listed in the accompanying README.

%% SUMMARY OF FEATURES:
%%
%% All environments, commands, and options provided by the `ut-thesis'
%% class will be described below, at the point where they should appear
%% in the document.  See the file `ut-thesis.cls' for more details.
%%
%% To explicitly set the pagestyle of any blank page inserted with
%% \cleardoublepage, use one of \clearemptydoublepage,
%% \clearplaindoublepage, \clearthesisdoublepage, or
%% \clearstandarddoublepage (to use the style currently in effect).
%%
%% For single-spaced quotes or quotations, use the `longquote' and
%% `longquotation' environments.


%%%%%%%%%%%%         PREAMBLE         %%%%%%%%%%%%

%%  - Default settings format a final copy (single-sided, normal
%%    margins, one-and-a-half-spaced with single-spaced notes).
%%  - For a rough copy (double-sided, normal margins, double-spaced,
%%    with the word "DRAFT" printed at each corner of every page), use
%%    the `draft' option.
%%  - The default global line spacing can be changed with one of the
%%    options `singlespaced', `onehalfspaced', or `doublespaced'.
%%  - Footnotes and marginal notes are all single-spaced by default, but
%%    can be made to have the same spacing as the rest of the document
%%    by using the option `standardspacednotes'.
%%  - The size of the margins can be changed with one of the options:
%%     . `narrowmargins' (1 1/4" left, 3/4" others),
%%     . `normalmargins' (1 1/4" left, 1" others),
%%     . `widemargins' (1 1/4" all),
%%     . `extrawidemargins' (1 1/2" all).
%%  - The pagestyle of "cleared" pages (empty pages inserted in
%%    two-sided documents to put the next page on the right-hand side)
%%    can be set with one of the options `cleardoublepagestyleempty',
%%    `cleardoublepagestyleplain', or `cleardoublepagestylestandard'.
%%  - Any other standard option for the `report' document arclass can be
%%    used to override the default or draft settings (such as `10pt',
%%    `11pt', `12pt'), and standard LaTeX packages can be used to
%%    further customize the layout and/or formatting of the document.

%% *** Add any desired options. ***
%PDF
%\documentclass[a4,narrowmargins,12pt,oneside,draft,onehalfspaced,singlespacednotes]{fcup-thesis}
%\documentclass[a4,narrowmargins,12pt,oneside,onehalfspaced,singlespacednotes]{fcup-thesis}
%Print
%\documentclass[draft,a4,narrowmargins,12pt,twoside,openright,onehalfspaced,singlespacednotes]{fcup-thesis}
\documentclass[a4,narrowmargins,12pt,twoside,openright,onehalfspaced,singlespacednotes]{fcup-thesis}

%% *** Add \usepackage declarations here. ***
%% The standard packages `geometry' and `setspace' are already loaded by
%% `ut-thesis' -- see their documentation for details of the features
%% they provide.  In particular, you may use the \geometry command here
%% to adjust the margins if none of the ut-thesis options are suitable
%% (see the `geometry' package for details).  You may also use the
%% \setstretch command to set the line spacing to a value other than
%% single, one-and-a-half, or double spaced (see the `setspace' package
%% for details).
% Overfull statements
\pretolerance=150
\setlength{\emergencystretch}{3em}
% Overfull end
\usepackage[english]{babel}
\usepackage{lipsum}
\usepackage[utf8]{inputenc}


%%% Additional useful packages
%%% ----------------------------------------------------------------
\usepackage{array}
\usepackage{amsmath}  
\usepackage{amssymb}
\usepackage{amsfonts}
\DeclareFontFamily{OT1}{pzc}{}
\DeclareFontShape{OT1}{pzc}{m}{it}{<-> s * [0.900] pzcmi7t}{}
\DeclareMathAlphabet{\mathpzc}{OT1}{pzc}{m}{it}
\usepackage{amsthm}      
\usepackage[ruled,algochapter]{algorithm2e}
\usepackage{algorithmic}
\usepackage{bm}
\usepackage[mathscr]{euscript}
\usepackage{graphicx}       
\usepackage{psfrag}         
\usepackage{fancyvrb}    
\usepackage{float}
\usepackage{ltablex}
\usepackage[square,sort,comma,numbers]{natbib}        
\usepackage{bbding}         
\usepackage{dcolumn}        
\usepackage{booktabs} 
\usepackage{multirow}
\usepackage{paralist}     
\usepackage{ifdraft}  
\usepackage{indentfirst}    
\usepackage[nottoc,notlof,notlot]{tocbibind}
\usepackage{url}
\usepackage{tabularx}
\usepackage{subcaption}
\usepackage[unicode]{hyperref}
\usepackage{xcolor}

\hypersetup{pdftitle=LiDAR obstacle detection and avoidance, 
            pdfauthor=Alojz Gomola,
            colorlinks=false,
            urlcolor=blue,
            pdfstartview=FitH,
            pdfpagemode=UseOutlines,
            pdfnewwindow,
            breaklinks
          }
\usepackage{array}
\newcolumntype{L}[1]{>{\raggedright\let\newline\\\arraybackslash\hspace{0pt}}m{#1}}
\newcolumntype{C}[1]{>{\centering\let\newline\\\arraybackslash\hspace{0pt}}m{#1}}
\newcolumntype{R}[1]{>{\raggedleft\let\newline\\\arraybackslash\hspace{0pt}}m{#1}}         
\newcolumntype{B}{X}
\newcolumntype{S}[1]{>{\hsize=#1\textwidth}X}

\newcommand{\FIGDIR}{./Pics}    %%% directory containing figures
\newcommand{\twolinecellr}[2][r]{%
  \begin{tabular}[#1]{@{}r@{}}#2\end{tabular}}
\newcommand{\secState}[1]{
	\ifdraft{(#1) }{}
}
\theoremstyle{plain}
\newtheorem{theorem}{Theorem}
\newtheorem{lemma}[theorem]{Lemma}
\newtheorem{proposition}[theorem]{Proposition}

\theoremstyle{plain}
\newtheorem{definition}{Definition}
\newtheorem{problem}{Problem}
\newtheorem{example}{Example}
\newtheorem{assumption}{Assumption}

\theoremstyle{remark}
\newtheorem*{corollary}{Corollary}
\newtheorem*{note}{Note}




\newenvironment{dokaz}{
  \par\medskip\noindent
  \textit{Proof}.
}{
\newline
\rightline{\SquareCastShadowBottomRight}
}

\newenvironment{constraints}[1]{
  \par\medskip\noindent
  \textit{Constraints #1} \\
}{
\newline
\rightline{\SquareCastShadowBottomRight}
}


%\bibliographystyle{plainnat}     %% Author (year) style
\bibliographystyle{unsrt}        %% [number] style
\setcitestyle{square}

% Section  3.7 Challenge list
\newif\ifproblemchallenge   %# Build block for problem challenges
\problemchallengetrue       %# Show comments

\newcommand{\R}{\mathbb{R}}
\newcommand{\N}{\mathbb{N}}

\DeclareMathOperator{\pr}{\textsf{P}}
\DeclareMathOperator{\E}{\textsf{E}\,}
\DeclareMathOperator{\var}{\textrm{var}}
\DeclareMathOperator{\sd}{\textrm{sd}}


\newcommand{\T}[1]{#1^\top}        

\newcommand{\goto}{\rightarrow}
\newcommand{\gotop}{\stackrel{P}{\longrightarrow}}
\newcommand{\maon}[1]{o(n^{#1})}
\newcommand{\abs}[1]{\left|{#1}\right|}
\newcommand{\dint}{\int_0^\tau\!\!\int_0^\tau}
\newcommand{\isqr}[1]{\frac{1}{\sqrt{#1}}}
\newcommand{\norm}[1]{\left\lVert#1\right\rVert}


\newcommand{\pulrad}[1]{\raisebox{1.5ex}[0pt]{#1}}
\newcommand{\mc}[1]{\multicolumn{1}{c}{#1}}
\newcommand{\TBD}[1]{\color{red}\emph{--TBD:}#1\color{black}}

%%%%%%%%%%%%%%%%%%%%%%%%%%%%%%%%%%%%%%%%%%%%%%%%%%%%%%%%%%%%%%%%%%%%%%%%
%%                                                                    %%
%%                   ***   I M P O R T A N T   ***                    %%
%%                                                                    %%
%%  Fill in the following fields with the required information:       %%
%%   - \degree{...}       name of the degree obtained                 %%
%%   - \department{...}   name of the graduate department             %%
%%   - \gradyear{...}     year of graduation                          %%
%%   - \author{...}       name of the author                          %%
%%   - \title{...}        title of the thesis                         %%
%%%%%%%%%%%%%%%%%%%%%%%%%%%%%%%%%%%%%%%%%%%%%%%%%%%%%%%%%%%%%%%%%%%%%%%%

%% *** Change this example to appropriate values. ***
\degree{Doctor of Philosophy}
\department{Departamento de Matem\'{a}tica}
\gradyear{2019}
\author{Alojz Gomola}
\title{Obstacle Avoidance Framework based on Reach Sets}

%% *** NOTE ***
%% Put here all other formatting commands that belong in the preamble.
%% In particular, you should put all of your \newcommand's,
%% \newenvironment's, \newtheorem's, etc. (in other words, all the
%% global definitions that you will need throughout your thesis) in a
%% separate file and use "\input{filename}" to input it here.


%% *** Adjust the following settings as desired. ***

%% List only down to subsections in the table of contents;
%% 0=chapter, 1=section, 2=subsection, 3=subsubsection, etc.
\setcounter{tocdepth}{3}

%% Make each page fill up the entire page.
\flushbottom


%%%%%%%%%%%%      MAIN  DOCUMENT      %%%%%%%%%%%%

\begin{document}


%%%%%%%%%%%%%%%%%%%%%%%%%%%%%%%%%%%%%%%%%%%%%%%%%%%%%%%%%%%%%%%%%%%%%%%%
%%  Put your Chapters here; the easiest way to do this is to keep     %%
%%  each chapter in a separate file and `\include' all the files.     %%
%%  Each chapter file should start with "\chapter{ChapterName}".      %%
%%  Note that using `\include' instead of `\input' will make each     %%
%%  chapter start on a new page, and allow you to format only parts   %%
%%  of your thesis at a time by using `\includeonly'.                 %%
%%%%%%%%%%%%%%%%%%%%%%%%%%%%%%%%%%%%%%%%%%%%%%%%%%%%%%%%%%%%%%%%%%%%%%%%

%% *** Include chapter files here. ***

\setcounter{chapter}{6}
\setcounter{section}{0}

%07-Simulations
    \chapter{Simulations}\label{Simulations}


\noindent The chapter presents the set of simulations developed according to a test plan (sec. \ref{s:testPlan}). Test configuration (sec. \ref{sec:testingConfiguration}) targets at exercising and evaluating proposed framework. The test cases are grouped in following sections:
\begin{enumerate}
    \item \emph{Non-cooperative test cases} (sec. \ref{s:noncooperativeTestCases}).
    \item \emph{Cooperative test cases} (sec. \ref{s:cooperativeTestCases}).
    \item \emph{Test cases conclusion} (sec. \ref{s:testCasesConclusion}).
    \item \emph{Reach set approximation performance tests} (sec. \ref{sec:reducedReachSetPerformance}).
\end{enumerate}
    \section{\secState{D}Test Plan} \label{s:testPlan}

\noindent The \emph{Avoidance requirements} are given in (sec. \ref{s:AvoidanceRequirements}), namely:

\begin{enumerate}
    \item\emph{Safety Margin Enforcement} (sec. \ref{s:Safety}) - keep UAS safe depending on situation.
    
    \item\emph{Path Tracking} (sec. \ref{s:trajectoryTracking}) - track mission is given by a set of \emph{waypoints} in the manner of \emph{Energy Efficiency} (sec. \ref{s:EnergyEfficiency}).
\end{enumerate}

These are given as nominal behavior (sec. \ref{s:aviudabceGridRun}), further enhanced by rule-based behavior (sec. \ref{sec:ruleImplementation}).

The \emph{Navigation requirements}, out of this scope, are given in (sec. \ref{s:navigationRequirements}). These are satisfied by \emph{Mission Control Run} (sec. \ref{s:missionControlRun}).


\subsection{\secState{D}Testing approach}\label{s:testingApproach}

\noindent The purpose of this section is to show complex scenarios, not unit testing of framework functionality. The focus is on \emph{borderline} cases for typical situations in an \emph{expected environment}. The \emph{mode switch} between \emph{Navigation} and \emph{Emergency Avoidance}.

\noindent The \emph{Tests} are designed to focus on particular functionality in specific \emph{operational environment} with main \emph{obstacle/weather/intruder feature} with environment induced \emph{constraints}. There is also \emph{UTM} factor and \emph{Navigation penalty}.

\paragraph{Operational Environment} is classified according to:

\begin{enumerate}
    \item \emph{Operation space} - important for \emph{Low Altitude Operations}, the difficulty of \emph{Avoidance Maneuvers} is proportionally increasing with \emph{Obstacle density}. There are following main categories
    \begin{enumerate}[a.]
        \item \emph{Rural environment} - the relief and man-made structures are sparsely spread around the \emph{operation space};  the UAS is operating on \emph{very low altitude} ($\le 50$ feet).
        
        \item \emph{Urban environment} - the concentration of the man-made structures are much higher, and they are more incorporated info land relief pattern, the UAS is operating on \emph{very low altitude}.
        
        \item \emph{Open air} - the concentration of ground structures is very low, the concentration of \emph{cooperative} and \emph{non-cooperative intruders} is increased, the UAS is operating in altitude ranging from \emph{50 feet} to \emph{space border}. This brings us to:
    \end{enumerate}
    
    \item \emph{Airspace category} -  when \emph{Operation Space} pattern is categorized as \emph{Open air} and depending on \emph{altitude above mean sea level}. The UTM  is \emph{designed authority} for controlled airspace in current \emph{F/G class airspace}.
    \begin{enumerate}[a.]
        \item \emph{Controlled} - Open air where authority is present. The cases when \emph{Authority} is not enforced due to the UTM malfunction, $C2$ link loss or other cause are not considered.
        
        \item \emph{Non-Controlled} - Open air operation space where is no central arbiter to determine or enforce traffic attendants behavior.
        
    \end{enumerate}
\end{enumerate}

\paragraph{Static obstacles:}  Static obstacles with various features detectable by main \emph{LiDAR} sensor. The main purpose is to show avoidance capabilities combined with heavy restrictions imposed by \emph{soft} and \emph{hard} constraints. The original purpose of our approach was to provide robust framework for static obstacle avoidance. Three tests with increasing obstacle density and navigation complexity are delivered.

\paragraph{Operational Space Constraints} depends mainly on the  \emph{operational environment}.  The standard set of constraints were taken into account for our test cases:
\begin{enumerate}
    \item \emph{Rural, Urban environment (low altitude)} are geo-fencing zones, ground (hard constraints), non-controlled airspace altitudes (soft constraints).
    
    \item \emph{Non-controlled airspace constraints (open air)} are  geo-fencing zones (hard constraints), restricted airspace (hard constraint), weather (soft/hard constraint), controlled airspace (hard constraint), very low altitude border (soft constraints).
    
    \item \emph{Controlled airspace constraints (open air)} are  restricted airspace (hard constraint), weather (soft/hard constraint), non-controlled airspace boundary (hard constraints), UTM Directives (hard constraints).
\end{enumerate}

\paragraph{Air Traffic Attendants:} 
\begin{enumerate}
    \item \emph{Non-cooperative UAS} (Intruder) -  there are some intruders with some degree of authority, size and \emph{severity}. There were three test cases for non-cooperative intrudes. Non-cooperative Intruders can be categorized as following based on behavior:
    \begin{enumerate}[a.]
        \item\emph{Chaotic} intruders usually tend to behave unpredictable, for example, bird or \emph{UAS in distress}, for this type of intruders \emph{Maneuver Uncertainty  Intersection Model} is used (app. \ref{s:uncertaintyIntersection}).
        
        \item\emph{Harmonic} intruder usually follows long straight paths, for example, UAS converging to waypoint, for this type of intruder \emph{Body Volume Intersection Model} is used. (app. \ref{s:bodyvolumeIntersection}).
    \end{enumerate}

    \emph{Cooperative UAS} (Intruder) -  there are cooperative intruders who are obeying authority (UTM) or follow \emph{common consensus}. The work focus on \emph{UTM} authority implementation in four test cases. These test cases are reflecting the traffic management situations essential for successful UTM collision management
\end{enumerate}
    
\paragraph{Weather} impose  \emph{soft} and \emph{hard} space constraint, which can be moving or static. The \emph{soft constraint avoidance} is covered by \emph{hard constraint avoidance}. The \emph{static constrained area} is covered by \emph{static obstacle avoidance} capability due to the \emph{data fusion procedure} \cite{gomola2017probabilistic}. The only case which is not covered is \emph{Moving constrained area}; small constraints can be covered by intruder models. The ideal candidate is a \emph{storm}, because it covers quite a large area, the clouds are constantly moving, and severity is changing with time.



\paragraph{UTM:} The \emph{UAS Traffic Management} service should be implemented in \emph{controlled airspace} by 2035. It is necessary to study impact of UTM services on the \emph{Detect and Avoid} systems like ours. 

The most basic service is \emph{Identity provider} which should be implemented by 2020. 

Then there are \emph{location services}, which are necessary for coordinated collision avoidance, these were implemented in our solution up to necessary level for \emph{Rules Of the Air} implementation.

\noindent \emph{Mission tracking} is service tracking deviations from \emph{declared mission plan} and \emph{actual execution}. These statistics were used in all tests to track deviations from the reference trajectory.

\emph{Directives} for \emph{Traffic management} and \emph{Collision prevention} are implemented  as the functional life cycle of  \emph{Position notification} (sec. \ref{sec:positionNotification}), \emph{Collision Case} (sec. \ref{sec:collisionCase}) for UTM. The directive handling is implemented as \emph{Rule engine} (sec. \ref{sec:ruleImplementation}) on UAS side.
\newpage
\paragraph{Navigation:} Navigation algorithm is depending on \emph{Navigation mode}. UAS is usually in \emph{Navigation mode} most of the time, despite this fact, UAS was forced into \emph{Emergency Avoidance Mode} most of the time in test cases. The navigation complexity has been divined into following categories:

\begin{enumerate}
    \item \emph{Open space} - UAS has visibility to goal waypoint most of the time; there are no traps.
    
    \item \emph{Hidden waypoint} - UAS does not have visibility to goal waypoint, most of the time; there are irregular traps sometimes.
    
    \item \emph{Maze solving} - UAS line of sight for goal waypoint is hindered by multiple obstacles, there are irregular traps often.
    
    \item \emph{Rule following} - UAS navigation capabilities are constrained by rule enforcement.
\end{enumerate}

\newpage
\subsection{\secState{D}Test Cases Summary}\label{s:testCaseSummary}

\noindent \emph{Test cases} are summarized in (tab. \ref{tab:testCasesSummary}).

\begin{table}[H]
\scriptsize
\centering
\begin{tabular}{c||c|c|c|c|c|c}
\textit{\begin{tabular}[c]{@{}c@{}}Test Case\\ Name\end{tabular}}                                               & \textit{\begin{tabular}[c]{@{}c@{}}Operational\\ Environment\end{tabular}}                                               & \textit{\begin{tabular}[c]{@{}c@{}}Air Traffic\\Attendants\end{tabular}} & \textit{Weather}                                           & \textit{\begin{tabular}[c]{@{}c@{}}UTM\end{tabular}} & \textit{Navigation}                                        & \textit{Scenario}                                                                     \\\hline\hline
\begin{tabular}[c]{@{}c@{}}Building \\ Avoidance\end{tabular}                                                              & \begin{tabular}[c]{@{}c@{}}Non-controlled \\ (Rural) \\ $\begin{gathered}4 \times buildings \end{gathered}$\end{tabular} & \begin{tabular}[c]{@{}c@{}}-\end{tabular}                      & -                                                        & -                                                                & \begin{tabular}[c]{@{}c@{}}Open \\ space\end{tabular}      & \begin{tabular}[c]{@{}c@{}}Fly mission around \\ four buildings\end{tabular}          \\\hline
Slalom                                                                                                                      & \begin{tabular}[c]{@{}c@{}}Non-controlled\\(Rural) \\ $\begin{gathered}14 \times buildings\end{gathered}$\end{tabular}                                                             & \begin{tabular}[c]{@{}c@{}}-\end{tabular}                     & -                                                        & -                                                                & \begin{tabular}[c]{@{}c@{}}Hidden \\ waypoint\end{tabular} & \begin{tabular}[c]{@{}c@{}}Navigate to hidden\\ waypoint\end{tabular}                 \\\hline
Maze                                                                                                                       & \begin{tabular}[c]{@{}c@{}}Non-controlled\\(Urban) \\ $\begin{gathered}30 \times buildings\end{gathered}$\end{tabular}         & \begin{tabular}[c]{@{}c@{}}-\end{tabular}                      & -                                                        & -                                                                & \begin{tabular}[c]{@{}c@{}}Maze \\ structure\end{tabular}  & \begin{tabular}[c]{@{}c@{}}Solve maze with\\ multiple curves\end{tabular}             \\\hline
Storm                                                                                                                       & \begin{tabular}[c]{@{}c@{}}Non-controlled\\(Rural)\\ $0 \times buildings$\end{tabular} & -                                                                              & Storm                                                        & -                                                                & \begin{tabular}[c]{@{}c@{}}Open\\ Space\end{tabular}       & \begin{tabular}[c]{@{}c@{}}Avoid approaching\\ storm\end{tabular}                     \\\hline
\begin{tabular}[c]{@{}c@{}}Emergency\\ Converging\end{tabular}                                                              & \begin{tabular}[c]{@{}c@{}}Non-controlled\\ (Open air)\end{tabular} & \begin{tabular}[c]{@{}c@{}}Non-cooperative\\ UAS (1x)\end{tabular}                                & \begin{tabular}[c]{@{}c@{}}-\end{tabular} & -                                                       & \begin{tabular}[c]{@{}c@{}}Open\\ Space\end{tabular}       & \begin{tabular}[c]{@{}c@{}}Converging situation\\ resolution w. o. UTM\end{tabular} \\\hline
\begin{tabular}[c]{@{}c@{}}Emergency\\ Head on\end{tabular}                                                               & \begin{tabular}[c]{@{}c@{}}Non-controlled\\ (Open air)\end{tabular} & \begin{tabular}[c]{@{}c@{}}Non-cooperative\\ UAS (1x)\end{tabular}                                & \begin{tabular}[c]{@{}c@{}}-\end{tabular} & -                                                       & \begin{tabular}[c]{@{}c@{}}Open\\ Space\end{tabular}       & \begin{tabular}[c]{@{}c@{}}Head on situation\\ resolution w. o.  UTM\end{tabular}    \\\hline
\begin{tabular}[c]{@{}c@{}}Emergency\\ Multiple\end{tabular}                                                              & \begin{tabular}[c]{@{}c@{}}Non-controlled\\ (Open air)\end{tabular} & \begin{tabular}[c]{@{}c@{}}Non-cooperative\\ UAS (3x)\end{tabular}                          & \begin{tabular}[c]{@{}c@{}}-\end{tabular} & -                                                       & \begin{tabular}[c]{@{}c@{}}Open\\ Space\end{tabular}       & \begin{tabular}[c]{@{}c@{}}Multi-collision case\\ resolution w. o.  UTM\end{tabular} \\\hline
\begin{tabular}[c]{@{}c@{}}Rule-based\\ Converging\end{tabular}                                                             & \begin{tabular}[c]{@{}c@{}}Controlled \\ (Open air)\end{tabular}    & \begin{tabular}[c]{@{}c@{}}Cooperative\\ UAS(1x)\end{tabular}                                & \begin{tabular}[c]{@{}c@{}}-\end{tabular}        & Full                                                              & \begin{tabular}[c]{@{}c@{}}Follow\\ Rules\end{tabular}     & \begin{tabular}[c]{@{}c@{}}Converging situation\\ resolution with UTM\end{tabular}    \\\hline
\begin{tabular}[c]{@{}c@{}}Rule-based\\ Head on\end{tabular}                                                                & \begin{tabular}[c]{@{}c@{}}Controlled \\ (Open air)\end{tabular}    & \begin{tabular}[c]{@{}c@{}}Cooperative\\UAS(1x)\end{tabular}                                & \begin{tabular}[c]{@{}c@{}}-\end{tabular}        & Full                                                              & \begin{tabular}[c]{@{}c@{}}Follow\\ Rules\end{tabular}     & \begin{tabular}[c]{@{}c@{}}Head on situation\\ resolution with UTM\end{tabular}       \\\hline
\begin{tabular}[c]{@{}c@{}}Rule-based\\ Multiple\end{tabular}                                                              & \begin{tabular}[c]{@{}c@{}}Controlled \\ (Open air)\end{tabular}    & \begin{tabular}[c]{@{}c@{}}Cooperative\\UAS(3x)\end{tabular}                                & \begin{tabular}[c]{@{}c@{}}-\end{tabular}        & Full                                                              & \begin{tabular}[c]{@{}c@{}}Follow\\ Rules\end{tabular}     & \begin{tabular}[c]{@{}c@{}}Multi-collision case \\ resolution with UTM\end{tabular}   \\\hline
\begin{tabular}[c]{@{}c@{}}Rule-based\\ Overtake\end{tabular}                                                              & \begin{tabular}[c]{@{}c@{}}Controlled \\ (Open air)\end{tabular}    & \begin{tabular}[c]{@{}c@{}}Cooperative\\ UAS (1x)\end{tabular}                                & \begin{tabular}[c]{@{}c@{}}-\end{tabular}        & Full                                                              & \begin{tabular}[c]{@{}c@{}}Follow\\ Rules\end{tabular}     & \begin{tabular}[c]{@{}c@{}}Overtake by UAS\\ different speed ratio\end{tabular}      
\end{tabular}
\normalsize
\caption{Test Cases Summary.}
\label{tab:testCasesSummary}
\end{table}
    	\subsection{(W) Performance Evaluation}\label{s:performanceEvaluation}
\paragraph{Evaluation method:} \emph{Test cases} were evaluated according to performance requirements defined in (sec. \ref{s:AvoidanceRequirements}). The method was tracking critical parameter for \emph{Safety} (sec . \ref{s:Safety}) (primary) and \emph{Trajectory Tracking} (sec. \ref{s:trajectoryTracking}) (secondary) including \emph{Energy Efficiency} (sec. \ref{s:EnergyEfficiency}).

\paragraph{Safety Margin Performance Evaluation:} The \emph{safety of UAS} is main concern of \emph{DAA system}. The common concept of \emph{safety margin} is evaluated. 

The \emph{threat} is multidimensional, there are often multiple \emph{static obstacles, intruders} or \emph{weather constraints}. To reduce the multidimensional threats to one dimensional value \emph{crash distance} concept is used:

\begin{multline}\label{eq:crashDistance}
    crashDistance(t) =  distance(UAScenter(t),threat) \\\text{  where \emph{selection criterion} is:  }\\ \min \left\{\begin{gathered}\left( \begin{gathered} distance(UAScenter(t),threat)-\dots\\\dots-threat.SafetyMargin\end{gathered}\right)\\:\forall threat \in KnownWorld (t)\end{gathered}\right\}
\end{multline}

The \emph{crash distance} (eq. \ref{eq:crashDistance}) for given time is evaluated as shortest distance between UAS center and threat. The threat origins from known world (sec. \ref{s:KnownWorld}). The \emph{threat} have safety margin. The distance to safety margin is used as prioritization criterion in our test cases (tab. \ref{tab:testCasesSummary}).


The \emph{safety margin} evolution over time (eq. \ref{eq:safetyMarginOverTimeEvolution}) is calculated similar to \emph{crash distance}. The most dangerous threat is selected based on \emph{distance to safety margin} criterion. The value of \emph{safety margin} property is then used.

\begin{multline}\label{eq:safetyMarginOverTimeEvolution}
    safety Margin(t) =  threat.SafetyMargin\\\text{  where \emph{selection criterion} is:  }\\ \min \left\{\begin{gathered}\left( \begin{gathered} distance(UAScenter(t),threat)-\dots\\\dots-threat.SafetyMargin\end{gathered}\right)\\:\forall threat \in KnownWorld (t)\end{gathered}\right\}
\end{multline}

The \emph{distance to safety margin} (eq. \ref{eq:distanceToSafetyMargin}) is calculated as a difference between \emph{crash distance} (eq. \ref{eq:crashDistance}) and \emph{safety margin} (eq. \ref{eq:safetyMarginOverTimeEvolution}). The \emph{acceptance criteria} for safety is \emph{distance to safety margin} $\ge$ 0.

\begin{equation}\label{eq:distanceToSafetyMargin}
    distanceToSafetyMargin(t) =  crashDistance(t) - safetyMargin(t) \ge 0
\end{equation}

\paragraph{Distance to Safety Margin:}
\begin{enumerate}
    \item \emph{Minimal}
    \item \emph{Maximal}
\end{enumerate}

\paragraph{Breach Indicator:} 
\begin{enumerate}
    \item \emph{Yes/No}
    \item \emph{Distance to Safety Margin Reference}
\end{enumerate}

\paragraph{Trajectory Tracking Evaluation:}

\begin{enumerate}
    \item \emph{Waypoint reach:}
    \begin{enumerate}[a.]
        \item \emph{Yes/No}
        \item \emph{UAS ID}
        \item \emph{Trajectory tracking figure reference}
    \end{enumerate}
    \item \emph{Reference deviation:}
    \begin{enumerate}[a.]
        \item \emph{Waypoint ID}
        \item \emph{Peak deviation value}
    \end{enumerate}
    \item \emph{Acceptable deviation:}
    \begin{enumerate}[a.]
        \item \emph{Yes/No}
        \item \emph{Trajectory tracking performance table reference}
    \end{enumerate}
\end{enumerate}
    
    %Observations moved to Conclusion - no longer in use
    %\section{(W) Simulation Observations Summary}\label{sec:SimulationObservationsSummary}
    \noindent Use summary of this section in Conclusion and future work on specific 

\subsection{(W) Static Obstacles Avoidance}\label{sec:staticObstacleAvoidanceSummary}
    \noindent TODO: main points of building, slalom, maze scenarios - link artifacts and performance criteria

\subsection{(W) Constraints Avoidance}\label{sec:constraintAvoidanceSummary}
    \noindent TODO constraints main point, main loop processing, breach chance ? etc...
    
\subsection{(W) Unsupervised Intruder Avoidance}\label{sec:unsupervisedIntruderAvoidance}
    \noindent TODO emergency intruder avoidance emphasis navigation, Emergency avoidance contribution main points

\subsection{(W) Supervised (UTM) Intruder Avoidance}\label{sec:supervisedIntruderAvoidance}
    \noindent TODO UTM contribution main points
    
 

%% This adds a line for the Bibliography in the Table of Contents.
\addcontentsline{toc}{chapter}{Bibliography}
%% *** Set the bibliography style. ***
%% (change according to your preference/requirements)
%\bibliographystyle{plain}
%% *** Set the bibliography file. ***
%% ("thesis.bib" by default; change as needed)
\bibliography{thesis}

%% *** NOTE ***
%% If you don't use bibliography files, comment out the previous line
%% and use \begin{thebibliography}...\end{thebibliography}.  (In that
%% case, you should probably put the bibliography in a separate file and
%% `\include' or `\input' it here).

\end{document}
