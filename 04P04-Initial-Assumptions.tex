\newpage
\section{Initial Assumptions} \label{s:initialAssumptions}
    \noindent\emph{Initial assumptions} are the following:

    \begin{assumption}
        {Filtered sensor readings are available}\label{ass:filteredSensors}.\\
        \emph{SensorObservation} (\ref{eq:observationClassification}) for a given \emph{position}, \emph{time} returns classification of \emph{Space} which is corresponding with the real situation.
    \end{assumption}
    
    \begin{assumption}
        {There are no moving obstacles}\label{ass:noMovingObstacles}.\\
        The initial \emph{Space Classification Function} (\ref{eq:spaceCassificationFunction}) is static for all observation times $t \in (-\infty,\infty)$. Moreover, there are no \emph{intruders} or \emph{adversaries} present. 
    \end{assumption}

    \begin{assumption}
        {The movement takes place in the unrestricted airspace.}\label{ass:openAir}
    \end{assumption}

    \begin{assumption}
        {The mission consists of a set of reachable waypoints}\label{ass:reachableWaypoints}.\\
        For specific \emph{UAS system} (\ref{eq:vehicleModelAbstract}) and  \emph{Mission} (\ref{eq:missionAbstractSet}), there exists a control which satisfies \emph{Waypoint passing} (\ref{eq:waypointPassingFunction}) criterion and \emph{SafetyMargin} (\ref{eq:safetyMarginAbstract}) condition.
    \end{assumption}
    
    \begin{assumption}
        {The UAS is moving with constant velocity}\label{ass:constantVelocity}.\\
        For given \emph{UAS system} (\ref{eq:vehicleModelAbstract}) there is a subset of state $velocity(t)\subset x(t)$ which contains velocity parameters. Then there exist transformation function $LinearVelocity(\circ)$ which maps $velocity(t)$  to \emph{linear velocity} $\in\R^1$. For time $t$ in \emph{missionStart} and \emph{missionEnd} in \emph{Mission} (\ref{eq:missionAbstractSet}) constraint (\ref{eq:constantVelocityAssmunption}) with some $constantVelocity \in \R^+$ holds.

        \begin{equation}\label{eq:constantVelocityAssmunption}
            \forall t \in \left[\begin{aligned}&missionStart,\\&missionEnd\end{aligned}\right]: LinearVelocity(velocity(t))=constantVelocity
        \end{equation}    
    \end{assumption}

    \begin{note}
        \emph{Initial assumptions} \ref{ass:filteredSensors}., \ref{ass:noMovingObstacles}., \ref{ass:openAir}., \ref{ass:reachableWaypoints}, and \ref{ass:constantVelocity}. will be relaxed in \emph{Incremental problem definition}.
    \end{note}