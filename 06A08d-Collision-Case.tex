\subsection{\secState{R}Collision Case}\label{sec:collisionCase}


\paragraph{Collision Case Purpose:} There is a need for detection and tracking of possible \emph{controlled airspace traffic attendants} collisions.  The presented \emph{collision case structure} (tab. \ref{tab:collisionCase}) is a minimalist reflection of \emph{ATM} requirements. Following aspects of  \emph{collision case} life cycle are explained in this section:
\begin{enumerate}
    \item \emph{Base terminology} - the definition of \emph{enforcement procedure} and difference between \emph{Resolution} and \emph{Mandate} from UTM authority. The \emph{severity issue} is open.
    
    \item \emph{Calculation of single case for single decision frame} - step by step calculation and threat evaluation. Prequel to the \emph{life-cycle}.
    
    \item \emph{Life cycle} gives outlook on how collision case data are handled through a longer period, notably: \emph{Opening}, \emph{collision point handling}, \emph{safety margin handling}, and, \emph{Closure}.
    
    \item \emph{Merge procedure for multiple cases in a single cluster} - the naive \emph{merge procedure} to solve \emph{multiple collision cases} via the \emph{virtual roundabout}.
\end{enumerate}


\paragraph{Resolution/Mandate Enforcement:}
\emph{Enforcement procedure} is consisting from \emph{Threat detection phase} and \emph{Mitigation phase}. The \emph{mitigation phase} is a time interval when \emph{UTM} decision is enforced. The decision the UTM is enforcing is delivered in the form of \emph{Resolutions} and \emph{Mandates}.


A \emph{Resolution} is an order from the \emph{UTM} authority which is followed by subjected UAS. The \emph{subjected UAS} can determine own behavior to some extent. When there is an emerging threat or another destructive event, like a new non-cooperative adversary, the UAS is allowed to broke \emph{resolution}.  

A \emph{Mandate} is an order from the \emph{UTM} authority which can not be broken at any cost. The example of the \emph{mandate}: UAS is flying in the airspace, the passenger in distress needs it to safely land. The UAS must obey mandate even at the event of own destruction.

\paragraph{Threat Severity Evaluation:} The threat severity evaluation is omitted partially, all threats are considered as equal. All commands from \emph{UTM authority} will be considered as \emph{resolutions}.

\paragraph{Calculation procedure:} Collision case is calculated for two \emph{Registered UAS systems} in \emph{Unified UTM time-frame}. The \emph{unified UTM time-frame} is a short period in future when the anticipated situations are predicted. 

\paragraph{1\textsuperscript{st}} The \emph{position} and \emph{orientation} are adjusted according to the \emph{mission plan}. Our implementation uses \emph{Movement Automaton} as a predictor:
\begin{equation}
\begin{gathered}
    adjustedPosition = Position(Trajectory(notifiedState, futureMovements))\\
    adjustedOrientation = Orientation(Trajectory(notifiedState, futureMovements))
\end{gathered}
\end{equation}

\noindent For other cases standard linear prediction can be used:
\begin{equation}
    \begin{gathered}
        adjustedPosition = notificationPosition \times notificationVelocity \times timeDifference\\
        adjustedOrientation = notificationOrientation
    \end{gathered}
\end{equation}

\paragraph{2\textsuperscript{nd}} The \emph{maneuverability}, \emph{craft category}, \emph{registration ID} are taken from \emph{position notification}.

\paragraph{3\textsuperscript{rd}} \emph{Collision case check procedure} goes like follows:
\begin{enumerate}
    \item \emph{Operation space checks} - the controlled airspace and flight level must match for proceeding.
    
    \item \emph{Maneuverability/Category check} - the maneuverability and UAS category must match. If there is mismatch, then the right of the way is forced to the vehicle with higher priority.
\end{enumerate}

\paragraph{4\textsuperscript{th}} \emph{Linear Intersection test} is designed to calculate \emph{closest distance} and \emph{time} of \emph{linear trajectory projections}.  First, for given \emph{velocity} and \emph{position} for UAS1 and UAS2 the helper variables are calculated:
\begin{equation}
    \begin{aligned}
        A&=\norm{velocity_1}^2\\
        B&=2*({velocity_1}^T\times position_1-{velocity_2}^T\times position2)\\
        C&=2\times {velocity_1}^T *velocity_2\\
        D&=2*({velocity_2}^T \times position_2 - {velocity_2}' \times {position_1});\\
        E&=\norm{velocity_2}^2;\\
        F&=\norm{position_1}^2 + \norm{position_2}^2;\\
    \end{aligned}\\
\end{equation}
\noindent Then the projection parameters can be calculated:
\begin{equation}
    \begin{aligned}
    time& = \frac{-B-D}{2 \times A- 2 \times C+ 2 \times E}\\
    destination_i &= position_i + velocity_i \times time, \quad i \in \{1,2\}\\
    collisionPoint &= \frac{destination_1 + destination_2}{2}\\
    collisionDistance &= \norm{destination_1 - destination_2}\\
    \end{aligned}
\end{equation}

\noindent If $time < 0$ the trajectories are diverging from each other (because the closest points already occurred). The procedure ends, the \emph{collision flag} is not raised.

If $time > time Margin$ the trajectories will get close to each other, but in further future and changes are anticipated. The procedure ends, the \emph{collision flag} is not raised.

If $0 \le time \le timeMargin$ the trajectories are converging to each other and distance needs to be checked. If $distance \le collisionMargin$ then \emph{collision flag} is raised and \emph{collision point} is set.

\begin{note}
    \emph{Collision Margin} is some number which is determined based on aircraft category and maneuverability. Our work defines collision margin as follow:
    \begin{equation}
        collisionMargin = \forall situation : \max\left\{\begin{gathered}safetyMargin(situation,UAS1)\\ +safetyMargin(situation,UAS2) \end{gathered}\right\}
    \end{equation}
    
    Where the \emph{safety margin} for every possible situation is evaluated for both \emph{UAS}.
\end{note}

\paragraph{5\textsuperscript{th}} The \emph{trajectory} intersection is \emph{Movement Automaton} specific collision detection method. Its based on the assumption that \emph{UTM} has the following information from \emph{mission plan}:
\begin{enumerate}
    \item \emph{UAS state} - not only \emph{position}, \emph{orientation}, and, \emph{velocity} vectors, but other mathematical model parameters mandatory for \emph{movement automaton}.
    
    \item \emph{Movement Automaton} - movement automaton for our UAS system, so that UTM can use it in predictor mode.
    
    \item \emph{Future Movements set} - up to reasonable prediction horizon $timeMargin$. 
\end{enumerate}

The \emph{Movement Automaton} can be used as trajectory prediction for initial system state and future movements.  The prediction function (eq. \ref{eq:statePredictionCollisionCase}).
\begin{equation}\label{eq:statePredictionCollisionCase}
    Prediction: UAS \times state \times future Movements \to [x,y,z,t] \in \R^4
\end{equation}

\begin{note}
    Then prediction for UAS1 is $Prediction_1$, and for UAS 2 $Prediction_2$, the predictions are synchronized meaning that time at position $i$ is equal in both discrete trajectory matrices.
\end{note}

\noindent The \emph{collision distance} for predictor (eq. \ref{eq:statePredictionCollisionCase}) is given as minimal distance of projected synchronized trajectories for UAS1 and UAS2. In our discrete enviroment, the \emph{collision distance} is given as (eq. \ref{eq:TrajectoryPredictionMinimalDistance}).

\begin{equation}\label{eq:TrajectoryPredictionMinimalDistance}
    collisionDistance = \min\left\{\norm{point_1-point_2}:\forall \left(\begin{gathered}point_1 \in Prediction_1,\\ point_2, \in Prediction_2,\\ t_1 \sim t_2 \end{gathered}\right)\right\} 
\end{equation}

If $collisionDistance \le collision Margin$  condition is met, \emph{collision flag} is set.  

The collision point is then calculated  as mean of \emph{UAS positions} in prediction at a time when the distance is minimal.  The final collision point is arithmetic mean of two positions (eq. \ref{eq:collisionPointTrajectoryPrediction}).
\begin{equation}\label{eq:collisionPointTrajectoryPrediction}
    collisionPoint= \frac{point_1 - point_2}{2}:\left(\begin{gathered}point_1 \in Prediction_1,\\ point_2, \in Prediction_2,\\ t_1 \sim t_2 \text{ at minimal distance}\end{gathered}\right)
\end{equation}

\begin{note}
    Collision point is overwritten by trajectory intersection (specific) method; the \emph{linear intersection} is considered a \emph{general collision detection method}. The collision detection method in future UTM system needs to be determined. The \emph{Trajectory intersection} method presented in this work is one of the possible candidates. 
\end{note}

\paragraph{6\textsuperscript{th}} \emph{Role determination} phase is invoked if and only if previous conditions are met and \emph{collision flag} with \emph{collision point} exists.

There is \emph{adjusted position} of each UAS used as verticals and \emph{collision point} used as a center. The first step is normalization of adjusted position around collision point for both UAS:
\begin{equation}
    normalized_i =  adjustedPosition_i - collisionPoint,\quad i \in \{1, 2\}
\end{equation}

Then the right-hand coordinate system internal angle calculation method is used:


\begin{equation}
    angleOfApproach = \left|\text{atan2}\left(\begin{gathered}normalized_1 \times normalized_2, \\normalized_1 \circ normalized_2\end{gathered}\right)\right|
\end{equation}

\noindent Based on the \emph{angle of approach} the \emph{scenario type} is  decided like follows:
\begin{enumerate}
    \item $130^\circ \le angle Of Approach  \le 180^\circ$ - the scenario type is set as \emph{Head On Approach} (sec.\ref{sec:handlingHeadOnApproach})
    \item $70^\circ \le angle Of Approach  < 130^\circ$ - the scenario type is set as \emph{Converging Maneuver} (sec.\ref{sec:handlingConvergingManuever})
    \item $0^\circ \le angle Of Approach  < 70^\circ$ and \emph{different speed} -   - the scenario type is set as \emph{Overtake Maneuver} (sec.\ref{sec:handlingOvertakeManuever})
\end{enumerate}

\noindent Based on \emph{relative position} and \emph{scenario type}, the \emph{avoidance role} like follows:
\begin{enumerate}
    \item \emph{Head On Approach} enforces the following:
    \begin{enumerate}[a.]
        \item The \emph{avoidance role} is set as \emph{RoundAbounting} for both UAS.
        
        \item None of the \emph{UAS} does have the \emph{Right Of the Way}.
    \end{enumerate}
    
    \item \emph{Converging Maneuver} enforces the following:
    \begin{enumerate}[a.]
        \item \emph{UAS} without free right side has a role set as \emph{Converging}.
        
        \item \emph{UAS} with free right side has the \emph{Right Of the Way}.
    \end{enumerate}
    
    \item \emph{Overtake Maneuver}  enforces the following:
    \begin{enumerate}[a.]
        \item \emph{Slower UAS} has \emph{Overtaken} role with \emph{Right Of the Way}.
        
        \item \emph{Faster UAS} has \emph{Overtaking} without 
        \emph{Right Of the Way}.
        
        \item \emph{Faster UAS} mission plan is altered with \emph{divergence and convergence waypoints}.
    \end{enumerate} 
\end{enumerate}

\paragraph{7\textsuperscript{th}} \emph{Safety Margin Calculation} Is invoked when the collision case is \emph{Active}. The \emph{Active Collision Case} in this time-frame means that \emph{Collision Flag} is raised. The \emph{avoidance role} determines \emph{safety margin calculation}.

If \emph{Head-On Approach} is case type of \emph{Head collision case} then \emph{safety margin} is calculated as the maximum of the sum of \emph{default} margins or \emph{head on} margins:
\begin{equation}
    safetyMargin = \max\left\{\begin{aligned}&default(UAS1)+default(UAS2),\\ &headOn(UAS_1)+headOn(UAS_2)\end{aligned}\right\}
\end{equation}

If \emph{Converging Maneuver} is case type of \emph{Head collision case} then \emph{safety margin} is calculated based on \emph{avoiding UAS} as the maximum of opposing UAS \emph{default margin} and avoiding \emph{converging margin}:
\begin{equation}
    safetyMargin = 
    \begin{cases}
        uas1.role = Converging :& \max\left\{\begin{aligned}&default(UAS2),\\&converging(UAS1)\end{aligned}\right\} \\
        uas1.role = Converging :&   \max\left\{\begin{aligned}&default(UAS1),\\&converging(UAS2)\end{aligned}\right\} \\
    \end{cases}
\end{equation}

If \emph{Overtake maneuver} is case type of \emph{Head collision case} then \emph{safety margin} is calculated as the maximum of \emph{default, overtaking, overtaken} margins of both UAS:

\begin{equation}
    safetyMargin = \max\left\{\begin{aligned}&default(UAS1),default(UAS2),\\ &overtaken(UAS_1),overtaking(UAS_2),\\&overtaking(UAS_1),overtaken(UAS_2)\end{aligned}\right\}
\end{equation}

\paragraph{Collision Case Chaining} is procedure when multiple active collision cases for different \emph{time-frame} are chained and creates the time ordered series of \emph{collision cases}. There are two notable instances in the \emph{chain}:
\begin{enumerate}
    \item \emph{Head Collision Case} - Collision case when the first danger was detected. The notable parameters are \emph{collision point} and UAS \emph{avoidance roles} because these are enforced by the \emph{Rule engine} (sec. \ref{sec:ruleEngine}). The \emph{head collision case} is first in the chain.
    
    \item \emph{Tail Collision Case} -  Collision case when the \emph{collision danger} was not detected. The \emph{tail collision case} is last in the chain.  
\end{enumerate}

\begin{note}
    The \emph{Chaining} of \emph{collision cases} is rather primitive and sensitive for errors/noise.
    
    The \emph{Consistency of Avoidance Maneuver} is ensured by enforcing \emph{head collision case} parameters. 
\end{note}

\begin{tabularx}{\textwidth}{S{0.25}|X}
     \multicolumn{2}{c}{\textbf{Data for both attendants}}\\\hline
     adjusted position &  predicted from previous \emph{position notifications} (\ref{tab:positionNotification}) data at the time of \emph{UTM decision frame} start.\\
     adjusted orientation & predicted from previous \emph{position notifications} (\ref{tab:positionNotification}), \emph{mission plan}, and \emph{expected velocity}.\\
     velocity& proclaimed velocity for given \emph{UTM decision time frame}.\\
     registration ID &  is unique registration number issued by the local aviation authority\\
     craft category & from \emph{position notifications} (\ref{tab:positionNotification}).\\
     maneuverability &  from \emph{position notifications} (\ref{tab:positionNotification}).\\
     mission plan & is acquired from \emph{allowed mission registers} where it has been  registered prior UAS flight\\
     safety margins & list of all safety margins derived based or craft categorization or overridden by \emph{position notifications} (\ref{tab:positionNotification}).\\
     avoidance role & is given based on situation evaluation.\\
     trajectory prediction & simulated based on \emph{position notification} (\ref{tab:positionNotification}) and \emph{mission plan}.\\
     \caption{Collision case structure attendant data.}
    \label{tab:dataForBothAttendants}
\end{tabularx} 

\paragraph{Collision Cases Merge} also known as \emph{Collision Point Adjustment Procedure} purpose it to \emph{merge} multiple collision cases into one general collision case. The clustering is used to identify \emph{airspace congestion events} \cite{bilimoria2005analysis}. Example of \emph{airspace clustering} is given it \cite{brinton2008airspace}.

The main idea is to \emph{encapsulate multiple collision cases} into one virtual roundabout to ease \emph{traffic load} \cite{fouladvand2004characteristics}. The potential risk on \emph{turbo roundabouts} have been outlined in \cite{mauro2010potential}.

There are \emph{active collision cases} in a focused \emph{cluster} in \emph{controlled airspace}. The multiple collision cases can pop up at different \emph{start times}, and they can be active for a different \emph{period}. 

The \emph{Collision point} is replaced with the \emph{roundabout center} point (eq. \ref{eq:aggregatedCollisionCaseCenter}). The \emph{roundabout center} is calculated as weighted average of \emph{active collision cases} collision points. The $weight \in [0,1]$ depending on severity rating of collision case.

\begin{equation}\label{eq:aggregatedCollisionCaseCenter}
    roundaboutCenter=\frac{\sum_{ \in Cluster}^{\forall collisionCase} collisionCase.collisionPoint \times weight}{\left | collisionCase \in Cluster \right |}
\end{equation}

\begin{note}
    The weight in (eq. \ref{eq:aggregatedCollisionCaseCenter}) is set to 1 for all time; the weight calculation needs to be determined in future works. 
\end{note}

The \emph{smallest circle problem} defined and solved in \cite{ritter1990efficient,welzl1991smallest} is used to determine the safety margin in our approach. The \emph{naive approach} determining \emph{roundabout safety margin} is to take the maximum of all open case \emph{safety margins} including default ones (eq. \ref{eq:naiveSafetyMarginAgregation}).

\begin{multline}\label{eq:naiveSafetyMarginAgregation}
    safetyMargin = \max \left\{\begin{aligned}&case.UAS_i.roundabout Safety Margin,\\&case.UAS_i.default Safety Margin\\\end{aligned}\right \},\\
    \forall case \in Cluster,\quad UAS_i \in \{1,2\}
\end{multline}

%\begin{note}
%    The \emph{naive approach} to \emph{roundabout} safety margin is bloated, and do not respect original collision points. The minimal circle problem is minimal roundabout design. The issue if there is feasible dynamic for all roundabout attendees is not addressed. 
%\end{note}

%The parameters of single \emph{Collision Case} are given by (tab. \ref{tab:collisionCase}):

\newpage    

\begin{tabularx}{\textwidth}{S{0.25}|X}
     \multicolumn{2}{c}{\textbf{Collision case calculated data}}\\\hline
     linear intersection & is predicted on attendants \emph{position}, \emph{heading}, \emph{velocity}, based on \emph{maneuverability} certain thresholds are applied  to determine safety properties.\\
     trajectory intersection & is predicted on attendants \emph{position}, \emph{velocity}, \emph{heading}, and \emph{related mission plans}, based on \emph{maneuverability} certain thresholds are applied  to determine safety properties.\\
     collision point & is created if there is the risk of medium/short period collision, if head collision case has not been closed, collision point is inherited.\\
     adj. collision point & is created if there exists at least one active collision case in the nearby surroundings of this case collision point (cluster). \\
     angle of approach($\alpha$) & is calculated based on attendants \emph{velocity} and \emph{position}, the range is $[0^\circ,180^\circ]$, it determines \emph{primary avoidance roles}.\\
     safety margin & is calculated based on \emph{avoidance roles}, \emph{maneuverability}, collision indicators, and \emph{angle of approach}.\\
     margin adjustment & is calculated based on \emph{linked collision cases}, \emph{estimation errors} and \emph{weather}.\\
     linked cases & contains a list of collision cases which are active and can have an  impact on this \emph{collision case}.\\
     head case & is a reference to collision case in the time frame when it was first opened.\\
     \multicolumn{2}{c}{\textbf{Collision case indicators}}\\\hline
     linear intersection & indicates if there was a safety breach on linear trajectories estimation with the risk of direct collision.\\
     trajectory intersection & indicates if there was a breach on trajectory estimation, with the risk of direct collision.\\
     well clear breach & indicates if \emph{linear projection} or \emph{trajectory projection} breaches \emph{well clear barrel} in \emph{controlled airspace}.\\
     active case & indicates if the case is still open.\\
    \caption{Collision case structure for given decision time-frame.}
    \label{tab:collisionCase} 
\end{tabularx}


