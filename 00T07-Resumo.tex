\noindent Esse trabalho aborda o \textit{problema de desviar de obstáculos baseado em eventos e de forma reactiva} para um sistema autónomo não tripulado que opera em \textit{espaço aéreo não segregado}. O sistema é controlado através de um autômato de movimento, dessa forma é possível d\textit{iscretizar a trajetória e controlar o veículo de forma independente}. O autômato actua como uma interface  para controlar o sistema autónomo e gerar referências de trajetórias para um controlador de baixo nível.

As leituras do sensor e outras fontes de informações são combinadas através de uma técnica de fusão de dados baseado em escala, dessa forma o método é independente da plataforma. A avaliação situacional é projetada no espaço operacional.


O espaço operacional do sistema autónomo é representado em uma grelha planar, separada em células não uniformes. Os riscos são rastreados para cada célula, nomeadamente obstáculos e a presença de intrusos, geo-fencing ou distúrbios atmosféricos.


A\textit{ estratégia de evasão ou navegação} do sistema autônomo é representada como um conjunto alcançável no espaço operacional. O conjunto alcançável é aproximado como uma árvore na qual a raiz representa o estado inicial do sistema e os nós são os estados esperados após aplicar os movimentos. O conjunto alcançável é calculado para um conjunto de estados iniciais antes da execução da missão. Devido a baixa carga computacional, é possível implementar a estratégia em plataformas embarcadas.

A \textit{aproximação do conjunto alcançável} inclui diversas propriedades de manobra, como grande cobertura de regiões ou suavidade de trajetórias, para tarefas de navegação e evasão. A customização é utilizada para integrar o sistema autónomo no espaço de controlo aéreo, onde os requisitos de separação estão incluídos no conjunto alcançável.

Os \textit{serviços básicos do sistema de controlo de tráfego aéreo} como posição, como notificação da posição, restrição do espaço aéreo, diretivas e administração local são implementadas para provar a possibilidade da estratégia ser implementada no espaço de controlo aéreo.


A verificação da \textit{estratégia foi comprovada através} de cenários chaves obtidos de exercícios de  aviação e experiências. O ambiente de simulação completo com uma vasta gama de opções de customização é apresentado.

\vskip 16pt
\noindent\emph{Palavras-chave: Evitação de obstáculos, aproximação de alcance, exploração rápida de trajeto, autômato híbrido, controle de movimento, espaço aéreo controlado.}