\noindent Este trabalho aborda uma questão de prevenção de obstáculos reativa / baseada em eventos para sistemas autônomos não tripulados (UAS) operando em espaço aéreo não-segregado.

O UAS é controlado por meio de autômato de movimento, isso permite a discretização da trajetória e a independência do controle. O autômato de movimento atua como uma interface consumindo cadeia de comando de movimento para controlar UAS ou gerar uma trajetória de referência para controle de baixo nível.

As leituras do sensor e as fontes de informação são fundidas através da fusão de dados baseada em classificação, o que proporciona independência da plataforma do sensor. A avaliação situacional é projetada no espaço operacional.

O espaço operacional UAS é representado como uma grade planar, isso é separado em células não uniformes. As ameaças são rastreadas para cada célula, ou seja, presença de obstáculos ou intrusos, geofencing ou impacto climático.

A estratégia de evitação ou navegação do UAS é representada como um conjunto de alcance no espaço operacional. O conjunto de alcance é aproximado como uma árvore onde a raiz é o estado inicial do sistema, os nós são estados esperados após a aplicação de movimentos. O conjunto de alcance é calculado para uma gama de estados iniciais antes do voo, dando uma baixa pegada computacional, permitindo a implementação da abordagem em plataformas incorporadas.

A aproximação do conjunto de alcance pode incluir várias propriedades de manobra, como alta cobertura de espaço ou suavidade de trajetória, para evitar tarefas de navegação. A personalização é usada para integrar o UAS no espaço aéreo controlado, onde os requisitos de separação são incluídos no conjunto de alcance.

Os serviços básicos de Gerenciamento de Tráfego da UAS, como notificação de posição, restrição de espaço aéreo, diretivas e microgestão, são implementados para comprovar a viabilidade operacional da abordagem em espaço aéreo controlado.

A verificação da viabilidade da abordagem foi comprovada através de cenários de teste de caso de linha de fronteira, retirados das práticas e da experiência da aviação geral. 