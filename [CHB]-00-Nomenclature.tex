%% fcup-thesis.tex -- document template for PhD theses at FCUP
%%
%% Copyright (c) 2015 João Faria <joao.faria@astro.up.pt>
%%
%% This work may be distributed and/or modified under the conditions of
%% the LaTeX Project Public License, either version 1.3c of this license
%% or (at your option) any later version.
%% The latest version of this license is in
%%     http://www.latex-project.org/lppl.txt
%% and version 1.3c or later is part of all distributions of LaTeX
%% version 2005/12/01 or later.
%%
%% This work has the LPPL maintenance status "maintained".
%%
%% The Current Maintainer of this work is
%% João Faria <joao.faria@astro.up.pt>.
%%
%% This work consists of the files listed in the accompanying README.

%% SUMMARY OF FEATURES:
%%
%% All environments, commands, and options provided by the `ut-thesis'
%% class will be described below, at the point where they should appear
%% in the document.  See the file `ut-thesis.cls' for more details.
%%
%% To explicitly set the pagestyle of any blank page inserted with
%% \cleardoublepage, use one of \clearemptydoublepage,
%% \clearplaindoublepage, \clearthesisdoublepage, or
%% \clearstandarddoublepage (to use the style currently in effect).
%%
%% For single-spaced quotes or quotations, use the `longquote' and
%% `longquotation' environments.


%%%%%%%%%%%%         PREAMBLE         %%%%%%%%%%%%

%%  - Default settings format a final copy (single-sided, normal
%%    margins, one-and-a-half-spaced with single-spaced notes).
%%  - For a rough copy (double-sided, normal margins, double-spaced,
%%    with the word "DRAFT" printed at each corner of every page), use
%%    the `draft' option.
%%  - The default global line spacing can be changed with one of the
%%    options `singlespaced', `onehalfspaced', or `doublespaced'.
%%  - Footnotes and marginal notes are all single-spaced by default, but
%%    can be made to have the same spacing as the rest of the document
%%    by using the option `standardspacednotes'.
%%  - The size of the margins can be changed with one of the options:
%%     . `narrowmargins' (1 1/4" left, 3/4" others),
%%     . `normalmargins' (1 1/4" left, 1" others),
%%     . `widemargins' (1 1/4" all),
%%     . `extrawidemargins' (1 1/2" all).
%%  - The pagestyle of "cleared" pages (empty pages inserted in
%%    two-sided documents to put the next page on the right-hand side)
%%    can be set with one of the options `cleardoublepagestyleempty',
%%    `cleardoublepagestyleplain', or `cleardoublepagestylestandard'.
%%  - Any other standard option for the `report' document arclass can be
%%    used to override the default or draft settings (such as `10pt',
%%    `11pt', `12pt'), and standard LaTeX packages can be used to
%%    further customize the layout and/or formatting of the document.

%% *** Add any desired options. ***
%PDF
%\documentclass[a4paper,narrowmargins,11pt,oneside,draft,onehalfspaced,singlespacednotes]{fcup-thesis}
%\documentclass[a4paper,narrowmargins,11pt,oneside,onehalfspaced,singlespacednotes]{fcup-thesis}
%Print
%\documentclass[draft,a4paper,narrowmargins,11pt,twoside,openright,onehalfspaced,singlespacednotes]{fcup-thesis}
\documentclass[a4paper,narrowmargins,11pt,twoside,openright,onehalfspaced,singlespacednotes]{fcup-thesis}

%% *** Add \usepackage declarations here. ***
%% The standard packages `geometry' and `setspace' are already loaded by
%% `ut-thesis' -- see their documentation for details of the features
%% they provide.  In particular, you may use the \geometry command here
%% to adjust the margins if none of the ut-thesis options are suitable
%% (see the `geometry' package for details).  You may also use the
%% \setstretch command to set the line spacing to a value other than
%% single, one-and-a-half, or double spaced (see the `setspace' package
%% for details).
% Overfull statements
\pretolerance=150
\setlength{\emergencystretch}{3em}
% Overfull end
\usepackage[english]{babel}
\usepackage{helvet} %To replace arial fonts
\usepackage{lipsum}
\usepackage[utf8]{inputenc}


%%% Additional useful packages
%%% ----------------------------------------------------------------
\usepackage{array}
\usepackage{amsmath}  
\usepackage{amssymb}
\usepackage{amsfonts}
\DeclareFontFamily{OT1}{pzc}{}
\DeclareFontShape{OT1}{pzc}{m}{it}{<-> s * [0.900] pzcmi7t}{}
\DeclareMathAlphabet{\mathpzc}{OT1}{pzc}{m}{it}
%Titles need to be 14 pt => Large in \normaltext 11pt
\usepackage{titlesec}
\titleformat*{\section}{\Large\bfseries}
\titleformat*{\subsection}{\Large\bfseries}
\titleformat*{\subsubsection}{\Large\bfseries}
%Titles need to be 14 pt => Large in \normaltext 11pt
\usepackage{amsthm}      
\usepackage[ruled,algochapter]{algorithm2e}
\usepackage{algorithmic}
\usepackage{bm}
\usepackage[mathscr]{euscript}
\usepackage{graphicx}       
\usepackage{psfrag}         
\usepackage{fancyvrb}    
\usepackage{float}
\usepackage{ltablex}
\usepackage[square,sort,comma,numbers]{natbib}        
\usepackage{bbding}         
\usepackage{dcolumn}        
\usepackage{booktabs} 
\usepackage{multirow}
\usepackage{paralist}     
\usepackage{ifdraft}  
\usepackage{indentfirst}    
\usepackage[nottoc,notlof,notlot]{tocbibind}
\usepackage{url}
\usepackage{tabularx}
%use font size for captions like 8pt -> normalisize 11pt, scriptsize->8pt
\usepackage[font={scriptsize}]{caption}
\usepackage[font={scriptsize}]{subcaption}
\captionsetup{font=scriptsize}

\usepackage[unicode]{hyperref}
\usepackage{xcolor}


\hypersetup{pdftitle=Obstacle avoidance framework based on reach sets, 
            pdfauthor=Alojz Gomola,
            colorlinks=false,
            urlcolor=blue,
            pdfstartview=FitH,
            pdfpagemode=UseOutlines,
            pdfnewwindow,
            breaklinks
          }
\usepackage{array}
\newcolumntype{L}[1]{>{\raggedright\let\newline\\\arraybackslash\hspace{0pt}}m{#1}}
\newcolumntype{C}[1]{>{\centering\let\newline\\\arraybackslash\hspace{0pt}}m{#1}}
\newcolumntype{R}[1]{>{\raggedleft\let\newline\\\arraybackslash\hspace{0pt}}m{#1}}         
\newcolumntype{B}{X}
\newcolumntype{S}[1]{>{\hsize=#1\textwidth}X}

\newcommand{\FIGDIR}{./Pics}    %%% directory containing figures
\newcommand{\twolinecellr}[2][r]{%
  \begin{tabular}[#1]{@{}r@{}}#2\end{tabular}}
\newcommand{\secState}[1]{
	\ifdraft{(#1) }{}
}
\theoremstyle{plain}
\newtheorem{theorem}{Theorem}
\newtheorem{lemma}[theorem]{Lemma}
\newtheorem{proposition}[theorem]{Proposition}

\theoremstyle{plain}
\newtheorem{definition}{Definition}
\newtheorem{problem}{Problem}
\newtheorem{example}{Example}
\newtheorem{assumption}{Assumption}

\theoremstyle{remark}
\newtheorem*{corollary}{Corollary}
\newtheorem*{note}{Note}




\newenvironment{dokaz}{
  \par\medskip\noindent
  \textit{Proof}.
}{
\newline
\rightline{\SquareCastShadowBottomRight}
}

\newenvironment{constraints}[1]{
  \par\medskip\noindent
  \textit{Constraints #1} \\
}{
\newline
\rightline{\SquareCastShadowBottomRight}
}


%\bibliographystyle{plainnat}     %% Author (year) style
\bibliographystyle{unsrt}        %% [number] style
\setcitestyle{square}

% Section  3.7 Challenge list
\newif\ifproblemchallenge   %# Build block for problem challenges
\problemchallengetrue       %# Show comments

\newcommand{\R}{\mathbb{R}}
\newcommand{\N}{\mathbb{N}}

\DeclareMathOperator{\pr}{\textsf{P}}
\DeclareMathOperator{\E}{\textsf{E}\,}
\DeclareMathOperator{\var}{\textrm{var}}
\DeclareMathOperator{\sd}{\textrm{sd}}


\newcommand{\T}[1]{#1^\top}        

\newcommand{\goto}{\rightarrow}
\newcommand{\gotop}{\stackrel{P}{\longrightarrow}}
\newcommand{\maon}[1]{o(n^{#1})}
\newcommand{\abs}[1]{\left|{#1}\right|}
\newcommand{\dint}{\int_0^\tau\!\!\int_0^\tau}
\newcommand{\isqr}[1]{\frac{1}{\sqrt{#1}}}
\newcommand{\norm}[1]{\left\lVert#1\right\rVert}


\newcommand{\pulrad}[1]{\raisebox{1.5ex}[0pt]{#1}}
\newcommand{\mc}[1]{\multicolumn{1}{c}{#1}}
\newcommand{\TBD}[1]{\color{red}\emph{--TBD:}#1\color{black}}
%%%%%%%%%%%%%%%%%%%%%%%%%%%%%%%%%%%%%%%%%%%%%%%%%%%%%%%%%%%%%%%%%%%%%%%%
%%                                                                    %%
%%                   ***   I M P O R T A N T   ***                    %%
%%                                                                    %%
%%  Fill in the following fields with the required information:       %%
%%   - \degree{...}       name of the degree obtained                 %%
%%   - \department{...}   name of the graduate department             %%
%%   - \gradyear{...}     year of graduation                          %%
%%   - \author{...}       name of the author                          %%
%%   - \title{...}        title of the thesis                         %%
%%%%%%%%%%%%%%%%%%%%%%%%%%%%%%%%%%%%%%%%%%%%%%%%%%%%%%%%%%%%%%%%%%%%%%%%

%% *** Change this example to appropriate values. ***
\degree{Doctor of Philosophy}
\department{Departamento de Matem\'{a}tica}
\gradyear{2019}
\author{Alojz Gomola}
\title{Obstacle Avoidance Framework based on Reach Sets}

%% *** NOTE ***
%% Put here all other formatting commands that belong in the preamble.
%% In particular, you should put all of your \newcommand's,
%% \newenvironment's, \newtheorem's, etc. (in other words, all the
%% global definitions that you will need throughout your thesis) in a
%% separate file and use "\input{filename}" to input it here.


%% *** Adjust the following settings as desired. ***

%% List only down to subsections in the table of contents;
%% 0=chapter, 1=section, 2=subsection, 3=subsubsection, etc.
\setcounter{tocdepth}{3}

%% Make each page fill up the entire page.
\flushbottom


%%%%%%%%%%%%      MAIN  DOCUMENT      %%%%%%%%%%%%

\begin{document}

%% This sets the page style and numbering for preliminary sections.
\begin{preliminary}

%% This generates the title page from the information given above.
\maketitle

%% There should be NOTHING between the title page and abstract.
%% However, if your document is two-sided and you want the abstract
%% _not_ to appear on the back of the title page, then uncomment the
%% following line.
\cleardoublepage


%% This generates a "dedication" section, if needed
%% (uncomment to have it appear in the document).
\begin{dedication}
%% *** Put your Dedication here. ***
\noindent This work is a commutation of a five year long journey. A lot of sacrifices were made along the way, at least now you may hold the summary of it. The book in your hands is applicable in many areas of land/sea/air transportation. Please proceed with care and apply the knowledge to empower human kind.  
\\ \\
I \emph{dedicate} this work to anyone who is seeking knowledge. I would be glad if it can help you to find missing piece in jigsaw of science. You can expect a detailed cookbook with many useful ideas which needs to reach maturity. 
\\\\
The best is yet to come in field of autonomous systems, the full autonomy is in our grasp. To get there its geed to know what are the limits of your maneuverability in given situation. This work offers you that.
\\ 
\\
\begin{flushright}
Feel free to \emph{reach} it !
\end{flushright}
\end{dedication}

%% This generates an "acknowledgements" section, if needed
%% (uncomment to have it appear in the document).
\begin{acknowledgements}
% *** Put your Acknowledgements here. ***
\paragraph{This thesis was developed under} \emph{MarineUAS} - Innovative Training Network on Autonomous Unmanned Aerial Systems for Marine and Coastal Monitoring.

\paragraph{This project has received funding from} the European Union`s Horizon 2020 research and innovation programme, under the Marie Sklodowska-Curie grant agreement No. 642153. 

\paragraph{Acknowledgements:} The author acknowledges the support received from following parties and organizations/scientists:

\begin{itemize}
    \item \emph{Laboratório de Sistemas e Tecnologias Subaquáticas} (LSTS FEUP):
    \begin{itemize}
        \item João Tasso de Figueiredo Borges de Sousa (supervisor)
    \end{itemize}
    
    \item Departamento de Engenharia Eletrotécnica e de Computadores (DEEC FEUP):
    \begin{itemize}
        \item Fernando Manuel Ferreira Lobo Pereira (co-supervisor)
        \item António Pedro Rodrigues Aguiar (lecturer)
    \end{itemize}
    
    \item \emph{Unmanned Aerial Vehicles Laboratory} (UAV-Lab ITK NTNU):
    \begin{itemize}
        \item Tor Arne Johansen (project manager)
        \item Kristian Klausen (researcher)
    \end{itemize}
    
    \item \emph{Department of Electrical Engineering} (ISY LIU):
    \begin{itemize}
        \item Martin Enqvist (professor)
        \item Gustaf Hendeby (associated professor)
    \end{itemize}
    
    \item \emph{Honeywell International} (HISRO):
    \begin{itemize}
        \item Tomáš Kábrt (Technical Manager)
        \item Václav Mareš (Research \& Development)
        \item Milan Hrusecky (Supervisor)
    \end{itemize}
\end{itemize}
\end{acknowledgements}



%% This generates the abstract page, with the line spacing adjusted
\begin{abstract}
%% *** Put your Abstract here. ***
\noindent This work addresses an issue of \emph{event-based/reactive obstacle avoidance} for \emph{Unmanned Autonomous Systems} (UAS) operating in non-segregated airspace. 

The \emph{UAS} is controlled through \emph{movement automaton}; this enables trajectory discretization and \emph{control independence}. The movement automaton acts as an \emph{interface} consuming movement \emph{command chain} to control UAS or generate a reference trajectory for low-level control.

The \emph{sensor readings} and \emph{information sources} are fused through rating-based \emph{data fusion}; this provides \emph{sensor-platform independence}. The situational assessment is projected into operational space.

The UAS \emph{operational space} is represented as a \emph{planar grid}; this is separated into non-uniform cells. The \emph{threats} are tracked for each cell, namely obstacles or intruders presence, geo-fencing or weather impact. 

The \emph{avoidance} or \emph{navigation strategy} of UAS is represented as a \emph{reach set} in operational space. The \emph{reach set} is approximated as a tree where the root is initial system state; the nodes are expected states after movements application. The reach set is calculated for a range of initial states prior the flight, giving a low computational footprint, enabling approach implementation on embedded platforms. 

The reach set approximation can include various \emph{maneuvering properties}, like \emph{high space coverage} or \emph{trajectory smoothness}, for avoidance or navigation tasks. The \emph{customization} is used to integrate UAS into \emph{controlled airspace}, where \emph{separation requirements} are included in \emph{reach set}.

The basic services of \emph{UAS Traffic Management} like position notification, airspace restriction, directives, and micromanagement are implemented to prove the operational feasibility of approach in controlled airspace. 

The \emph{verification of approach feasibility} was proven through \emph{border-line case test scenarios} taken from general aviation practices and experience. The complete simulation environment with wide customization options is presented.
%% (At most 150 words for M.Sc. or 350 words for Ph.D.)
\end{abstract}

\begin{abstract-pt}
%% *** Put your Abstract here. ***
\noindent Este trabalho aborda uma questão de prevenção de obstáculos reativa / baseada em eventos para sistemas autônomos não tripulados (UAS) operando em espaço aéreo não-segregado.

O UAS é controlado por meio de autômato de movimento, isso permite a discretização da trajetória e a independência do controle. O autômato de movimento atua como uma interface consumindo cadeia de comando de movimento para controlar UAS ou gerar uma trajetória de referência para controle de baixo nível.

As leituras do sensor e as fontes de informação são fundidas através da fusão de dados baseada em classificação, o que proporciona independência da plataforma do sensor. A avaliação situacional é projetada no espaço operacional.

O espaço operacional UAS é representado como uma grade planar, isso é separado em células não uniformes. As ameaças são rastreadas para cada célula, ou seja, presença de obstáculos ou intrusos, geofencing ou impacto climático.

A estratégia de evitação ou navegação do UAS é representada como um conjunto de alcance no espaço operacional. O conjunto de alcance é aproximado como uma árvore onde a raiz é o estado inicial do sistema, os nós são estados esperados após a aplicação de movimentos. O conjunto de alcance é calculado para uma gama de estados iniciais antes do voo, dando uma baixa pegada computacional, permitindo a implementação da abordagem em plataformas incorporadas.

A aproximação do conjunto de alcance pode incluir várias propriedades de manobra, como alta cobertura de espaço ou suavidade de trajetória, para evitar tarefas de navegação. A personalização é usada para integrar o UAS no espaço aéreo controlado, onde os requisitos de separação são incluídos no conjunto de alcance.

Os serviços básicos de Gerenciamento de Tráfego da UAS, como notificação de posição, restrição de espaço aéreo, diretivas e microgestão, são implementados para comprovar a viabilidade operacional da abordagem em espaço aéreo controlado.

A verificação da viabilidade da abordagem foi comprovada através de cenários de teste de caso de linha de fronteira, retirados das práticas e da experiência da aviação geral. 
%% (At most 150 words for M.Sc. or 350 words for Ph.D.)
\end{abstract-pt}



%% Anything placed between the abstract and table of contents will
%% appear on a separate page since the abstract ends with \newpage and
%% the table of contents starts with \clearpage.  Use \cleardoublepage
%% for anything that you want to appear on a right-hand page.


%% This generates the Table of Contents (on a separate page).
\tableofcontents

%% This generates the List of Tables (on a separate page), if needed
%% (uncomment to have it appear in the document).
\listoftables

%% This generates the List of Figures (on a separate page), if needed
%% (uncomment to have it appear in the document).
\listoffigures

%% You can add commands here to generate any other material that belongs
%% in the head matter (for example, List of Plates, Index of Symbols, or
%% List of Appendices).
\listofalgorithms

\newpage
\cleardoublepage
\section*{\secState{R}Nomenclature}
\noindent
This chapter summarize used symbols (tab. \ref{tab:symbols}), acronyms (tab. \ref{tab:acronym}), terminology (tab. \ref{tab:TerminologyExplanation}) and, organizations (tab. \ref{tab:organizations}) mentioned in work. 

\begin{tabularx}{\textwidth}{l|X} 
    Acronym & Meaning\\ \hline\hline
    UAV & Unmanned Aerial Vehicle\\ 
    UAS & Unmanned Autonomous System(including naval vehicles)\\ 
    RPAS & Remotely Piloted Aerial System(lesser degree of autonomy)\\ 
    OPA & Optionally Piloted Aircraft\\
    RPV & Remotely Piloted Vehicle\\
    PIC & Pilot-in-Command\\\hline
    LOS & Line Of Sight\\ 
    VLOS & Visual Line Of Sight\\ 
    BLOS & Behind Line Of Sight\\ \hline
    SAA & Sense And Avoid\\ 
    DAA & Detect And Avoid \\ 
    MAC & Mid-Air Collision \\
    OAC & Off-Air Collision \\
    ABSAA & Airborne Sense and Avoid\\
    GBSAA & Ground Based Sense and Avoid\\
    POA & Preemptive Obstacle Avoidance\\
    ROA & Reactive Obstacle Avoidance \\\hline
    TCAS &Traffic Alert and Collision Avoidance System\\
    ACAS X & Airborne Collision Avoidance System X\\
    ACAS XU & Airborne Collision Avoidance System X for UAS\\
    CD\&R & Collision Detection and Resolution\\ \hline
    RTK & Real-Time Kinematik\\ 
    GPS & Global Positioning System\\ 
    IMU & Internal Measurement Unit\\ 
    LiDAR &  Light Detection and Ranging \\ 
    ADS-B & Automatic Dependent Surveillance – Broadcast\\ 
    GSE & Ground Support Equipment\\\hline
    ATC & Air Traffic Control \\
    ATO & Air Traffic Organization\\
    C2 & Control and Communications\\\hline
    MASPS & Minimum Aviation System Performance Standard\\
    MOPS & Minimum Operational Performance Standard\\
    RVSM & Reduced Vertical Separation Minimum\\
    RHSM & Reduced Horizontal Separation Minimum\\
    SM & Safety Margin \\
    \caption{List of Acronyms}
    \label{tab:acronym}
\end{tabularx}

\begin{tabularx}{\textwidth}{l|X}
    Acronym & Organization name \\ \hline\hline
    ICAO & International Civil Aviation Organization (UN)\\
    ITU & International Telecommunication Union (UN)\\\hline
    EASA & European Aviation Safety Agency (EU)\\ 
    JARUS&  Joint Authorities for Regulation of Unmanned Systems (EU)\\ \hline
    FAA & Federal Aviation Administration (USA)\\
    FCC & Federal Communications Commission (USA)\\\hline
    LSTS & Laboratório de Sistemas e Tecnologia Subaquática (PT)\\ 
    FEUP &Faculdade de Engenharia da Universidade do Porto (PT)\\ 
    ITK & Institutt for teknisk kybernetikk NTNU (NO)\\ 
    NTNU& Norges teknisk-naturvitenskapelige universitet (NO)\\ 
    IST & Instituto Superior Técnico - Universidade de Lisboa (PT)\\ 
    HWI & Honeywell International (CZ/USA)\\ 
    \caption{List of Organizations}
    \label{tab:organizations}
\end{tabularx} 
    



\begin{tabularx}{\textwidth}{l|X}  
    Symbol & Explanation \\ \hline\hline
    $A,B,C,D,\dots$ & Capital letters are used for matrices\\
    $A(\dots),B(\dots),\dots$ & Functional matrices, $(\dots)$ denotes parameters\\\hline
    $f(\dots),g(\dots),\dots$ & Vector or scalar functions $(\dots)$ denotes parameters\\
    $\vec{f}(\dots),\vec{g}(\dots),\dots$ & Explicit vector functions, when equation contains both types of scalar and vector functions\\\hline
    $t,x,y,z,\dots$ & Vectors or scalar coefficients \\
    $\vec{x},\vec{o},\vec{g},\dots$ & Explicit vectors, when function contains both types of scalar and vector parameters.\\\hline
    $\theta,\varphi$ & Greek letters denoting angles in radians\\
    \caption{List of symbols}
    \label{tab:symbols}
\end{tabularx} 



\begin{tabularx}{\textwidth}{S{0.22}|X} 
    \toprule
     Terminology &Definition  \\\hline
    \midrule
    \endhead
     Air Traffic Control & A service operated by appropriate authority to promote the safe, orderly, and expeditious flow of air traffic\\\hline
     Aircraft & A device that is used or intended to be used for flight in the air\\\hline
     Airspace & Any portion of the atmosphere sustaining aircraft flight and which has defined boundaries and specified dimensions. Airspace may be classified as to the specific types of flight allowed, rules of operation, and restrictions in accordance with International Civil Aviation Organization standards or State regulation\\\hline
     Civil Aircraft & Aircraft other than public aircraft. \\\hline
     Collision \mbox{Avoidance} & The Sense and Avoid system function where the UAS takes appropriate action to prevent an intruder from penetrating the collision volume. Action is expected to be initiated within a relatively short time horizon before closest point of approach. The collision avoidance function engages when all other modes of separation fail.\\\hline
     Communication Link & The voice or data relay of instructions or information between the UAS pilot and the air traffic controller and other NAS users.\\\hline
     Control Station & The equipment used to maintain control, communicate with, guide, or otherwise pilot an unmanned aircraft.\\\hline
     Crewmember (UAS) & In addition to the crewmembers identified in 14 CFR Part 1, a UAS flightcrew member includes pilots, sensor/payload operators, and visual observers, but may include other persons as appropriate or required to ensure safe operation of the aircraft.\\\hline
     Data Link & A ground-to-air communications system which transmits information via digital coded pulses.\\\hline
     Detect and Avoid & Term used instead of Sense and Avoid in the Terms of Reference for RTCA Special Committee 228. This new term has not been defined by RTCA and may be considered to have the same definition as Sense and Avoid when used in this document.\\\hline
     ICAO & International Civil Aviation Organization is a specialized agency of the United Nations whose objective is to develop the principles and techniques of international air navigation and to foster planning and development of international civil air transport.\\\hline
     Manned Aircraft & Aircraft piloted by a human onboard.\\\hline
     Model Aircraft & An unmanned aircraft that is capable of sustained flight in the atmosphere; flown within visual line-of-sight of the person operating the aircraft and flown for hobby or recreational purposes.\\\hline
     Optionally Piloted Aircraft & An aircraft that is integrated with UAS technology and still retains the capability of being flown by an onboard pilot using conventional control methods.\\\hline
     Pilot-in-Command & Pilot-in-command means the person who:\\
     &1.~ has final authority and responsibility for the operation and safety of the flight;\\
     &2.~has been designated as pilot-in-command before or during the flight; and\\
     &3.~ holds the appropriate category, class, and type rating, if appropriate, for the conduct of the flight.\\\hline
     Public Aircraft & An aircraft operated by a governmental entity (including federal, state, or local governments, and the U.S. Department of Defense and its military branches) for certain purposes\\\hline 
     RTCA & RTCA, Inc. is a private, not-for-profit corporation that develops consensus-based recommendations regarding communications, navigation, surveillance, and air traffic management system issues. RTCA functions as a Federal Advisory Committee. Its recommendations are used by the FAA as the basis for policy, program, and regulatory decisions and by the private sector as the basis for development, investment and other business decisions  (\url{www.rtca.org})\\\hline
     See and Avoid & When weather conditions permit, pilots operating instrument flight rules or visual flight rules are required to observe and maneuver to avoid another aircraft. \\\hline
     Self-Separation & Sense and Avoid system function where the UAS maneuvers within a sufficient time-frame to remain well clear of other airborne traffic.\\\hline 
     Sense and Avoid & The capability of a UAS to remain well clear from and avoid collisions with other airborne traffic. Sense and Avoid provides the functions of self-separation and collision avoidance to establish an analogous capability to “see and avoid” required by manned aircraft.\\\hline
     Unmanned Aircraft & 1.~ A device used or intended to be used for flight in the air that has no onboard pilot. This devise excludes missiles, weapons, or exploding warheads, but includes all classes of airplanes, helicopters, airships, and powered-lift aircraft without an onboard pilot.\\
     &2.~An aircraft that is operated without the possibility of direct human intervention from within or on the aircraft.\\\hline
     Unmanned Aircraft System & An unmanned aircraft and its associated elements related to safe operations, which may include control stations (ground, ship, or air-based), control links, support equipment, payloads, flight termination systems, and launch/recovery equipment.\\
     &~An unmanned aircraft and associated elements (including communications links and the components that control the unmanned aircraft) that are required for the pilot-in-command to operate safely and efficiently in the national airspace system. \\\hline
     Visual Line of Sight & Unaided (corrective lenses and/or sunglasses exempted) visual contact between a pilot-incommand or a visual observer and a UAS sufficient to maintain safe operational control of the aircraft, know its location, and be able to scan the airspace in which it is operating to see and avoid other air traffic or objects aloft or on the ground.\\
     \caption{Terminology}
     \label{tab:TerminologyExplanation}
\end{tabularx}

\begin{note}
\emph{Acronyms} (tab. \ref{tab:acronym}) and \emph{Terminology} (tab. \ref{tab:TerminologyExplanation}) are in compliance with \emph{ICAO}, \emph{FAA}, and, \emph{EASA} definitions, refer to  \cite{huerta2013integration} for more detailed information.
\end{note}
%% End of the preliminary sections: reset page style and numbering.
\end{preliminary}


%%%%%%%%%%%%%%%%%%%%%%%%%%%%%%%%%%%%%%%%%%%%%%%%%%%%%%%%%%%%%%%%%%%%%%%%
%%  Put your Chapters here; the easiest way to do this is to keep     %%
%%  each chapter in a separate file and `\include' all the files.     %%
%%  Each chapter file should start with "\chapter{ChapterName}".      %%
%%  Note that using `\include' instead of `\input' will make each     %%
%%  chapter start on a new page, and allow you to format only parts   %%
%%  of your thesis at a time by using `\includeonly'.                 %%
%%%%%%%%%%%%%%%%%%%%%%%%%%%%%%%%%%%%%%%%%%%%%%%%%%%%%%%%%%%%%%%%%%%%%%%%

%% *** Include chapter files here. ***

		
%% This adds a line for the Bibliography in the Table of Contents.
%\addcontentsline{toc}{chapter}{Bibliography}
%% *** Set the bibliography style. ***
%% (change according to your preference/requirements)
%\bibliographystyle{plain}
%% *** Set the bibliography file. ***
%% ("thesis.bib" by default; change as needed)
\bibliography{thesis}

%% *** NOTE ***
%% If you don't use bibliography files, comment out the previous line
%% and use \begin{thebibliography}...\end{thebibliography}.  (In that
%% case, you should probably put the bibliography in a separate file and
%% `\include' or `\input' it here).

\end{document}
