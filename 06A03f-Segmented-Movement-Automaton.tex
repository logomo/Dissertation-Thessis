\section{Segmented Movement Automaton}\label{s:segmentedMovementAutomaton}
\paragraph{Motivation:} Constructing \emph{Movement Automaton} for the more complex system can be tedious. Used \emph{Movement Automaton} for \emph{UAS system} (\ref{eq:UASNonlinearModelSimple}) has decoupled control which is not true for most of the copters/planes \cite{fossen2011mathematical}.

\paragraph{Partitioning UAS State Space:} Proposed movement automaton is defined by its Movement set (tab. \ref{tab:movements1},\ref{tab:movements2}). Those can be scaled depending on maneuverability in the  \emph{Initial state} $state(0)$:
\begin{enumerate}
    \item \emph{Climb/Descent Rate} $\delta pitch_{max}(k)$ - the maximal climb or descent rate for Up/Down movements.
    \item \emph{Turn Rate} $\delta yaw_{max}(k)$ - the maximal turn rate for Left/Right movement.
    \item \emph{Acceleration} $\delta v_{max}(k)$ - the maximal acceleration in cruising speed range.
\end{enumerate}

\begin{definition}{State Space partition}\label{def:stateSpacePartition}
    \emph{Maneuverability} is depending on \emph{Initial State}. There cannot be the infinite count of \emph{Movement Automatons}.
    
    The state space $State Space \in \R^n$ can be separated into two exclusive subsets:
    \begin{equation}
        StateSpace = [ImpactStates, NonImpactingStates]
    \end{equation}
    
    The \emph{Impacting states} are states which bounds the \emph{Maneuverability}: $\delta pitch_{max}(k)$, $\delta yaw_{max}(k)$, $\delta v_{max}(k)$. For each \emph{impact state} is possible to define upper and lower boundary:
    \begin{multline}
        \forall impactState\in ImpactStates, \exists:\\ lower(impactState) \le value(impactState) \le upper(impactState) 
    \end{multline}
    
	\noindent    The bounded interval of impact state can be separated into distinctive \emph{impact state segments} like follow:
    \begin{multline}
        impactState\in [lower,upper]:\\ \{[lower,separator_1[\dots\cup\dots[separator_i,separator_{i+1}[\dots\cup\dots\\\dots\cup\dots[separator_n,upper]\}=\\
        = impactStateIntervals(impactState)
    \end{multline}
    \begin{note}
        The interval length depends on model dynamics. The rule of thumb is to keep maximal climb/descend/turn/acceleration rates near constant value. 
    \end{note}
      
\noindent When partitioning of \emph{all impact States} finishes, the count of partitions is given as the product of \emph{count of partitions} for each member of \emph{Impact States}:
    
    \begin{equation}
        partition Count = \prod_{impactState \in}^{ImpactStates} |impactStateIntervals(impactState)| 
    \end{equation}
    
    \begin{note}
        Try to keep the count of partitions to a minimum; each new interval increases the count of partitions geometrically. 
    \end{note}
    
    There is finite number $n$ of \emph{Impacting States}, these are separated into $impactState-$ $Intervals_i$ with respective index $i \in 1\dots n$. The \emph{segment} with index defining position used \emph{impacting state} intervals is given as \emph{constrained space}:
    
    \begin{equation}
        Segment(index) = \left[
            \begin{gathered}
                impactState_1 \in impactStateIntervals_1[index_1],\\
                \vdots\\
                impactState_n \in impactStateIntervals_n[index_n],\\
                \vdots\\
                NonImpactingStates    
                \end{gathered}\right]
    \end{equation}
    
    Each \emph{Segment} covers one of impacting state intervals combination because the original intervals are exclusive, also \emph{Segments} are exclusive. The \emph{union} of all segments covers \emph{State Space}:
    
    \begin{equation}\label{eq:segmentedStateSpace}
        StateSpace = \bigcup_{\forall\quad index \in |impactStateIntervals|^n} Segment(index)
    \end{equation}
\end{definition}

\paragraph{Segmented Movement Automaton:} The segmentation of \emph{state space} is done  in (def. \ref{def:stateSpacePartition}) any \emph{state} belongs exactly to \emph{Segment} of \emph{State Space}. For each \emph{Segment} in \emph{State Space}it is possible to assess: \emph{Climb/Descent Rate} $\delta pitch_{max}(k)$, \emph{Turn Rate} $\delta yaw_{max}(k)$, and, \emph{Acceleration} $\delta v_{max}(k)$.


\begin{definition}{Movement Automaton for Segment(index)}\label{def:segmentMovementAutomaton}
    

\noindent For for Model(eq. \ref{eq:uasLinearDiscreteModel}) with State (eq. \ref{eq:ourState}) the input vector (eq. \ref{eq:ourImput}) is for position $[x,y,z]$ and velocity defined like: 

\begin{equation}
    \begin{aligned}
        \delta x(k)& = \left(v(k)+\delta v(k)\right) \cos(\delta pitch(k)) \cos(\delta yaw(k))\\
        \delta y(k)& = \left(v(k)+\delta v(k)\right) \cos(\delta pitch(k)) \sin(\delta yaw(k))\\
        \delta z(k)& = -\left(v(k)+\delta v(k)\right)\cos(\delta pitch(k))\\
        \delta v(k)&\in [-\delta v(k)_{max},\delta v(k)_{max}]
    \end{aligned}
\end{equation}
\end{definition}

\noindent The acceleration $\delta v(k)$ is in interval $[-\delta v(k)_{max},\delta v(k)_{max}]$, usually set to 0 $ms^{-1}$. The change of the orientation angles for \emph{Movement Set} (eq. \ref{eq:OurMovementSet}) is given in (tab. \ref{tab:movements3},\ref{tab:movements4}).

\begin{table}[H]
    \centering
    \begin{tabular}{r||r|r|r|r|r}
    
        $input(movement)$           &    Straight  & Down & Up & Left  & Right   \\\hline\hline
        $\delta  roll(k) [^\circ]$	   &    0	      & 0	  & 0	  & 0    & 0     \\\hline
        $\delta pitch(k) [^\circ]$     &    0	      & $\delta pitch_{max}$  & -$\delta pitch_{max}$ & 0	 & 0     \\\hline
        $\delta   yaw(k) [^\circ]$     &    0	      & 0	  & 0	  & $\delta yaw_{max}$ & -$\delta yaw_{max}$ \\
    \end{tabular}
    \caption{Orientation input values for main axes movements.}
    \label{tab:movements3}
\end{table}


\begin{table}[H]
    \centering
    \begin{tabular}{r||r|r|r|r}
        $input(movement)$             & Down-Left & Down-Right & Up-Left  & Up-Right   \\\hline\hline

        $\delta  roll(k) [^\circ]$	& 0	    & 0	    & 0    & 0     \\\hline
        $\delta pitch(k) [^\circ]$     & -$\delta pitch_{max}$ & -$\delta pitch_{max}$ & $\delta pitch_{max}$ & $\delta pitch_{max}$     \\\hline
        $\delta   yaw(k) [^\circ]$    & $\delta yaw_{max}$	& -$\delta yaw_{max}$	& $\delta yaw_{max}$ & -$\delta yaw_{max}$ \\
    \end{tabular}
    \caption{Orientation input values for diagonal axes movements.}
    \label{tab:movements4}
\end{table}

\begin{note}
    The \emph{Trajectory} is calculated the same as in (eq. \ref{eq:ourTrajectoryImplementation}). The \emph{State Projection} is given as in (eq. \ref{eq:ourStateProjection}).
\end{note}

\noindent Then the \emph{Movement Automaton} for \emph{Segment} $\in$ \emph{State Space} is defined.

\begin{definition}{Segmented Movement Automaton}\label{def:segmentedMovementAutomaton}
    For system with segmented state space (eq. \ref{eq:segmentedStateSpace}) there is for each $state(k)$ in $StateSpace$ injection function:
    \begin{equation} \label{eq:activeMovementAutomaton}
        ActiveMovementAutomaton:StateSpace\to MovementAutomaton
    \end{equation}

\noindent Trajectory of non-segmented movement automaton (eq. \ref{eq:ourTrajectoryImplementation}) is then given as the time-series of discrete states:
\begin{equation}\label{eq:ourTrajectoryImplementation}
    Trajectory(state(0),Buffer)= \left\{\begin{gathered}state(0)+\sum_{j=0}^{i-1} input(movement(j)):\\i \in\left\{1\dots |Buffer|+1\right\}, \\movement(\cdot) \in Buffer\end{gathered}\right\}
\end{equation}
    
\noindent Selecting appropriate \emph{movement automaton} implementation (def. \ref{def:segmentMovementAutomaton}) for \emph{state(k)} $\in$ \emph{Segment} $\subset$ \emph{State Space}. The mapping function (eq. \ref{eq:activeMovementAutomaton}) is injection mapping every state(k) to Segment then \emph{Movement Automaton Implementation}. The trajectory generated is then given:
    
    \begin{equation}\label{eq:ourTrajectoryImplementationSegmented}
        Trajectory\left(\begin{gathered}state(0),\\Buffer\end{gathered}\right)= 
        \left\{
            \begin{gathered}
                state(0)+\dots\\\sum_{j=0}^{i-1} 
                    \begin{aligned} 
                        &ActiveMovementAutomaton(state(j-1)).\\
                        &\quad.input(movement(j))
                    \end{aligned}:\\
                i \in\left\{1\dots |Buffer|+1\right\}, \\
                movement(\cdot) \in Buffer
            \end{gathered}
        \right\}
    \end{equation}
    
\end{definition}


