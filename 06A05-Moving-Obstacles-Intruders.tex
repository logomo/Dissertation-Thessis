\subsection{Intruders}\label{s:intruders}

\paragraph{Summary:} The intruder information coming from ADS-B needs to be assessed in relationship to the Avoidance Grid.  The final assessment consists of time encounter and space encounter ratings. The space encounter rating describes the probability of UAS meeting intruder in the same space. The time encounter rating is reflecting simulations time in the same cell. 

\paragraph{Intruder behavior:} \emph{Adversarial behavior} of moving obstacle is trying to destroy avoiding our UAS.  The \emph{Intruder} UAS \cite{fiorini1998motion} is not trying to hurt our \emph{UAS} actively. The \emph{Adversarial behaviour} is neglected in this work. The non-cooperative avoidance is assumed, it can be relaxed to \emph{cooperative avoidance} in \emph{UTM controlled airspace}.

\paragraph{Intruder information:} The \emph{observable intruder information set} for any kind of intruder, obtained through the sensor/C2 line, is following:
\begin{enumerate}
    \item\emph{Position} - position of an intruder in the \emph{local} or \emph{global} coordinate frame, which can be transformed into \emph{avoidance grid coordinate frame}.
    
    \item\emph{Heading and Velocity} - intruder heading and linear velocity in avoidance grid coordinate frame.
    
    \item\emph{Horizontal/Vertical Maneuver Uncertainty Spreads} - how much can an \emph{intruder} deviate from the \emph{original linear path} in a \emph{horizontal/vertical} plane in \emph{Global coordinate Frame}.
\end{enumerate}

 

\begin{figure}[H]
    \centering
    \includegraphics[width=0.7\textwidth]{\FIGDIR/TE052AdversaryProbabilitySpread}
    \caption{Intruder UAS intersection rate along the expected trajectory.}
    \label{fig:intruderProbabiltySpreadTheoretical}
\end{figure}   

\paragraph{Example of Intruder Intersection:} Let us neglect the \emph{time-impact} aspect on the \emph{intersection}.  The \emph{intruder} (black "I" circle) is intersecting one \emph{avoidance grid horizontal slice} (fig. \ref{fig:intruderProbabiltySpreadTheoretical}).  The intruder is moving along linear path approximation based on velocity (middle green line). The \emph{Horizontal Maneuver Uncertainty spread} is in \emph{green line boundary area} \emph{intruder intersection rating} is denoted as green-orange-red cell fill reflecting intersection severity:  red is a high rate of the intersection, orange is the medium rate of the intersection and green is a low rate of the intersection.
    


\paragraph{Moving Threats:} The \emph{UAS} can encounter following threats during the \emph{mission execution}:
\begin{enumerate}
    \item \emph{Non-cooperative Intruders} - the intruders who does not implement any approach to ensure mutual avoidance efficiency.
    
    \item \emph{Cooperative Intruders} - the intruders whom actively communicate or follow common agreed behavior pattern (ex. Rules of the Air).
    
    \item \emph{Moving Constraints} - the constrained portion of \emph{free} space which is shifting its boundary over time (ex. Short term bad weather).
\end{enumerate}
    
\begin{note}
    Our approach considers only \emph{UAS} intruders because \emph{Data Fusion} considers data received through \emph{ADS-B} messages. The \emph{Intruders} extracted from \emph{LiDAR} scan were not considered (ex. birds). The proposed \emph{intruder intersection models} are reusable for other \emph{intruder sources}.
\end{note}

\paragraph{Approach Overview:} The \emph{Avoidance Grid} (def. \ref{def:AvoidanceGrid}) is adapted to the \emph{LiDAR} sensor. The \emph{Euclidean grid intersections} are fairly simple. The \emph{polar coordinates grid} is not. The need to keep \emph{polar coordinates grid} is prevalent, because of fast \emph{LiDAR} reading assessment.  There are following commonly known methods to address this issue:

\begin{enumerate}
    \item\emph{Point-cloud Intersections} - the \emph{threat impact area} is discredited into a sufficiently thick point cloud. This point-cloud have \emph{point impact rate} and \emph{intersection time} assigned to each point. The \emph{point-cloud} is projected to \emph{Avoidance Grid}. If the \emph{impact point} hits $cell_{i,j,k}$ the cell`s impact rate is increased by the amount of \emph{point impact rate}. The final \emph{threat impact rate} in $cell_{i,j,k}$ is given when \emph{all} points from point cloud are consumed. Close point problem \cite{shamos1975closest} was solved by the application of method  \cite{bentley1980optimal}.
    
    \item\emph{Polygon Intersections} - the \emph{threat impact area} is modeled as a polygon, each $cell_{i,j,k}$ in \emph{Avoidance Grid} is considered as a \emph{polygon}. There is a possibility to calculate cell space geometrical inclusive intersection. The \emph{impact rate} is then given as rate between \emph{intersection volume} and $cell_{i,j,k}$ volume. The algorithm used for intersection selected based on:\citep{bentley1979algorithms} the selected algorithm  \emph{Shamos-Hoey} \cite{shamos1976geometric}.
\end{enumerate}

\begin{note}
    The \emph{Intruder Intersection} models are based on \emph{analytical geometry} for \emph{cones and ellipsoids} taken from \cite{sommerville2016analytical}.
\end{note}
