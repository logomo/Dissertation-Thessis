\subsection{\secState{D}Specialization of Hybrid Automaton}\label{s:MovementAutomatonBuidlingBlocks}
    \begin{definition}{Movement Primitive:}\label{def:MovementPrimitive}\\\emph{States} from \emph{Hybrid automaton} can be taken as \emph{Movements} in \emph{Movement Automaton}. \emph{MovementPrimitive} (eq. \ref{eq:movementPrimitive}) is describing the \emph{Movement} behaviour as transfer function \emph{VectorField} enriched with parameters. 

    \begin{equation}\label{eq:movementPrimitive}
        \begin{aligned}
            &MovementPrimitive(vectorField,minimalDuration,parameters)\\
            &VectorField:SystemState\times parameters \to SystemState
        \end{aligned}
    \end{equation}
    \end{definition}



    \paragraph{Example: }Let say that \emph{UAS} system is given as $\dot{position}=velocity$, then let us have two \emph{MovementPrimitives}:
    
    \begin{enumerate}
        \item \textit{Stay} - $minimalTime=1s$, $parameters=\{\}$, $VectorField:\dot{position}=0$.
        \item \textit{Move} - $minimalTime=1s$, $parameters=\{velocity\}$, $VectorField:\dot{position}=velocity$.
    \end{enumerate}
    
    \paragraph{Trajectory from Movement Primitives:} The \emph{UAS} should \emph{Move} for $5s$ with velocity $10 m/s$, then \emph{Stay} for $10s$, then move for $7s$ with velocity $4 m/s$, with initial position $position_0=0$ and initial time $t_0=1$ The standard approach is to derive transfer function $position = \Theta(\dots)$
    \begin{equation}\label{eq:trajectoryExample}
        position(t)=\Theta(\dots)
        \begin{cases}
            t \in [0,5] &: 10\times t + position(0)\\
            t \in (5,15] &: 0\times (t-5) + position(5)\\
            t \in (15,22]&: 4\times (t-15) + position(15)
        \end{cases}
    \end{equation}

    The \emph{example} given by (eq. \ref{eq:trajectoryExample}) is fairly primitive, but imagine UAS system given by nonlinear dynamics \cite{fossen2011mathematical}. Then defining transfer function for given command chain can be impossible.

    \begin{definition}{Movement Transition:}\label{def:movementTransition}\\
        \emph{System state} can be different than intended movement application, the notion of \emph{Transition} is therefore introduced as stabilizing element in movement chaining (eq. \ref{eq:movementTransition}).
        \begin{equation}\label{eq:movementTransition}
            Transition:MovementPrimitive\times SystemState \to MovementPrimitive    
        \end{equation}
    \end{definition}

    \paragraph{Trajectory with Transitions:} Introducing two transitions $Transition(Move,Stay)$ and $Transition(Stay,Move)$ reflecting periods when vehicle stop moving or speed-up to desired velocity. The transfer function (eq. \ref{eq:trajectoryExample}) can be rewritten as combination of \emph{MovementPrimitives} (eq. \ref{eq:movementPrimitive}) and \emph{Transitions} (eq. \ref{eq:movementTransition}):
    
    \begin{multline}
        Transition(Stay,Move), Move(5s,10m/s),\\
        Transition(Move,Stay), Stay(10s),\\ 
        Transition(Stay,Move), Move(7s,4m/s)
    \end{multline}.

    \begin{note} There are two types of \emph{MovementPrimitives}:
    \begin{enumerate}
        \item \emph{Stationary} - when system state is considered neutral and they are considered as entry point for automaton.
        \item \emph{Dynamic} - when the system state is considered evolving and they needs to be terminated with \emph{stationary} transition.
    \end{enumerate}
    \end{note}

    \paragraph{Movement Mapping Example:} Transition/MovementPrimitive pairs (eq. \ref{eq:movementTransition}) can be mapped into movements (eq. \ref{eq:movementMappingExample}).
    
    \begin{equation}\label{eq:movementMappingExample}
    \begin{aligned}
        Move(5s,10m/s) &:Transition(Stay,Move), Move(5s,10m/s),\\
        Stay(10s) &: Transition(Move,Stay), Stay(10s),\\ 
        Move(7s,4m/s) &: Transition(Stay,Move), Move(7s,4m/s)
    \end{aligned}    
    \end{equation}

    \begin{definition}{Movement:}\label{def:Movement}\\
        Movement can consist from multiple \emph{Transitions} (eq. \ref{eq:movementTransition}) and one \emph{MovementPrimitive} (eq. \ref{eq:movementPrimitive}), the duration of \emph{MovementPrimitive} can be shortened by \emph{Transitions} duration. \emph{Movement} is defined as follows:
        
        \begin{equation}
            \small Movement \left(
                \begin{gathered}
                    \scriptstyle initialState,\\
                    \scriptstyle initialTime[0..1],\\ 
                    \scriptstyle duration,\\ 
                    \scriptstyle parameters[0..1]
                \end{gathered}\right)
            = \small Chain \left(
            \begin{gathered}
            \small InitialTransition(\dots)[0..*],\\
            \small MovementPrimitive\left(
            \begin{gathered}
                \scriptstyle transitionState,\\
                \scriptstyle remainingDuration,\\
                \scriptstyle parameters
            \end{gathered}\right)\\
            \small LeaveTransition(\dots)[0..*],\\
            \end{gathered}
            \right)
        \end{equation}
        
        \emph{Chain function} connects multiple \emph{initial Transitions} which are appliead at \emph{initialState} at \emph{initialTime}. Then own \emph{MovementPrimitive} (eq. \ref{eq:movementPrimitive}) is invoked with \emph{transitionnsState}. \emph{Transitions state} is state changed by \emph{Initial Transitions}. After \emph{Movement Primitive} there can be \emph{Leave Transitions Movement}
    \end{definition}

    \paragraph{Minimal Movement Time:} Given by (eq. \ref{eq:minimalMovementTime}) for \emph{movement} is given as sum of \emph{MovementPrimitive} (eq. \ref{eq:movementPrimitive}) minimal time, and \emph{Transition} (eq. \ref{eq:movementTransition}) in/out combined minimal time.
    
    \begin{equation}\label{eq:minimalMovementTime}
        minimalTime(Movement)=
        \begin{aligned}
        &minimalTime(MovementPrimitive) +\\ &\text{max}_{in/out}\left\{time(Transition)\right\}
        \end{aligned}
    \end{equation}
   
    \paragraph{Movement Chaining:}\emph{Movements} can be \emph{chained} and applied to initial \emph{system state} to generate \emph{system trajectory}. Example of trajectory is given by (eq. \ref{eq:trajectoryExample}). Movements are reversibly obtained by participation such \emph{trajectory} into \emph{Movement primitives} and \emph{Transitions}. Then sample \emph{Trajectory} for $n\in \N^+$ movements looks like (eq. \ref{eq:movementChaining}).
    \begin{equation}\label{eq:movementChaining}
        \begin{aligned}
        &Trajectory(t_0)=State(t_0)\\
        &Trajectory(t_0,t_1]=Movement_1(Trajectory(t_0),t_0,duration_1,parameters_1)\\
        &Trajectory(t_1,t_2]=Movement_2(Trajectory(t_1),t_1,duration_2,parameters_2)\\
        &Trajectory(t_2,t_3]=Movement_3(Trajectory(t_2),t_2,duration_3,parameters_3)\\
        &\vdots\\
        &Trajectory(t_{n-1},t_n]=Movement_n(Trajectory(t_{n-1}),t_{n-1},duration_n,parameters_n)\\
        \end{aligned}
    \end{equation}

    Given \emph{Trajectory} at time $t_0$ is given as initial \emph{State} of \emph{System}. For time interval $(t_0,t_1)$, which length is equal to $duration_1$, the \emph{State} is given by $Movement_1$ with $parameters_1$ and base time $t_0$. This behaviour continues for movements $2,\dots,n$. 

    \begin{definition}{Movement Buffer:}\label{def:MovementBuffer}\\
        \noindent\emph{Movements} can be chained into \emph{Buffer} with assumption of \emph{continuous movement execution}. \emph{Continuous movement executions} each movement in chain (eq. \ref{eq:movementChaining}) is executed in time interval $\tau_i=(t_{i-1},t_{i}]$ where $i$ is movement order and $\forall$ $Movement_i$ starting time is $t_0$ or $t_{i-1}$ from previous movement. With given assumption \emph{Buffer} is given as (eq. \ref{eq:movementBuffer}) with parameters $t_{i-1},t_{i}$ omitted, due $t_0$ and $duration_i$ dependency.
        
        \begin{equation}\label{eq:movementBuffer}
            Buffer = \left\{Movement_i(duration_i,parameters_i)\right\}i\in\N^+
        \end{equation}
    \end{definition}
    
    \begin{definition}{Movement Automaton Trajectory:}\label{def:MovementAutomatonTrajectory}\\
        Let say system \emph{State}$\in\R^n$ which \emph{Trajectory} is defined by movement chaining (eq. \ref{eq:movementChaining}), applied on some \emph{initial time} $t_0\in\R^+$ and final time $t_f=t_0+\sum_{i=1}^{I}duration_i$, with movements contained in \emph{Buffer} (eq. \ref{eq:movementBuffer}) is given as \emph{Trajectory} (eq. \ref{eq:TrajectoryDefinition}).
        
        \begin{equation}\label{eq:TrajectoryDefinition}
            Trajectory(t_0,State(t_0),Buffer)\text{ or } Trajectory(State_0,Buffer) \text{ if } t_0=0
        \end{equation}
    \end{definition}


    \begin{note}
        The space dimension of \emph{Trajectories} is $\R^{n+1}$ if the space dimension of state \emph{Space} is $R^n$, because \emph{Trajectory space} contains evolution of \emph{Space} in time interval $T[t_0,t_f]$.
        
        \noindent The transformation from \emph{transfer function} (eq. \ref{eq:trajectoryExample}) to \emph{trajectory} (eq. \ref{eq:TrajectoryDefinition}) is natural, only set of \emph{Movement primitives} (eq. \ref{eq:movementPrimitive}) and set of \emph{Transitions} (eq. \ref{eq:movementTransition}) is required.
    \end{note}

    \paragraph{State Projection:} \emph{Trajectory} (eq. \ref{eq:TrajectoryDefinition})is naturally evolution of space over time, then there exists \emph{StateProjection} function (eq. \ref{eq:stateprojection}) which returns \emph{State} for specific \emph{Time}.
    
    \begin{equation}\label{eq:stateprojection}
        StateProjection:Trajectory\times Time \to State(Time)
    \end{equation}

