%% fcup-thesis.tex -- document template for PhD theses at FCUP
%%
%% Copyright (c) 2015 João Faria <joao.faria@astro.up.pt>
%%
%% This work may be distributed and/or modified under the conditions of
%% the LaTeX Project Public License, either version 1.3c of this license
%% or (at your option) any later version.
%% The latest version of this license is in
%%     http://www.latex-project.org/lppl.txt
%% and version 1.3c or later is part of all distributions of LaTeX
%% version 2005/12/01 or later.
%%
%% This work has the LPPL maintenance status "maintained".
%%
%% The Current Maintainer of this work is
%% João Faria <joao.faria@astro.up.pt>.
%%
%% This work consists of the files listed in the accompanying README.

%% SUMMARY OF FEATURES:
%%
%% All environments, commands, and options provided by the `ut-thesis'
%% class will be described below, at the point where they should appear
%% in the document.  See the file `ut-thesis.cls' for more details.
%%
%% To explicitly set the pagestyle of any blank page inserted with
%% \cleardoublepage, use one of \clearemptydoublepage,
%% \clearplaindoublepage, \clearthesisdoublepage, or
%% \clearstandarddoublepage (to use the style currently in effect).
%%
%% For single-spaced quotes or quotations, use the `longquote' and
%% `longquotation' environments.


%%%%%%%%%%%%         PREAMBLE         %%%%%%%%%%%%

%%  - Default settings format a final copy (single-sided, normal
%%    margins, one-and-a-half-spaced with single-spaced notes).
%%  - For a rough copy (double-sided, normal margins, double-spaced,
%%    with the word "DRAFT" printed at each corner of every page), use
%%    the `draft' option.
%%  - The default global line spacing can be changed with one of the
%%    options `singlespaced', `onehalfspaced', or `doublespaced'.
%%  - Footnotes and marginal notes are all single-spaced by default, but
%%    can be made to have the same spacing as the rest of the document
%%    by using the option `standardspacednotes'.
%%  - The size of the margins can be changed with one of the options:
%%     . `narrowmargins' (1 1/4" left, 3/4" others),
%%     . `normalmargins' (1 1/4" left, 1" others),
%%     . `widemargins' (1 1/4" all),
%%     . `extrawidemargins' (1 1/2" all).
%%  - The pagestyle of "cleared" pages (empty pages inserted in
%%    two-sided documents to put the next page on the right-hand side)
%%    can be set with one of the options `cleardoublepagestyleempty',
%%    `cleardoublepagestyleplain', or `cleardoublepagestylestandard'.
%%  - Any other standard option for the `report' document arclass can be
%%    used to override the default or draft settings (such as `10pt',
%%    `11pt', `12pt'), and standard LaTeX packages can be used to
%%    further customize the layout and/or formatting of the document.

%% *** Add any desired options. ***
%PDF
%\documentclass[a4paper,narrowmargins,12pt,oneside,draft,onehalfspaced,singlespacednotes]{fcup-thesis}
%\documentclass[a4paper,narrowmargins,12pt,oneside,onehalfspaced,singlespacednotes]{fcup-thesis}
%Print
%\documentclass[draft,a4paper,narrowmargins,12pt,twoside,openright,onehalfspaced,singlespacednotes]{fcup-thesis}
\documentclass[a4paper,narrowmargins,12pt,twoside,openright,onehalfspaced,singlespacednotes]{fcup-thesis}

%% *** Add \usepackage declarations here. ***
%% The standard packages `geometry' and `setspace' are already loaded by
%% `ut-thesis' -- see their documentation for details of the features
%% they provide.  In particular, you may use the \geometry command here
%% to adjust the margins if none of the ut-thesis options are suitable
%% (see the `geometry' package for details).  You may also use the
%% \setstretch command to set the line spacing to a value other than
%% single, one-and-a-half, or double spaced (see the `setspace' package
%% for details).
% Overfull statements
\pretolerance=150
\setlength{\emergencystretch}{3em}
% Overfull end
\usepackage[english]{babel}
\usepackage{lipsum}
\usepackage[utf8]{inputenc}


%%% Additional useful packages
%%% ----------------------------------------------------------------
\usepackage{array}
\usepackage{amsmath}  
\usepackage{amssymb}
\usepackage{amsfonts}
\DeclareFontFamily{OT1}{pzc}{}
\DeclareFontShape{OT1}{pzc}{m}{it}{<-> s * [0.900] pzcmi7t}{}
\DeclareMathAlphabet{\mathpzc}{OT1}{pzc}{m}{it}
\usepackage{amsthm}      
\usepackage[ruled,algochapter]{algorithm2e}
\usepackage{algorithmic}
\usepackage{bm}
\usepackage[mathscr]{euscript}
\usepackage{graphicx}       
\usepackage{psfrag}         
\usepackage{fancyvrb}    
\usepackage{float}
\usepackage{ltablex}
\usepackage[square,sort,comma,numbers]{natbib}        
\usepackage{bbding}         
\usepackage{dcolumn}        
\usepackage{booktabs} 
\usepackage{multirow}
\usepackage{paralist}     
\usepackage{ifdraft}  
\usepackage{indentfirst}    
\usepackage[nottoc,notlof,notlot]{tocbibind}
\usepackage{url}
\usepackage{tabularx}
\usepackage{subcaption}
\usepackage[unicode]{hyperref}
\usepackage{xcolor}

\hypersetup{pdftitle=LiDAR obstacle detection and avoidance, 
            pdfauthor=Alojz Gomola,
            colorlinks=false,
            urlcolor=blue,
            pdfstartview=FitH,
            pdfpagemode=UseOutlines,
            pdfnewwindow,
            breaklinks
          }
\usepackage{array}
\newcolumntype{L}[1]{>{\raggedright\let\newline\\\arraybackslash\hspace{0pt}}m{#1}}
\newcolumntype{C}[1]{>{\centering\let\newline\\\arraybackslash\hspace{0pt}}m{#1}}
\newcolumntype{R}[1]{>{\raggedleft\let\newline\\\arraybackslash\hspace{0pt}}m{#1}}         
\newcolumntype{B}{X}
\newcolumntype{S}[1]{>{\hsize=#1\textwidth}X}

\newcommand{\FIGDIR}{./Pics}    %%% directory containing figures
\newcommand{\twolinecellr}[2][r]{%
  \begin{tabular}[#1]{@{}r@{}}#2\end{tabular}}
\newcommand{\secState}[1]{
	\ifdraft{(#1) }{}
}
\theoremstyle{plain}
\newtheorem{theorem}{Theorem}
\newtheorem{lemma}[theorem]{Lemma}
\newtheorem{proposition}[theorem]{Proposition}

\theoremstyle{plain}
\newtheorem{definition}{Definition}
\newtheorem{problem}{Problem}
\newtheorem{example}{Example}
\newtheorem{assumption}{Assumption}

\theoremstyle{remark}
\newtheorem*{corollary}{Corollary}
\newtheorem*{note}{Note}




\newenvironment{dokaz}{
  \par\medskip\noindent
  \textit{Proof}.
}{
\newline
\rightline{\SquareCastShadowBottomRight}
}

\newenvironment{constraints}[1]{
  \par\medskip\noindent
  \textit{Constraints #1} \\
}{
\newline
\rightline{\SquareCastShadowBottomRight}
}


%\bibliographystyle{plainnat}     %% Author (year) style
\bibliographystyle{unsrt}        %% [number] style
\setcitestyle{square}

% Section  3.7 Challenge list
\newif\ifproblemchallenge   %# Build block for problem challenges
\problemchallengetrue       %# Show comments

\newcommand{\R}{\mathbb{R}}
\newcommand{\N}{\mathbb{N}}

\DeclareMathOperator{\pr}{\textsf{P}}
\DeclareMathOperator{\E}{\textsf{E}\,}
\DeclareMathOperator{\var}{\textrm{var}}
\DeclareMathOperator{\sd}{\textrm{sd}}


\newcommand{\T}[1]{#1^\top}        

\newcommand{\goto}{\rightarrow}
\newcommand{\gotop}{\stackrel{P}{\longrightarrow}}
\newcommand{\maon}[1]{o(n^{#1})}
\newcommand{\abs}[1]{\left|{#1}\right|}
\newcommand{\dint}{\int_0^\tau\!\!\int_0^\tau}
\newcommand{\isqr}[1]{\frac{1}{\sqrt{#1}}}
\newcommand{\norm}[1]{\left\lVert#1\right\rVert}


\newcommand{\pulrad}[1]{\raisebox{1.5ex}[0pt]{#1}}
\newcommand{\mc}[1]{\multicolumn{1}{c}{#1}}
\newcommand{\TBD}[1]{\color{red}\emph{--TBD:}#1\color{black}}

%%%%%%%%%%%%%%%%%%%%%%%%%%%%%%%%%%%%%%%%%%%%%%%%%%%%%%%%%%%%%%%%%%%%%%%%
%%                                                                    %%
%%                   ***   I M P O R T A N T   ***                    %%
%%                                                                    %%
%%  Fill in the following fields with the required information:       %%
%%   - \degree{...}       name of the degree obtained                 %%
%%   - \department{...}   name of the graduate department             %%
%%   - \gradyear{...}     year of graduation                          %%
%%   - \author{...}       name of the author                          %%
%%   - \title{...}        title of the thesis                         %%
%%%%%%%%%%%%%%%%%%%%%%%%%%%%%%%%%%%%%%%%%%%%%%%%%%%%%%%%%%%%%%%%%%%%%%%%

%% *** Change this example to appropriate values. ***
\degree{Doctor of Philosophy}
\department{Departamento de Matem\'{a}tica}
\gradyear{2019}
\author{Alojz Gomola}
\title{Obstacle Avoidance Framework based on Reach Sets}

%% *** NOTE ***
%% Put here all other formatting commands that belong in the preamble.
%% In particular, you should put all of your \newcommand's,
%% \newenvironment's, \newtheorem's, etc. (in other words, all the
%% global definitions that you will need throughout your thesis) in a
%% separate file and use "\input{filename}" to input it here.


%% *** Adjust the following settings as desired. ***

%% List only down to subsections in the table of contents;
%% 0=chapter, 1=section, 2=subsection, 3=subsubsection, etc.
\setcounter{tocdepth}{3}

%% Make each page fill up the entire page.
\flushbottom


%%%%%%%%%%%%      MAIN  DOCUMENT      %%%%%%%%%%%%

\begin{document}

	\setcounter{chapter}{4}
	\setcounter{equation}{9}
	\begin{equation}\label{eq:spaceCassificationFunction}
        SpaceClassification: y \in \emph{Space} \mapsto s \in \{Free, Restricted, Occupied, Uncertain \}
    \end{equation}
	
	\setcounter{chapter}{4}
	\setcounter{equation}{18}
	\begin{equation}\label{eq:DataFusionFunction}
        DataFusion:
        \left[
        \begin{aligned}
            &InformationSource_1 \times\\
            &InformationSource_2 \times\\
            &\times\dots\times\\
            &InformationSource_l \times\\
            &SensorFusion(\dots)\times\\
            &Weather(\dots)\\
            &position\times\\
            &\textit{fixed time }t_{fix}\times\\
            &dataFusionParameters
        \end{aligned}
        \right]
        \to 
        \begin{cases}
            Free(t_{fix})\\
            Occupied(t_{fix})\\
            Restricted(t_{fix})\\
            Uncertain(t_{fix})
        \end{cases}
    \end{equation}

	\setcounter{chapter}{4}
	\setcounter{equation}{5}
	\begin{equation}\label{eq:missionAbstractSet}
        Mission = \left\{
        \begin{aligned}
            &waypoint_1, waypoint_2, \dots,waypoint_m:\\
            &\forall_{i=1\dots m} waypoint_i \in  Space
        \end{aligned}
        \right\}, \quad m\in\N^{+},m\ge2
    \end{equation}
	
	\setcounter{chapter}{4}
	\setcounter{equation}{20}
	\begin{multline}\label{eq:safetyMarginAbstract}
        \forall t\in [missionStart,missionEnd]:\\distance(x(t),Occupied(t),t) \ge safetyMargin
    \end{multline}
	
	\setcounter{chapter}{6}
	\setcounter{equation}{0}
	\begin{equation}\label{eq:simple3dStatevector}
		state = \left [ x,y,z, roll, pitch, yaw \right]^T
	\end{equation}
	
	\setcounter{chapter}{6}
	\setcounter{equation}{1}
	\begin{equation}\label{eq:simple3dInputVector}
		input = \left [ v, \omega_{roll}, \omega_{pitch},\omega_{yaw}\right ]^T
	\end{equation}
	
	\setcounter{chapter}{6}
	\setcounter{equation}{3}
	\begin{equation}\label{eq:UASNonlinearModelSimple}
		\begin{aligned}
			\frac{\text{d} x}{\text{d} time} &= v\cos(pitch)\cos(yaw);\\
			\frac{\text{d} y}{\text{d} time} &= v\cos(pitch)\sin(yaw);\\
			\frac{\text{d} z}{\text{d} time} &= -v\sin(pitch);\\
		\end{aligned}\\\quad\quad
		\begin{aligned}
			\frac{\text{d} roll}{\text{d} time} &= \omega_{roll};\\
			\frac{\text{d} pitch}{\text{d} time} &= \omega_{pitch};\\
			\frac{\text{d} yaw}{\text{d} time} &= \omega_{yaw};\\
		\end{aligned}
	\end{equation}
	
	\setcounter{chapter}{6}
	\setcounter{equation}{10}
	\begin{equation}\label{eq:OurMovementSet}
        Movement Set= \left\{
        \begin{gathered}
            Straight, Left,Right, Up, Down,\\
            Down Left, Down Right,  Up Left,   Up Right
        \end{gathered}
        \right\}
    \end{equation}
	
	\setcounter{chapter}{6}
	\setcounter{equation}{11}
	\begin{equation}\label{eq:ourBuffer}
		Buffer = \left\{
					movement(j):
					\begin{aligned}
						&movement(j)\in Movement Set (eq. \ref{eq:OurMovementSet}),\\
						& j \in 1\dots n, n \in N^+
					\end{aligned}
				\right\}
	\end{equation}

	\setcounter{chapter}{6}
	\setcounter{equation}{12}	
	\begin{equation}\label{eq:ourTrajectoryEvolution}
		\begin{aligned}
		T&rajectory(state(0),Buffer)=\\
			&\left\{
			\begin{aligned}
				state(0) &= state(0),\\
				state(1) &= apply Movement\left(state(0), movement(1)\right),  \\
				state(2) &= apply Movement\left(state(1), movement(2)\right),  \\
				 \vdots  &= \vdots\\
				state(n-1) &= apply Movement\left(state(n-2), movement(n-1)\right),  \\
				state(n)   &= apply Movement\left(state(n-1), movement(n)\right)  \\
			\end{aligned}
			\right\}
		\end{aligned}
	\end{equation}
	
	
	\setcounter{chapter}{6}
	\setcounter{equation}{14}
	\begin{multline}\label{eq:boundedSpaceCell}
        cell.space Portion = \dots\\
            \left \{
                \begin{aligned}
                point& \in \R^3 \text{ where}:\\
                    &\left(\begin{aligned}
                        cell.distance_{start} &<& point.distance &\le& cell.distance_{end},\\
                        cell.horizontal^\circ_{start} &<& point.horizontal^\circ &\le&  cell.horizontal^\circ_{end},\\
                        cell.vertical^\circ_{start} &<& point.vertical^\circ &\le& cell.vertical^\circ_{end}\\
                    \end{aligned}\right)
                \end{aligned}
            \right\}
    \end{multline}
	
	\setcounter{chapter}{6}
	\setcounter{equation}{19}
	\begin{equation}\label{eq:avoidanceGridCellSpace}
    Avoidance Grid = \left\{
    					cell_{i,j,k}:
    					\begin{aligned}
    						& i \in 1 \dots layer Count\\
    						& j \in 1 \dots horizontal Count\\
    						& k \in 1 \dots vertical Count
    					\end{aligned} 
                     \right\}
	\end{equation}
	
	\setcounter{chapter}{6}
	\setcounter{equation}{20}
	\begin{equation}
		\forall cell_{i,j,k}, cell_{m,n,o}:
		\begin{aligned}
		    &cell_{i,j,k}\cap cell_{m,n,o} = \varnothing,
		    i \neq o \lor j \neq n \lor k \neq o
		\end{aligned}
	\end{equation}
\end{document}
