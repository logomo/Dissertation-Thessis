\section{(W) Incremental problem definition}

\noindent\emph{UAV system} defined by model (??) is executing \emph{Mission} (??) in \emph{KnownWordl} (??) $\subset$ (??) \emph{Space}. Solving avoidance problem defined in section ??.

\begin{constraints}{for all problems:}
    % General criteria of Feasibility/Stability/Scalability/Determinism/Guarantees
    %\begin{enumerate}
    %    \itemit{Calculation complexity}
    %    \itemit{Estimation precision}
    %\end{enumerate}
    %kron kron
    Bla bla
\end{constraints}

\begin{equation}
    \forall \, Trajectory(initialState,Buffer), \forall \, HFlightConstraints:
\end{equation}

\begin{problem}{Basic problem}\label{pro:basicProblem}
    is to navigate trough \emph{KnownWorld} under assumption that every \emph{waypoint} in \emph{Mission} is reachable.
    \begin{equation}\label{eq:basicProblemDefinition}
        \begin{aligned}
            KnownWorld:&= Space, \forall point\in KnownWorld=Free \cup Occupied\\
            Mission:&= \forall waypoint\in Mission \text{ are reachable}\\
            HFlightConstraints:&=\{\text{vehicle dynamic}\}\\
        \end{aligned}
    \end{equation}
    \ifproblemchallenge
    \noindent \emph{Challenges for problem  \ref{pro:basicProblem}. :}
    \begin{enumerate}
        \item \emph{Control concept} -  method for vehicle control using \emph{abstract control} representaiton.
        \item \emph{Navigation framework} - techniques and constraints to execute mission with emphasis on \emph{SafetyMargin}(??) to \emph{Occupied} (??) set.
    \end{enumerate}
    \fi
\end{problem}



\begin{problem}{Known world evolution,}\label{pro:knownWorldEvolution}
    in addition to \emph{Basic problem} (pr. \ref{pro:basicProblem}) The \emph{KnownWorld} is fed trough \emph{SensorFusion} function which is joining \emph{LiDAR} scanning into \emph{Free(t), Occupied(t),} and, \emph{Unknown(t)} sets in \emph{real time} $t$. 

    \begin{equation}\label{eq:knownworldEvolutionProblemDefinition}
        \begin{aligned}
            KnownWorld:&= SensorFusion(t)\forall point\in KnownWorld(t)\\
                       &=Free(t) \cup Occupied(t) \cup Unknown(t)\\
            Mission:&= \forall waypoint\in Mission \text{ are reachable}\\
            Sensors:&= \{LiDAR\}\\
            SensorFusion:&= \{\text{Simple joint sets (??)}\}\\
            HFlightConstraints:&=\{\text{vehicle dynamic}\}\\
        \end{aligned}
    \end{equation}
    
    \ifproblemchallenge
    \noindent \emph{Challenges for problem  \ref{pro:knownWorldEvolution}. :}
    \begin{enumerate}
        \item \emph{Evolving sets representation} - continuous LiDAR scanning are imposing problem of abstract data representation requiring introduction of some sort of \emph{Grid} representation.
        \item \emph{Evolving navigation} - imposes problem of path re-planning including \emph{Decision point determination}, such as point when next decision should be made to guarantee return path in \emph{Known world}.
    \end{enumerate}
    \fi
\end{problem}


\begin{problem}{Intruder detection}\label{pro:intruderDetection}
    in addition to \emph{Known world evolution} (pr.\ref{pro:knownWorldEvolution}) the \emph{ADS-B} sensor is introduced into \emph{Sensors} array, this imposes \emph{HardConstraint} of \emph{Flight corridor} for detected intruders impacting the evolution of \emph{Free}(t) and \emph{Occupied}(t) sets significantly.
    
    \begin{equation}\label{eq:intruderDetectionProblemdefinition}
        \begin{aligned}
            KnownWorld:&= SensorFusion(t)\forall point\in KnownWorld(t)\\
                       &=Free(t) \cup Occupied(t) \cup Unknown(t)\\
            Mission:&= \forall waypoint\in Mission \text{ are reachable}\\
            Sensors:&= \{LiDAR,ADS-B\}\\
            SensorFusion:&= \{\text{Advanced joint sets}\}\\
            HFlightConstraints:&=\{\text{vehicle dynamic}\}\\
            HardConstraints:&=\{\text{intruder corridors}\}\\
        \end{aligned}
    \end{equation}
    
    \ifproblemchallenge
    \noindent \emph{Challenges for problem  \ref{pro:intruderDetection}. :}
    \begin{enumerate}
        \item \emph{Moving obstacles} - continuous ADS-B updates reveals adversaries which can significantly impact the situation along planned path.
        \item \emph{Advanced joint sets} - the \emph{Known world} is impacted two sources of information, therefore \emph{SensorFusion} needs to be adapted to feed existing unified grid with relevant data. The information priority sensor fusion needs to be introduced. 
    \end{enumerate}
    \fi
\end{problem}

\begin{problem}{Terrain map}\label{pro:terrainMap},
    in addition to \emph{Intruder Detection} (pr. \ref{pro:intruderDetection}) \emph{TerrainMap} information source is introduced. This change impacts \emph{KnowWord} by changing the final set distribution to \emph{DatFusion}. For \emph{DataFusion} simple data fusion from one information source is used. The \emph{terrain} from \emph{TerrainMap} extends \emph{HardConstraints}. 
    \begin{equation}\label{eq:terainMapProblemDefinition}
        \begin{aligned}
            KnownWorld:&= DataFusion(t)\forall point\in KnownWorld(t)\\
                       &=Free(t) \cup Occupied(t) \cup Unknown(t)\\
            Mission:&= \forall waypoint\in Mission \text{ are reachable}\\
            Sensors:&= \{LiDAR,ADS-B\}\\
            SensorFusion:&= \{\text{Advanced joint sets}\}\\
            InformationSources:&=\{Terrain Map\}\\
            DataFusion:&= \{\text{Simple data fusion}\}\\
            HFlightConstraints:&=\{\text{vehicle dynamic}\}\\
            HardConstraints:&=\{\text{intruder corridors,terrain}\}\\
        \end{aligned}
    \end{equation}
    
    \ifproblemchallenge
    \noindent \emph{Challenges for problem  \ref{pro:terrainMap}. :}
    \begin{enumerate}
        \item \emph{Data fusion} - introduce \emph{post-SensorFusion} processing \emph{Free(t), Occupied(t),  Unknown(t)} sets.
        \item \emph{Terrain map clustering} - fetching only relevant \emph{TerrainMap} space portion to \emph{KnownWorld}.
    \end{enumerate}
    \fi
\end{problem}


\begin{problem}{Static restrictions}\label{pro:staticRestrictions},
    in addition to \emph{TerrainMap} problem (pr. \ref{pro:terrainMap}) the \emph{InformationSources} are expanded by \emph{static restriction} sources: 
    \begin{enumerate}
        \item \emph{ObstacleDatabase} - database containing notable landmarks, buildings, structures, with well defined \emph{protection zones}.
        \item \emph{FlightRestrictions} - database containing ATM flight restrictions in non-segregated airspace for UAV relevan airspace categories. 
    \end{enumerate}
    \noindent This change impacts \emph{DataFusion} by splitting \emph{Free(t)} set into \emph{Free(t)} and \emph{Restricted(t)} disjoint sets. In addition \emph{SoftConstraints} are introduced which contain restricted areas from relevant information sources.  
    
    \begin{equation}\label{eq:staticRestrictionsProblemDefinition}
        \begin{aligned}
            KnownWorld:&= DataFusion(t)\forall point\in KnownWorld(t)\\
                       &=Free(t) \cup Occupied(t) \cup Unknown(t)\cup Restricted(t)\\
            Mission:&= \forall waypoint\in Mission \text{ are reachable}\\
            Sensors:&= \{LiDAR,ADS-B\}\\
            SensorFusion:&= \{\text{Advanced joint sets}\}\\
            InformationSources:&=\{Terrain Map,Obstacle Database,Flight Restriction\}\\
            DataFusion:&= \{\text{Advanced data fusion}\}\\
            HFlightConstraints:&=\{\text{vehicle dynamic}\}\\
            HardConstraints:&=\{\text{intruder corridors,terrain,obstacles}\}\\
            Softconstraints:&=\{\text{protection zones}\}
        \end{aligned}
    \end{equation}
    
    \ifproblemchallenge
    \noindent \emph{Challenges for problem  \ref{pro:staticRestrictions}. :}
    \begin{enumerate}
        \item \emph{Restricted zones} - \emph{Obstacle map} introduces protection zones around notable landmarks, there is hierarchy given by law, other source of restrictions is \emph{Flight restrictions}, which also has hierarchy. 
        \item \emph{Navigation improvement} - Extension to problem is given by virtual obstacles form \emph{Restricted zones}. These \emph{restrictions} can be broken under certain condition, the rule engine must consider implementation of legal measurements to allow such navigation. 
    \end{enumerate}
    \fi
\end{problem}

\begin{problem}{Dynamic restrictions}\label{pro:dynamicRestrictions}
    in addition to \emph{Static restrictions} (pr. \ref{pro:staticRestrictions}), the \emph{Weather} as information source is introduced. \emph{Soft constraints} are extended by medium level dangerous zones from weather map. \emph{Hard constraints} are expanded by protection zones where the \emph{weather} conditions are harsh. Overall \emph{Weather} constraints are dynamic and changing position and shape over mission time. Modern weather systems can provide streamline overview of weather situation. 
    
    \begin{equation}\label{eq:dynamicRestrictionsProblemDefinition}
        \begin{aligned}
            KnownWorld:&= DataFusion(t)\forall point\in KnownWorld(t)\\
                       &=Free(t) \cup Occupied(t) \cup Unknown(t) \cup Restricted(t)\\
            Mission:&= \forall waypoint\in Mission \text{ are reachable}\\
            Sensors:&= \{LiDAR,ADS-B\}\\
            SensorFusion:&= \{\text{Advanced joint sets}\}\\
            InformationSources:&=\{Terrain Map,Obstacle Database,\\
                               &\quad Flight Restriction,Weather\}\\
            DataFusion:&= \{Advanced data fusion\}\\
            HFlightConstraints:&=\{\text{vehicle dynamic}\}\\
            HardConstraints:&=\{\text{intruder corridors,terrain,obstacles, protection zones}\}\\
            Softconstraints:&=\{\text{protection zones}\}
        \end{aligned}
    \end{equation}
    
    \ifproblemchallenge
    \noindent \emph{Challenges for problem  \ref{pro:dynamicRestrictions}. :}
    \begin{enumerate}
        \item \emph{Changing Restricted zones} -  Weather is changing with time, that means the weather restriction zones are changing their position, shape and \emph{severity} over time. 
        \item \emph{Sensor impact} - \emph{Weather} severely impacts functionality of some sensor, at this point scalable avoidance method to address this fact needs to be introduced
        \item \emph{Navigation algorithm} - \emph{Restricted areas} from \emph{Weather} are changing their \emph{severity}. This imposes dynamic changes to maneuverability and virtual obstacles priority. 
    \end{enumerate}
    \fi
\end{problem}

\begin{problem}{Rules of the air}\label{pro:rulesOfTheAir}, 
    in addition to \emph{Dynamic restrictions} (pr. \ref{pro:dynamicRestrictions}), \emph{Rules of the air} framework introduction, inducing new \emph{SFlightConstraints} including air-spaces and rules of air impact on control mechanism.
    \begin{equation}\label{eq:rulesOfTheAir}
        \begin{aligned}
            KnownWorld:&= DataFusion(t)\forall point\in KnownWorld(t)\\
                       &=Free(t) \cup Occupied(t) \cup Unknown(t) \cup Restricted(t)\\
            Mission:&= \forall waypoint\in Mission \text{ are reachable}\\
            Sensors:&= \{LiDAR,ADS-B\}\\
            SensorFusion:&= \{\text{Advanced joint sets}\}\\
            InformationSources:&=\{Terrain Map,Obstacle Database,\\
                               &\quad Flight Restriction,Weather\}\\
            DataFusion:&= \{Advanced data fusion\}\\
            HFlightConstraints:&=\{\text{vehicle dynamic}\}\\
            SFlightConstratins:&=\{\text{airspaces, rules of the air}\}\\
            HardConstraints:&=\{\text{intruder corridors,terrain,obstacles, protection zones}\}\\
            Softconstraints:&=\{\text{protection zones}\}
        \end{aligned}
    \end{equation}
    
    \ifproblemchallenge
    \noindent \emph{Challenges for problem  \ref{pro:rulesOfTheAir}. :}
    \begin{enumerate}
        \item \emph{Soft constraints on control} - Control mechanism which follows constraints on control to execute maneuvers is biased by implementation of situation based maneuvers. 
    \end{enumerate}
    \fi
\end{problem}