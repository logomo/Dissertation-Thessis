\cleardoublepage
\chapter{\secState{R}Background Theory}\label{ch:backGroundTheory}

\paragraph{Motivation:} Cooperative and Non-Cooperative \emph{Sense and Avoid} (SAA) systems are key enablers for the \emph{Unmanned Aerial Systems} (UAS) to routinely access non-segregated airspace \cite{spriesterbach2013unmanned}. Both cooperative and non-cooperative SAA systems are being developed to address this integration requirement.

The \emph{DAA capability} is defined as the automatic detection of possible conflicts by the UAS platform under consideration and performing avoidance maneuvers to prevent the identified collisions.

An analysis of the available SAA candidate technologies and the associated sensors for both cooperative and non-cooperative SAA systems is presented in \cite{muraru2011critical}. 

Non-cooperative \emph{Collision Detection and Resolution} (CD\&R) for UAS is considered as one of the major challenges that needs to be addressed \cite{lai2012see} for the insertion of UAVs in non-segregated air space. As a result, a number of non-cooperative sensors for the SAA system have been adopted. Light Detection and Ranging (LIDAR)is used for detecting, warning and avoiding obstacles for low-level flying \cite{sabatini2014lidar}.

An approach to the definition of encounter models and their applications to SAA strategies is presented in \cite{kochenderfer2008encounter} for both cooperative and non-cooperative scenarios.

Since 2014, there is a visible strong political support for developing rules on drones but regulations are harmonizing slowly. The European Aviation Safety Agency (EASA) has been tasked to develop a regulatory framework for drone operations and proposals for the regulation of "low-risk" UAV operations. In achieving this, EASA is working closely with the Joint Authorities for Regulation of Unmanned Systes (JARUS) \cite{jarus2016regulations}.

\paragraph{Background Areas:} Following Areas are introduced in this chapter:
\begin{enumerate}
    \item \emph{UAS System Model} (sec. \ref{s:uavMotionModel}) - continuous and discrete mathematical models.
    
    \item \emph{Reach Sets} (sec. \ref{s:ReachSets}) - introduction to \emph{Reach set} representation and calculation methods.
    
    \item \emph{Hybrid Automaton} (sec. \ref{s:HybridAutomaton}) - intuitive definition and establishment of \emph{hybrid automaton}.
    
    \item \emph{LiDAR} (sec. \ref{sec:LiDARStateOfArt}) - short summary of \emph{LiDAR} technology and terminology introduction.
\end{enumerate}
