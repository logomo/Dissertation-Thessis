\subsection{Handling Head-on Approach}\label{sec:handlingHeadOnApproach}

\paragraph{Summary:} Two UAS are facing each other head-on. There is a need to define triggers for detection and resolution approach for autonomous UAS.  Rules for VFR/IFR modes in manned aviation are the base for the autonomous collision resolution. The concept of the virtual roundabout is introduced.

\paragraph{Goal:} Identify required parameters sufficient for automatic solution of \emph{Head-on collision} situation.

\paragraph{VFR:} The \emph{Visual Flight Rules} (VFR) are specified in annex 2 \cite{icaoAnnex2}, and there is a \emph{Head-on} approach for two or more air crafts. The definition is rather vague: "The pilot should diverge from original heading to the right to create sufficient, safe space for avoidance." 

\paragraph{IFR:} The \emph{Instrument Flight Rules} in annex 2. \cite{icaoAnnex2} and 11. \cite{icaoAnnex11} are defining the boundaries and events for success full \emph{Head-on resolution} in larger detail. 

The parameter values are useless due to the UAS scaling factor; the following parameters can be used in UTM:

\begin{enumerate}
    \item The \emph{angle of approach $\ge 130^\circ$} - the minimal planar angle between aircraft positions and expected collision point is in the interval $[130^\circ,180^\circ]$.
    
    \item \emph{Minimal detection range} - the minimal detection range of head-on collision is $2\times turning Radius + safety Margin$.
    
    \item \emph{Safety margin} - during avoidance all aircraft keeps mutual distance at least the value of safety margin.
\end{enumerate}

\paragraph{Triggering Events:} The \emph{head-on approach} (fig. \ref{fig:HeadOnApproachTheoretical}) \emph{triggering events} are the following:
\begin{enumerate}
    \item \emph{Detection} (fig. \ref{fig:HeadOnApproachTheoreticalDetection}) - the \emph{collision case} is open when \emph{collision point} with the respective angle of approach is detected. This must happen until the \emph{point of no return} is achieved. 
    
    \item \emph{Resolution} (fig. \ref{fig:HeadOnApproachTheoreticalResolution}) - the \emph{virtual} roundabout is enforced until the closing condition is met. 
    
    \item \emph{Closing} (fig. \ref{fig:HeadOnApproachTheoreticalResolution}) - based on the condition that all vehicles are heading away from \emph{collision point} and their mutual heading is neutral or opposite.
\end{enumerate} 


\begin{figure}[H]
	\centering
    \begin{subfigure}{0.45\textwidth}
    	\centering
        \includegraphics[width=0.9\linewidth,height=95pt,keepaspectratio]{\FIGDIR/RE008HeadOnApproach01} 
        \caption{Detection.}
        \label{fig:HeadOnApproachTheoreticalDetection}
    \end{subfigure}
    \begin{subfigure}{0.45\textwidth}
    	\centering
        \includegraphics[width=0.9\linewidth,height=95pt,keepaspectratio]{\FIGDIR/RE009HeadOnApproach02} 
        \caption{Resolution/Closing.}
        \label{fig:HeadOnApproachTheoreticalResolution}
    \end{subfigure}
    \caption{Head-on approach detection/resolution/Closing}
    \label{fig:HeadOnApproachTheoretical}
\end{figure}

\paragraph{Virtual roundabout:} The \emph{flight levels} can be abstracted as the  \emph{virtual 2D surface}. The \emph{airspace attendants} are moving on virtual routes which can cross each other. The idea is to create virtual roundabout with enforced velocity to enable smooth collision avoidance.

\begin{enumerate}
    \item \emph{Center} - the center defined in \emph{airspace cluster} local coordinate system (flight level defining the horizontal placement).
    
    \item \emph{Diameter} - the minimal distance to \emph{center}, accounting the \emph{wake turbulence} and other phenomena. 
    
    \item \emph{Enforced velocity} - all attendants at \emph{virtual roundabout} keeps the same velocity. It helps to keep constant mutual distances.
\end{enumerate}



\subsection{Handling Converging Maneuver}\label{sec:handlingConvergingManuever}

\paragraph{Summary:} Two planned trajectories of the UAS are perpendicular, thus resulting in a protentional collision.  There is a need to define triggers for detection and resolution approach for autonomous UAS.  Rules for VFR/IFR modes in manned aviation are the base for the autonomous collision resolution.

\paragraph{Goal:} Identify \emph{required parameters} sufficient for automatic solution of \emph{Converging Maneuver}.

\paragraph{VFR:} The \emph{Visual Flight Rules} (VFR) are specified in annex 2 \cite{icaoAnnex2}. The rule is different from \emph{Head-on Approach} (sec. \ref{sec:handlingHeadOnApproach}) because multiple roles are depending on the relative aircraft position:
\begin{enumerate}
    \item \emph{Avoiding Aircraft} - there is an aircraft on the relative right side (blue). 
    \item \emph{Right Of the Way (ROA) Aircraft} - there is an aircraft on the relative left side (red). 
\end{enumerate}

The \emph{avoiding aircraft} should take the \emph{right of the way aircraft} from behind, with sufficient \emph{safety margin}, and return to original \emph{heading} afterward. The \emph{magnitude} of \emph{avoidance curve} must consider \emph{wake turbulence} and other impacts of \emph{avionic properties}.

\begin{note}
    This rule is applied only when both \emph{aircraft} belong to the same  \emph{maneuverability class} \cite{icaoAnnex2}.
\end{note}

\paragraph{IFR:} The \emph{Instrument Flight Rules} in annex 2. \cite{icaoAnnex2} and 11. \cite{icaoAnnex11} are defining \emph{converging maneuver} in detail. The \emph{parameters} from a \emph{head-on approach} can be reused:

\begin{enumerate}
    \item $70^\circ$ $\le$ the \emph{Angle of Approach} $<$ $130^\circ$ - the minimal planar angle between aircraft position and expected collision point is in the interval $[70^\circ , 130^\circ[$.
    
    \item\emph{Minimal detection range} - given as $turning Radius + safety Margin$, while \emph{safety margin} is accounting all impact factors. 
    
    \item\emph{Safety margin} - during avoidance all aircraft keeps mutual distance at least on the value of \emph{Safety Margin}.
\end{enumerate}

\begin{note}
The lesser \emph{angle of approach} induces stronger wake turbulence impact on avoiding aircraft. This results in an increase of \emph{safety margin}. 

The \emph{wake turbulence} is represented as a droplet at the back of the plane. \emph{Wake turbulence range} can be calculated based on wake turbulence cone.
\end{note}

\begin{figure}[H]
	\centering
    \begin{subfigure}{0.32\textwidth}
    	\centering
        \includegraphics[width=0.9\linewidth,height=105pt,keepaspectratio]{\FIGDIR/RE005ConvergingManeuver01} 
        \caption{Detection.}
        \label{fig:ConvergingManeuverTheoreticalDetection}
    \end{subfigure}
    \begin{subfigure}{0.32\textwidth}
        \centering
        \includegraphics[width=0.9\linewidth,height=105pt,keepaspectratio]{\FIGDIR/RE006ConvergingManuever02} 
        \caption{Resolution.}
        \label{fig:ConvergingManeuverTheoreticalResolution}
    \end{subfigure}
    \begin{subfigure}{0.32\textwidth}
        \centering
        \includegraphics[width=0.9\linewidth,height=105pt,keepaspectratio]{\FIGDIR/RE007ConvergingManuever03} 
        \caption{Closing}
        \label{fig:ConvergingManeuverTheoreticalClosure}
    \end{subfigure}
    \caption{Converging maneuver Detection/Resolution/Closing}
    \label{fig:ConvergingManeuverTheoretical}
\end{figure}

\paragraph{Triggering Events:} The \emph{converging maneuver} (fig. \ref{fig:ConvergingManeuverTheoretical}) \emph{triggering events} are the following:

\begin{enumerate}
    \item \emph{Detection} (fig. \ref{fig:ConvergingManeuverTheoreticalDetection}) -  The \emph{avoiding airplane} (blue) detects \emph{collision point} (blue circle) which satisfy the \emph{converging maneuver conditions}. The distance between \emph{aircraft position} and \emph{collision point} is lesser than the \emph{detection range}.
    
    \item \emph{Resolution} (fig. \ref{fig:ConvergingManeuverTheoreticalResolution}) - the \emph{Right Of the Way aircraft} (red) stays at the original course. The \emph{avoiding aircraft} (blue) follows the \emph{parallel} to another \emph{plane}. The distance of \emph{avoiding plane} to \emph{other plane trajectory} is greater or equal to \emph{safety margin}.
    
    \item \emph{Closing} (fig. \ref{fig:ConvergingManeuverTheoreticalClosure}) - when both planes have an opposite heading, and they miss each other the converging maneuver can be closed. The \emph{avoiding airplane} will return to \emph{original trajectory}  while keeping the distance from \emph{another plane} (red) at greater or equal to \emph{safety margin}.
\end{enumerate}


\subsection{Handling Overtake Maneuver}\label{sec:handlingOvertakeManuever}

\paragraph{Summary:} Two UAS are on the same airway, flying in the same direction. The slower UAS is in front of the faster UAS. The slower UAS has the right of way, and the faster UAS needs to make an overtake. There is a need to define triggers for detection and resolution approach for autonomous UAS.  Rules for VFR/IFR modes in manned aviation are the base for the autonomous collision resolution.

\paragraph{Goal:} Identify \emph{required parameters} sufficient for automatic solution of \emph{Overtake Maneuver}

\paragraph{VFR:} The \emph{Visual Flight Rules} (VFR) are specified in annex 2 \cite{icaoAnnex2}. The rule states that faster air traffic attendant may overtake slower one, from right side keeping sufficient distance (\emph{safety margin}). There are two forced roles:

\begin{enumerate}
    \item \emph{Overtaking} - faster aircraft with similar heading cruising in similar altitude than \emph{overtaken} (blue). It is expected that \emph{faster aircraft} has maneuvering capability to avoid slower aircraft.
    
    \item \emph{Overtaken} - slower aircraft which keeps the \emph{Right of the way}
\end{enumerate}

\begin{note}
    This rule is applied only when both aircraft have the same maneuverability class \cite{icaoAnnex2}. The overtake is considered \emph{borderline emergency maneuver} in controlled airspace because the aircraft tend to keep similar velocity in similar cruising altitude. The overtake is usual in \emph{non-controlled airspace}.
\end{note}

\paragraph{IFR:} The \emph{Instrument Fight Rules} in annex 2. \cite{icaoAnnex2} and 11. \cite{icaoAnnex11} are defining the converging manual in detail:

\begin{enumerate}
    \item 0$^\circ \le$ the \emph{Angle of Approach} $<$ 130$^\circ$ - the minimal planar angle between aircraft position and expected collision point is in the interval $[0^\circ,70^\circ[$
    
    \item \emph{Minimal Detection Range} - given as $2 \times  reaction Time \times speed Difference$. 
    
    \item \emph{Safety Margin} - during avoidance the overtaking aircraft keeps the minimal distance of \emph{wake turbulence} of overtaken aircraft in own flight altitude. 
\end{enumerate}

\begin{note}
    The \emph{Safety Margin} is sufficiently small because speed difference is usually much lesser than in case of  \emph{Head-on approach}. The \emph{Wake turbulence} can be avoided completely by taking the higher altitude level than overtaken aircraft.
\end{note}



\begin{figure}[H]
	\centering
    \begin{subfigure}{0.32\textwidth}
        \includegraphics[width=0.9\linewidth,height=142pt,keepaspectratio]{\FIGDIR/RE010OvertakeMAnuever01} 
        \caption{Detection.}
        \label{fig:OvertakeManeuverTheoreticalDetection}
    \end{subfigure}
    \begin{subfigure}{0.32\textwidth}
        \includegraphics[width=0.9\linewidth,height=142pt,keepaspectratio]{\FIGDIR/RE011OvertakeMAnuever02} 
        \caption{Resolution.}
        \label{fig:OvertakeManeuverTheoreticalResolution}
    \end{subfigure}
    \begin{subfigure}{0.32\textwidth}
        \includegraphics[width=0.9\linewidth,height=142pt,keepaspectratio]{\FIGDIR/RE012OvertakeMAnuever03} 
        \caption{Closing.}
        \label{fig:OvertakeManeuverTheoreticalClosure}
    \end{subfigure}
    \caption{Overtake maneuver Detection/Resolution/Closing}
    \label{fig:OvertakeManeuverTheoretical}
\end{figure}

\newpage
\paragraph{Triggering events:}
\begin{enumerate}
    \item \emph{Detection} (fig. \ref{fig:OvertakeManeuverTheoreticalDetection}) - occurs when the distance between \emph{overtaking} (blue) and overtaken (red) is approaching \emph{minimal detection range} or double of \emph{safety margin}. If the performance of \emph{overtaking aircraft} (blue) allows taking \emph{sharp right side to overtake} the \emph{Maneuver starts}, otherwise \emph{overtaking aircraft} (blue slows down) and keeps at least \emph{safety margin distance} to avoid \emph{wake turbulence}.
    
    \item \emph{Resolution} (fig. \ref{fig:OvertakeManeuverTheoreticalResolution}) - \emph{overtaken} (red) is keeping same heading and \emph{speed} during overtake maneuver. The \emph{overtaking} (blue) projects two waypoints: \emph{Divergence} and \emph{Convergence} keeping the required separation minimum during overtake. Then the \emph{overtaking} (blue) diverges heading to \emph{Divergence waypoint}. When the \emph{Divergence waypoint} is reached by \emph{overtaking} (blue) aircraft, it changes to \emph{original heading}.
    
    \item \emph{Closing} (fig. \ref{fig:OvertakeManeuverTheoreticalClosure}) - the \emph{closing} of \emph{Overtake} starts when \emph{overtaking} aircraft (blue) have sufficient lead over \emph{overtaken} aircraft (red). The \emph{overtaking} aircraft (blue) can safely change the heading to the original waypoint.
\end{enumerate}


\paragraph{Constant Cruising Speed:} Most of the traffic attendants at same flight level have similar (close to constant) cruising speed. Lower flight levels are for slower turbo-prop planes, and higher altitudes are for jet planes. It is stated that this principle will persist even when UAS will be integrated \cite{bayen2005langrangian,kopardekar2002dynamic,helme1992optimization} in multiple air-traffic models.


