\subsection{\secState{R}Mid-Air Collision Prevention}\label{sec:MidairCollisionPrevention}
\paragraph{Idea:} The first fatal mid-air collision occurred in 1912. The occurrence rate increased with technological progress. The \emph{European airspace management} is thoroughly analyzed in \cite{cook2007european}. 

\emph{Mid-Air Collision Situations:} The most common situations when \emph{Mid Air Collision} occurs are:

\begin{enumerate}
    \item \emph{Approach} (airport landing sequence) - the traffic density is increasing with proximity to traffic hub (airport). The aircraft lowers velocity in early approach phase to critical level which significantly reduces maneuverability. The \emph{final approach} phase is most dangerous, because the aircraft is close to the ground prepare to the landing.  
    
    \item \emph{Descent} (decrease of altitude) - the pilot is heading plane down, the dead angle is much greater than in other situations.
    
    \item \emph{Cruise}  (keeping same altitude) - the pilot is  keeping the altitude and heading, the awareness usually decreases significantly in this slight phase. 

    \item \emph{Climb} (increase of altitude) - the pilot is heading plane up, the dead angle is increased significantly. 
\end{enumerate}

\begin{note}
    The \emph{Mid-Air Collision} occurrence is strongly correlating with \emph{traffic density}. Therefore the most of \emph{near miss /collision cases} happens in vicinity of airport. It is expected to elevate the risk of \emph{Mid-Air Collision} by enabling \emph{UAS} into \emph{B, C, D, airports class}  airspace.
\end{note}

\paragraph{Collision Situation Awareness:} The surveillance capability of manned aviation is limited by \emph{pilot`s field of vision} in case of VFR (sec. \ref{sec:VisualFlightRules}) and by \emph{technical limitations} in case of IFR (sec. \ref{sec:InstrumentalFlightRules}). The \emph{surveillance and avoidance} support systems like TCAS (sec. \ref{sec:TCAS}) and ACAS (sec. \ref{sec:ACASX}) can be used as a base of future \emph{DAA} system.

The \emph{UAS} collision situation awareness system is taking the \emph{mid-air} collision prevention to another level, additional functionality needs to be implemented:

\begin{enumerate}
    \item \emph{Intruder trajectory prediction} - anticipate future \emph{intruder actions} based on gathered knowledge. Recognize dangerous actions or triggering situations. The triggering events of future dangerous situations are processed by pilot in manned aviation.
    
    \item \emph{Intruder intersection model} - anticipate future maneuverability and physical properties (turbulence, body size) in intersection model of the intruder and try to estimate an impact area in future. This process is done by pilot in manned aviation. Some supplementary information like \emph{aircraft type} and some properties are provided by surveillance system. The final decision and estimate needs to be automatized in form of impact probability or impact rating. 
    
    \item \emph{Decision making process} - the situation assessment gives an outline of the surrounding space properties, this process is well covered by \emph{surveillance}. The decision making process needs to be flexible and adaptable, but the limited computational resources needs to be taken into account (discretization problem).
\end{enumerate}

\begin{note}
    The \emph{example} of \emph{enclosed operational space} which is necessary for \emph{intruder intersection} detection is given in \cite{welzl1991smallest}.
\end{note}