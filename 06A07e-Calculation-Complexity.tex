\newpage
\subsection{\secState{R}Computation Complexity}\label{sec:MCRcomputationalComplexity}
\paragraph{Introduction:}The \emph{Computation Complexity} one mission control run assessment is necessary to identify the strong and weak points of approach. Lets get trough modules to assess notable calculations/algorithms complexity on high abstraction level.

\paragraph{Navigation Loop:} I the navigation loop, the \emph{waypoint reach condition} (eq. \ref{eq:waypointReachCondition}) is checked, this is unitary operation with worst complexity $\mathscr{O}(1)$. The selection process of the next \emph{Goal Waypoint} can get trough all waypoints in the mission if they are all unreachable the complexity is $\mathscr{O}(|waypoints|)$.

The \emph{notable steps} complexity is following:
\begin{equation*}
    \begin{aligned}
        \texttt{Reach Condition: }& \mathscr{O}(1)\\
        \texttt{Select Next Waypoint: }&\mathscr{O}(|waypoints|)
    \end{aligned}
\end{equation*}

\paragraph{Data Fusion:} The \emph{data fusion} is all about \emph{threat selection}. 

If \emph{UAS} is in \emph{controlled airspace} it needs to iterate over received \emph{collision Cases} to select \emph{active ones}. The complexity of this step is linear, therefore boundary is given as $\mathscr{O} (|collision Cases|)$.

Thresholding \emph{Detected Obstacles} is done by simple comparison of \emph{LiDAR ray hits} in given $cell_{i,j,k}$ of \emph{Avoidance Grid}.

Any loading of \emph{threats} from \emph{information sources} is depending on clustering. The \emph{Airspace Clustering} is considered as static for our setup. Therefore the \emph{count of active airspace clusters} has main impact on complexity. The \emph{count of information sources} is static and not changing over mission time. Information sources usually implement \emph{Hash search function} with complexity $\mathscr{O}\ln|searched Item Set|$.

\noindent The \emph{computation complexity} boundaries for \emph{Data fusion} in  our setup are following:
\begin{equation*}
    \begin{aligned}
        \texttt{Select Active Collision Cases: }& \mathscr{O} (|collision Cases|)\\
        \texttt{Threshold Detected Obstacles: }& \mathscr{O}(|cells|)\\
        \texttt{Load Map Obstacles: }& \mathscr{O}(\ln|activeClusters|\times|information Sources|)\\
        \texttt{Load Hard Constraints: }& \mathscr{O}(\ln|activeClusters|\times|information Sources|)\\
        \texttt{Load Soft Constraints: }& \mathscr{O}(\ln|activeClusters|\times|information Sources|)
    \end{aligned}
\end{equation*}

\begin{note}
    The \emph{real-time clustering} is \emph{hard non-polynomial problem} \cite{kleinberg1998microeconomic}.  Usually all information sources and sensor have \emph{polynomial complexity} of processing. The \emph{controlled airspace clusters} are usually set for very long period of time. Therefore \emph{Obstacle Map}, \emph{Airspace Constraints}, and, \emph{Weather Constraints} can be considered as preprocessed
\end{note}

\paragraph{Situation Assessment:} The \emph{Situation Assessment} is evaluating triggering events. The \emph{evaluation} is usually simple existence question without further calculations. The \emph{complexity} of \emph{event evaluation} for our case is $\mathscr{O}(1)$. There are 8 triggers. The count of \emph{triggers} needs to be accounted in complexity boundary:

\begin{equation*}
    \mathscr{O}(|triggers|\times event Evaluation Complexity)    
\end{equation*}

\begin{note} The \emph{trigger calculation complexity} needs to stay low, because the \emph{triggers} are verified every \emph{Mission Control Run}. The \emph{Avoidance Run} trigger frequency should be very low under normal conditions.  
\end{note}


\paragraph{Avoidance Run:} The \emph{Avoidance run} is most critical part of \emph{Mission Control Run}, because \emph{Avoidance Path} calculation. The \emph{Navigation Path} calculation is less complex (Rule engine is not accounted), therefore \emph{Emergency Avoidance Mode} is assumed. 

The \emph{threat insertion} is realized in 7\textsuperscript{th} to 10\textsuperscript{th} step. The first is \emph{Avoidance Grid} filled with \emph{Static Obstacles}. The \emph{Avoidance Grid} is designed to separate rotary  \emph{LiDAR} ray space into hit count even cells. Insertion of \emph{LiDAR} scan into \emph{Avoidance Grid} complexity depends on \emph{total cell count}. The \emph{upper boundary} for \emph{insert obstacles} is given like follow:

\begin{equation*}
    \texttt{Insert Obstacles: } \mathscr{O}(|cells|)
\end{equation*}

\noindent The \emph{intruders intersection model} type impact the insertion complexity. The \emph{linear intersection} (sec. \ref{s:linearIntersectionModel}) is going trough maximum of \emph{layers count} cells. 

The \emph{body volume intersection model} (sec. \ref{s:bodyvolumeIntersection}) can check the \emph{simple intersection condition} over all \emph{Avoidance Grid} in worst case, therefore complexity for this check is bounded by \emph{count of cells}. 

The \emph{Maneuverability Uncertainty Intersection} (sec. \ref{s:uncertaintyIntersection}) can hit all cells in \emph{Avoidance Grid}. The calculation complexity boundary is exponential depending on \emph{horizontal/vertical} spread in $[rad]$. The \emph{intersection} implementation was done \emph{ad-hoc}. The impact of \emph{intersection application} is visible only when there is more than \emph{4} concurrence intruders (fig. \ref{fig:emergencyHeadOnMultipleComputationTime}).

\noindent The \emph{complexity boundary for \emph{intruder insertion}} is given like follow:

\begin{equation*}
    \texttt{Insert Intruders: }
    \mathscr{O}\left(\sum \begin{bmatrix}
        |linear Intersections| \times |layers|\\
        |body volume Intersections| \times  |cells|\\
        |cells|^{horizontal Spread \times vertical Spread}\\
    \end{bmatrix}\right)
\end{equation*}

\begin{note}
    The \emph{intruder intersection} is critical in \emph{non-controlled airspace}. The main complexity gain in \emph{controlled airspace} is from \emph{rule application}. Our \emph{rule complexity} is in worst case depending on \emph{Reach Set} node count and \emph{Active Collision Cases} count.
    
    \begin{equation*}
        \texttt{Apply Our Rules: } \mathscr{O}(|active Collision Cases| \times |nodes|)
    \end{equation*}
\end{note}

\noindent For \emph{Hard/Soft Constraints} The algorithm used for intersection polygons was selected based on study \citep{bentley1979algorithms}, the selected algorithm  \emph{Shamos-Hoey} \cite{shamos1976geometric}. The \emph{calculation complexity} boundary is given like follow:

\begin{multline*}
    \texttt{Hard Constraints Intersection:}\\ \mathscr{O}(|cells|\times|hard Constraints| \times \max |constraint Points|^2)
\end{multline*}
\begin{multline*}
    \texttt{Soft Constraints Intersection:}\\ \mathscr{O}(|cells|\times|soft Constraints| \times \max |constraint Points|^2)
\end{multline*}

\noindent Each \emph{threat} category application in \emph{Mission Control Run} is done after \emph{each intersection} in 7\textsuperscript{th} to 10\textsuperscript{th} step. All ratings (tab. \ref{tab:defuzificationRatings}) expect $Reachibility(cell_{ij,k})$ and $Reachibility(Trajectory)$  are calculated. The \emph{calculation complexity} boundary for one \emph{reachibility rating} is $\mathscr{O}(1)$. (eq. \ref{eq:trajectoryReachibility}, \ref{eq:cellReachibility}). The \emph{Recalculate Reachibility} operation applied $4\times$ have maximal \emph{complexity} boundary given as follow:

\begin{equation*}
    \texttt{Recalculate Reachibility: } \mathscr{O}(4 \times (|nodes| + |cells|))
\end{equation*}

\noindent Each time at the end of in 7\textsuperscript{th} to 10\textsuperscript{th} step the \emph{Avoidance Path is Selected}. The \emph{Worst Case} (expected) scenario is to \emph{select} four paths for each \emph{treath} application. The algorithm for \emph{best path selection} (alg. \ref{alg:FindBestPathAvoidanceGrid}) iterates over all \emph{cells} in avoidance grid and over all \emph{trajectories} passing trough that cell. The complexity boundary for \emph{path selection} is given as follow:

\begin{equation*}
    \texttt{Select Path: } \mathscr{O}\left(4 \times \left(|cells|+\frac{|nodes|}{|cells|}\right)\right)
\end{equation*}


\paragraph{Conclusion:}  Overall approach complexity is \emph{low}. If proper \emph{Information Sources} with efficient clustering and \emph{intersection models for intruders} are used, the approach will stay within \emph{non-polynomial complexity}. 
The average load time for \emph{testing scenarios} is summarized in (tab. \ref{tab:computationLoadStatistics}).

\begin{note}
    The calculation of \emph{Reach Set} is eliminated by pre-calculation for \emph{state range} \cite{gomola2017obstacle}.
\end{note}