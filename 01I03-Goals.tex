\section{Goals}\label{s:goals}
\paragraph{Situation:} The \emph{UAS} equipped with cooperative/non-cooperative surveillance sensors, with prior knowledge of operation space has to fly a mission represented ordered set of waypoints. The set of sensors can change depending on UAS construction. The minimal airworthiness for a given operation is assumed.

\paragraph{Problem:} Given environment and artifact definitions (sec. \ref{s:basicDefinitions}) with \emph{initial assumptions} (sec. \ref{s:initialAssumptions}) and \emph{incremental problem definition} (sec. \ref{s:IncrementalProblemDefinition}) develop \emph{obstacle avoidance framework} which will satisfy \emph{avoidance} (sec. \ref{s:AvoidanceRequirements}) and \emph{navigation} (sec. \ref{s:navigationRequirements}).

\paragraph{Expected Solution:} Define an approach based on \emph{reach sets} which are capable of:

\begin{enumerate}
    \item \emph{Static obstacle avoidance} - to avoid the ground, man-made structures in open terrain. 
    
    \item \emph{Intruders avoidance} - to avoid flying objects which does not have the intention to harm our UAS, detected in sufficient distance. 

    \item \emph{Geo-fencing support} - to avoid known zones/airspace portions, which have forbidden entry.

    \item \emph{Weather avoidance} - to avoid known zones of harmful weather conditions.

    \item \emph{Cooperative conflict resolution} - to communicate own position to authority and to follow authority orders.
    
    \item \emph{Treat prioritization} - to assess avoidance according to natural or man-made priorities. (Rather break geo-fence, than crashing into the ground).
\end{enumerate}

\paragraph{Validation:} Develop test-framework to showcase approach properties. Define \emph{test scenarios} (sec. \ref{s:testingApproach}) to validate \emph{Expected Solution Performance} (sec.  tab. \ref{tab:testCasesSummary}) concerning \emph{avoidance capability} (sec. \ref{s:performanceEvaluationTable}) and \emph{computational feasibility} (sec. \ref{s:ComputaitonFootprint}).

\paragraph{Application Requirements:} There are  following application requirements, based on similar applications for \emph{manned aviation} and \emph{industry expectations}:

\begin{enumerate}
    \item \emph{Low-performance requirements} - the computational footprint of the approach should be polynomial. The most of actual UAS systems have \emph{embedded computer} with low computation power.
    
    \item \emph{Deterministic} - the \emph{avoidance strategy} should be achieved in finite time frame. The mandatory requirement for \emph{airborne operation support application}, the advice needs to be reproducible under the same conditions.
    
    \item \emph{Scalability} - the \emph{avoidance framework} should be portable to the different platforms, and it should work with different sensor array. The interface requirement for \emph{control} and \emph{data fusion} coming from other \emph{collision avoidance systems}.
    
    \item \emph{Adaptability} - the \emph{avoidance process} should have tuning points where is possible to change behavior according to UAS context. The regulations are changing with location, time and circumstances, the part of calculation/control process needs to be implemented dynamically.
\end{enumerate}