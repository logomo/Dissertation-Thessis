\chapter{Approach}

\section{Avoidance Framework}\label{s:AvoidanceFramework}
    \emph{To be done here:}
    \begin{itemize}
        \item Avoidance framework introduction, containing main advantages of avoidance framework:
        \item Control layer interface - via movement automaton.
        \item Sensor layer interface  - via custom made data fusion block and rules.
        \item Concept of virtual obstacles to accommodate various restrictions.
        \item Incorporation of rules of the air (This must be developed).
    \end{itemize}
        
\subsection{Framework overview}\label{s:FrameworkOverview}
    \emph{To be done here:}
    \begin{itemize}
        \item Enhanced picture from \emph{Optimal control report} with valid and proper markings and nomenclature established in \emph{02ProblemStatement} and \emph{03StateOfArt}.
        \item Event base layer components description + references to proper definitions overall text
        \item Discrete layer components description + references to proper definitions in overall text
        \item Continious layer components description + reference to proper defitintions in overall text
        \item all previously mentioned definitions in compact from can be taken from \emph{Article}.
    \end{itemize}
    
\section{Avoidance Grid}\label{s:AvoidanceGrid}
    \emph{To be done here:}
    \begin{itemize}
        \item define cell $c_{i,j,k}$ segmentation,
        \item define trajectory representation for system,
        \item define membership function,
        \item define rating binding to cell and trajectory segment.
    \end{itemize}
    
\subsection{Data Fusion}\label{s:DataFusion}
    \emph{To be done here:}
    \begin{itemize}
        \item Introduce concept of data fusion in terms of general algorithm \emph{Probabilistic approach report} (sec. 5).
        \item Define \emph{obstacle, visibility, reachability} ratings for cell/trajectory just as concepts.
        \item Define general fusion formula for deterministic approach.
    \end{itemize}
    
\subsection{Detected Obstacle}\label{s:DetectedObstacle}
    \emph{To be done here:}
    \begin{itemize}
        \item LiDAR tresholding to determine obstacle in cell
        \item Visibility principle to mark hindered space.
    \end{itemize}

\subsection{Map Obstacle}\label{s:Map Obstacle}
    \emph{To be done here:}
    \begin{itemize}
        \item Map obstacle data fusion.
        \item Different modes of fusion for charted obstacles and virtual obstacles.
    \end{itemize}

\subsection{Intruder Intersection Model}\label{s:IntruderIntersectionModel}
    \emph{To be done here:}
    \begin{itemize}
        \item define intruder parameters to be considered in model: position, velocity, class size (vehicle body radius).
        \item define various intersection models: a. Linear intersection model, b. Tubular intersection model (To be implmemented), c. Body intersection model, d. Uncertaininty spread intersection model.
        \item Introduce timed/untimed intersection model
        \item Compare intruder intersection model with virtual obstacle intersection model. 
    \end{itemize}
    
\section{Decision Making}\label{s:Decision making}
    \emph{To be done here:}
    \begin{itemize}
        \item Emphatize the reason of decision time existence.
        \item Define formal rule structure which will be later used in subchapters (TODO)
        \item Elaborate on feasibility of rule appliance (indirect via virtual obstacles) direct via forced movement set constraints.
    \end{itemize}

\subsection{Rules of the Air and Right of the Way}\label{s:RulesofTheAir}
    \emph{To be done here:}
    \begin{itemize}
        \item Define aerial vehicle classes (up to UAVS)
        \item Elaborate on need to introduce autonomous and piloted UAS classes.
        \item Introduce rules of the air emphatize the manned aviation as adversary
        \item introduce \emph{Right of the way}: collision, clash and overtake
        \item Propose movement set restrictions or other means to implement these simple rules.
        \item Discuss the under-performance of these rules in terms of manned aviation safety and try to modify them to fit needs of obstacle avoidance systems.
    \end{itemize}
    
\subsection{Air Traffic Management Restrictions}\label{s:AirTrafficManagementRestrictions}
    \emph{To be done here:}
    \begin{itemize}
        \item Define reference altitudes, emphatize different altitudes measurements in terms of flight levels
        \item introduce rules of different flight levels , focusing on, B, C F, G zones 
        \item propose mechanism of restricted flight areas as virtual obstacles implementation
        \item discuss the mechanism how to address breaching into such airspaces (if NASA UTM will mandate some at time of dissertation).
    \end{itemize}
    
\subsection{Weather Forecast Restrictions}\label{s:WeatherForecastRestrictions}
    \emph{To be done here:}
    \begin{itemize}
        \item introduce weather conditions varying with time 
        \item introduce concept of moving dynamic obstacles 
        \item emphatize necessity of weather avoidance (manned aviation viewpoint modification)
    \end{itemize}
    
\section{Vehicle Model}
    \emph{To be done here:}
    \begin{itemize}
        \item just model of plant description
    \end{itemize}
    
\section{Movement Automaton Control}
    \emph{To be done here:}
    \begin{itemize}
        \item  Movement automaton definition and justification of such movement set
    \end{itemize}

\section{Platform Specific Rules}
    \emph{To be done here:}
    \begin{itemize}
        \item Identification, Elaboration and discussion of platform specific rules.
    \end{itemize}