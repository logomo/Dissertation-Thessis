\subsection{Movement Automaton Applications}\label{sec:MovementAutomatonBackground}

    \noindent\emph{Movement Automaton} is a basic interface approach for discretization of \emph{trajectory evolution}  or \emph{control input} for any \emph{continuous or discrete system model}.
    
    \emph{Main function} of \emph{Movement Automaton is} for system given by equation $\dot{state}=f(time,state,input)$ with initial state $state_0$ to generate \emph{reference trajectory} $\hat{state}(t)$ or \emph{control signal} $input(t)$.
    
    Using \emph{Movement Automaton} as \emph{Control Proxy} will provide us with \emph{discrete command chain} interface. This will reduce the \emph{non-deterministic} element from \emph{Evasive trajectory} generation, by reducing infinite maneuver set to finite \emph{movement set}.
    
    \emph{Non-determinism} of \emph{Avoidance Maneuver} has been discussed as an issue in following works:
    \begin{enumerate}
        \item Newton gradient method for evasive car maneuvers \cite{vsantin2011combined}.
        \item Non-holistic methods for trajectory generation \cite{pin1990autonomous}.
        \item Stochastic approach to elliptic trajectories generation \cite{andrzejak2001epileptic}.
    \end{enumerate}
    
\noindent\emph{Examples} of \emph{Movement Automaton Implementation} as \emph{Control Element} can be mentioned as follows:
    \begin{enumerate}
        \item Control of traffic flow \cite{kuwata2009real}.
        \item Complex air traffic collision situation resolution system  \cite{frazzoli2001robust,frazzoli2000trajectory}.
        \item SAA/DAA capable avoidance system \cite{gomola2017obstacle}.
    \end{enumerate}


