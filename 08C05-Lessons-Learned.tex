\section{\secState{R}Lessons Learned}\label{s:lessonsLearned}
\noindent During the approach development, some mistakes were made. This section summarizes the most notable design choices with the reasoning behind them.

\paragraph{Euclidean vs. Planar Space Partitioning:} The \emph{operational space discretization} was planned from the beginning. The initial approach was to use a \emph{uniform Euclidean  grid} where the smallest space portion was represented as cubic cell.

\paragraph{Euclidean grid shortcomings:} The problem was with \emph{LiDAR} reading assessment because for each point it was necessary to do \emph{transformation} defined as  a series of the functions (eq. \ref{eq:cpt01}, \ref{eq:cpt02}, \ref{eq:cpt03}, \ref{eq:cpt04}). The continuous updating of the \emph{space assessment} usually timed-out. 

The other issue was with wavefront algorithm (alg. \ref{alg:Wavefront Propagation}). The intersection (def. \ref{def:ContainedReducedReachSet}) function was not problematic, due the unlimited estimation time of \emph{reach set approximation}. The \emph{reach set approximation} is shaped usually like the cone; therefore each propagation caused hitting more cells in Euclidean  grid. This caused the following challenges:

\begin{enumerate}
    \item \emph{An exponential growth of trajectory count} - the \emph{trajectory footprint} (def. \ref{def:trajectoryFootprint}) is almost always unique in Euclidean Grid. The problem is that two very similar trajectories concerning avoidance were considered unique.
    
    \item \emph{Low avoidance strategies variability} - the generated trajectories by \emph{harmonic} (sec. \ref{s:harmonicReachSet}) or \emph{chaotic} (sec. \ref{s:chaoticReachSet}) reach set approximation did not follow the desired properties of coverage.
    
    \item \emph{Threat impact reflection} - the space which is closer to UAS needs to be surveillance with greater detail than space which is further away from UAS. The Euclidean grid does not reflect this.
\end{enumerate}

\paragraph{Euclidean Grid Benefits:} The intersection model performance was good because of most of the numeric/analytically intersection solutions for:
\begin{enumerate}
    \item \emph{Line} - used in \emph{linear intruder intersection model} (app. \ref{s:linearIntersectionModel}).
    
    \item \emph{Tunnel} - used in \emph{body-volume intersection model} (app. \ref{s:bodyvolumeIntersection}).
    
    \item \emph{Polygon} - used in constraints (sec. \ref{s:virtualConstraints}, def. \ref{def:movingConstraint})
\end{enumerate}

\paragraph{Planar Grid Deployment:} The shortcomings and benefits of \emph{Euclidean grid} are obvious, the main reason why the \emph{planar gird} was employed is the \emph{performance of the wave-front algorithm} (alg. \ref{alg:Wavefront Propagation}) for UAS system dynamic. The main focus of this work lies on \emph{Reach Set} properties, the planar grid (fig. \ref{fig:LidarSpaceSegmentation}) properties:

\begin{equation*}
    \uparrow distance \implies \uparrow cell Size \implies \downarrow importance
\end{equation*}

\noindent are ideal for \emph{avoidance problems}. The cell size is increasing proportionally with increasing distance. The space partitioning at the lower distance is dense, reflecting the importance of proximity threats. The space partitioning at the greater distance is getting looser, reflecting the unimportant of threats at greater distance. 

\noindent The \emph{planar grid} employment has following benefits:
\begin{enumerate}
    \item Fast \emph{LiDAR} (and other ranging sensors) clustering.
    \item Reflection of space importance.
    \item A significant increase of \emph{wave-front} algorithm.
    \item A significant decrease in reach set approximation node count.
\end{enumerate}

\noindent The \emph{planar grid} employment has the following shortcomings:
\begin{enumerate}
    \item Implementation complexity of intruder intersection models.
    \item Implementation complexity of constraints intersection models.
\end{enumerate}

\begin{note}
    The count of intruder and constraints is not so great concerning \emph{aerial transpiration}; the situation is opposite in case of \emph{ground transportation} especially road vehicles. 
\end{note}

\paragraph{Intruder Intersection Models:} The \emph{implemented intruder intersection models} (sec. \ref{s:intruderBehaviourPrediction}) are basic and sufficient. There are few ideas which do not make it into the final concept:

\begin{enumerate}
    \item \emph{Intruder as line/curve intersection} - the idea is to use a parametric curve instead of the \emph{linear intersection model} (app. \ref{s:linearIntersectionModel}). The problem is those curve parameters are hard to obtain; they are not mandatory part of any position notification proposal so far.
	
	\item \emph{Intruder as movement automaton simulated trajectory} - the idea is to have access to movement automaton of an intruder in \emph{predictor mode} (sec. \ref{s:referenceTrajectoryGenerator}) to generate future expected trajectory. The problem is in some situations the trajectory of intruder changes a lot and \emph{maneuverability uncertainty intersection} (app. \ref{s:uncertaintyIntersection}) outperforms this approach significantly. The other problem is that most of the \emph{general aviation} does not want to share this kind of information due to security reasons.  This concept makes it into the final solution, but not in \emph{avoidance framework}, but as \emph{UTM collision case intersection calculation} (sec. \ref{sec:collisionCase}).
	
	\item \emph{Intruder well clear implementation as a rule} - the idea is to implement \emph{potential fields principle} as core mechanism of avoidance. The problem is determinism of this approach, and the idea was scrapped.
\end{enumerate}

\paragraph{Approach to data fusion:} To make our approach universal, the decision to use input/output interfaces were made (fig. \ref{fig:AvoidanceFrameworkConceptNew}). The choice of the \emph{output control interface} is clear, the movement automaton (sec. \ref{sec:MovementAutomatonBackground}) was performing well from the beginning \cite{gomola2017optimal,gomola2017mpc}.

The \emph{input interface} represented as \emph{data fusion} (theory sec. \ref{s:dataFusionDefinition}, implementation sec \ref{s:sensorFusion}), was more problematic.

\paragraph{Deterministic approach:} The article \cite{gomola2017obstacle} and technical reports \cite{gomola2017mpc,gomola2017optimal} are using a deterministic approach where the property of the avoidance grid cell is given as a Boolean value. This approach is too restrictive because all the values needed to be determined on the spot. The overall data fusion was nonexistent because the approaches were used only static obstacles and \emph{LiDAR} sensor as the base. 

\paragraph{Probabilistic approach:} The technical report \cite{gomola2017probabilistic} introduced the \emph{intruders} and \emph{map obstacles} \cite{cernamaria2018}. This requires\emph{probabilistic implementation}. The probabilistic density functions with all that theoretical apparatus are established in order to cover the processes of the data fusion and decision making. The problem was the formalization level; much \emph{effort} is wasted to formalize very simple calculations and thresholds. The tuning parameters were nonexistent, preventing the customization.

\paragraph{Final Approach as a Synthesis:} The benefits of both approaches for data fusion are used in final implementation (sec. \ref{s:sensorFusion}), the useful aspects of probabilistic approach like ratings and addition rules persisted. The useful aspects of deterministic, like threshold, is kept through predicates (tab. \ref{tab:defuzificationRatings}). The concept is the following:

\begin{enumerate}
        \item \emph{Ratings reflects probabilities} - the property of visibility (eq. \ref{eq:visibilityForCell}), obstacle encounter in cell (eq. \ref{eq:detectedObstacleRatingForCell}), map obstacle in cell (eq. \ref{eq:mapObstacleRatingForCell}), intruder intersection with cell (eq. \ref{eq:intruderRatingForCell}), constraint intersection of the cell (eq. \ref{eq:constraintRatingForCell}) are formalized as threat property (def. \ref{def:threat}). 
	    
	    \item \emph{Man-made rules incorporation} - man-made rules can be incorporated into any part of the data fusion process.
	    
	    \item \emph{Situation assessment} - the known world classification through data fusion (eq. \ref{eq:UncertainDataFusion}) is implemented as cell set classification:
	    \begin{enumerate}[a.]
	        \item \emph{Uncertain} (eq. \ref{eq:UncertainDataFusion}) - the state of this space cannot be determined by sensors or prior knowledge.
	        
	        \item \emph{Occupied} (eq. \ref{eq:ocuupiedDataFusion}) -  these cells are occupied by a physical object, entering into this space will cause real damage to UAS.
	        
 	        \item \emph{Constrained} (eq. \ref{eq:constrainedDataFusion}) - intruder or other kinds of threat may be present, this space is better to avoid.
 	        
	        \item \emph{Free} (eq. \ref{eq:freeDataFusion}) - these cells are free of any threat, this fact is verified by sensors and prior knowledge.
	        
	        \item \emph{Reachable} (eq. \ref{eq:ReachableDataFusion}) - these cells are free and moreover there exists at least one \emph{safe trajectory} leading there, the safe trajectory is a trajectory from \emph{reach set approximation} which goes through free space.
	    \end{enumerate}
\end{enumerate}	    

\begin{note}
    The formal definition for data fusion (sec. \ref{s:dataFusionDefinition}), known world (sec. \ref{s:KnownWorld}) and their role in problem solution (sec. \ref{s:BasicProblemDefinition}) are impacting the final data fusion implementation.
\end{note}
