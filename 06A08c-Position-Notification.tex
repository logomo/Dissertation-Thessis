\subsection{(W) Position Notification}\label{sec:positionNotification}
    \noindent Process of creation for data structure from table \ref{tab:positionNotification}.
    \begin{tabularx}{\textwidth}{S{0.25}|X}
         \multicolumn{2}{c}{\textbf{Position}}  \\\hline
         latitude & based on GPS/IMU sensor fusion.\\
         longitude & based on GPS/IMU sensor fusion.\\
         altitude & barometric altitude \emph{Above Mean Sea Level} (AMSL). \\         
         \multicolumn{2}{c}{\textbf{Heading}}  \\\hline
         orientation & orientation in standard North-East coordinate frame.\\
         velocity & relative UAS velocity.\\
         \multicolumn{2}{c}{\textbf{Flight Levels}}\\\hline
         main & flight level, where UAS mass center belongs\\
         passing & flight level, during climb/ascend, or when distance of UAS mass center to flight level boundary $\le 250 ft.$ .\\
         \multicolumn{2}{c}{\textbf{Registration}}\\\hline
         registration ID& is unique registration number \emph{to be issued} by local aviation authority for UTM communications purposes.\\
         flight code& or mission code is unique identification number for approved mission plan which is going to be flown by UAS.\\
         UAS name & optional UAS identifier to increase human recognition. \\
         \multicolumn{2}{c}{\textbf{Categorization}}\\\hline
         craft category & ICAO main category, based on vehicle type.\\
         maneuverability& secondary categorization specifying size class, horizontal/vertical turning radius, minimal and maximal cruising speed.\\
         \multicolumn{2}{c}{\textbf{Safety margins}}\\\hline
         universal & minimal safety margin for any avoidance situation\\
         head on & minimal distance from other similar maneuverability class aircraft in case of head on approach.\\
         converging & minimal distance from other similar maneuverability class aircraft in case of head of converging maneuver.\\
         overtake & minimal distance from other similar maneuverability class aircraft in case of overtake maneuver.\\
         wake angle & for wake turbulence cone.\\
         wake radius & for wake turbulence cone.\\
        \caption{Time-stamped \emph{position notification} structure.}
        \label{tab:positionNotification}
    \end{tabularx}

