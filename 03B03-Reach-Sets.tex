\section{Reach Sets}\label{s:ReachSets}
    \noindent Informally, the \emph{Reach Set} of a system described by a differential equation is the \emph{set of all states that can be reached from an initial state within a given time interval}. Similar definitions applies to the systems with different representation and control, such as \emph{hybrid automaton}.


\subsection{Definitions}\label{sec:reachSetIncrementalDefinition}
    \paragraph{For following definitions} consider \emph{nonlinear UAS system} described in (sec. \ref{s:uavMotionModel}).
    
    \begin{definition}[Reach set starting at a given point]\label{def:reachset01}
        Suppose the initial position
        and time $(state_0, time_0)$ are given. The reach set $ReachSet[\tau, time_0, state_0]$ of \emph{nonlinear system} at time $\tau \ge time_0$, starting at $(state_0, time_0)$ is given by:
        \begin{equation}
            ReachSet[\tau, time_0, state_0] = \bigcup \{state(\tau):input(s)\in Inputs(s),s \in (time_0,\tau]\}
        \end{equation}
    \end{definition}
    
    \paragraph{Reach set starting at given set} can be used to determine reach set in case of \emph{hybrid system} input control switch and it is defined as follow:
    \begin{definition}\label{def:ReachSetBasic} set starting at a given set]
        The reach set at time $\tau > t_0$ starting from set $States_0$ is defined as:
        \begin{equation}
            ReachSet[\tau, time_0, States_0] = \bigcup \{ReachSet[\tau, time_0, state_0]: state_0 \in States_0\}
        \end{equation}
    \end{definition}
    
    \paragraph{Reach set for adversarial behavior} can be used to calculate possible escape routes from pursuer and it is defined as follow:

    \begin{definition}[Reach set under adversarial behavior]
        Consider now the case of adversarial behavior(\cite{game1987,game1988}).
        where $input(t)$ is our control and $adversary(t)$ is adversary control which is independent of $input(t)$, let $differentialControl(t)=input(t)-$ $\sup_{{state} \in state(t)}$\\ $adversary(t)$, which represents worst possible input change in given state and time, then \emph{reach set for system} is represented as:
        \begin{equation}
            ReachSet\left[\begin{gathered}\tau,\\ time_0,\\ state_0\end{gathered}\right] = \bigcup \left\{state(\tau): 
                \begin{aligned}
                    differentialControl(s)\in\\  DifferentialControlSet(s)
                \end{aligned}
            ,s \in (time_0,\tau]\right\}
        \end{equation}
    \end{definition}

    \paragraph{Reach set under state constraints} are usable to define state constrained systems in terms of dynamics and technical capabilities.
    
    \begin{definition}[Reach set under state constraints]
        Suppose the initial position and time $(state_0, time_0)$ and $state$ constraints are given $state(t) \in \mathbb{A} \subset \R^n, \dot{x}(t) \in \mathbb{B} \subset \R^n$. The reach set $ReachSet[\tau, time_0, \vec{state}_0]$ of \emph{nonlinear UAS system} at time $\tau \ge time_0$, starting at position and time $(state_0, time_0)$ is given by:
        
        \begin{equation}
            ReachSet\left[\begin{gathered}\tau,\\ time_0,\\ state_0\end{gathered}\right] = \bigcup 
            \left\{
                state(\tau):
                \begin{gathered}
                    \forall s\in (time_0,\tau], state(s) \in \mathbb{A},\\ 
                    \dot{state}(s) \in \mathbb{B},\\ 
                    \exists input(s) \in Inputs(s)
                \end{gathered}
            \right\}
        \end{equation}
        
    \end{definition}


    \subsection{Computation of Reach Sets}\label{sec:ReachSetComputationMethodsASIS}
    \noindent   Several techniques for reachability analysis of systems have been proposed. They can be (roughly) classified into two kinds:
    
    \begin{enumerate}
    \item Purely symbolic methods based on: 
        \begin{enumerate}[a.]
        \item the existence of analytic solutions of the differential equations and 
    
        \item the representation of the state space in a decidable theory of the real numbers.
        \end{enumerate}
    \item Methods that combine 
    \begin{enumerate}[a.]
        \item numeric integration of the differential equations
    
        \item symbolic representations of approximations of state space typically using (unions of) polyhedra or ellipsoids.
        \end{enumerate}
    \end{enumerate}
    These techniques provide the algorithmic foundations for the tools that are available for computer-aided verification of hybrid systems (\cite{daws1996tool}, \cite{henzinger1994symbolic}, \cite{henzinger1995hytech}).

    The set-valued Lebesgue integral provides a conceptual tool for the direct computation of the reach set. In what follows we describe techniques from dynamic optimization which are used to compute reach sets for dynamic systems.

    The relation between dynamic optimization and reachability was first observed in \cite{leitmann1982optimality}. A typical problem of optimal control can be formulated as follows:

    \begin{equation}
        \max \left(
        \begin{gathered}
            \int_{initialTime}^{finalTime} cost(time,state(time),contro(time)) \textnormal{d}time + \dots \\ \dots +FinalCost(state(finalTime))\end{gathered}
        \right)
    \end{equation}
    
    \noindent For nonlinear system:
    \begin{equation}
        \dot{state}(t) = f(t,state(t),control(t)), control(t) \in ControlSet(t) \subset \R^p
    \end{equation}
    
    Where $cost$ is given as a cost function of time, state and input and $FinalCost$ represent cost functional.
    
    There are two main techniques to solve this problem, the maximum principle, and dynamic programming. 
    
    The maximum principle gives necessary conditions of optimality. Dynamic programming may be used to derive sufficient conditions of optimality.

    A good reference on the maximum principle is \cite{pontryagin1962ef}. A less known reference with detailed geometric interpretations is \cite{girsanov2012lectures}. A good reference on dynamic programming is given in  \cite{bardi2008optimal}.
    
