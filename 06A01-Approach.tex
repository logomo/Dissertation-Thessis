\chapter{(R) Approach}\label{ch:approach}

\noindent There are few levels of \emph{Avoidance} based on the \emph{remaining time to collision}. These levels are summarized in (fig. \ref{s:approachOverview}).

\begin{figure}[H]
    \centering
    \includegraphics[width=0.7\linewidth]{\FIGDIR/RE001AvoidanceLevelsBasedOnReactionTime} 
    \caption{Avoidance levels based on reaction time.}
    \label{fig:AvoidanceLevels}
\end{figure}

This work will focus on handling \emph{Event Avoidance} and \emph{Reactive Avoidance} and the \emph{Avoidance Path} will be calculated using \emph{Reach set Based Methods}. 

The \emph{Preemptive Avoidance} is trying to remove any possible threat prior the flight. The risk mitigation is tedious and its done only when necessary. Even the best \emph{preemptive} avoidance could fail.

The \emph{Reactive Avoidance} is solving most urgent situations with very short reaction opportunity. This work focus on physical obstacles and terrain. Non cooperative intruders are considered partially. The adversary behaviour was omitted.

The \emph{Event Avoidance} has more opportunity to react. Some threats are know prior the flight (geo-fenced areas, ...). The future UTM implementation is also considered as \emph{Event Avoidance}, due the time horizon and authority enforcement. 

\paragraph{Basic Idea:} Create deterministic finite-time \emph{Reactive Avoidance} based on \emph{Reach sets} to assure \emph{trajectory feasibility}. Enhance method with set of the rules to enable handling more complex situations.

The \emph{Discretization} is the key to assure calculation in finite time. Finite \emph{partition} of \emph{operational space (Known World)} and finite representation of \emph{Reach set} guarantees finite count of calculation steps. Aircraft conflict prediction mentioned in \cite{prandini2008application}.

    
\section{(R) Overview}\label{s:approachOverview}

\noindent The \emph{Overview} is based on \emph{Existing} Emergency avoidance framework \cite{gomola2017obstacle} (fig. \ref{fig:avoidanceConcept}). To achieve goals defined in \emph{Problem Definition} (sec. \ref{s:BasicProblemDefinition}, \ref{s:IncrementalProblemDefinition}) following \emph{Avoidance Framework Concept} (fig. \ref{fig:AvoidanceFrameworkConceptNew}) is proposed:

\begin{figure}[H]
    \centering
    \includegraphics[width=0.95\linewidth]{\FIGDIR/TE037ConceptualSchemeNew} 
    \caption{Avoidance Framework Concept.}
    \label{fig:AvoidanceFrameworkConceptNew}
\end{figure}


\paragraph{Structure of Avoidance Framework:}

\begin{enumerate}
    \item \emph{Unmanned Aircraft System} (UAS) (Role: Controlled Plant) - the \emph{UAS} is controlled via \emph{interface} implemented as \emph{Movement Automaton}. The model used is described in (sec. \ref{s:UASNonlinearModel}).
    
    \item \emph{Movement Automaton} (Role: Control Interface/Predictor) - consumes \emph{Discrete Command Chain} to generate discrete \emph{reference trajectory}, it can be also used as a predictor of \emph{future UAS states} (sec. \ref{s:referenceTrajectoryGenerator}). The movement Automaton used in this work is given in (sec. \ref{s:movementAutomatonDefinition}). 
    
    \item \emph{Sensor Field} (Role: Surveillance Providers), following sensors were considered in this work:
        \begin{enumerate}[a.]
            \item \emph{LiDAR} (Static obstacle detection) - detection of physical obstacles (sec. \ref{s:detectedObstacles})
            
            \item \emph{ADS-B} (Intruder UAS/Plane detection) - detection of intruders whom are broadcasting their position and heading sometimes with future plans and additional parameters. The \emph{intersection models} are given in (sec. \ref{s:intruderBehaviourPrediction}, \ref{s:linearIntersectionModel}, \ref{s:bodyvolumeIntersection}, \ref{s:uncertaintyIntersection}).
        \end{enumerate}

    
    \item \emph{Information Sources} (Role: Known World Information Enhancers): 
        \begin{enumerate}[a.]
            \item \emph{Obstacle Map} (Static Restriction Source) - imposing static soft/hard constraints on \emph{Known Word}/\emph{Operational Space}. Static constraints are given in (sec. \ref{s:virtualConstraints}).
            
            \item \emph{Weather Information} (Static/Dynamic Restriction Source) - imposing static/moving soft/hard constraints on \emph{Known World}/\emph{Operational Space}. Moving constraints are given in (sec. \ref{s:MovingVirtualConstraints}).
            
            \item \emph{Other Airspace Restrictions} - like restricted airspace, geo-fencing and other future constraint sources, all of them are covered by \emph{Static/Dynamic Constraints} for now.
        \end{enumerate}
    
    \item \emph{Data Fusion} (Role: Sensor Input Interface) - is the unifying interface to asses \emph{Operational State Properties} mainly \emph{Obstacle Rating}, \emph{Visibility}, \emph{Map Obstacle Rating}, \emph{Intruder Rating} for portion of the space. The partial \emph{ratings} are proposed in related sections. The data fusion procedure with \emph{defuzzyfication} and final assessment into space sets is outlined in (sec. \ref{s:sensorFusion})  
    
    \item \emph{Reach Set Approximation} (Role: Reachability Estimator) - as \emph{data fusion} is providing the situation assessment, the \emph{Reach set} is providing maneuvering capability assessment. The introduction is given in (sec. \ref{s:reachSet}), the properties are defined in (sec. \ref{s:ReachSetPerformanceCriteria}), the approximation methods with constrained expansion are outlined in (sec. \ref{s:chaoticReachSet}, \ref{s:harmonicReachSet}, \ref{s:combinedReachSet}, \ref{s:acasReachSet}). The reach set estimation is main contribution of this work.
    
    \item \emph{Grids: Navigation/Avoidance} (Role: Operation Space Segmentation \& Situation Evaluation) - space discretization in polar coordinates grid, different reach sets are used for different grid type, defined in (sec. \ref{s:AvoidanceGrid}).
    
    \item \emph{Avoidance loop} (Role: Short Term Decision Maker) - using data from \emph{Sensor fusion} in \emph{Avoidance/Navigation Grid} trimming \emph{Reachable Space} approximated by \emph{Reach Set} generating feasible \emph{Avoidance Path}. \emph{Avoidance Path} is fed to controlling \emph{Movement Automaton}. The Goal is given by \emph{Navigation Loop}. Avoidance loop is given in (sec. \ref{s:aviudabceGridRun}).
    
    \item \emph{Navigation loop} (Role: Long Term Decision Maker) - using data from \emph{Avoidance Loop}, \emph{Mission plan} and \emph{UTM} directives defines the current long term navigation goal. Details given in (sec. \ref{s:missionControlRun}).
    
    \item \emph{Command and Control Communication Link} (C2 Link) (Role: Communication Link) - standard communication link with sufficient reliability.
    
    \item \emph{UAS Traffic Management} (UTM) (Controlled Airspace Authority) - checking possible collisions and enforces counter-measurements. Details given in (sec. \ref{sec:UASTrafficManagement}).
    
\end{enumerate}

\paragraph{Communication in Avoidance Framework:}
\begin{enumerate}
    \item \emph{UAS $\leftrightarrow$ Movement Automaton} - sharing \emph{actual system state}, commanding the UAS platform.
    
    \item \emph{Reach Set $\leftrightarrow$ Movement Automaton} - predicting set of feasible trajectories for given situation.
    
    \item \emph{Reach Set $\leftrightarrow$ Grids} - providing trajectory set depending on active mode (Navigation/Emergency Avoidance).
    
    \item \emph{Avoidance Loop $\leftrightarrow$ Data Fusion} - assessing the situation in \emph{operational space} based on sensor readings/information sources.
    
    \item \emph{Avoidance Loop $\leftrightarrow$ Navigation Loop} - determining long term goal based on situation assessment and UTM directives. 
    
    \item \emph{Avoidance Loop $\to$ Grids} - feeding assessment data and constraints into selected operational space Grid.
    
    \item \emph{Grids $\to$ Avoidance Loop} - returning feasible and \emph{cost effective} avoidance path after situation assessment and \emph{Reach set} pruning.
    
    \item \emph{Avoidance Loop $\to$ Movement Automaton} - issuing and monitoring movement commands based on actual \emph{avoidance strategy}.
    
    \item \emph{Navigation Loop $\leftrightarrow$ C2 Link $\leftrightarrow$ UTM} - communication to receive directives and send fulfillment. 
\end{enumerate}