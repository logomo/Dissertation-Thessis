\section{\secState{R}Approach Reusability}\label{s:approachReusability}
\noindent The  \emph{avoidance framework} can be used as a whole, or some of the concepts can be transferred as modules to other approaches. This section picks such reusable concepts,

\paragraph{UTM Services:} The constrained \emph{UTM functionality} is outlined in (sec. \ref{sec:UASTrafficManagement}) including:
\begin{enumerate}
    \item \emph{Future UTM Communication Architecture} (fig. \ref{fig:UTMArchitectureOverview}) as the authority over \emph{airspace segment} (fig. \ref{fig:DAMExample})\cite{gerdes2016dynamic}.
    
    \item \emph{Cooperative Conflict Resolution Under UTM Supervision} (fig. \ref{fig:CooperativeConflictResolutionUTM}) designed as mild/feasible directives (commands) with \emph{constant supervision}.
    
    \item \emph{Rules of the Air Enforcement} (sec. \ref{sec:handlingHeadOnApproach}, \ref{sec:handlingConvergingManuever}, \ref{sec:handlingOvertakeManuever}) including designs of \emph{Position Notification} (sec. \ref{sec:positionNotification}) and \emph{Collision Case Structure/Calculation} (sec. \ref{sec:collisionCase}).
    
    \item \emph{Divergence/Convergence Waypoints} concept is showcased in \emph{Overtake Rule} (rule \ref{tab:ruleOvertakeDefinition}). 
    
    \item \emph{Weather Avoidance} (sec. \ref{sec:weatherCase}) is using a similar concept to \emph{Collision Case}: \emph{Weather Case}. The information is provided by \emph{Local Airspace Authority}.
    
\end{enumerate}

\paragraph{Emergency Avoidance Functionality:}  The standard framework implementation (fig. \ref{fig:missionControlRunActivityDiagram}) can handle the situations given in non-cooperative test cases (sec. \ref{s:noncooperativeTestCases}). The list of threats is given by (tab. \ref{tab:uncontrolledAirspaceViolations}). 

\paragraph{Event-Based Avoidance Functionality:} The standard framework implementation (fig. \ref{fig:missionControlRunActivityDiagram}) with active $C2$ link and rules setup (fig. \ref{fig:RuleEngineInstanceLevels}) can handle the situations given in cooperative test cases (\ref{s:cooperativeTestCases}). The list of threats is given by (tab. \ref{tab:controlledAirspaceViolations}). The \emph{Avoidance Mode Concept} enables to switch between \emph{Event-Based Avoidance} (Navigation) and \emph{Emergency Avoidance.}
\begin{note}
    The emergency Avoidance Functionality is included in \emph{Event-Based Avoidance} (Navigation) mode.
    The prioritization of \emph{threats} may differ (tab. \ref{tab:controlledAirspaceViolations}).
\end{note}

\paragraph{Reusability for More Complex Systems:} The framework (fig. \ref{fig:avoidanceConcept}) with implemented rule engine (fig. \ref{fig:RuleEngineBasicArchitecture}) can be used on \emph{any system}, with appropriate \emph{Movement automaton} (sec. \ref{s:segmentedMovementAutomaton}) enabling \emph{wave-front} propagation (alg. \ref{alg:Wavefront Propagation}) for reach set estimation. Following artifacts needs to be delivered for concept reuse:

\begin{enumerate}
    \item The \emph{Movement Automaton} is used to generate \emph{thick series of waypoints} which guarantees desired degree of safety.
    
    \item The \emph{complex UAS system} is following the \emph{reference trajectory} (sec. \ref{s:referenceTrajectoryGenerator}).
    
    \item The \emph{Sensor Fusion} (sec. \ref{s:SensorFusionDefinition}) implementation including classification to \emph{Free}, \emph{Occupied}, \emph{Restricted} space type.
    
    \item The \emph{sensor field} is supporting detection of threats. There should be at least one sensor with the capability of feeding \emph{Avoidance Grid}. Our implementation was based on LiDAR/ADS-B feeds.
    
    \item The \emph{Information Sources} are supporting the online/offline threat processing. This one is completely optional.
\end{enumerate}

\begin{note}
    \emph{On UTM integration:} The future UTM system will not be giving the extreme commands, the directives are more like constraints; therefore our system can provide the guidance and constraint evaluation
\end{note}

\begin{note}
    \emph{On Safety Margin:} The disparity between real flown trajectory (nonlinear dynamics) and planned trajectory (Movement Automaton) needs to be accounted into \emph{Safety Margin}.
\end{note}

\paragraph{Reach Set Approximations:} The \emph{wave-front} approach (alg. \ref{alg:Wavefront Propagation}) can be used with \emph{Constrained expansion function} (sec. \ref{s:constrainedTrajectoryExpansion}) to create own \emph{Reach set Approximation Method}. Existing  reach set approximation methods are always following a different goal; they can be reused for other tasks (perf. \ref{sec:reducedReachSetPerformance}):

\begin{enumerate}
    \item \emph{Chaotic} (def. \ref{s:chaoticReachSet}) - high space coverage, ideal for unpredictable and complex avoidance maneuvers.
    
    \item \emph{Harmonic} (def. \ref{s:harmonicReachSet}) - smooth trajectories, medium space coverage, ideal for navigation maneuvers.
    
    \item \emph{Combined} (def. \ref{s:combinedReachSet}) - a combination of the \emph{harmonic} and \emph{chaotic} approximations, the cost function defines preferred trajectories. The procedure is reusable for any reach set approximation types ($2^+$) combination.
    
    \item \emph{ACAS-X Like} (def. \ref{s:acasReachSet}) - following \emph{TCAS/ACAS separation modes}, can be used as an alternative for \emph{controlled avoidance} and \emph{navigation}.
\end{enumerate}