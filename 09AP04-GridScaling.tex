\chapter{Grid Size Calculation}\label{app:gridSizeCalculation}

\noindent The grid size calculation is done by hand. The following approach has been used in our work. 

For \emph{Sensor Field} there is \emph{effective sensor boundary} given as set:
\begin{equation}
    Boundary(Sensor \in Sensor Field) = \{points \in polarCoordinates\}
\end{equation}

\noindent The \emph{Boundary} for sensor fields is then given as \emph{union of all singe sensor boundaries}:

\begin{equation}
     Boundary(Sensor Field) = \bigcap_{\forall Sensors} Boundary(Sensor \in Sensor Field)
\end{equation}

\noindent Depending on boundary properties it can be projected into maximal avoidance grid boundary values:
\begin{equation}
    Boundary(Sensor Field) \to Avoidance Grid : 
    \begin{gathered}
        \max(distanceRange)\\
        \max(horizontalRange)\\
        \max(verticalRange)\\
    \end{gathered}
\end{equation}

\noindent Our approach taken worst LiDAR performance into account \cite{sabatini2014lidar} and following parameters for avoidance grid were calculated:

\begin{enumerate}
    \item distance range $[0m,10m]$,
    \item horizontal range $]-180^\circ,180^\circ]$,
    \item vertical range $[-30^\circ,30^\circ]$.
\end{enumerate}

\noindent The \emph{count of layers} is derived from \emph{average distance traveled by one movement application}:

\begin{equation}
    layer Count = \frac{|distance Range|}{\text{avg.}\quad length(movement\in Movement Set)}
\end{equation}

\noindent The \emph{layer length} is based on \emph{our movement set} (tab. \ref{tab:movements1}, \ref{tab:movements2}) the average movement length is 1 m; therefore the \emph{layer count} is 10.

\noindent The \emph{efficient boundary} is given by  \emph{Reach Set}. Estimate reach set coverage space using \emph{ellipsoidal toolbox} \cite{kurzhanskiy2006ellipsoidal} up to given \emph{sensor field} maximal distance:

\begin{equation}
    Boundary(Reach Set) = Ellipsoid(UAS System,distance)
\end{equation}

The values for \emph{Reach Set Boundary} with distance 10 m was following:
\begin{enumerate}
    \item distance range $[0m,10m]$,
    \item horizontal range $[-45^\circ,45^\circ]$,
    \item vertical range $[-45^\circ,45^\circ]$,
\end{enumerate}

\noindent The \emph{Avoidance Grid} boundary is given as \emph{intersection} of all boundaries:

\begin{equation}
    Boundary(Avoidance Grid) =  Boundary(Reach Set) \cap Boundary(Sensor Field)
\end{equation}

\noindent The values for \emph{Avoidance Grid Boundary} for our UAS system (sec. \ref{s:UASNonlinearModel}) following:
\begin{enumerate}
    \item distance range $[0m,10m]$,
    \item horizontal range $[-45^\circ,45^\circ]$,
    \item vertical range $[-45^\circ,45^\circ]$,
    \item layer count $10$, layer distance 1m.
\end{enumerate}

The \emph{horizontal cell count} and \emph{vertical cell count} was estimated by the  \emph{rule of thumb} to have value 7 and 5.

