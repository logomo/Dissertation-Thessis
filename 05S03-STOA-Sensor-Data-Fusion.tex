\section{\secState{R}Surveillance}\label{s:dataFusionProbabilisticModelTheory}
\paragraph{Idea:}  There is a \emph{need for the abstract representation} of \emph{operational space}, which is independent of used sensors, technologies, information sources. The universal obstacle avoidance system should be \emph{portable} between various platforms. Our previous work \emph{Obstacle avoidance framework based on reach sets} \cite{gomola2017obstacle} has introduced a concept of the \emph{control interface}. The concept of \emph{control interface} increases portability of the solution.

The original concept used cell status interpretation (boolean values), which is hardwired to LiDAR technology. The basic methods for \emph{Statistical Sensor Fusion} were outlined in \cite{gustafsson2010statistical}. The result of the application of the method is \emph{data fusion interface} (sec. \ref{s:dataFusionDefinition}) - interface to fuse sense data from various online, offline, cooperative, non-cooperative sources into single scalable {space and trajectory evaluation procedure}.
    
\paragraph{Related work:} \noindent UAS specific sensor fusion has been proposed by Ramsay in \cite{ramasamy2014avionics}. \emph{Next generation avoidance concept} \cite{ramasamy2014next} is introducing concept of higher-level sensor fusion called \emph{data fusion}. 

The uncertainty and properties in \emph{Remotely Piloted Systems} have been discussed in \cite{chynchenko2016remotely}. The work provided the concept of various performance ratings like visibility and obstacle rating; more details have been given in \cite{shmelova2016modeling}. These ratings were modeled only for operator decision making \cite{kharchenko2017modelling}, results are usable for automated decision making and space assessment. 




