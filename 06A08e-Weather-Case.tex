\section{Weather Case Implementation}\label{sec:weatherCase}
\paragraph{Motivation:}  The weather, as defined in (eq. \ref{eq:weatherProjection}), impacts flight and system dynamics; therefore it impacts the \emph{reach set} is impacted. The \emph{weather impact} can be solved by policy application:

\begin{enumerate}
    \item \emph{Weather Acceptance} - for bigger \emph{UAS} the normal weather impact does not pose a significant risk.  The \emph{segmented movement automaton} (def. \ref{def:segmentMovementAutomaton}) with \emph{Weather situation} as the discrete state is used.
    
    \item \emph{Weather Avoidance} - all \emph{weather} impact zones are considered as hard constraints with protective \emph{soft-constraint} around.
    
    \item \emph{Combined approach} - depending on the type of impact and declared UAS impact resistance the zones are divined into \emph{soft} and \emph{hard} constraints.
\end{enumerate}

\begin{note}
    This work handles small \emph{UAS} avoidance; these are very sensitive to any weather impact; therefore \emph{Weather impacted areas} will be considered as \emph{hard constraints with soft constraint protection zone}. 
    
    The original \emph{weather impact zone} is considered as obstacle body and enforces the body margin.
    
    The surroundings of \emph{weather impact zone} up to \emph{safety margin} distance are considered as \emph{soft constraint zone} (implemented as a bloated polygon).
\end{note}

\paragraph{Purpose:} The \emph{weather case} (tab. \ref{tab:weatherConstraint}) is broadcasted by \emph{Airspace Authority} to \emph{impacted area}, each \emph{UAS} then change their mission according to \emph{their maneuvering capabilities}.  Each trajectory must lead away from the \emph{constrained area}. The algorithm used for intersection selected based on \citep{bentley1979algorithms} the selected algorithm  \emph{Shamos-Hoey} \cite{shamos1976geometric}.

\paragraph{Constrained Area:} Constrained area can be defined as \emph{static} (sec. \ref{s:virtualConstraints}) or dynamic constraint (def. \ref{def:movingConstraint}).  The \emph{constraint center} is defined on horizontal plane like follow:

\begin{equation}
    Constraint Center = center \in \left [latitude, longitude\right]
\end{equation}

\noindent The \emph{Convex Polygon} boundary is defined on horizontal plane, contains at least 3 vertexes:

\begin{equation}
    Convex Polygon = \left\{point_i:point_i\in \left [latitude, longitude\right], i \ge 3\right\}
\end{equation}

\noindent The \emph{Vertical constraint} is defined as \emph{range of barometric altitude} (Above Mean Sea Level):

\begin{equation}
    Vertical Constraint = \left [ start Altitude, end Altitude \right ]
\end{equation}

\begin{tabularx}{\textwidth}{S{0.25}|X}
     \multicolumn{2}{c}{\textbf{Constrained area}}\\\hline
     center position & is given as a geometrical \emph{center point of the boundary}.  \\
     boundary & is represented as a \emph{convex polygon} on the latitude-longitude plane.\\
     start altitude & is lover boundary barometric altitude given at above mean sea level, where given weather factor has a significant impact.\\
     end altitude & is upper boundary barometric altitude given at above mean sea level, where given weather factor has a significant impact.\\
	
     \multicolumn{2}{c}{\textbf{Additional parameters}}\\\hline
     type(s) & lists weather events occurring in the \emph{constrained area}.\\
     severity list & is recorded for each plane \emph{category}\\
     start & indicates when weather constraint was established. \\
     expected end & of weather constraint.\\
     velocity & indicates if weather phenomenon is moving.\\
     \multicolumn{2}{c}{\textbf{Miscellaneous}}\\\hline
     previous & reference to \emph{weather constraint} decision time-frame data.\\
     impacted & list of possibly impacted attendees (planes which obtained divergence order or warning from UTM).\\
    \caption{Static/Dynamic weather constraint for given decision time-frame.}
    \label{tab:weatherConstraint} 
\end{tabularx}

\paragraph{Additional parameters}: Following additional parameters with additional purpose can be attached to \emph{Weather Constraint}.
\begin{enumerate}
    \item \emph{Type} - defines required resistance - moisture, temperature, wind.
    
    \item \emph{Severity} - defines the impact for each \emph{aircraft category}, this is used in soft/hard type assessment. 
    
    \item \emph{Duration} - start and end of \emph{constraint} validity, if not defined valid for all \emph{UAS mission time}.
    
    \item \emph{Velocity} - velocity and last position assessment time.
\end{enumerate}

\begin{note}
    Our implementation does not consider the \emph{type} or \emph{severity}. All \emph{weather impact} is considered as a \emph{hard constraint}. The velocity differentiates \emph{static} ($=0$)/\emph{moving} ($>0$) \emph{constraints}. 
\end{note}

\paragraph{Avoidance System:} Resolve similar to \emph{Converging/Overtake Maneuver} depending on the \emph{angle of approach}. The \emph{virtual roundabout} is utilized for \emph{static constraints}; the \emph{intruder model} is utilized for \emph{dynamic constraints}.
