\cleardoublepage
\chapter{\secState{R}Collision Avoidance}\label{ch:CollisionAvoidance}
\noindent The context of Collision Avoidance is introduced in (tab. \ref{tab:CASContext}), the structure was taken from Gardi \cite{gardi2015automated}. The \emph{state of art} changes was incorporated into the table.

\begin{tabularx}{\textwidth}{S{0.20}||S{0.55}} 
    \centering \emph{Function} &  \emph{Equipment/Task}\\ \hline\hline
    \centering Communication & Telecommunication datalinks,\newline Controller Pilot Data Link-Control (CPDLC), \newline Voice Communication\\\hline
    \centering Navigation & Navigation sensors including GNSS, INS, etc. providing 3D/4D navigation capabilities.\\\hline
    \centering Surveillance & Cooperative Systems (TCAS, ACAS, etc.)\newline Non-cooperative Sensors (LiDAR,Cameras, etc.)\\\hline
    \centering Situation\newline Awareness& Early Warning Systems, \newline CDTI Display\\\hline
    \centering{Autonomous\newline{Decision}\newline{Making}}& Strategic, Tactical, Emergency Flight Planning,\newline Intelligent Collision Detection,\newline Conflict Resolution and Prevention,\newline Weather/Terrain/Constraints Avoidance\\
    \caption{Collision avoidance systems context overview \cite{gardi2015automated}.}
    \label{tab:CASContext}
\end{tabularx}

\section{\secState{R}Overview}\label{s:collisionAvoidanceOverview}
\noindent The \emph{Detect and Avoid}, as a part of \emph{Collision Avoidance}, impacts all collision avoidance aspects (tab. \ref{tab:CASContext}). This work focuses on the \emph{Reach Sets} which gives us the following focus area:

\begin{enumerate}
    \item \emph{Communication} - it is assumed the command \& control communication link is stable. This aspect is not affected by reach sets.
    
    \item \emph{Navigation} - minimal navigation framework needs to be implemented for full experimentation with the navigation capabilities of the \emph{Reach Set} based trajectory generation.
    
    \item \emph{Surveillance} - the surveillance will be covered with necessary low-cost technologies, the simulated sensor inputs for following surveillance equipment is considered:
    \begin{enumerate}[a.]
        \item \emph{Non-cooperative} - LiDAR Sensor.
        \item \emph{Cooperative} - ADS-B In/Out.
    \end{enumerate}
    
    \item \emph{Situation  awareness} - the situation awareness focuses on \emph{space segmentation} and \emph{safety evaluation} to support proper safe trajectory selection from \emph{reach set}.
    
    \item \emph{Autonomous decision making} - the \emph{reach set} covers all possible avoidance strategies, to know how to select proper strategy is key in successfully avoidance maneuvering.
\end{enumerate}

\paragraph{Communication:} An overview elaboration on capability, reliability, security, architecture have been summarized  by Johansen et al. in \cite{johansenetal2018surveyCommunicaiton}.

The current state of art \emph{communication lines} and relay approaches are sufficient to provide necessary utilities. The use of an existing 4G/3G mobile network is the most probable candidate for low altitude UAS operations. The necessity to build a back-up network for communication is still an open topic.

\paragraph{Navigation:} An overview is given by Nex \cite{nex2014uav} \emph{Waypoint planning in a 3D environment} is elaborated in \cite{bodin2007navigating}. \emph{Waypoint Tracking and Test Environments} are thoughtfully discussed in \cite{how2008real,girard2004border,andrade2017autonomous,klausen2017nonlinear}. 

All navigation methods are fairly similar. Consisting of the following steps in the loop:
\begin{enumerate}
    \item Select goal waypoint.
    \item Evaluate feasible navigation strategies (cost function).
    
    \item Select navigation strategy and generate reference trajectory.
    
    \item Follow the reference trajectory with UAS system.
\end{enumerate}
The \emph{evaluation process} and selection criteria need to be designed in the context of \emph{reach sets}.

\paragraph{Surveillance:} TCAS and ACAS systems cover the cooperative surveillance, an interesting aspect of these systems are \emph{Resolution Advisories} \cite{kennedy1995resolution} for TCAS \cite{marston2015acas}, for ACAS.  These advisories are giving the suggestions for the pilot to avoid an occurring collision. The responsibility for following advisories and avoiding collision is on the pilot.  

This mechanism needs to be changed to increase the determinism of UAS behavior. The voluntary approach of advisories needs to be replaced with a mandatory approach (directives).

\paragraph{Situation awareness:} The aspect of the situation awareness of surroundings has been introduced in \cite{blaskovich2007declutter}. \emph{LiDAR}-based \emph{SAA} system has been introduced by Sabatini \cite{sabatini2014lidar} further enhanced by Ramasay \cite{ramasamy2016lidar}. Other \emph{Non-Cooperative} sensors and their feasibility have been outlined in Ramasay work \cite{ramasamy2014avionics}. 

The common ground of these works is an operational space discretization into various forms of finite discrete sets to enable deterministic decision making. The key issue is to find a good rate between space democratization and solution precision. The large cells in the grid usually hide many escape routes. The small cells in grid usually increase the computation complexity and diminish computation time optimal solution.

Examples of \emph{situation awareness:} implementation can be found mainly in \emph{human-centered} systems, \emph{Early Warning System} has been proposed by Lee \cite{lee2002collision} and an adaptive version by Miller \cite{miller2002adaptive}. Effects of \emph{CDTI Display} visualization and human decision impact have been examined by Thomas \cite{thomas2005effects}. \emph{Self Separation} aspect has been examined by Williams \cite{williams1983self}.

\noindent
The important concept for \emph{UAS} is internal data representation and autonomous situation resolution. The autonomous situation resolution (decision-making process) can be extracted from human pilot operation procedures.  

\paragraph{Manned Aviation Concepts:} The introduction of necessary concepts from manned aviation is organic in UAS concept understanding. 
Many of the concepts are taken directly from manned aviation. The main contribution is to change the \emph{human decisions} into \emph{autonomous system decisions}.

\paragraph{Airspace Classification:} For integration of the UAS systems into non -segregated airspace it is necessary to know the classification of the \emph{operational space}. Who is the authority, in which space, and when the authority is enforced. The general overview of airspace classes and concepts accepted by ICAO/FAA/EASA are outlined in (sec. \ref{sec:AirspaceClassification}). The common viewpoint is emphasized. 
    
\paragraph{Aircraft Operational Rules:} It is necessary to know the basic rules in controlled/uncontrolled airspace. What is expected to be done by the aircraft in various flight modes. What is minimal equipment's, what is airworthiness and so on. The basic regulations are outlined in (sec. \ref{sec:AircraftOperationRules}). Visual Flight Rules (VFR) interesting parts can be found in (sec. \ref{sec:VisualFlightRules}). Instrumental Flight Rules interesting parts can be found in (sec. \ref{sec:InstrumentalFlightRules}).
        
\paragraph{Active/Passive Separation and Self-Separation:} The \emph{safe navigation} in \emph{airspace} have multiple levels, going from least strict to very strict and keeping aircraft or UAS \emph{well clear} of all threats. There is first protective barrel known as \emph{well clear}; then there is a smaller protective barrel representing \emph{near miss}, then the smallest protective barrel representing \emph{crash zone}. The \emph{Well clear} state of aircraft/UAS in airspace  important parts are mentioned in (sec.\ref{sec:WellClear}). 

The important role of \emph{Air Traffic Control} for manned aircraft is introduced in  (sec. \ref{sec:AirTrafficControl}). The general aviation \emph{routing} principles can be used on the various scale for \emph{UAS routing}. The form of \emph{ATC} commands and directives must persist in future UAS traffic management, for compatibility reasons.

The current Collision Avoidance Systems systems TCAS (\ref{sec:TCAS}) and ACAS-X (\ref{sec:ACASX}) which can be used as unmanned approach base are introduced.

\paragraph{UAS Traffic Management:} The traffic management functionality is analyzed in (sec. \ref{sec:UTM}), two major movements EU USPACE (\ref{sec:USpace}) and US NASA UTM (\ref{sec:NASAUtm}) exists. The most notable information from operation specification is extracted there.

\emph{Event-Based Avoidance} (sec. \ref{sec:EventBasedAvoidance}) defines basic event-based control invoked by \emph{UTM}; two major categories are analyzed in \emph{Mid-Air Collision Prevention} (sec. \ref{sec:MidairCollisionPrevention}) and \emph{Weather Impact} (sec. \ref{sec:WeatherImpact}).