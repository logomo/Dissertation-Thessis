\subsection{Combined Reach Set Approximation - Tree Merge}\label{s:combinedReachSet}
\paragraph{Motivation:} Turn-Minimizing Reach Set Approximation (sec. \ref{s:harmonicReachSet}) is \emph{efficient} for \emph{Navigation in \emph{Controlled Airspace}}. Coverage-Maximizing Reach Set Approximation (sec. \ref{s:chaoticReachSet}) is good for \emph{Emergency avoidance}. The need for the differentiation between \emph{Navigation} and \emph{Emergency Avoidance} mode is necessary for \emph{Controlled Airspace}, but not for \emph{Non-controlled Airspace}. The combination of \emph{Turning-Minimizing} and \emph{Coverage-Maximizing} reach set approximations is an obvious solution. 

\emph{Automatic mode switch} can be provided by a combination of \emph{Navigation Reach Set} and \emph{Avoidance Reach Set} with an elevated cost function. Overall having a method to merge multiple trees would be beneficial.

\paragraph{Background:} If two \emph{Reach Set Approximation} were calculated for the same \emph{Avoidance Grid} and \emph{Initial State}, using same \emph{Movement Automaton} and \emph{UAS model} are possible to merge. 

The \emph{Reach Set Approximation} is a \emph{tree} with \emph{Root Node} in \emph{initial state} with movement buffer = $\varnothing$. The \emph{movement buffer} in each node can be used as \emph{route trace} during the merging procedure. The example two reach set merge can be given as follow, where only the \emph{latest} applied movement is taken into account.

\begin{equation}\label{eq:mergeTreeFunctionExample}
    \left[
    \begin{aligned}
    \texttt{Fi}&\texttt{rst Reach Set}\\
        &\varnothing \to 
            \left<
                \begin{aligned}
                    &left \to \left<
                                \begin{aligned}
                                    &left\\
                                    &right
                                \end{aligned}
                              \right.\\
                    &\varnothing\\
                \end{aligned}
             \right.\\ 
    \texttt{Se}&\texttt{cond Reach Set}\\
        &\varnothing \to 
            \left<
                \begin{aligned}
                    & \varnothing \\
                    & right \to \left< 
                                    \begin{aligned}
                                        &left\\
                                        &right
                                    \end{aligned}
                                \right.\\
                \end{aligned}
            \right.\\
    \end{aligned}
    \right]
    \to
    \left[
    \begin{aligned}
    \texttt{Co}&\texttt{mbined Reach Set}\\
    &\varnothing\to
        \left< 
            \begin{aligned}
                &left &\to   \left< 
                                \begin{aligned}
                                    &left\\
                                    &right
                                \end{aligned}    
                            \right.\\
                &right &\to   \left< 
                                \begin{aligned}
                                    &left\\
                                    &right
                                \end{aligned}    
                            \right.\\
            \end{aligned}    
        \right.
    \end{aligned}
    \right]
\end{equation}

\noindent\emph{First Reach Set} contains two trajectories given by buffers \emph{\{left,left\}} and \emph{\{left,right\}}. \emph{Second Reach Set} contains two trajectories given by buffers \emph{\{right,left\}} and \emph{\{right,right\}}. The \emph{Combined Reach Set} contains all four trajectories.

\begin{note}
    The combined tree \cite{o1996log} does not need to a have combined amount of original \emph{Reach Sets} trajectories. There can be \emph{Duplicity} which means that any bounded property like \emph{Cost} must be \emph{calculated} again.
\end{note}

\paragraph{Combined Reach Set Calculation Function} (alg. \ref{alg:ReachSetMerge}) is implemented as function $Node combinedReachSet(\dots)$ which takes root Node with \emph{initial State}, \emph{Avoidance Grid} and respective parameters for each calculation method. \emph{turn-minimizing spread} for \emph{Turn-Minimizing Reach set calculation} and \emph{coverage spread}, \emph{Footprint Length} for \emph{Coverage-Maximizing Reach Set Approximation}.

\emph{Separate Reach Sets} are calculated using \emph{Wave-front propagation} (alg. \ref{alg:Wavefront Propagation}) using respective \emph{Constrained Expansion} functions for \emph{Turn-Minimizing} (alg. \ref{alg:ExpansionConstraintFunctionForHarmonicReachSet}) and \emph{Coverage-Maximizing} (alg. \ref{alg:ExpansionConstraintFunctionForChaoticReachSet}) reach sets.

\emph{Combined Reach Set} is created using \emph{Node mergeTree($\dots$)} function, because different cost function or \emph{Bounded Parameters Calculation} may be applied on \emph{Original Reach Sets}.

\emph{Cost} for \emph{each node} needs to be recalculated due to original reach sets disparity. Function \emph{combined.applyCostFunction()} will recalculate the new cost for each node. 

The Goal is to have a penalization for \emph{non-turn-minimizing behavior}, implementation of \emph{Automatic Mode Switch} can be done like follows:
\begin{enumerate}
    \item \emph{Calculate Normal Cost} for Node $Cost(Node)$ for the associated trajectory:\\ $Cost(Node.Trajectory)$.
    
    \item \emph{Calculate Penalization for \emph{additional maneuvering}}, calculate \emph{Smoothness Rating for Trajectory} (def. \ref{def:SmoothnessRatingForTrajectory}) in the interval $[0,1]$, introduce penalization with base $100 \%$.
\end{enumerate}

\noindent The final $Cost(Node)$ function is applied to each \emph{Combined Reach Set Node} and look like follows:

\begin{multline}
    Cost(Node) = Cost(Node.Trajectory) \times\dots\\\dots\times \left(1+ \left(1-SmoothnessRate(N ode.T rajectory)\right)\right)
\end{multline}

\newpage

\paragraph{Tree Merge Function} $mergeTree(\dots)$ implements \emph{Outer Join} operation on two trees. Example was given in (eq. \ref{eq:mergeTreeFunctionExample}).
Function is applied on \emph{root Node} iterating over \emph{Movements in Movement Set}, because \emph{Movement is pivot}.

\begin{algorithm}[H]
    \SetKwProg{Fn}{}{}{}\SetKwFunction{FRecurs}{Node mergeTree}
    \SetKwProg{Fn}{}{}{}\SetKwFunction{FMain}{Node combinedReachSet}
    
    \BlankLine
    \# Tree merge function\;
    \Fn(){\FRecurs{Node firstNode, Node secondNode}}{
        \BlankLine
        \# Try to copy reference node or return null\;
        Node referenceNode = (firstNode?:(secondNode?: return null))\;
        Node merged =  new Node(referenceNode)\;
        merged.leafs= []\;
        \BlankLine
        \# Try to fetch movement nodes if exist in any sub tree\;
        \For{movement $\in$ Movements}{
            firstLeaf = firstNode.getLeafFor(movement)\;
            secondLeaf = secondNode.getLeafFor(movement)\;
            newLeaf = mergeTree(firstLeaf,secondLeaf)\;
            \If{newLeaf $\sim = $ null}{
                merged.leafs.append(newLeaf)\;
            }
        }
        \Return{merged}
    }{}
    
    \BlankLine
    \# Combined Reach Set calculation function\;
    \Fn(){\FMain{Node root, AvoidanceGrid grid,int$^+$ coverageSpread, int$^+$ turnSpread, int$^+$ footprintLength}}{
        Node cmrsa = chaoticReachSet(root,grid, footprintLength,coverageSpread)\;
        Node tmrsa = harmonicReachSet(root,grid, turnSpread)\;
        Node combined = mergeTree(cmrsa,tmrsa)\;
        combined.applyCostFunction()\;
        \Return{combined}
    }

    
    \caption{Reach Set Merge Function and Combined Reach Set calculation}
    \label{alg:ReachSetMerge}
\end{algorithm}

\paragraph{Example:} for \emph{Avoidance Grid} with \emph{Distance 10 m}, \emph{Layer count 10}, \emph{Horizontal range $[-45^\circ,+45^\circ]$}, \emph{Horizontal Cell Count 7}, \emph{Vertical range $[-30^\circ,+30^\circ]$}, and \emph{Vertical Cell Count 5}. Is given in (fig. \ref{fig:combinedReachSetApproximation}). The UAS is at \emph{Back-side} of \emph{Figure} (initial state is at all \emph{Trajectory Origins}). The \emph{black dashed line} marks \emph{Avoidance Grid} space boundary. Each trajectory has its own color and ends at \emph{Front-side} of \emph{Avoidance Grid Boundary}. The \emph{Coverage-Maximizing Spread} was set to 8, \emph{Footprint Length} to 3 and \emph{Turn-Minimizing Spread} to 1.

\begin{note}
    Notice there are typical trajectories from both \emph{Turn-Minimizing} (fig. \ref{fig:harmonicReachSetApproximation}) and \emph{Coverage-Maximizing} (fig. \ref{fig:chaoticReachSetApproximation}) \emph{Reach Set Approximations}.
\end{note}

\begin{figure}[H]
    \centering
    \includegraphics[width=0.7\linewidth]{\FIGDIR/RS004CombinedReachSetEstimationMethod} 
    \caption{\emph{Combined \emph{reach set} approximation}.}
    \label{fig:combinedReachSetApproximation}
\end{figure}


\paragraph{Pros and Cons:} It can be seen from example (fig. \ref{fig:combinedReachSetApproximation}) that \emph{Combined Reach Set Approximation} (alg. \ref{alg:ReachSetMerge}) contains both types of maneuvers. \emph{Cheaper turn-minimizing} for navigation and \emph{More Expensive Coverage-Maximizing} for \emph{Emergency Avoidance}. The upper limit of trajectories is given as follow:

\begin{multline}
    countTrajectories(ReachSet) \le countTrajectories(CM-RSA) \\+ countTrajectories(TM-RSA)
\end{multline}

\noindent The \emph{upper limit of nodes} is given as follow:

\begin{equation}
    countNodes(ReachSet) \le countNodes(CM-RSA)+ countNodes(TM-RSA)
\end{equation}

\noindent \emph{Turn-Minimizing Reach Set} is ideal for \emph{Non-controlled Airspace} missions because it contains \emph{Automatic Mode Switch} between \emph{Navigation} and \emph{Emergency Avoidance}.


