\section{\secState{R}Future Work}\label{s:futureWork}

\noindent The future work needs to address issues of \emph{adversarial avoidance} (sec. \ref{s:adversadialBehaviourImpact}) and \emph{practical implementation} (sec. \ref{s:TestingFrameworkTheory}).

\paragraph{Adversarial Avoidance:} The counter example (sec. \ref{fig:adversarialAttackNotableMoments}) showed that the approach is vulnerable to \emph{adversarial behaviour}. The adversarial UAS just simply avoided our field of the vision to proceed with side-hit.

This opens the possibility to solve the problem as \emph{differential game} \cite{game1987,game1988}. Our UAS should pose as the \emph{defender}, adversary should pose as \emph{attacker}. The \emph{reach set} of the system given as:

\begin{equation*}
    differential Game = model(positions, defender, attacker)
\end{equation*}

\noindent This reach set  will give us the options in our decision making process to avoid the pursuer. This will bring additional scientific challenges which can yield interesting results.

\paragraph{Real System Implementation:} The testing proved \emph{capability of approach} for wide range of applications (tab. \ref{tab:testCasesPerformacneEvaluation}). To proceed further with comparative testing, beyond theoretical implementations (sec. \ref{s:conservativeComparison}), it is necessary to deploy it in real environment.

The \emph{ideal candidate} is \emph{LSTS-tooolchain} \cite{pinto2013lsts}. The LSTS offers all necessary base for \emph{approach software architecture} (fig. \ref{fig:AvoidanceFrameworkConceptNew}). The parts of our approach can be distributed over LSTS-toolchain in following manner:

\begin{enumerate}
    \item \emph{LSTS Dune} (UAS on-board control) - the implementation of main \emph{navigation loop} (sec. \ref{s:missionControlRun}), including sensor integration with data fusion (sec. \ref{s:sensorFusion}). The \emph{rule engine} (sec. \ref{s:RuleEngineArchitecture}) can be deployed after UTM services implementation, to support \emph{cooperative maneuvers}.
    
    \item \emph{LSTS Neptus} (UTM equivalent) - the implementation of \emph{UTM} services and calculations (sec. \ref{sec:utmArchitecture}).
    
    \item \emph{LSTS IMC} (Messaging implementation) -  the messaging support for cooperative (fig. \ref{fig:CooperativeConflictResolutionUTM}) and non-cooperative (fig. \ref{fig:NonCooperativeConflictResolutionUTM}) communication schemes.
    
    \item \emph{LSTS Ripples} (Long term data storage) -  the flight-log storage.
    
\end{enumerate}

\noindent The real system implementation will enable to:
\begin{enumerate}
    \item \emph{Compare approach performance} - the theoretical performance of approach is good, only the real challenges can show the flaws and aves of approach. 
    
    \item \emph{Develop fault-tolerant and recovery procedures} - the process of \emph{event-based} and \emph{reactive} avoidance have been developed and tested in theoretical environment. The real implementation can improve the process weak points by \emph{"learning by doing"} method
    
    \item \emph{Advertise the approach benefits} - the successful implementation on real system will increase the outreach and visibility of the approach significantly.
\end{enumerate}