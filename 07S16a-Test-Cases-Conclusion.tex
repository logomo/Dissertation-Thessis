\cleardoublepage
\section{\secState{R}Test Cases Conclusion}\label{s:testCasesConclusion}
\noindent This section contains summary of performance evaluation (sec. \ref{s:performanceEvaluation}), adversary behavior impact on our approach (sec. \ref{s:adversadialBehaviourImpact}), \emph{calculation load} in (sec. \ref{s:ComputaitonFootprint}).

\subsection{\secState{R}Performance Evaluation}\label{s:performanceEvaluationTable}
\noindent Performance of test cases was evaluated according to criteria given by (sec. \ref{s:performanceEvaluation}). The performance for \emph{test cases} from test plan (tab. \ref{tab:testCasesSummary}) has been summarized in (tab. \ref{tab:testCasesPerformacneEvaluation}).

\begin{table}[H]
    \scriptsize
    \centering
    \begin{tabular}{c|c|c|c|c|c|c|c}
    
    %Header - do not touch the magic
    \multirow{3}{*}{\begin{tabular}[c]{@{}c@{}}Scenario\\name\end{tabular}}& \multicolumn{3}{c|}{Safety Margin} & \multicolumn{3}{c|}{Trajectory tracking} & \multirow{3}{*}{\begin{tabular}[c]{@{}c@{}}Pass\\ /\\ Fail\end{tabular}} \\ \cline{2-7}
    &\multicolumn{2}{c|}{Distance} & \multirow{2}{*}{Breach} & \multirow{2}{*}{\begin{tabular}[c]{@{}c@{}}Waypoint \\ Reach\end{tabular}} & \multirow{2}{*}{\begin{tabular}[c]{@{}c@{}}Reference\\ Deviation\end{tabular}} & \multirow{2}{*}{\begin{tabular}[c]{@{}c@{}}Acceptable\\ Deviation\end{tabular}} &  \\ \cline{2-3}
    &min & max &  &  &  &  &  \\ \hline\hline
    
    % Building avoidance
    \begin{tabular}[c]{@{}c@{}}Building\\avoidance\\(sim. \ref{fig:testCaseBuildingAvoidanceSituation})\end{tabular} & 
    \begin{tabular}[c]{@{}c@{}}0.69 m\\ UAS 1\end{tabular} & 
    \begin{tabular}[c]{@{}c@{}}24.98 m\\ UAS 1\end{tabular} & 
    \begin{tabular}[c]{@{}c@{}}No\\(\ref{fig:testCaseBuildingAvoidancePerformance})\end{tabular} & 
    \begin{tabular}[c]{@{}c@{}}Yes/UAS 1/(\ref{fig:testCaseBuildingAvoidancePathTracking})\end{tabular} & 
    \begin{tabular}[c]{@{}l@{}r@{}}
        $\mathscr{WP}_1:$& $107.05m$\\
        $\mathscr{WP}_2:$&$\quad 86.20m$\\
        $\mathscr{WP}_3:$&$28.70m$\\ 
        $\mathscr{WP}_4:$&$32.84m$
    \end{tabular} & 
    \begin{tabular}[c]{@{}c@{}}Yes\\(\ref{tab:pathTrackingParametersForBuildingAvoidance})\end{tabular} & 
    Pass \\ \hline
    
    % Slalom
    \begin{tabular}[c]{@{}c@{}}Slalom\\(sim. \ref{fig:testCaseSlalomwithHiddenWaypoint})\end{tabular} & 
    \begin{tabular}[c]{@{}c@{}}0.09 m\\ UAS 1\end{tabular} & 
    \begin{tabular}[c]{@{}c@{}}3.74 m\\ UAS 1\end{tabular} & 
    \begin{tabular}[c]{@{}c@{}}No\\(\ref{fig:testCaseSlalomAvoidancePerformance})\end{tabular} & 
    \begin{tabular}[c]{@{}c@{}}Yes/UAS 1/(\ref{fig:testCaseSlalomPathTracking})\end{tabular} & 
    \begin{tabular}[c]{@{}c@{}r@{}}$\mathscr{WP}_1$:&$\quad 20.06m$\end{tabular} & 
    \begin{tabular}[c]{@{}c@{}}Yes\\ (\ref{tab:pathTrackingParametersForSlalomAvoidance})\end{tabular} & 
    Pass \\ \hline
    
    % Maze
    \begin{tabular}[c]{@{}c@{}}Maze\\(sim \ref{fig:testCaseMazeSolver})\end{tabular} & 
    \begin{tabular}[c]{@{}c@{}}0.01 m\\ UAS 1\end{tabular} & 
    \begin{tabular}[c]{@{}c@{}}2.95 m\\ UAS 1\end{tabular} &
    \begin{tabular}[c]{@{}c@{}}No\\(\ref{fig:testCaseMazeAvoidancePerformance})\end{tabular} & 
    \begin{tabular}[c]{@{}c@{}}Yes/UAS 1/(\ref{fig:testCaseMazePathTracking})\end{tabular} & 
    \begin{tabular}[c]{@{}c@{}r@{}}$\mathscr{WP}_1:$&$\quad 28.06m$\end{tabular} & 
    \begin{tabular}[c]{@{}c@{}}Yes\\(\ref{tab:pathTrackingParametersForMazeAvoidance})\end{tabular} & 
    Pass \\ \hline
    
    % Storm
    \begin{tabular}[c]{@{}c@{}}Storm\\(sim. \ref{fig:testCaseStormAvoidance})\end{tabular} & 
    \begin{tabular}[c]{@{}c@{}}0.04 m\\ UAS 1\end{tabular} & 
    \begin{tabular}[c]{@{}c@{}}34.99 m\\ UAS 1\end{tabular} &
    \begin{tabular}[c]{@{}c@{}}No\\ (\ref{fig:testCaseStormAvoidancePerformance})\end{tabular}& 
    \begin{tabular}[c]{@{}c@{}}Yes/UAS 1/(\ref{fig:testCaseStormPathTracking})\end{tabular} & 
    \begin{tabular}[c]{@{}c@{}r@{}}$\mathscr{WP}_1$:&$\quad15.76m$\end{tabular} & 
    \begin{tabular}[c]{@{}c@{}}Yes\\(\ref{tab:pathTrackingParametersForStormAvoidance})\end{tabular} & 
    Pass \\ \hline\hline
    
    % Emergency Converging
    \begin{tabular}[c]{@{}c@{}}Emergency\\Converging\\(sim. \ref{fig:testCaseEmergencyConverging})\end{tabular} & 
    \begin{tabular}[c]{@{}c@{}}1.67 m\\ UAS 1-2\end{tabular} & 
    \begin{tabular}[c]{@{}c@{}}27.08 m\\ UAS 1-1\end{tabular} &
    \begin{tabular}[c]{@{}c@{}}No\\ (\ref{fig:testCaseEmergencyConvergingAvoidancePerformance})\end{tabular} & 
    \begin{tabular}[c]{@{}c@{}}Yes/UAS 1/(\ref{fig:emergencyConvergingUAS1PathTracking})\\\hline Yes/UAS 2/(\ref{fig:emergencyCovnergingUAS2PathTracking})\end{tabular} & 
    \begin{tabular}[c]{@{}c@{}r@{}}$\mathscr{WP}_1:$&$\quad 3.25m$\\\hline $\mathscr{WP}_1:$&$0.00m$\end{tabular} & 
    \begin{tabular}[c]{@{}c@{}}Yes\\(\ref{tab:pathTrackingParametersForEmergencyConverging})\end{tabular} & 
    Pass \\ \hline
    
    % Emergency Head On
    \begin{tabular}[c]{@{}c@{}}Emergency\\Head On\\(sim. \ref{fig:testCaseEmergencyHeadOnApproach})\end{tabular} & 
    \begin{tabular}[c]{@{}c@{}}0.38 m\\ UAS 1-2\end{tabular} & 
    \begin{tabular}[c]{@{}c@{}}38.00 m\\ UAS 1-2\end{tabular} & \begin{tabular}[c]{@{}c@{}}No\\(\ref{fig:testCaseHeadOnAvoidancePerformance})\end{tabular} &
    \begin{tabular}[c]{@{}c@{}}Yes/UAS 1/(\ref{fig:emergencyHeadOnUAS1PathTracking})\\\hline Yes/UAS 2/(\ref{fig:emergencyHeadOnUAS2PathTracking})\end{tabular} & 
    \begin{tabular}[c]{@{}c@{}r@{}}$\mathscr{WP}_1:$&$\quad 3.25m$\\\hline $\mathscr{WP}_1:$&$0.00m$\end{tabular} & 
    \begin{tabular}[c]{@{}c@{}}Yes\\(\ref{tab:pathTrackingParametersForEmergencyHeadOn})\end{tabular} & 
    Pass \\ \hline
    
    % Emergency Multiple
    \begin{tabular}[c]{@{}c@{}}Emergency\\Multiple\\(sim. \ref{fig:testCaseEmergencyMixed})\end{tabular} & 
    \begin{tabular}[c]{@{}c@{}}0.20 m\\ UAS 2-4\end{tabular} & 
    \begin{tabular}[c]{@{}c@{}}45.46 m\\ UAS 3-4\end{tabular} & 
    \begin{tabular}[c]{@{}c@{}}No\\(\ref{fig:testCaseMultipleAvoidancePerformance})\end{tabular} &
    \begin{tabular}[c]{@{}c@{}}Yes/UAS 1/(\ref{fig:emergencyMixedPathTrackingUAS1})\\\hline Yes/UAS 2/(\ref{fig:emergencyMixedPathTrackingUAS2})\\\hline Yes/UAS 3/(\ref{fig:emergencyMixedPathTrackingUAS4}) \\\hline Yes/UAS 4/(\ref{fig:emergencyMixedPathTrackingUAS3})\end{tabular} & 
    \begin{tabular}[c]{@{}c@{}r@{}}$\mathscr{WP}_1:$&$\quad 4.84m$\\\hline $\mathscr{WP}_1:$&$1.83m$\\\hline$\mathscr{WP}_1:$&$3.45m$\\\hline $\mathscr{WP}_1:$&$2.05m$\end{tabular} & 
    \begin{tabular}[c]{@{}c@{}}Yes\\(\ref{tab:pathTrackingParametersForEmergencyMixed})\end{tabular} & 
    Pass \\ \hline\hline
    
    % Rule based Converging
    \begin{tabular}[c]{@{}c@{}}Rule-based\\Converging\\(sim. \ref{fig:testCaseRuleBasedConverging})\end{tabular} & 
    \begin{tabular}[c]{@{}c@{}}1.22 m\\ UAS 1-2\end{tabular} & 
    \begin{tabular}[c]{@{}c@{}}20.28 m\\ UAS 1-2\end{tabular} & 
    \begin{tabular}[c]{@{}c@{}}No\\(\ref{tab:testCaseRuleBasedConvergingSafetyMarginDistances})\end{tabular} &
    \begin{tabular}[c]{@{}c@{}}Yes/UAS 1/(\ref{fig:ruleBasedConvergingUAS1PathTracking})\\\hline Yes/UAS 2/(\ref{fig:ruleBasedCovnergingUAS2PathTracking})\end{tabular} & 
    \begin{tabular}[c]{@{}c@{}r@{}}$\mathscr{WP}_1:$&$ 10.22m$\\\hline $\mathscr{WP}_1:$&$\quad0.00m$\end{tabular} & 
    \begin{tabular}[c]{@{}c@{}}Yes\\(\ref{tab:pathTrackingParametersForRuleBasedConverging})\end{tabular} & 
    Pass \\ \hline
    
    % Rule based Head On
    \begin{tabular}[c]{@{}c@{}}Rule-based\\Head On\\(sim. \ref{fig:testCaseRuleBasedHeadOnApproach})\end{tabular} & 
    \begin{tabular}[c]{@{}c@{}}0.21 m\\ UAS 1-2\end{tabular} & 
    \begin{tabular}[c]{@{}c@{}}36.33 m\\ UAS 1-2\end{tabular} &
    \begin{tabular}[c]{@{}c@{}}No\\(\ref{fig:testCaseRuleBasedHeadOnAvoidancePerformance})\end{tabular} &
    \begin{tabular}[c]{@{}c@{}}Yes/UAS 1/(\ref{fig:ruleBasedHeadOnUAS1PathTracking})\\\hline Yes/UAS 2/(\ref{fig:ruleBasedHeadOnUAS2PathTracking})\end{tabular} & 
    \begin{tabular}[c]{@{}c@{}r@{}}$\mathscr{WP}_1:$&$\quad 5.40m$\\\hline $\mathscr{WP}_1:$&$5.40m$\end{tabular} & 
    \begin{tabular}[c]{@{}c@{}}Yes\\(\ref{tab:pathTrackingParametersForRuleBasedHeadOn})\end{tabular} & 
    Pass \\ \hline
    
    % Rule based Multiple
    \begin{tabular}[c]{@{}c@{}}Rule-based \\Multiple\\ (sim. \ref{fig:testCaseRuleBasedMixed}) \end{tabular}& 
    \begin{tabular}[c]{@{}c@{}}0.54 m\\ UAS 2-3\end{tabular} & 
    \begin{tabular}[c]{@{}c@{}}32.24 m\\ UAS 1-2\end{tabular} &
    \begin{tabular}[c]{@{}c@{}}No\\(\ref{fig:testRuleBasedMultipleAvoidancePerformance})\end{tabular} &
    \begin{tabular}[c]{@{}c@{}}Yes/UAS 1/(\ref{fig:ruleBasedMixedPathTrackingUAS1})\\\hline Yes/UAS 2/(\ref{fig:ruleBasedMixedPathTrackingUAS2}) \\\hline Yes/UAS 3/(\ref{fig:ruleBasedMixedPathTrackingUAS3})\\\hline Yes/UAS 4/(\ref{fig:ruleBasedMixedPathTrackingUAS4})\end{tabular} & 
    \begin{tabular}[c]{@{}c@{}r@{}}$\mathscr{WP}_1:$&$\quad 11.40m$\\\hline $\mathscr{WP}_1:$&$11.40m$\\\hline$\mathscr{WP}_1:$&$11.40m$\\\hline $\mathscr{WP}_1:$&$11.40m$\end{tabular} & 
    \begin{tabular}[c]{@{}c@{}}Yes\\(\ref{tab:pathTrackingParametersForRuleBasedMixed})\end{tabular} & 
    Pass \\ \hline
    
    % Rule Based Overtake
    \multirow{2}{*}{\begin{tabular}[c]{@{}c@{}}Rule-based\\Overtake\\(sim. \ref{fig:testCaseRuleBasedOvertake2xSpeed})\end{tabular}} & 
    \multirow{2}{*}{\begin{tabular}[c]{@{}c@{}}0.80 m\\ UAS 1-2\end{tabular}} & 
    \multirow{2}{*}{\begin{tabular}[c]{@{}c@{}}48.85 m\\ UAS 1-2\end{tabular}} &
    \multirow{2}{*}{\begin{tabular}[c]{@{}c@{}}No\\(\ref{fig:testRuleBasedOvertakeSafetyMarginForDifferentSpeedPerformance})\end{tabular}} &
    \begin{tabular}[c]{@{}c@{}}Yes/UAS 1/(\ref{fig:ruleBasedOvertake2xdPathTrackingUAS1})\end{tabular} & 
    \begin{tabular}[c]{@{}c@{}r@{}}$\mathscr{WP}_1:$&$24.00m$\\$\mathscr{WP}_2:$&$0.00m$\\ $\mathscr{WP}_3:$&$4.00m$\\ $\mathscr{WP}_4:$&$\quad5.00m$\end{tabular} & 
    \multirow{2}{*}{\begin{tabular}[c]{@{}c@{}}Yes \\ (\ref{tab:pathTrackingParametersForRuleBasedOvertake2x})\end{tabular} }& 
    \multirow{2}{*}{Pass} \\ 
     & & & & \begin{tabular}[c]{@{}c@{}} \hline Yes/UAS 2/(\ref{fig:ruleBasedOvertake2xPathTrackingUAS2})\end{tabular}& \begin{tabular}[c]{@{}c@{}r@{}}\hline $\mathscr{WP}_1:$&$\quad 0.00m$\end{tabular} & \\
    \end{tabular}
    
    \caption{Test cases \emph{performance evaluation}.}
    \label{tab:testCasesPerformacneEvaluation}
\end{table}

\paragraph{Highlights:} Each \emph{scenario} contains the reference to notable simulation moments and results. The scenarios were grouped according to the \emph{Operational Space} category, and each category is separated by strike line.

\noindent\emph{Non-cooperative test cases for the Rural/Urban environment:}
\begin{enumerate}
    \item\emph{Static obstacle avoidance} (Building/Slalom/Maze) - the buildings were correctly avoided without security breach; navigation algorithm was sufficient for given scenarios and obstacle density.
    
    \item\emph{Weather avoidance} (Storm) - the moving \emph{storm} have been avoided in both \emph{soft constraint} and \emph{hard constraint} state. The assumption of \emph{early detection/notification} is key in successful weather avoidance.
\end{enumerate}

\noindent\emph{Non-cooperative test cases for Intruder Avoidance} - the key assumptions are early intruder detection in \emph{Avoidance Grid} and \emph{non-adversarial} behavior. Each UAS was running own instance of \emph{Navigation loop} (fig. \ref{fig:missionControlRunActivityDiagram}). The summary of test cases is going like follow:

\begin{enumerate}
    \item\emph{Emergency converging} - both UAS identified correct roles according to the rules of the air. The UAS 2 kept \emph{right of the way}.
    
    \item\emph{Emergency head on} - both UAS identified correct roles according rules of the air, both of them uses full separation with \emph{Combined Reach Set Approximation} (sec. \ref{s:combinedReachSet}).
    
    \item\emph{Emergency mixed} -  all four UAS enters into emergency avoidance mode intermediately after intruders detection. The \emph{non-cooperative} consensus of separation is reached (fig. \ref{fig:NonCooperativeConflictResolutionUTM})
\end{enumerate}

\noindent\emph{Cooperative test cases with UTM supervision} are working according to \emph{UTM architecture} (fig. \ref{fig:UTMArchitectureOverview}), where the \emph{UTM} is considered as main authority. The key assumptions are UTM Resolution fulfillment and \emph{non-adversary behavior}. Each UAS was running own instance of \emph{Navigation loop} (fig. \ref{fig:missionControlRunActivityDiagram}) with enabled \emph{Rule Engine} (sec. \ref{sec:ruleEngine}). The summary of test cases is going like follow:

\begin{enumerate}
    \item \emph{Rule-based converging} - correct handling of \emph{converging maneuver} (fig. \ref{fig:ConvergingManeuverTheoretical}), proper rule invocation (rule \ref{tab:ruleConvergingManuever}) on UAS side.
    
    \item \emph{Rule-based head on} - correct handling of \emph{head on maneuver} (fig. \ref{fig:HeadOnApproachTheoretical}), proper rule invocation (rule \ref{tab:ruleHeadonApproach}) on UAS side. 
    
    \item \emph{Rule-based multiple} - proper \emph{Collision case Merge} (tab. \ref{tab:collisionCasesRuleBasedMixed}) with new collision point (eq. \ref{eq:aggregatedCollisionCaseCenter}) and \emph{safety margin calculation} (eq. \ref{eq:naiveSafetyMarginAgregation}).
    
    \item \emph{Rule-based overtake} - correct handling of \emph{overtaking maneuver} (fig. \ref{fig:OvertakeManeuverTheoretical}), proper rule invocation (rule \ref{tab:ruleOvertakeDefinition}). Divergence/Convergence (eq. \ref{eq:overtakeDivergenceLocal},\ref{eq:overtakeConvergenceLocal}) for multiple waypoints calculation works for various speed difference (fig. \ref{fig:testRuleBasedOvertakeSafetyMarginForDifferentSpeedPerformance}).
\end{enumerate}

\subsection{\secState{R}Adversary Behaviour Impact}\label{s:adversadialBehaviourImpact}

\noindent The \emph{abuse} of UAS for \emph{ill intentions} realization is expected. The \emph{UAS} is cheap, disposable and does not have ethic boundaries. 

One of the \emph{assumptions} was that there are only intruders who do not actively look to harm our \emph{UAS}. Breaking this assumption can be lethal for our system and also for other systems. 

Let us take \emph{Rule-based Head on} test case (sec. \ref{s:testRuleHeadOn}), changing only following aspects:
\begin{enumerate}
	\item \emph{UAS 2 position spoofing} - the adversarial vehicle is \emph{faking its position} according to expected behavior.
	\item \emph{UAS 2 Navigation goal} - set as \emph{UAS 1 position} from intercepted \emph{position notifications} (tab. \ref{tab:positionNotification}).
\end{enumerate}


\paragraph{Simulation:} The \emph{simulation} (fig. \ref{fig:adversarialAttackNotableMoments}) have been run with defined condition. UAS 2 (magenta) has been chosen as the \emph{adversary}. UTM sees the expected trajectory of UAS 2 (grey plane/trajectory) based on spoofed \emph{position notifications}. The \emph{navigation/avoidance grid} range (black dashed line boundary) is shown. The notable moment of the simulation are: 

\begin{enumerate}
    \item \emph{Deviation detection ($UAS2\leftrightarrow UTM$)} (fig. \ref{fig:adversaryDetectedDeviation}) - the \emph{collision case} (tab. \ref{tab:collisionCasesRuleBasedHeadon}) is active and \emph{enforced} by UTM. The \emph{adversary} UAS 2 (magenta) starts deviating from expected trajectory (grey). UAS 1 (blue) does not register any foreign object in \emph{avoidance grid range} (black dashed line). 
    
    \item \emph{Adversary attacking ($UAS 2 \to UAS 1$)} (fig. \ref{fig:adversaryAttacking}) - the adversary UAS 2 (magenta) starts actively pursuing UAS 1 (blue) by changing the original heading. This can be considered as the beginning of \emph{active pursuit}. UAS 1 (blue) does not detect any foreign object in \emph{avoidance grid} (black dashed line boundary). \emph{UTM} is receiving expected UAS position (grey plane/line).
    
    \item \emph{Emergency avoidance} ($UAS 1 \to UAS 2$) (fig. \ref{fig:adversaryEmergencyAvoidance}) following happens:
    \begin{enumerate}[a.]
        \item \emph{Adversary UAS 2} (magenta) is spotted by \emph{UAS 1}(blue), it entered into UAS 1 avoidance grid (black dashed line boundary).
        
        \item \emph{UAS 1} (blue) enters into \emph{Emergency Avoidance Mode} because there is a \emph{foreign object} in \emph{avoidance grid}.
        
        \item \emph{UTM} notices a warning to \emph{UAS 1} (blue) because it entered into \emph{Emergency Avoidance Mode}. UTM is not aware of any breach, because of expected UAS 2 position (grey plane/line)
        
        \item \emph{Adversary UAS 2} (magenta) has UAS 1 (blue) locked in \emph{navigation grid} as thegoal (which guarantees optimal pursuit).
    \end{enumerate}
    
    \item \emph{Blind spot ($\circlearrowright UAS 1$)} (fig. \ref{fig:adversaryBlindSpot}) following happens:
    \begin{enumerate}[a.]
        \item \emph{UAS 1} (blue) returns to \emph{Navigation Mode} because there is no \emph{foreign object} in \emph{avoidance grid} (black dashed line boundary).
        
        \item \emph{UTM} receives the mode change, and it starts enforcing \emph{resolutions} for \emph{collision case}, Adversary \emph{UAS 2} is considered clear due to \emph{expected position} (grey plane/line) compliance with the resolution.
        
        \item \emph{Adversary UAS 2} (magenta) is on UAS 1 blind spot. The target UAS 1 (blue) is locked in the UAS 2 navigation grid (black dashed line boundary).
    \end{enumerate}
    
    \item \emph{Collision detail ($UAS1\leftrightarrow UAS2$)}(fig. \ref{fig:adversaryCollisionDetail}) - Target \emph{UAS 1} (blue) is hit by \emph{Adversary UAS 2} (magenta) on left wing tip. Both UAS are going down. UTM will detect sudden loss of both UAS systems. 
\end{enumerate}
\begin{figure}[H]
    \centering
    \begin{subfigure}{0.48\textwidth}
    	\centering
        \includegraphics[width=0.9\linewidth]{\FIGDIR/NS086UTMAdversary00011}
        \caption{Deviation detection (UAS 2).}
        \label{fig:adversaryDetectedDeviation}
    \end{subfigure}
    \begin{subfigure}{0.48\textwidth}
    	\centering
        \includegraphics[width=0.9\linewidth]{\FIGDIR/NS087UTMAdversary-00014} 
        \caption{Adversary attacking (UAS 2).}
        \label{fig:adversaryAttacking}
    \end{subfigure}
    \\
    \begin{subfigure}{0.48\textwidth}
    	\centering
        \includegraphics[width=0.9\linewidth]{\FIGDIR/NS088UTMAdversary-00017} 
        \caption{Emergency avoidance (UAS 1).}
        \label{fig:adversaryEmergencyAvoidance}
    \end{subfigure}
    \begin{subfigure}{0.48\textwidth}
    	\centering
        \includegraphics[width=0.9\linewidth]{\FIGDIR/NS089UTMAdversary-00020} 
        \caption{Blind spot (UAS 1).}
        \label{fig:adversaryBlindSpot}
    \end{subfigure}
    \\
    \begin{subfigure}{0.40\textwidth}
    	\centering
        \includegraphics[width=0.9\linewidth]{\FIGDIR/NS090UTMAdversaryCrash} 
        \caption{Collision detail.}
        \label{fig:adversaryCollisionDetail}
    \end{subfigure}
    \caption{Adversarial behaviour of \emph{UAS 2} (magenta) to compliant \emph{UAS 1} (blue)}
    \label{fig:adversarialAttackNotableMoments}
\end{figure}

\paragraph{Performance Parameters Evaluation:} Performance parameters (y-axis) are tracked over \emph{UTM time} (x-axis). The evolution of \emph{performance} (fig. \ref{fig:adversaryDistanceToSafetyMargins}) is tracking following parameters:

\begin{enumerate}
    \item \emph{Expected crash distance} (gray line) - defined  as (eq . \ref{eq:crashDistance}) between UAS 1  (blue)and expected UAS 2 position (grey plane/line) over mission time $t\in[0,22]$.
    
    \item \emph{Crash distance} (blue line) - defined  as (eq . \ref{eq:crashDistance}) between UAS 1 (blue) and real UAS 2 position (magenta plane/line) over mission time $t\in[0,22]$.
    
    \item \emph{Safety margin} (yellow line) - constant value according to \emph{collision case}  (tab. \ref{tab:collisionCasesRuleBasedHeadon}) as the value of \emph{10 m}. The safety margin is considered as a \emph{soft constraint}.
    
    \item \emph{Body margin} (red line) - constant value according to (tab. \ref{tab:aboidanceParametersForEmergencyHeadOnScenario}) as value of  \emph{1.2 m}. The body margin is considered as a \emph{hard constraint}. The breaking of \emph{body} margin means an effective \emph{collision} UAS 1 and UAS 2.
\end{enumerate}

\begin{figure}[H]
    \centering
    \includegraphics[width=0.95\linewidth]{\FIGDIR/NS091UTMAdversaryPerformance} 
    \caption{Expected/Real Distance to body/safety margin evolution for \emph{adversarial behavior} of UAS 2.}
    \label{fig:adversaryDistanceToSafetyMargins}
\end{figure}

\noindent\emph{Safety criteria} for both \emph{body} and \emph{safety margins} in case of \emph{expected behavior} are satisfied (eq. \ref{eq:expectedDistanceCondition}). This means that \emph{UAS 1} fulfilled the \emph{UTM directive} even though it entered \emph{Emergency Avoidance Mode} (fig. \ref{fig:adversaryEmergencyAvoidance}).

\begin{equation}\label{eq:expectedDistanceCondition}
    \begin{aligned}
    expectedDistanceToSafetyMargin(t) &\ge 0,\quad &\forall t \in [0,22]\\
    expectedDistanceToBodyMargin(t)  &\ge 0 \quad &\forall t \in [0,22]
    \end{aligned}
\end{equation}

\emph{Safety Margin} is broken at UTM time \emph{15 s}, \emph{body margin} is broken at UTM time \emph{21 s}, the collision happens at  UTM time \emph{22 s}. This is summarized in \emph{Distance Condition Breach} (eq. \ref{eq:distanceConditionBreach}).

\begin{equation}\label{eq:distanceConditionBreach}
    \begin{aligned}
    distanceToSafetyMargin(t) &< 0,\quad &\forall t \in [21,22]\\
    distanceToBodyMargin(t)  &< 0 \quad &\forall t \in [15,22]
    \end{aligned}
\end{equation}

\begin{note}
    An \emph{adversary behavior} needs to be addressed on:
    \begin{enumerate}
        \item  \emph{UAS Traffic Management Level} -  our UTM implementation failed to detect \emph{deviation} (fig. \ref{fig:adversaryDetectedDeviation}) and \emph{start of attack} (fig. \ref{fig:adversaryAttacking}). UAS 2 (magenta) had clean intention from the beginning and did not change pursuit even when \emph{safety margin} was breached. 
        
        \item  \emph{Emergency Avoidance Level} - our \emph{navigation loop implementation} does not consider the \emph{ill-intentions}. The UAS 1 (blue) properly switched to \emph{Emergency avoidance mode} (fig. \ref{fig:adversaryEmergencyAvoidance}) after detection of UAS2 (magenta). UAS 2 (magenta) then used the blind spot to exploit UAS 1 vulnerability.   
    \end{enumerate}
\end{note}

\subsection{\secState{R}Computation Footprint}\label{s:ComputaitonFootprint}

\noindent The \emph{computation footprint} is summarized in computation  load (tab. \ref{tab:computationLoadStatistics}). The \emph{computation load} (eq. \ref{eq:computationLoad}) was calculated for each \emph{time-frame} in scenarios. There is the summary of \emph{minimal, maximal, average} and \emph{median} values.

The \emph{computational load} never exceed more than $55.95\%$ in case of  \emph{emergency Head On} (eq. \ref{eq:computationFeasibilityCriterion}), which means that \emph{every path} was calculated on time.

\begin{table}[H]
    \centering
    \begin{tabular}{r||r|r|r|r}
    
    \multirow{2}{*}{Scenario} & \multicolumn{4}{c}{Computation load} \\ \cline{2-5} 
    & min. & max. & avg. & med. \\ \hline\hline
    Building avoidance (fig. \ref{fig:buildingAvoidanceComputationTime})		
                            &   2.20\% &  27.40\% &  12.11\%  & 13.20\% \\\hline
    Slalom (fig. \ref{fig:slalomComputationTime})				    
                            &  12.20\% &  30.50\% &  21.42\%  & 21.50\% \\\hline
    Maze (fig. \ref{fig:mazeComputationTime})					
                            &  24.90\% &  46.10\% &  31.51\%  & 30.80\% \\\hline
    Storm (fig. \ref{fig:stormComputationTime})					
                            &   2.60\% &  26.90\% &  11.57\%  & 13.90\% \\\hline\hline
    
    Emergency Converging (fig. \ref{fig:emergencyConvergingComputationTime})    
                            &   2.75\% &  16.50\% &   5.84\%  &  4.95\% \\\hline
    Emergency Head On (fig. \ref{fig:emergencyHeadOnComputationTime})	 	
                            &	3.90\% &  55.95\% &  13.19\%  &  6.90\% \\\hline
    Emergency Multiple (fig. \ref{fig:emergencyHeadOnMultipleComputationTime})	 	
                            &	5.90\% &  52.35\% &  12.77\%  &  8.56\% \\\hline\hline
    
    Rule-based Converging (fig. \ref{fig:ruleBasedCConvergingComputationTime})	
                            &   3.60\% &  13.50\% &   7.32\%  &  5.97\% \\\hline
    Rule-based Head on (fig. \ref{fig:ruleBasedHeadOnComputationTime})		
                            &   4.65\% &  41.60\% &  13.64\%  &  9.30\% \\\hline
    Rule-based Multiple	(fig. \ref{fig:ruleBasedMultipleComputationTime})	
                            &   4.37\% &  23.30\% &  11.96\%  & 10.93\% \\\hline
    Rule-based Overtake	(fig. \ref{fig:ruleBasedOvertakeComputationTime})	
                            &   3.85\% &  13.40\% &   7.62\%  &  6.70\% 
    
    \end{tabular}
    \caption{\emph{Computation load statistics} for all test cases.}
    \label{tab:computationLoadStatistics}
\end{table}

\noindent \emph{Following observations can be made:}

\begin{enumerate}
    \item \emph{Building avoidance}, \emph{Slalom}, and \emph{Maze} scenarios - the computation load is increasing with the \emph{number of static obstacles}. The \emph{average load} for \emph{Emergency avoidance mode} in \emph{clustered environment} is $31.51\%$ (Maze).
    
    \item \emph{Storm scenario} - the overall \emph{computation load} is very low due to the \emph{moving constraint implementation} (sec. \ref{s:MovingVirtualConstraints}).
    
    \item\emph{Emergency Converging/Head On/Multiple} scenarios - the \emph{overall computation load} is quite high due to the ineffective \emph{body volume intersection} (sec. \ref{s:bodyvolumeIntersection}) implementation.
    
    \item \emph{Rule-based Converging/Head On/Multiple} scenarios - the \emph{median computational} load is low, because of the linear \emph{rule implementation} (sec. \ref{sec:ruleImplementation})
    
    \item \emph{Rule-based Overtake} - the \emph{average computation load} is very low because only \emph{divergence/convergence} (rule. \ref{tab:ruleOvertakeDefinition}) waypoints are calculated and UAS stays in \emph{navigation mode}.
\end{enumerate}