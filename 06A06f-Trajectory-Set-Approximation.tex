\subsection{Trajectory Set Approximation}\label{sec:trajectory set Reach set approximation}

	\paragraph{Summary:} To achieve finite representation, it is necessary to establish a general relationship between the reach sets and trajectory trees, built from a set of prototypical movements.
	
	
    \paragraph{Discretization of Reach set:} There is a need for a discrete finite \emph{Reach Set approximation} to enable \emph{Avoidance Strategy Evaluation} in finite-time. Replacing \emph{Continuous Control Set} \emph{Inputs(t)} by \emph{Movement Automaton} is feasible:
    
    
    \begin{definition}[Reach set Approximation by Movement Automaton]\label{def:ReachSetApproximationByMovementAutomaton}
        \emph{A trajectory} (def. \ref{def:MovementAutomatonTrajectory}) for system $\dot{state}=f(time,state,input)$ under control of the movement automaton $\mathscr{MA}$ is given as execution of movement buffer (def. \ref{def:MovementBuffer}) with an initial state of system $state_0$.  Therefore notation $Trajectory(state_0,buffer)$ is used.
    
        
        \paragraph{The Complete Reach Set} (\ref{eq:reachSetFormalDefintion}) for system with initial state $state_0$ with existing \emph{control strategy} $control(time)\in Controls(time)$. for time  $\tau > time_0$.
        \begin{equation}\label{eq:reachSetFormalDefintion}
            ReachSet(\tau,time_0,state_0) = \bigcup \left\{state(s):control(s)\in Controls(s), s\in (time_0,\tau]\right\} 
        \end{equation}
        
        
        \paragraph{The Reach Set Approximation by Movement Automaton} (\ref{eq:ReachSetDefinitionDiscrete}) of the system under the control of the movement automation $\mathscr{MA}$ consist from the set of trajectories $Trajectory$ $(state_0,$ $Buffer)$, which are executed in constrained time $\tau > time_0$.
        \begin{equation}\label{eq:ReachSetDefinitionDiscrete}
             ReachSet(\tau,time_0,state_0)=\left\{Trajectory(state_0,buffer):
             \begin{gathered}
                \text{duration}(buffer)\\ \le\\ (time_0-\tau)
             \end{gathered}\right\}
        \end{equation}
    \end{definition}
    
    \begin{note}
        \emph{Reach Set Approximation} (def. \ref{def:ReachSetApproximationByMovementAutomaton}) is a subset of \emph{Full Reach Set} (def. \ref{def:ReachSetBasic}) in continuous space $\R^n$ it inherits all important properties, like \emph{Invariance} \cite{blanchini1999set}.
        
        \emph{Discretization} of \emph{Reach Set} have been achieved leaving us with a \emph{finite count} of \emph{Trajectories}, instead of \emph{Infinite subspace or $\R^N$}
    \end{note}

    \paragraph{Approximated Reach Set Containment:} The \emph{Approximated Reach Set} introduced in (def. \ref{def:ReachSetApproximationByMovementAutomaton}) is constrained only by \emph{future expansion time} $\tau$. UAS makes space assessment in \emph{Avoidance Grid}. There is no point to consider Trajectories outside of \emph{Avoidance Grid}
    
    \begin{definition}[Contained Approximated Reach Set]\label{def:ContainedReducedReachSet}
        For a pair ($state_0$, $AvoidanceGrid_0$) at time $time_0$ and \emph{prediction horizon} $\tau=\infty$ there is \emph{Contained Reduced Reach Set:}
        \begin{equation}\label{eq:containedReachSet}
            ReachSet\left(\begin{gathered}time_0,\\state_0,\\AvoidanceGrid_0\end{gathered}\right) = 
            \left\{
                \begin{gathered}
                Trajectory(\dots) \\\in\\ ReachSet(\ref{eq:ReachSetDefinitionDiscrete})
                \end{gathered}
                :
                \begin{gathered}
                \forall segment \in AvoidanceGrid_0,\\ segment \in Trajectory(\dots)    
                \end{gathered}
            \right\}
        \end{equation}
        
        \paragraph{Properties:} \emph{Container Approximated Reach Set} contains only trajectories where all segments belong to \emph{Avoidance Grid}, there are following functions:
        \begin{enumerate}
            \item The \emph{membership function} for any \emph{Trajectory} in \emph{Constrained Reduced Reach set} returns \emph{Ordered Set} of \emph{Passing Cells}. 
            
            \item The \emph{cost function} for any \emph{Trajectory Portion} in \emph{Constrained Reduced Reach Set} return \emph{Cost of Execution}
        \end{enumerate}
        
        \paragraph{Passing cell:} \emph{Cell} of \emph{Avoidance Grid} which has some intersection with {Trajectory}.
    \end{definition}
    
    \begin{note}
        \emph{Contained Reduced Reach Set} (eq. \ref{eq:containedReachSet}) which is contained in the \emph{Avoidance Grid} and have a \emph{Membership Function} enable Property transition between Reach set and \emph{Avoidance grid}. 
        
        \emph{Example:} Visibility from cells along \emph{Trajectory} can be gathered to calculate \emph{Trajectory`s} feasibility.
    \end{note}
    
    \paragraph{Reach Set Pruning:} There is a need to implement \emph{Set Difference} between \emph{Reach Set} and 
    \emph{Constraint Set}. Constraint Set can be an \emph{Obstacle Set} from \emph{Known World} (sec. \ref{s:KnownWorld}) and other different constraints.
    
    \paragraph{Reach Set Trajectory Tree:} (\ref{eq:trajectoryTree}) \emph{Any Reach Set} where \emph{Control Strategy Constraint} is implemented as \emph{Movement Automaton}, with defined \emph{Movements} set and for single initial $state_0$. The \emph{Reach Set} is given as discrete tree with root $Trajectory(state_0,\varnothing)$. 
    
    \begin{equation}\label{eq:trajectoryTree}
        ReachSet(state_0,\dots) = \left\{Trajectory(state_0,buffer):
        \begin{gathered}
            buffer\in Movements^i,\\
            i\in\{1,\dots,k\}
        \end{gathered}\right\}
    \end{equation}
    
    \noindent For each \emph{Trajectory Segment}, there exists \emph{intersection function} which evaluates as true if there exists at least one point in \emph{Segment} which belongs to \emph{Constraint Set}. Formally: 
    \begin{equation}\label{eq:reachsetIntersectionConstraintSet}
        intersection(segment,Set):
        \begin{cases}
            \begin{aligned}\exists &point \in segment,\\ &point \in Set \end{aligned}&: true \\
            Otherwise &: false 
        \end{cases}
    \end{equation}
    
    \begin{definition}[Pruned Reach Set]\label{def:PrunedReachSet} 
        For \emph{Reach set} represented as \emph{Trajectory Tree} (eq. \ref{eq:trajectoryTree}) and some constraint set ($Set$) where exist \emph{intersection function} (eq. \ref{eq:reachsetIntersectionConstraintSet}). The \emph{Pruned Reach set} is given as follows:
        
        \begin{equation}\label{eq:PrunedReachSet}
            Prune(ReachSet,Set) = 
            \left\{
                Trajectory(\dots):
                \begin{gathered} 
                \forall segment \in Trajectory,\\ \neg intersection(segment, Set) 
                \end{gathered}
            \right\}
        \end{equation}
    \end{definition}
    
    
    \begin{note} 
        Pruning(def. \ref{def:PrunedReachSet}) \cite{birmingham1988tree} is applied multiple times for various \emph{Constraints Set}. 
    
        Example of \emph{Approximated Reach set Calculation} (def. \ref{def:ReachSetApproximationByMovementAutomaton}), \emph{Reach Set Containment} (def. \ref{def:ContainedReducedReachSet}), and, \emph{Pruning} is given in \cite{gomola2017obstacle}.
    \end{note}
