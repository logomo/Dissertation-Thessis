\cleardoublepage
\section{Avoidance Concept}\label{s:avoidanceConcept}

\paragraph{Summary:} There is a need for a functional orchestration of previous concepts to achieve avoidance and navigation capabilities. The avoidance grid threat assessment done in (sec. \ref{s:sensorFusion}) needs to be applied on the RSA of choice to produce a safe trajectory for one fixed time. This procedure is described in the avoidance grid run. There is a need to join output multiple avoidances runs over the time to achieve the required avoidance/navigation capabilities.  This procedure is described in the navigation run. There is a need to assess the computational complexity of the approach to show implementation feasibility. 


This section introduces \emph{Platform Independent Avoidance Concept} core functionality (fig. \ref{fig:AvoidanceFrameworkConceptNew}) modules responsible for \emph{pathfinding} and \emph{navigation}. The sections are organized like follow:

\begin{enumerate}
    %\item \emph{Data Fusion} (sec. \ref{s:sensorFusion}) - implementation details of \emph{input interface} responsible for \emph{processing partial known world data} into final visibility, obstacle, intruder, and, constraints ratings.
    
    \item \emph{Avoidance Grid Run} (sec.\ref{s:aviudabceGridRun}) (inner avoidance run) - the \emph{best pathfinding} in one \emph{Avoidance Grid} with \emph{situation assessment} done.
    
    \item \emph{Mission Control Run} (sec . \ref{s:missionControlRun}) (outer navigation run) - main navigation and decision making an algorithm for \emph{non-cooperative obstacle avoidance}.
    
    \item \emph{Computational Complexity} (sec. \ref{sec:MCRcomputationalComplexity}) - the \emph{computational feasibility study} and \emph{weak point identification} of our approach.
    
    %\item \emph{Safety Margin Calculation} (app. \ref{s:safetyMarginCalculation}) - the boundaries of \emph{Safety Margin} and identified \emph{impact factors}.
\end{enumerate}
