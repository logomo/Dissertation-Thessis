\cleardoublepage
\section{\secState{R}UAS Model and Control}\label{s:modelMAImplementation}

\noindent The key feature of \emph{Movement Automaton} is to interface \emph{continuous-control signal} as the \emph{discrete command chain}. Following topics are introduced in this section:

\begin{enumerate}
	\item \emph{Movement Automaton Background} (sec. \ref{sec:MovementAutomatonBackground}) - the listing of related work and similar approaches to ours.
	
	\item \emph{Specialization of Hybrid Automaton} (sec. \ref{s:MovementAutomatonBuidlingBlocks}) - the specialization of the hybrid automaton to fulfill control/approximation roles in our approach.
	
	\item \emph{Formal Movement Automaton Definition} (sec. \ref{s:MovementAutomatonDefinitionAndProperties}) - the formal definition of \emph{movement automaton} used in our approach.
	
    \item \emph{Used UAS Nonlinear Model} (sec. \ref{s:UASNonlinearModel}) - simple plane model used in this work as \emph{controlled plant}.
    
    \item \emph{Used Movement Automaton} (sec. \ref{s:movementAutomatonDefinition}) - movement automaton for \emph{UAS Nonlinear Model} constructed from scratch.
    
    \item \emph{Segmented Movement Automaton} (sec. \ref{s:segmentedMovementAutomaton}) - for more complex systems the \emph{State Space} can be \emph{separated into Segments} and \emph{segment movement automaton} is used to generate \emph{thick reference trajectory}.
    
    \item \emph{Reference Trajectory Generator} (sec. \ref{s:referenceTrajectoryGenerator}) - other use of \emph{Movement Automaton} as predictor for \emph{reference trajectory calculation}.
\end{enumerate}



