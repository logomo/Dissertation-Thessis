\section{\secState{R}Summary}\label{s:conclusionSummary}
\noindent The approach presented in (ch. \ref{ch:approach}) is covering the challenges (fig. \ref{fig:AvoidanceLevels}) related to:

\begin{enumerate}
    \item \emph{Reactive avoidance} - covering static obstacles, moving intruders, geo-fencing, weather effects.
    
    \item \emph{Event-based avoidance} - covers core UTM functionality and cooperative avoidance.
    
    \item \emph{Pre-emptive avoidance} - not covered in this work, the assumption about initial waypoint reachability still holds (ass. \ref{ass:reachableWaypoints}).
\end{enumerate}

\paragraph{Reactive Avoidance Test Coverage:} Testing avoidance capabilities against \emph{static obstacles}, \emph{non-cooperative intruders}, \emph{moving hard constraints} are covered. Coverage of the \emph{soft constraints}, \emph{map obstacles}, and \emph{detected obstacles} are implicitly covered due to the properties of \emph{safety} and \emph{body} margins (tab. \ref{tab:uncontrolledAirspaceViolations}).
    
\begin{enumerate}
    \item \emph{Building Avoidance} (sec. \ref{s:testBuildingAvoidance}) covers \emph{static obstacles} explicitly and \emph{map obstacles, hard constraints, ground avoidance} implicitly.
    
    \item \emph{Slalom} (sec. \ref{s:testSlalom}) covers \emph{open space navigation capabilities}, showing the determinism of the \emph{avoidance loop run}, in addition to \emph{building avoidance}.
    
    \item \emph{Maze} (sec. \ref{s:testMaze}) covers \emph{closed space navigation capabilities}, showing the higher level navigation properties of primitive \emph{right-side} 2D maze solver. The main point is to show possibility to enrich the \emph{Navigation loop algorithm} (fig. \ref{fig:missionControlRunActivityDiagram}).
    
    \item \emph{Storm} (sec. \ref{s:testStorm}) covers \emph{hard moving constraints avoidance} explicitly and \emph{hard static constraints}, \emph{soft static constraints}, \emph{soft moving constraints} implicitly.
    
    \item \emph{Emergency converging} scenario (sec. \ref{s:testEmergencyConverging}) covers \emph{non-cooperative intruder with the right of the way} avoidance capability.
    
    \item \emph{Emergency head-on} scenario covers (sec. \ref{s:testEmergencyHeadOn}) \emph{non-cooperative intruder without right of the way} avoidance capability
    
    \item \emph{Emergency mixed} scenario (sec. \ref{s:testEmergencyMixed}) covers \emph{multiple intruders with/without right of the way} avoidance capability.
\end{enumerate}

\paragraph{Event-Based Avoidance Coverage:} Test cases covers \emph{well clear breach prevention}, \emph{situation-based avoidance}, and \emph{rules of the air enforcement}. Coverage of \emph{near miss situations}, \emph{clash incidents} is given implicitly by \emph{safety} and \emph{body} margins (tab. \ref{tab:controlledAirspaceViolations}).

\begin{enumerate}
    \item \emph{Rule-based converging} (sec. \ref{s:testRuleConverging}) covers \emph{well clear breach} and \emph{the converging rule of the air}, showing determinism and \emph{UTM resolution execution}.
    
    \item \emph{Rule-based head-on} (sec. \ref{s:testRuleHeadOn}) covers \emph{well clear breach} and \emph{head on rule of the air}, showing determinism and \emph{UTM resolution execution}.
    
    \item \emph{Rule-based mixed head on with converging} (sec. \ref{s:testRuleMixed}) covers \emph{well clear breach} and \emph{head on and converging rules of the air}. The main focus is on the \emph{virtual roundabout} concept when multiple collision cases are clustered into one avoidance maneuver. 
    
    \item \emph{Rule-based overtake} (sec. \ref{s:testRuleOvertake}) covers \emph{well clear breach} during \emph{overtaking} by faster UAS.
\end{enumerate}


\subsection{\secState{R} Hierarchical Decision Making}\label{s:conclusionAvoidanceNavigationRun}
\noindent The key feature of the approach is hierarchical decision making. The hierarchy is introduced not only to threat prioritization but also to the time sequence of the decisions. The framework scheme (fig. \ref{fig:AvoidanceFrameworkConceptNew}) shows that situational assessment in the form of \emph{data fusion} is done in (sec. \ref{s:sensorFusion}). The trajectory selection process is separated from the final avoidance strategy pick-up process. The decision making separation is done like follows:

\begin{itemize}
    \item[$\to$] \emph{Avoidance Run} (sec. \ref{s:aviudabceGridRun}) - for one specific data set, for a fixed time, for a fixed goal find the cheapest trajectory which is safe (reachable).
    
    \item[$\to$] \emph{Navigation Run} (sec. \ref{s:missionControlRun}) - for multiple data sets (threat hierarchy), for a fixed time, for context depending goal try to find existing solution as a safe trajectory. Select trajectory breaking least restrictions from candidates.
\end{itemize}

The decision making process on threat assessment level (fig. \ref{fig:missionControlRunActivityDiagram}) is the configuration of threat type application in 7\textsuperscript{th}-10\textsuperscript{th} step. The idea is that UAS:
\begin{enumerate}
    \item must avoid any \emph{static obstacles} (sec. \ref{s:detectedObstacles}),
    
    \item can skip \emph{intruders} (sec. \ref{s:intruders}) if there are no other options
    
    \item can enter into \emph{hard constraints} (sec. \ref{s:virtualConstraints}, \ref{s:MovingVirtualConstraints}) area if its necessary,
    
    \item can enter into \emph{soft constraints} (sec. \ref{s:virtualConstraints}, \ref{s:MovingVirtualConstraints}) area if its necessary.
\end{enumerate}

\begin{note}
    The \emph{implementation} of \emph{soft/hard constraints} and their evaluation process is same in \emph{avoidance run}; the only difference is when they are applied by \emph{navigation run}.
\end{note}
    
\subsection{\secState{R}Use of Developed Reach Set Approximations}\label{s:conclusionReachSet}

\noindent The reach set approximation (sec. \ref{s:reachSet}) represents a set of  \emph{maneuvering} strategies (avoidance/movement) which can be used in a different context. The properties of the trajectories are given based on \emph{constrained expansion} (sec. \ref{s:constrainedTrajectoryExpansion}) and performance parameters priority (sec. \ref{s:ReachSetPerformanceCriteria}).

The conditions and priorities are changing with the operational environment, to successful integration into nonsegregated airspace is necessary to reflect this.  For this purpose, the multiple reach set approximation approaches were developed. The performance tests and longer summary can be found in (sec. \ref{sec:reducedReachSetPerformance}). The summary of each reach set approximation method is given like follow:

\begin{itemize}
    \item[$\to$] \emph{Chaotic reach set approximation} (sec. \ref{s:chaoticReachSet}) - contains a lot of \emph{curved trajectories} with high space coverage ratio, used for \emph{emergency reactive avoidance} in non-controlled/controlled airspace, \emph{last resort reach set}.
    
    \item[$\to$] \emph{Harmonic reach set approximation} (sec. \ref{s:harmonicReachSet}) -  contains a lot of \emph{straight} trajectories with low space coverage ratio, used for navigation in \emph{non-controller airspace}.
    
    \item[$\to$] \emph{Combined reach set approximation} (sec. \ref{s:combinedReachSet}) - the reach set approximation is represented as a tree, there is a possibility to combine two or more reach sets with the same initial state, the outcome of such merge is a new reach state with different properties.
    
    \item[$\to$] \emph{ACAS-like reach set approximation} (sec. \ref{s:acasReachSet}) - contains \emph{mandatory separations} (depending on aircraft type), has low trajectory count and low coverage ratio, it emulates ACAS maneuver table in reach set data structure, used for navigation and avoidance in ATC controlled airspace.
\end{itemize}