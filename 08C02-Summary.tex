\section{\secState{W}Summary}\label{s:conclusionSummary}
\noindent The \emph{obstacle avoidance} is a complex topic the work has adressed:
\begin{itemize}
    \item Reactive obstacle avoidance (practically for large scale of threats) - obstacles, intruders, constraints
    
    \item Event Based obstacle avoidance (partially for smal subset of events +UTM) - there are issues which has not been adressed (and they will be later)
    
    \item Pre-emptive obstacle avoidance - this has not been adressed at all still holding the reachibility of waypoint assumption (link assumption) 
    \item link the avoidance levels from chapter 6
\end{itemize}

    
\subsection{\secState{W}Reach Set Role in Avoidance}\label{s:conclusionReachSet}

\noindent The reach set represents a set of avoidance/movement (manuevering) strategies which can be used in different context (situations)
\begin{itemize}
    \item Chaotic - intermediate emergency avoidance
    \item Harmonic - navigation
    \item Combined - non controlled airspace navigation + avoidance
    \item ACAS like  controlled airspace navigation + avoidance
\end{itemize}


\subsection{\secState{W}Avoidance/Navigation Run}\label{s:conclusionAvoidanceNavigationRun}

\noindent The principle of hierarchical decision making
\begin{itemize}
    \item Avoidance Run - one decision, one context
    \item Navigation Run - multiple avoidance runs with different context (hierarchical applicaiton), joining multiple decisions over time
    \item short summary with references
\end{itemize}


