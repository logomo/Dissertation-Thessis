%% fcup-thesis.tex -- document template for PhD theses at FCUP
%%
%% Copyright (c) 2015 João Faria <joao.faria@astro.up.pt>
%%
%% This work may be distributed and/or modified under the conditions of
%% the LaTeX Project Public License, either version 1.3c of this license
%% or (at your option) any later version.
%% The latest version of this license is in
%%     http://www.latex-project.org/lppl.txt
%% and version 1.3c or later is part of all distributions of LaTeX
%% version 2005/12/01 or later.
%%
%% This work has the LPPL maintenance status "maintained".
%%
%% The Current Maintainer of this work is
%% João Faria <joao.faria@astro.up.pt>.
%%
%% This work consists of the files listed in the accompanying README.

%% SUMMARY OF FEATURES:
%%
%% All environments, commands, and options provided by the `ut-thesis'
%% class will be described below, at the point where they should appear
%% in the document.  See the file `ut-thesis.cls' for more details.
%%
%% To explicitly set the pagestyle of any blank page inserted with
%% \cleardoublepage, use one of \clearemptydoublepage,
%% \clearplaindoublepage, \clearthesisdoublepage, or
%% \clearstandarddoublepage (to use the style currently in effect).
%%
%% For single-spaced quotes or quotations, use the `longquote' and
%% `longquotation' environments.


%%%%%%%%%%%%         PREAMBLE         %%%%%%%%%%%%

%%  - Default settings format a final copy (single-sided, normal
%%    margins, one-and-a-half-spaced with single-spaced notes).
%%  - For a rough copy (double-sided, normal margins, double-spaced,
%%    with the word "DRAFT" printed at each corner of every page), use
%%    the `draft' option.
%%  - The default global line spacing can be changed with one of the
%%    options `singlespaced', `onehalfspaced', or `doublespaced'.
%%  - Footnotes and marginal notes are all single-spaced by default, but
%%    can be made to have the same spacing as the rest of the document
%%    by using the option `standardspacednotes'.
%%  - The size of the margins can be changed with one of the options:
%%     . `narrowmargins' (1 1/4" left, 3/4" others),
%%     . `normalmargins' (1 1/4" left, 1" others),
%%     . `widemargins' (1 1/4" all),
%%     . `extrawidemargins' (1 1/2" all).
%%  - The pagestyle of "cleared" pages (empty pages inserted in
%%    two-sided documents to put the next page on the right-hand side)
%%    can be set with one of the options `cleardoublepagestyleempty',
%%    `cleardoublepagestyleplain', or `cleardoublepagestylestandard'.
%%  - Any other standard option for the `report' document arclass can be
%%    used to override the default or draft settings (such as `10pt',
%%    `11pt', `12pt'), and standard LaTeX packages can be used to
%%    further customize the layout and/or formatting of the document.

%% *** Add any desired options. ***
%PDF
%\documentclass[a4paper,narrowmargins,11pt,oneside,draft,onehalfspaced,singlespacednotes]{fcup-thesis}
%\documentclass[a4paper,narrowmargins,11pt,oneside,onehalfspaced,singlespacednotes]{fcup-thesis}
%Print
%\documentclass[draft,a4paper,narrowmargins,11pt,twoside,openright,onehalfspaced,singlespacednotes]{fcup-thesis}
\documentclass[a4paper,narrowmargins,11pt,twoside,openright,onehalfspaced,singlespacednotes]{fcup-thesis}

%% *** Add \usepackage declarations here. ***
%% The standard packages `geometry' and `setspace' are already loaded by
%% `ut-thesis' -- see their documentation for details of the features
%% they provide.  In particular, you may use the \geometry command here
%% to adjust the margins if none of the ut-thesis options are suitable
%% (see the `geometry' package for details).  You may also use the
%% \setstretch command to set the line spacing to a value other than
%% single, one-and-a-half, or double spaced (see the `setspace' package
%% for details).
% Overfull statements
\pretolerance=150
\setlength{\emergencystretch}{3em}
% Overfull end
\usepackage[english]{babel}
\usepackage{helvet} %To replace arial fonts
\usepackage{lipsum}
\usepackage[utf8]{inputenc}


%%% Additional useful packages
%%% ----------------------------------------------------------------
\usepackage{array}
\usepackage{amsmath}  
\usepackage{amssymb}
\usepackage{amsfonts}
\DeclareFontFamily{OT1}{pzc}{}
\DeclareFontShape{OT1}{pzc}{m}{it}{<-> s * [0.900] pzcmi7t}{}
\DeclareMathAlphabet{\mathpzc}{OT1}{pzc}{m}{it}
%Titles need to be 14 pt => Large in \normaltext 11pt
\usepackage{titlesec}
\titleformat*{\section}{\Large\bfseries}
\titleformat*{\subsection}{\Large\bfseries}
\titleformat*{\subsubsection}{\Large\bfseries}
%Titles need to be 14 pt => Large in \normaltext 11pt
\usepackage{amsthm}      
\usepackage[ruled,algochapter]{algorithm2e}
\usepackage{algorithmic}
\usepackage{bm}
\usepackage[mathscr]{euscript}
\usepackage{graphicx}       
\usepackage{psfrag}         
\usepackage{fancyvrb}    
\usepackage{float}
\usepackage{ltablex}
\usepackage[square,sort,comma,numbers]{natbib}        
\usepackage{bbding}         
\usepackage{dcolumn}        
\usepackage{booktabs} 
\usepackage{multirow}
\usepackage{paralist}     
\usepackage{ifdraft}  
\usepackage{indentfirst}    
\usepackage[nottoc,notlof,notlot]{tocbibind}
\usepackage{url}
\usepackage{tabularx}
%use font size for captions like 8pt -> normalisize 11pt, scriptsize->8pt
\usepackage[font={scriptsize}]{caption}
\usepackage[font={scriptsize}]{subcaption}
\captionsetup{font=scriptsize}

\usepackage[unicode]{hyperref}
\usepackage{xcolor}


\hypersetup{pdftitle=Obstacle avoidance framework based on reach sets, 
            pdfauthor=Alojz Gomola,
            colorlinks=false,
            urlcolor=blue,
            pdfstartview=FitH,
            pdfpagemode=UseOutlines,
            pdfnewwindow,
            breaklinks
          }
\usepackage{array}
\newcolumntype{L}[1]{>{\raggedright\let\newline\\\arraybackslash\hspace{0pt}}m{#1}}
\newcolumntype{C}[1]{>{\centering\let\newline\\\arraybackslash\hspace{0pt}}m{#1}}
\newcolumntype{R}[1]{>{\raggedleft\let\newline\\\arraybackslash\hspace{0pt}}m{#1}}         
\newcolumntype{B}{X}
\newcolumntype{S}[1]{>{\hsize=#1\textwidth}X}

\newcommand{\FIGDIR}{./Pics}    %%% directory containing figures
\newcommand{\twolinecellr}[2][r]{%
  \begin{tabular}[#1]{@{}r@{}}#2\end{tabular}}
\newcommand{\secState}[1]{
	\ifdraft{(#1) }{}
}
\theoremstyle{plain}
\newtheorem{theorem}{Theorem}
\newtheorem{lemma}[theorem]{Lemma}
\newtheorem{proposition}[theorem]{Proposition}

\theoremstyle{plain}
\newtheorem{definition}{Definition}
\newtheorem{problem}{Problem}
\newtheorem{example}{Example}
\newtheorem{assumption}{Assumption}

\theoremstyle{remark}
\newtheorem*{corollary}{Corollary}
\newtheorem*{note}{Note}




\newenvironment{dokaz}{
  \par\medskip\noindent
  \textit{Proof}.
}{
\newline
\rightline{\SquareCastShadowBottomRight}
}

\newenvironment{constraints}[1]{
  \par\medskip\noindent
  \textit{Constraints #1} \\
}{
\newline
\rightline{\SquareCastShadowBottomRight}
}


%\bibliographystyle{plainnat}     %% Author (year) style
\bibliographystyle{unsrt}        %% [number] style
\setcitestyle{square}

% Section  3.7 Challenge list
\newif\ifproblemchallenge   %# Build block for problem challenges
\problemchallengetrue       %# Show comments

\newcommand{\R}{\mathbb{R}}
\newcommand{\N}{\mathbb{N}}

\DeclareMathOperator{\pr}{\textsf{P}}
\DeclareMathOperator{\E}{\textsf{E}\,}
\DeclareMathOperator{\var}{\textrm{var}}
\DeclareMathOperator{\sd}{\textrm{sd}}


\newcommand{\T}[1]{#1^\top}        

\newcommand{\goto}{\rightarrow}
\newcommand{\gotop}{\stackrel{P}{\longrightarrow}}
\newcommand{\maon}[1]{o(n^{#1})}
\newcommand{\abs}[1]{\left|{#1}\right|}
\newcommand{\dint}{\int_0^\tau\!\!\int_0^\tau}
\newcommand{\isqr}[1]{\frac{1}{\sqrt{#1}}}
\newcommand{\norm}[1]{\left\lVert#1\right\rVert}


\newcommand{\pulrad}[1]{\raisebox{1.5ex}[0pt]{#1}}
\newcommand{\mc}[1]{\multicolumn{1}{c}{#1}}
\newcommand{\TBD}[1]{\color{red}\emph{--TBD:}#1\color{black}}

%%%%%%%%%%%%%%%%%%%%%%%%%%%%%%%%%%%%%%%%%%%%%%%%%%%%%%%%%%%%%%%%%%%%%%%%
%%                                                                    %%
%%                   ***   I M P O R T A N T   ***                    %%
%%                                                                    %%
%%  Fill in the following fields with the required information:       %%
%%   - \degree{...}       name of the degree obtained                 %%
%%   - \department{...}   name of the graduate department             %%
%%   - \gradyear{...}     year of graduation                          %%
%%   - \author{...}       name of the author                          %%
%%   - \title{...}        title of the thesis                         %%
%%%%%%%%%%%%%%%%%%%%%%%%%%%%%%%%%%%%%%%%%%%%%%%%%%%%%%%%%%%%%%%%%%%%%%%%

%% *** Change this example to appropriate values. ***
\degree{Doctor of Philosophy}
\department{Departamento de Matem\'{a}tica}
\gradyear{2019}
\author{Alojz Gomola}
\title{Obstacle Avoidance Framework based on Reach Sets}

%% *** NOTE ***
%% Put here all other formatting commands that belong in the preamble.
%% In particular, you should put all of your \newcommand's,
%% \newenvironment's, \newtheorem's, etc. (in other words, all the
%% global definitions that you will need throughout your thesis) in a
%% separate file and use "\input{filename}" to input it here.


%% *** Adjust the following settings as desired. ***

%% List only down to subsections in the table of contents;
%% 0=chapter, 1=section, 2=subsection, 3=subsubsection, etc.
\setcounter{tocdepth}{3}

%% Make each page fill up the entire page.
\flushbottom


%%%%%%%%%%%%      MAIN  DOCUMENT      %%%%%%%%%%%%

\begin{document}


%%%%%%%%%%%%%%%%%%%%%%%%%%%%%%%%%%%%%%%%%%%%%%%%%%%%%%%%%%%%%%%%%%%%%%%%
%%  Put your Chapters here; the easiest way to do this is to keep     %%
%%  each chapter in a separate file and `\include' all the files.     %%
%%  Each chapter file should start with "\chapter{ChapterName}".      %%
%%  Note that using `\include' instead of `\input' will make each     %%
%%  chapter start on a new page, and allow you to format only parts   %%
%%  of your thesis at a time by using `\includeonly'.                 %%
%%%%%%%%%%%%%%%%%%%%%%%%%%%%%%%%%%%%%%%%%%%%%%%%%%%%%%%%%%%%%%%%%%%%%%%%

%% *** Include chapter files here. ***

\setcounter{chapter}{7}
\setcounter{section}{1}

    \cleardoublepage
\section{\secState{D}Testing Configuration}\label{sec:testingConfiguration}

    \noindent All \emph{simulations} are run with configuration described in this \emph{section}. The UAS used for the purposes is given by \emph{model and control} (sec. \ref{s:modelMAImplementation}). 
    
    \emph{UAS parameters:} An \emph{UAS system} (tab. \ref{tab:testUASBasicParameters}) is modeled after small scale toy model with: maximal body radius $30$ $cm$, maximal speed $4$ $m.s^{-1}$,weight $450$ $g$., maximal flight duration $20$ $min$, maximal turning rate $15$ $deg.s^{-1}$. The \emph{body margin} is set to $0.3 m$, the \emph{near miss radius} is double of \emph{body margin}, thus $0.6$ $m$, the \emph{well clear radius} is set to $5$ $m$. Margins can be set to any value if they are complaint with condition (\ref{eq:marginsBoundary}).
    
    \begin{equation}\label{eq:marginsBoundary}
        0 < bodyMargin \le nearMissRadius \le wellClearRadius \le gridDistance
    \end{equation}   
    
    \begin{note}
        \emph{Safety margin} is broad term used to describe \emph{minimal distance} between UAS and \emph{adversarial object}. The \emph{Safety margin} is:
        \begin{enumerate}
            \item \emph{near miss radius} in case of \emph{non-controlled airspace} or \emph{emergency avoidance mode}.
            \item \emph{well clear radius} in case of \emph{controlled airspace} and \emph{navigation mode}.
        \end{enumerate}
    \end{note}
    
    \emph{Decision time:} Decision time can be set by the user to any positive non-zero value (\ref{eq:decisionTimeBoundary}). The \emph{Decision time} is equal $1$ $s$ and \emph{Decision frames} are synchronized.
    \begin{equation}\label{eq:decisionTimeBoundary}
        maxAlrogithmCalculationTime \le decisionTome \le \infty
    \end{equation}
    
    \emph{Speed:} For \emph{all movements} constant speed $1$ $m.s^{-1}$ is used. Speed can be changed to any value in given boundary (\ref{eq:speedBoundary}).
        \begin{equation}\label{eq:speedBoundary}
            0 \le speed    \le 
            \min\left(
            \begin{aligned}
                & 0.5\times(navigationGrid.distance/decisionFrame)\\
                & 0.5\times(avoidanceGrid.distance/decisionFrame)
            \end{aligned}
            \right)
        \end{equation}
    
    
    \emph{Movement automaton:} The \emph{movement set} is given in (tab. \ref{tab:testMovementOrientations}). The \emph{movement} set contains horizontal, vertical, and, combined movements. 
    
    \emph{Grids:} Used \emph{Navigation grid parameters} are given in (tab. \ref{tab:testNavigationGridBasic}).Selected \emph{Navigation Reach set} is \emph{ACAS-like} with enabled horizontal/vertical separation. Used \emph{Avoidance grid parameters} are given in (tab. \ref{tab:testAvoidanceGridBasic}). Selected \emph{Avoidance Reach set} is \emph{combined} because of high \emph{coverage ratio}. 
    
    User can define own grid parameters according to the \emph{space discretization rules} (sec. \ref{s:AvoidanceGrid}) and chose own \emph{reach set type} according to preference (sec. \ref{s:reachSet}).
    
    \begin{tabular}{cc}
    
    \begin{minipage}[t]{0.48\textwidth}
        \begin{table}[H]
            \centering
            \begin{tabular}{r||r|r|r}
             Movement  &  Roll         & Pitch             & Yaw          \\\hline\hline
             Straight  &  $0^\circ$    & $0^\circ$         & $0^\circ$    \\\hline
             Left      &  $0^\circ$    & $15^\circ$        & $0^\circ$    \\\hline
             Right     &  $0^\circ$    & $-15^\circ$       & $0^\circ$    \\\hline
             Up        &  $0^\circ$    & $0^\circ$         & $-15^\circ$  \\\hline
             Down      &  $0^\circ$    & $0^\circ$         & $15^\circ$   \\\hline
             UpLeft    &  $0^\circ$    & $15^\circ$        & $-15^\circ$  \\\hline
             UpRight   &  $0^\circ$    & $-15^\circ$       & $-15^\circ$  \\\hline
             DownLeft  &  $0^\circ$    & $15^\circ$        & $15^\circ$   \\\hline
             DownRight &  $0^\circ$    & $-15^\circ$       & $15^\circ$   \\
            \end{tabular}
            \caption{Movement orientations.}
            \label{tab:testMovementOrientations}
        \end{table}
    \end{minipage}
    &
    \begin{minipage}[t]{0.48\textwidth}
        \begin{table}[H]
            \centering
            \begin{tabular}{r|r}
            \multicolumn{2}{c}{UAS parameters}                  \\\hline\hline
             speed                  &  $1\,ms^{-1}$             \\\hline
             horizontal turning r.  &  $3.82\,m$                \\\hline
             vertical turning r.    &  $3.82\,m$                \\\hline
             body radius            &  $0.3\,m$                 \\\hline
             near miss r.           &  $0.6\,m$                 \\\hline
             well clear r.          &  $5\,m$                   \\
            \end{tabular}
            \caption{\emph{UAS} parameters.}
            \label{tab:testUASBasicParameters}
        \end{table}
    \end{minipage}\\
    \begin{minipage}[t]{0.48\textwidth}
        \begin{table}[H]
            \centering
            \begin{tabular}{r|r}
            \multicolumn{2}{c}{Navigation Grid}                 \\\hline\hline
             type                   &  ACAS-like                \\\hline
             distance range         &  $0-10\,m$                \\\hline
             layer step             &  $1\,m$                   \\\hline
             horizontal range       &  $\pm 45^\circ$           \\\hline
             horizontal cells       &  $7$                      \\\hline
             vertical range         &  $\pm 30^\circ$           \\\hline
             vertical cells         &  $5$                      \\
            \end{tabular}
            \caption{\emph{Navigation Space} parameters.}
            \label{tab:testNavigationGridBasic}
        \end{table}  
    \end{minipage}&
    \begin{minipage}[t]{0.48\textwidth}
        \begin{table}[H]
            \centering
            \begin{tabular}{r|r}
            \multicolumn{2}{c}{Avoidance Grid}                  \\\hline\hline
             type                   &  combined                 \\\hline
             distance range         &  $0-10\,m$                \\\hline
             layer step             &  $1\,m$                   \\\hline
             horizontal range       &  $\pm 45^\circ$           \\\hline
             horizontal cells       &  $7$                      \\\hline
             vertical range         &  $\pm 30^\circ$           \\\hline
             vertical cells         &  $5$                      \\
            \end{tabular}
            \caption{\emph{Avoidance Space} parameters.}
            \label{tab:testAvoidanceGridBasic}
        \end{table}
    \end{minipage}
    \end{tabular}
    \begin{table}[H]
        \centering
        \begin{tabular}{r|l|l}
        \multicolumn{3}{c}{Coloring}                 \\\hline\hline
         Airc. & Executed & Planned                  \\\hline\hline
         UAS 1 & blue     & red                      \\\hline
         UAS 2 & cyan     & magenta                  \\\hline
         UAS 3 & green    & yellow                   \\\hline
         UAS 4 & black    & green                    \\
        \end{tabular}
        \caption{\emph{UAS} coloring.}
        \label{tab:testUASColoring}
    \end{table}


    
    

%% This adds a line for the Bibliography in the Table of Contents.
\addcontentsline{toc}{chapter}{Bibliography}
%% *** Set the bibliography style. ***
%% (change according to your preference/requirements)
%\bibliographystyle{plain}
%% *** Set the bibliography file. ***
%% ("thesis.bib" by default; change as needed)
\bibliography{thesis}

%% *** NOTE ***
%% If you don't use bibliography files, comment out the previous line
%% and use \begin{thebibliography}...\end{thebibliography}.  (In that
%% case, you should probably put the bibliography in a separate file and
%% `\include' or `\input' it here).

\end{document}
