\section{\secState{W}Comparison to Other Methods}\label{s:OtherMethodsComparison}

\begin{enumerate}
	\item Vector field avoidance \cite{borenstein1991vector}
	\item Potential field \cite{koren1991potential}
\end{enumerate}

\subsection{\secState{W}Scalability}\label{s:conclusionScalability}

\begin{itemize}
    \item the scalability is a key for everchanging rules/regulations accomodations
    
    \item many concepts have margins hardwired
    
    \item our concept is limited by turning ratio + body radius < margin < max range of sensor field - avoidance grid
    
    \item our approach is scalable trough concept of multiple  margins:
    \begin{itemize}
        \item body margin
        \item safety margin
        \item soft constraitns - warning margin
        \item hard constraints - body margin
    \end{itemize}
    
\end{itemize}

\subsection{\secState{W}Conservative Method Comparison}\label{s:conservativeComparison}
\begin{itemize}
    \item Take notes from martin hrdlik work - compare the method 
    \item Key concept/Idea: keep awayat leas double of truning radius
    \item Show calculation/comparison
\end{itemize}

\subsection{\secState{W}Potential Field Method Comparison}\label{s:potentialComparison}
\begin{itemize}
    \item Take ntoes from martin hrdlik work - compare the methods
    \item key concept/Idea: Every obstacle have charge proportional to expected mass. our UAS is repelled by this charge, the charges can aslo have static/dinamic energy emulating obstacles/intruders
    \item method has good performance but do not guarantees the safety like ours
\end{itemize}

