\section{Aircraft Operational Rules}\label{sec:AircraftOperationRules}
\paragraph{Motivation:} The \emph{aircraft operation rules} are ranging from personal, through technical, to standardization category. In this section, the \emph{flight rules} will be outlined in necessary depth for \emph{collision avoidance}. The goal of this section is to give an overview of airspace constraints. 

\paragraph{Rules Origin:} The \emph{Rules of the Air} are provided by the following documents:

\begin{enumerate}
    \item \emph{SERA Regulation 923/2012} - laying down the common rules of the air and operational provisions regarding services and procedures in air navigation and amending Implementing Regulation (EU) No 1035/2011 and Regulations (EC) No 1265/2007, (EC) No 1794/2006, (EC) No 730/2006, (EC) No 1033/2006 and (EU) No 255/2010 \cite{rulesOfTheFlight2012} notable contributions:
    \begin{enumerate}[a.]
        \item \emph{Table of cruising levels} - Appendix III.
        \item \emph{ATS airspace classes — services provided and flight requirements} - Appendix IV.
    \end{enumerate}
    
    \item \emph{SERA Regulation 2016/1185} - Commission Implementing Regulation (EU) 2016/1185 of 20 July 2016 amending Implementing Regulation (EU) No 923/2012 as regards the update and completion of the common rules of the air and operational provisions regarding services and procedures in air navigation (SERA Part C) and repealing Regulation (EC) No 730/2006.

    \item \emph{ICAO  Annex II.} - the \emph{most accepted} rules of the air document \cite{icaoAnnex2}, providing general rules of the air (sec. \ref{sec:handlingHeadOnApproach}, \ref{sec:handlingConvergingManuever}, \ref{sec:handlingOvertakeManuever}).
\end{enumerate}

\begin{note}
    This section contains important parts from previously mentioned documents. 
\end{note}

\subsection{Visual Flight Rules}\label{sec:VisualFlightRules}
\paragraph{Motivation:} A \emph{Visual Flight Rules} (VFR) requires the pilot ability to see outside the cockpit to:

\begin{enumerate}
    \item \emph{Control an aircraft} - to check a responses to control input (UAS self-diagnostic).
    
    \item \emph{Check altitude} - to check and asses an altitude based on the estimated ground distance (UAS - barometric altimeter, ranging sensors).
    
    \item \emph{Navigate} - to steer aircraft for reaching long term goal, including position estimation. (UAS Navigation Module, GPS Module)
    
    \item \emph{Avoid other obstacles and intruders} - see and avoid procedures, following rules of the air in case of intruder avoidance. (UAS Detect And Avoid system).
\end{enumerate}

\begin{note}
    Each of VFR task has an equivalent task in IFR or UAS implementation. The system impact on aircraft airworthiness is interchangeable up to some degree.
\end{note}

\paragraph{See And Avoid:} The pilot has situations awareness of its surroundings and velocity. The \emph{horizontal/vertical} avoidance maneuvers are executed if necessary. 

\paragraph{Night VFR:} Some countries (ex. U. S.) allows flights under VFR when the sun is after horizon (astronomical night). The separation minimums are same. There is a \emph{clear sky requirement} (FAA) which disallows any clouds on higher flight levels.



\paragraph{Traffic Advisories:} The \emph{United States}, \emph{Australia}, and,  \emph{Canada} ATC provides the service of \emph{flight following}. A pilot can request the \emph{flight following} outside the \emph{B, C, D} class airspace, the ATC will communicate possible threats to pilot, the responsibility for safety is on the pilot.

\begin{note}
    The \emph{traffic advisories} are a weaker version of \emph{directives}; they can be used for RPAS systems communication.
\end{note}


\paragraph{Weather Separation:} VFR Weather Minimums – \emph{Visual Meteorological Conditions} (VMC).  Europe currently follows SERA (Standardised European Rules of the Air) rules, which are mostly the same as ICAO rules used throughout the world (local exceptions may apply). Current VFR Weather Minimums are:

\begin{enumerate}
    \item Altitude: at and above 10000 feet (3000 m), in every class of airspace – flight visibility 8000 m; 1500m horizontally from clouds, 1000 feet (300 m) vertically from clouds.
    
    \item Altitude: below 10000 feet (3000 m) and above 3000 feet (900 m) or above 1000 feet  (300 m) above terrain (whichever is higher) in every class of airspace – flight visibility 5000 m, 1500m horizontally from clouds, 1000 feet (300m) vertically from clouds.

    \item Altitude: at or below 3000 feet (900 m) or at or below 1000 feet (300m) above terrain in class A, B, C, D, E airspace (controlled) – flight visibility 5km and 1500 m horizontally from clouds 1000 feet (300 m) vertically from clouds.

    \item Altitude: at or below 3000 feet (900 m) or at or below 1000 feet (300 m) above terrain in class F and G airspace (uncontrolled) – flight visibility 5000 m, clear of cloud and with a sight of the surface. 
\end{enumerate}

\noindent There are exceptions from the last rule. ICAO rules allow for flights (at or below 3000 feet or at or below 1000 feet above terrain in F and G uncontrolled airspace) when flight visibility is no less than 1500m:
\begin{enumerate}
    \item at speeds that, in the prevailing visibility, will give adequate opportunity to observe other traffic or any obstacles in time to avoid a collision,
    
    \item in circumstances in which the probability of encounters with other traffic would normally be low, e.g., in areas of low volume traffic and for aerial work at low levels
\end{enumerate}

\noindent A similar exception (at or below 3000 feet or at or below 1000 feet above terrain) applies to helicopters, which can fly when flight visibility is less than 1500 m.

Refer home-country AIP\footnote{Czech republic AIP 1.1 document: \url{https://lis.rlp.cz/ais_data/aip/data/valid/e1-2.pdf}} (usually  or AIP ENR 1.1) for local restrictions.

\begin{note}
    The clouds are very dangerous for UAS because they impair sensors, causes freezing and damages the on-board electronic, the WMC can be used in \emph{weather safety handling definitions}.
\end{note}

\subsection{Instrumental Flight Rules}\label{sec:InstrumentalFlightRules}
\paragraph{Idea:} The key idea of \emph{Instrument Flight Rules} (IFR) is to provide additional surveillance resulting in better air traffic knowledge, weather and situation awareness. 

The situation is different in the case of UAS; single autonomous agent replaces the combination of human pilot decisions and surveillance provided information. The main challenge is to replicate the human pilot data fusion process leading to \emph{situation awareness} and later decision-making process leading to \emph{aircraft control}.

\paragraph{Instrument Flight Rules:} By definition \cite{icaoAnnex2}, \emph{Implementation of Visual Flight Rules under the weather or other conditions}. The main goal of IFR is to keep aircraft separated and with clearance.

The \emph{separation} of \emph{aircraft} is the main responsibility of \emph{Air Traffic Control} (ATC) which is providing IFR aircraft with guidance.

There is minimum equipment which needs to be carried by aircraft to be considered IFR airworthiness \cite{icao4444}. The minimal equipment to be carried for IFR flight in European Airspace (EuroControl) is given as follow:
\begin{enumerate}
    \item \emph{GPS} - mandatory for all flight levels in controlled airspace. 
    
    \item \emph{Transmitter} - the way to communicate with ATC/Ground. There can be digital transmitter equipment to receive automatic warnings from TCAS/ACAS systems, \emph{ATC directives}, and, notices to airman (NOTAMS).
    
    \item \emph{Transponder} - the broadcasting device is giving out aircraft position and additional mandatory information. The current plan is to make ADS-B In/Out mandatory for all air traffic attendants.
    
    \item \emph{Barometric Altimeter} - to measure precise AMSL altitude.
\end{enumerate}


\begin{note}
    Carried does not mean used. Most of the pilots in the UK believe that usage of GPS is illegal and they avoid it in small private flights in controlled airspace.
    
    The other popular practice is to turn off the \emph{transponder} because planes tend to hide their position and heading for safety reasons.
\end{note}

\paragraph{Required Navigation Performance:} The required navigation performance is depending on \emph{flight level} where the \emph{Flight Plan} (Mission) is executed. The \emph{planned trajectory deviation} is the key performance parameter \cite{icao4444}.