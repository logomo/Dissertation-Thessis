\newpage
\section{(R) UAS System Model}\label{s:uavMotionModel}
\noindent
This section strongly follows \cite{lee2011structure}.

\subsection{Continuous-time systems}\noindent

\noindent Consider a class of systems given by functions:
\begin{equation}
    \begin{aligned}
    State Evolution &: input(time)  \to state(state_0,time) \\
    input(time)&: [0,FinalTime] \to \R^p \\
    input(time)&\in \mathbb{R}^p , state(time) \in \mathbb{R}^n \\
    \end{aligned}
\end{equation}
Where $input(time)$ and $state(state_0,time)$ are a sets of continuous-time signals. These are often called continuous-time systems because they operate on continuous-time signals. 

Frequently, such systems can be defined by differential equations that relate the input signal to the output signal.

A prototypical description of a controlled (there is a control input signal) continuous-time system is:
\begin{multline}\label{eq:nonlinearsystem}
    \partial/\partial\text{t state}(time) =\\ f(time,state(time),input(time)), input(time) \in Inputs(time)
\end{multline}
Where $f:\mathbb{R}\times\mathbb{R}^n\times\mathbb{R}^p\to\mathbb{R}^n$ satisfies the conditions for existence and uniqueness of the ordinary differential equation and $u$ is our control \cite{butcher1987numerical}.

\subsection{Discrete-time Systems}
\noindent
\noindent Consider another class of systems given by functions
\begin{equation}\label{eq:Discretegenericuavmodel}
    \begin{aligned}
    State Evolution:& input(k)  \to state(k), \\
    k \in& \{0, t_s, 2.t_s, 3.t_s, \dots i.t_s\}, i \in \N^+\\
    input(k)\in& \mathbb{R}^p , state(k) \in \mathbb{R}^n\\
    \end{aligned}
\end{equation}
Where $input(k)$, $state(k)$ is a set of discrete-time signals. They can be represented by a function $f$ like $f:\{0, t_s, 2.t_s, 3.t_s, \dots i.t_s\} \to \R^n,  i \in \N^+$ where $t_s$ is sampling time and $i$ is discrete step \cite{shampine1997matlab}.

\subsection{Adversarial behaviour in continuous systems}
\noindent Consider a subclass of continuous time systems where are two sets of control signals $uas(time)$ and $adversary(time)$ which are accommodated in following system:
\begin{equation}\label{eq:Adversarial}
    \begin{aligned}
    \partial / \partial \text{t state}(time)&= f(t,state(time),uas(time),adversary(time)), \\
    uas(time) &\in UAS Inputs(time) \subset \R^u, \\
    adversary(time) &\in Adversary Inputs(time) \subset \R^v \\
    \end{aligned}
\end{equation}
This system representation is often used in definition of problem of pursuit/evasion problem. Krasovskii developed a solution approach to this problem in \cite{game1987}. A complex example of can be found in article \cite{game1988}.