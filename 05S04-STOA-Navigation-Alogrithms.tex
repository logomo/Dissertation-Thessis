\newpage
\section{Navigation Algorithms}\label{s:NavigationAlgorithms}
\paragraph{Idea:} The basic idea is to provide hierarchical \emph{navigation frame} with \emph{some optimal path search capabilities}. 


\paragraph{Space Assessment:}\emph{Probabilistic trajectory assessment} has been firstly proposed in \cite{kim2007uav} where trajectory was tracking and predicting \emph{safety properties} along. 

\emph{Game theory} viewpoint is firstly used in \cite{vidal2002probabilistic}. Probabilistic path planning using safety zones similar to cell classification of this work has been used in \cite{pfeiffer2005path}.

Probabilistic path search similar to our reach set representation using rapidly exploring path trees have been used in \cite{kothari2013probabilistically,blackmore2006probabilistic}. The relationship between clasic grid search and probabilistic lattice search have been established in \cite{lavalle2004relationship}. A probabilistic approach for trajectory estimation via reduced lattice search is known from 1986 from work of Gessel \cite{gessel1986probabilistic} lattice paths were enumerated via movement sequences and a similar technique is used in our reach set estimation method using movement automaton.  Pruning methods comparison and complexity can be found in \cite{esposito1997comparative}.

Overall concepts of probabilistic sets have been given by Hirota in \cite{hirota1981concepts}.  Free flight safety rating similar to our reachability concept has been presented in \cite{hoekstra2002designing}.

\paragraph{Standard Navigation:} The standard navigation is given as \emph{expected cost optimization problem} for \emph{future cost function} (eq. \ref{eq:costFunctionReachable}). The key concept of navigation algorithm was fully taken from \cite{gardi2018multi}. The decision was made based on navigation survey \cite{goerzen2010survey}. The \emph{descent} for landing is out of scope in this work, can be found in \cite{lim2018energy}. The navigation principle is roughly described in (sec. \ref{s:missionControlRun}).


\paragraph{Maze Solving Capabilities:} The \emph{maze solving capability} is usable in \emph{controlled airspace} where 2D maze solving algorithms are applicable. The notable implementation was for \emph{micro mouse robot} based on right-hand rule \cite{mishra2008maze}. Flood fill algorithm is partially usable for 3D environment \cite{elshamarka2012design}. The application of \emph{maze solving} was given in case study \cite{chatelais2014maze}.

\paragraph{Hybrid Automaton Path Planning:} A hybrid automaton path planning based on $A*$ algorithm was given by Richards in \cite{richards2004hybrid}. The key idea was to use \emph{hybrid automaton} (eq. \ref{eq:hybridAutomaton}) as a reference generator. This idea was taken and formulated as \emph{Movement Automaton Predictor mode}. 

The similar idea where \emph{potential fields} were used as the  \emph{intruder model} and path was re-planned  based on events is given in \cite{dong2011hybrid}.

\paragraph{Mode Switch:} The \emph{Mode Switch Control} idea has been presented in \cite{ryan2005mode}. There was the definition of behavioral switch between:

\begin{enumerate}
    \item \emph{Navigation Mode} - navigation control and behavior was used.
    
    \item \emph{Task-Specific Mode} - mode specific for tasks, authors were using modes for search and rescue. 
\end{enumerate}

This concept will be reused; the \emph{task-specific mode} will be \emph{Emergency Avoidance Mode} in Our case. The triggering events and switch conditions will be defined in (sec. \ref{s:missionControlRun}).

\paragraph{Used Concepts:} The \emph{Following concepts} were used in navigation loop:

\begin{enumerate}
    \item \emph{Standard navigation} took from \cite{gardi2018multi} minor implementation changes using offline optimization. The purpose of the navigation loop is to bring us closest to the waypoint if it is reachable. Navigation example (sec. \ref{s:testRuleMixed}).
    
    \item \emph{Maze solving capabilities} partially taken as secondary functionality based on \cite{elshamarka2012design}. The purpose is \emph{looping prevention}. The example was given in (sec. \ref{s:testMaze}).
    
    \item \emph{Mode switch} partially taken as the main feature from \cite{ryan2005mode}, the triggering events were identified and defined by author and can be found over (chapter \ref{ch:approach}).
\end{enumerate}

\paragraph{Shortcomings:} 

\begin{enumerate}
    \item \emph{Hierarchical calculation} - there is a need to calculate the \emph{avoidance trajectory} for incremental constraint applications. For example:
    \begin{enumerate}[a.]
        \item Calculate \emph{Minimal escape path} covering physical obstacles and intruders.
        \item Apply next level of constraints, like airspace restrictions and some virtual constraints. Then calculate path if exists, continue.
        \item Apply nice to have constraints, like non-lethal weather, recalculate the path.
    \end{enumerate}
    
    \item \emph{Source Reliability Evaluation} -  reliability evaluation is an empirical process usually done by hand. The result aggregation is not standardized. There can be multiple sources of the same rating, for example, visibility, which needs to be aggregated into one.  
    
    \item \emph{Ambiguous rating definition} - There are multiple definitions especially for \emph{Reachability rating} in works \cite{kothari2013probabilistically,blackmore2006probabilistic,gessel1986probabilistic}.
    
\end{enumerate}


