\cleardoublepage
\section{\secState{D}Cooperative Test Cases}\label{s:cooperativeTestCases}
    
    \noindent The \emph{main goal} of this section is to show the operational capabilities of \emph{approach} under \emph{UTM supervision}. The minimal UTM functionality set (sec. \ref{sec:UASTrafficManagement}) has been implemented, including \emph{position notifications mechanism, collision case calculation, resolution enforcement} components. 
    
    Test cases covers \emph{well clear breach prevention}, \emph{situation based avoidance}, and \emph{rules of the air enforcement}. 
    
    Coverage of \emph{near miss situations}, \emph{clash incidents} is given implicitly by \emph{safety} and \emph{body} margins (tab. \ref{tab:controlledAirspaceViolations}).
    
    \begin{enumerate}
        \item \emph{Rule based converging} (sec. \ref{s:testRuleConverging}) covers \emph{well clear breach} and \emph{converging rule of the air}, showing determinism and \emph{UTM resolution execution}.
        
        \item \emph{Rule based head on} (sec. \ref{s:testRuleHeadOn}) covers \emph{well clear breach} and \emph{head on rule of the air}, showing determinism and \emph{UTM resolution execution}.
        
        \item \emph{Rule based mixed head on with converging} (sec. \ref{s:testRuleMixed}) covers \emph{well clear breach} and \emph{head on and converging rules of the air}. The main focus is on \emph{virtual roundabout} concept, when multiple collision cases are clustered into one avoidance maneuver. 
        
        \item \emph{Rule based overtake} (sec. \ref{s:testRuleOvertake}) covers \emph{well clear breach} during \emph{overtake} by faster UAS.
    \end{enumerate}

