%% fcup-thesis.tex -- document template for PhD theses at FCUP
%%
%% Copyright (c) 2015 João Faria <joao.faria@astro.up.pt>
%%
%% This work may be distributed and/or modified under the conditions of
%% the LaTeX Project Public License, either version 1.3c of this license
%% or (at your option) any later version.
%% The latest version of this license is in
%%     http://www.latex-project.org/lppl.txt
%% and version 1.3c or later is part of all distributions of LaTeX
%% version 2005/12/01 or later.
%%
%% This work has the LPPL maintenance status "maintained".
%%
%% The Current Maintainer of this work is
%% João Faria <joao.faria@astro.up.pt>.
%%
%% This work consists of the files listed in the accompanying README.

%% SUMMARY OF FEATURES:
%%
%% All environments, commands, and options provided by the `ut-thesis'
%% class will be described below, at the point where they should appear
%% in the document.  See the file `ut-thesis.cls' for more details.
%%
%% To explicitly set the pagestyle of any blank page inserted with
%% \cleardoublepage, use one of \clearemptydoublepage,
%% \clearplaindoublepage, \clearthesisdoublepage, or
%% \clearstandarddoublepage (to use the style currently in effect).
%%
%% For single-spaced quotes or quotations, use the `longquote' and
%% `longquotation' environments.


%%%%%%%%%%%%         PREAMBLE         %%%%%%%%%%%%

%%  - Default settings format a final copy (single-sided, normal
%%    margins, one-and-a-half-spaced with single-spaced notes).
%%  - For a rough copy (double-sided, normal margins, double-spaced,
%%    with the word "DRAFT" printed at each corner of every page), use
%%    the `draft' option.
%%  - The default global line spacing can be changed with one of the
%%    options `singlespaced', `onehalfspaced', or `doublespaced'.
%%  - Footnotes and marginal notes are all single-spaced by default, but
%%    can be made to have the same spacing as the rest of the document
%%    by using the option `standardspacednotes'.
%%  - The size of the margins can be changed with one of the options:
%%     . `narrowmargins' (1 1/4" left, 3/4" others),
%%     . `normalmargins' (1 1/4" left, 1" others),
%%     . `widemargins' (1 1/4" all),
%%     . `extrawidemargins' (1 1/2" all).
%%  - The pagestyle of "cleared" pages (empty pages inserted in
%%    two-sided documents to put the next page on the right-hand side)
%%    can be set with one of the options `cleardoublepagestyleempty',
%%    `cleardoublepagestyleplain', or `cleardoublepagestylestandard'.
%%  - Any other standard option for the `report' document arclass can be
%%    used to override the default or draft settings (such as `10pt',
%%    `11pt', `12pt'), and standard LaTeX packages can be used to
%%    further customize the layout and/or formatting of the document.

%% *** Add any desired options. ***
%PDF
%\documentclass[a4paper,narrowmargins,12pt,oneside,draft,onehalfspaced,singlespacednotes]{fcup-thesis}
%\documentclass[a4paper,narrowmargins,12pt,oneside,onehalfspaced,singlespacednotes]{fcup-thesis}
%Print
%\documentclass[draft,a4paper,narrowmargins,12pt,twoside,openright,onehalfspaced,singlespacednotes]{fcup-thesis}
\documentclass[a4paper,narrowmargins,12pt,twoside,openright,onehalfspaced,singlespacednotes]{fcup-thesis}

%% *** Add \usepackage declarations here. ***
%% The standard packages `geometry' and `setspace' are already loaded by
%% `ut-thesis' -- see their documentation for details of the features
%% they provide.  In particular, you may use the \geometry command here
%% to adjust the margins if none of the ut-thesis options are suitable
%% (see the `geometry' package for details).  You may also use the
%% \setstretch command to set the line spacing to a value other than
%% single, one-and-a-half, or double spaced (see the `setspace' package
%% for details).
% Overfull statements
\pretolerance=150
\setlength{\emergencystretch}{3em}
% Overfull end
\usepackage[english]{babel}
\usepackage{lipsum}
\usepackage[utf8]{inputenc}


%%% Additional useful packages
%%% ----------------------------------------------------------------
\usepackage{array}
\usepackage{amsmath}  
\usepackage{amssymb}
\usepackage{amsfonts}
\DeclareFontFamily{OT1}{pzc}{}
\DeclareFontShape{OT1}{pzc}{m}{it}{<-> s * [0.900] pzcmi7t}{}
\DeclareMathAlphabet{\mathpzc}{OT1}{pzc}{m}{it}
\usepackage{amsthm}      
\usepackage[ruled,algochapter]{algorithm2e}
\usepackage{algorithmic}
\usepackage{bm}
\usepackage[mathscr]{euscript}
\usepackage{graphicx}       
\usepackage{psfrag}         
\usepackage{fancyvrb}    
\usepackage{float}
\usepackage{ltablex}
\usepackage[square,sort,comma,numbers]{natbib}        
\usepackage{bbding}         
\usepackage{dcolumn}        
\usepackage{booktabs} 
\usepackage{multirow}
\usepackage{paralist}     
\usepackage{ifdraft}  
\usepackage{indentfirst}    
\usepackage[nottoc,notlof,notlot]{tocbibind}
\usepackage{url}
\usepackage{tabularx}
\usepackage{subcaption}
\usepackage[unicode]{hyperref}
\usepackage{xcolor}

\hypersetup{pdftitle=LiDAR obstacle detection and avoidance, 
            pdfauthor=Alojz Gomola,
            colorlinks=false,
            urlcolor=blue,
            pdfstartview=FitH,
            pdfpagemode=UseOutlines,
            pdfnewwindow,
            breaklinks
          }
\usepackage{array}
\newcolumntype{L}[1]{>{\raggedright\let\newline\\\arraybackslash\hspace{0pt}}m{#1}}
\newcolumntype{C}[1]{>{\centering\let\newline\\\arraybackslash\hspace{0pt}}m{#1}}
\newcolumntype{R}[1]{>{\raggedleft\let\newline\\\arraybackslash\hspace{0pt}}m{#1}}         
\newcolumntype{B}{X}
\newcolumntype{S}[1]{>{\hsize=#1\textwidth}X}

\newcommand{\FIGDIR}{./Pics}    %%% directory containing figures
\newcommand{\twolinecellr}[2][r]{%
  \begin{tabular}[#1]{@{}r@{}}#2\end{tabular}}
\newcommand{\secState}[1]{
	\ifdraft{(#1) }{}
}
\theoremstyle{plain}
\newtheorem{theorem}{Theorem}
\newtheorem{lemma}[theorem]{Lemma}
\newtheorem{proposition}[theorem]{Proposition}

\theoremstyle{plain}
\newtheorem{definition}{Definition}
\newtheorem{problem}{Problem}
\newtheorem{example}{Example}
\newtheorem{assumption}{Assumption}

\theoremstyle{remark}
\newtheorem*{corollary}{Corollary}
\newtheorem*{note}{Note}




\newenvironment{dokaz}{
  \par\medskip\noindent
  \textit{Proof}.
}{
\newline
\rightline{\SquareCastShadowBottomRight}
}

\newenvironment{constraints}[1]{
  \par\medskip\noindent
  \textit{Constraints #1} \\
}{
\newline
\rightline{\SquareCastShadowBottomRight}
}


%\bibliographystyle{plainnat}     %% Author (year) style
\bibliographystyle{unsrt}        %% [number] style
\setcitestyle{square}

% Section  3.7 Challenge list
\newif\ifproblemchallenge   %# Build block for problem challenges
\problemchallengetrue       %# Show comments

\newcommand{\R}{\mathbb{R}}
\newcommand{\N}{\mathbb{N}}

\DeclareMathOperator{\pr}{\textsf{P}}
\DeclareMathOperator{\E}{\textsf{E}\,}
\DeclareMathOperator{\var}{\textrm{var}}
\DeclareMathOperator{\sd}{\textrm{sd}}


\newcommand{\T}[1]{#1^\top}        

\newcommand{\goto}{\rightarrow}
\newcommand{\gotop}{\stackrel{P}{\longrightarrow}}
\newcommand{\maon}[1]{o(n^{#1})}
\newcommand{\abs}[1]{\left|{#1}\right|}
\newcommand{\dint}{\int_0^\tau\!\!\int_0^\tau}
\newcommand{\isqr}[1]{\frac{1}{\sqrt{#1}}}
\newcommand{\norm}[1]{\left\lVert#1\right\rVert}


\newcommand{\pulrad}[1]{\raisebox{1.5ex}[0pt]{#1}}
\newcommand{\mc}[1]{\multicolumn{1}{c}{#1}}
\newcommand{\TBD}[1]{\color{red}\emph{--TBD:}#1\color{black}}

%%%%%%%%%%%%%%%%%%%%%%%%%%%%%%%%%%%%%%%%%%%%%%%%%%%%%%%%%%%%%%%%%%%%%%%%
%%                                                                    %%
%%                   ***   I M P O R T A N T   ***                    %%
%%                                                                    %%
%%  Fill in the following fields with the required information:       %%
%%   - \degree{...}       name of the degree obtained                 %%
%%   - \department{...}   name of the graduate department             %%
%%   - \gradyear{...}     year of graduation                          %%
%%   - \author{...}       name of the author                          %%
%%   - \title{...}        title of the thesis                         %%
%%%%%%%%%%%%%%%%%%%%%%%%%%%%%%%%%%%%%%%%%%%%%%%%%%%%%%%%%%%%%%%%%%%%%%%%

%% *** Change this example to appropriate values. ***
\degree{Doctor of Philosophy}
\department{Departamento de Matem\'{a}tica}
\gradyear{2019}
\author{Alojz Gomola}
\title{Obstacle Avoidance Framework based on Reach Sets}

%% *** NOTE ***
%% Put here all other formatting commands that belong in the preamble.
%% In particular, you should put all of your \newcommand's,
%% \newenvironment's, \newtheorem's, etc. (in other words, all the
%% global definitions that you will need throughout your thesis) in a
%% separate file and use "\input{filename}" to input it here.


%% *** Adjust the following settings as desired. ***

%% List only down to subsections in the table of contents;
%% 0=chapter, 1=section, 2=subsection, 3=subsubsection, etc.
\setcounter{tocdepth}{3}

%% Make each page fill up the entire page.
\flushbottom


%%%%%%%%%%%%      MAIN  DOCUMENT      %%%%%%%%%%%%

\begin{document}


%%%%%%%%%%%%%%%%%%%%%%%%%%%%%%%%%%%%%%%%%%%%%%%%%%%%%%%%%%%%%%%%%%%%%%%%
%%  Put your Chapters here; the easiest way to do this is to keep     %%
%%  each chapter in a separate file and `\include' all the files.     %%
%%  Each chapter file should start with "\chapter{ChapterName}".      %%
%%  Note that using `\include' instead of `\input' will make each     %%
%%  chapter start on a new page, and allow you to format only parts   %%
%%  of your thesis at a time by using `\includeonly'.                 %%
%%%%%%%%%%%%%%%%%%%%%%%%%%%%%%%%%%%%%%%%%%%%%%%%%%%%%%%%%%%%%%%%%%%%%%%%

%% *** Include chapter files here. ***

\setcounter{chapter}{6}
\setcounter{section}{7}
\setcounter{subsection}{3}

    %06-07 Avoidance Concept	
   
    	\newpage
\subsection{\secState{R}Computation Complexity}\label{sec:MCRcomputationalComplexity}
\paragraph{Introduction:}The \emph{Computation Complexity} one mission control run assessment is necessary to identify the strong and weak points of approach. Lets get through modules to assess notable calculations/algorithms complexity on high abstraction level.

\paragraph{Navigation Loop:} I the navigation loop, the \emph{waypoint reach condition} (eq. \ref{eq:waypointReachCondition}) is checked, this is unitary operation with worst complexity $\mathscr{O}(1)$. The selection process of the next \emph{Goal Waypoint} can get through all waypoints in the mission if they are all unreachable the complexity is $\mathscr{O}(|waypoints|)$.

The \emph{notable steps} complexity is following:
\begin{equation*}
    \begin{aligned}
        \texttt{Reach Condition: }& \mathscr{O}(1)\\
        \texttt{Select Next Waypoint: }&\mathscr{O}(|waypoints|)
    \end{aligned}
\end{equation*}

\paragraph{Data Fusion:} The \emph{data fusion} is all about \emph{threat selection}. 

If \emph{UAS} is in \emph{controlled airspace} it needs to iterate over received \emph{collision Cases} to select \emph{active ones}. The complexity of this step is linear, therefore boundary is given as $\mathscr{O} (|collision Cases|)$.

Thresholding \emph{Detected Obstacles} is done by simple comparison of \emph{LiDAR ray hits} in given $cell_{i,j,k}$ of \emph{Avoidance Grid}.

Any loading of \emph{threats} from \emph{information sources} is depending on clustering. The \emph{Airspace Clustering} is considered as static for our setup. Therefore the \emph{count of active airspace clusters} has main impact on complexity. The \emph{count of information sources} is static and not changing over mission time. Information sources usually implement \emph{Hash search function} with complexity $\mathscr{O}\ln|searched Item Set|$.

\noindent The \emph{computation complexity} boundaries for \emph{Data fusion} in  our setup are following:
\begin{equation*}
    \begin{aligned}
        \texttt{Select Active Collision Cases: }& \mathscr{O} (|collision Cases|)\\
        \texttt{Threshold Detected Obstacles: }& \mathscr{O}(|cells|)\\
        \texttt{Load Map Obstacles: }& \mathscr{O}(\ln|activeClusters|\times|information Sources|)\\
        \texttt{Load Hard Constraints: }& \mathscr{O}(\ln|activeClusters|\times|information Sources|)\\
        \texttt{Load Soft Constraints: }& \mathscr{O}(\ln|activeClusters|\times|information Sources|)
    \end{aligned}
\end{equation*}

\begin{note}
    The \emph{real-time clustering} is \emph{hard non-polynomial problem} \cite{kleinberg1998microeconomic}.  Usually all information sources and sensor have \emph{polynomial complexity} of processing. The \emph{controlled airspace clusters} are usually set for very long period of time. Therefore \emph{Obstacle Map}, \emph{Airspace Constraints}, and, \emph{Weather Constraints} can be considered as preprocessed
\end{note}

\newpage
\paragraph{Situation Assessment:} The \emph{Situation Assessment} is evaluating triggering events. The \emph{evaluation} is usually simple existence question without further calculations. The \emph{complexity} of \emph{event evaluation} for our case is $\mathscr{O}(1)$. There are 8 triggers. The count of \emph{triggers} needs to be accounted in complexity boundary:

\begin{equation*}
    \mathscr{O}(|triggers|\times event Evaluation Complexity)    
\end{equation*}

\begin{note} The \emph{trigger calculation complexity} needs to stay low, because the \emph{triggers} are verified every \emph{Mission Control Run}. The \emph{Avoidance Run} trigger frequency should be very low under normal conditions.  
\end{note}


\paragraph{Avoidance Run:} The \emph{Avoidance run} is most critical part of \emph{Mission Control Run}, because \emph{Avoidance Path} calculation. The \emph{Navigation Path} calculation is less complex (Rule engine is not accounted), therefore \emph{Emergency Avoidance Mode} is assumed. 

The \emph{threat insertion} is realized in 7\textsuperscript{th} to 10\textsuperscript{th} step. The first is \emph{Avoidance Grid} filled with \emph{Static Obstacles}. The \emph{Avoidance Grid} is designed to separate rotary  \emph{LiDAR} ray space into hit count even cells. Insertion of \emph{LiDAR} scan into \emph{Avoidance Grid} complexity depends on \emph{total cell count}. The \emph{upper boundary} for \emph{insert obstacles} is given like follow:

\begin{equation*}
    \texttt{Insert Obstacles: } \mathscr{O}(|cells|)
\end{equation*}

\noindent The \emph{intruders intersection model} type impact the insertion complexity. The \emph{linear intersection} (sec. \ref{s:linearIntersectionModel}) is going through maximum of \emph{layers count} cells. 

The \emph{body volume intersection model} (sec. \ref{s:bodyvolumeIntersection}) can check the \emph{simple intersection condition} over all \emph{Avoidance Grid} in worst case, therefore complexity for this check is bounded by \emph{count of cells}. 

The \emph{Maneuverability Uncertainty Intersection} (sec. \ref{s:uncertaintyIntersection}) can hit all cells in \emph{Avoidance Grid}. The calculation complexity boundary is exponential depending on \emph{horizontal/vertical} spread in $[rad]$. The \emph{intersection} implementation was done \emph{ad-hoc}. The impact of \emph{intersection application} is visible only when there is more than \emph{4} concurrence intruders (fig. \ref{fig:emergencyHeadOnMultipleComputationTime}).

\noindent The \emph{complexity boundary for \emph{intruder insertion}} is given like follow:

\begin{equation*}
    \texttt{Insert Intruders: }
    \mathscr{O}\left(\sum \begin{bmatrix}
        |linear Intersections| \times |layers|\\
        |body volume Intersections| \times  |cells|\\
        |cells|^{horizontal Spread \times vertical Spread}\\
    \end{bmatrix}\right)
\end{equation*}

\begin{note}
    The \emph{intruder intersection} is critical in \emph{non-controlled airspace}. The main complexity gain in \emph{controlled airspace} is from \emph{rule application}. Our \emph{rule complexity} is in worst case depending on \emph{Reach Set} node count and \emph{Active Collision Cases} count.
    
    \begin{equation*}
        \texttt{Apply Our Rules: } \mathscr{O}(|active Collision Cases| \times |nodes|)
    \end{equation*}
\end{note}

\newpage\noindent For \emph{Hard/Soft Constraints} The algorithm used for intersection polygons was selected based on study \citep{bentley1979algorithms}, the selected algorithm  \emph{Shamos-Hoey} \cite{shamos1976geometric}. The \emph{calculation complexity} boundary is given like follow:

\begin{multline*}
    \texttt{Hard Constraints Intersection:}\\ \mathscr{O}(|cells|\times|hard Constraints| \times \max |constraint Points|^2)
\end{multline*}
\begin{multline*}
    \texttt{Soft Constraints Intersection:}\\ \mathscr{O}(|cells|\times|soft Constraints| \times \max |constraint Points|^2)
\end{multline*}

\noindent Each \emph{threat} category application in \emph{Mission Control Run} is done after \emph{each intersection} in 7\textsuperscript{th} to 10\textsuperscript{th} step. All ratings (tab. \ref{tab:defuzificationRatings}) expect $Reachibility(cell_{ij,k})$ and $Reachibility(Trajectory)$  are calculated. The \emph{calculation complexity} boundary for one \emph{reachibility rating} is $\mathscr{O}(1)$. (eq. \ref{eq:trajectoryReachibility}, \ref{eq:cellReachibility}). The \emph{Recalculate Reachibility} operation applied $4\times$ have maximal \emph{complexity} boundary given as follow:

\begin{equation*}
    \texttt{Recalculate Reachibility: } \mathscr{O}(4 \times (|nodes| + |cells|))
\end{equation*}

\noindent Each time at the end of in 7\textsuperscript{th} to 10\textsuperscript{th} step the \emph{Avoidance Path is Selected}. The \emph{Worst Case} (expected) scenario is to \emph{select} four paths for each \emph{treath} application. The algorithm for \emph{best path selection} (alg. \ref{alg:FindBestPathAvoidanceGrid}) iterates over all \emph{cells} in avoidance grid and over all \emph{trajectories} passing through that cell. The complexity boundary for \emph{path selection} is given as follow:

\begin{equation*}
    \texttt{Select Path: } \mathscr{O}\left(4 \times \left(|cells|+\frac{|nodes|}{|cells|}\right)\right)
\end{equation*}


\paragraph{Conclusion:}  Overall approach complexity is \emph{low}. If proper \emph{Information Sources} with efficient clustering and \emph{intersection models for intruders} are used, the approach will stay within \emph{non-polynomial complexity}. 
The average load time for \emph{testing scenarios} is summarized in (tab. \ref{tab:computationLoadStatistics}).

\begin{note}
    The calculation of \emph{Reach Set} is eliminated by pre-calculation for \emph{state range} \cite{gomola2017obstacle}.
\end{note}

\setcounter{chapter}{7}
\setcounter{section}{5}
\setcounter{subsection}{2}   
 
 \newpage
 \subsection{\secState{R}Computation Footprint}\label{s:ComputaitonFootprint}

\noindent The \emph{computation footprint} is summarized in computation  load (tab. \ref{tab:computationLoadStatistics}). The \emph{computation load} (eq. \ref{eq:computationLoad}) was calculated for each \emph{time-frame} in scenarios. There is summary of \emph{minimal, maximal, average} and \emph{median} values.

The \emph{computational load} never exceed more than $55.95\%$ in case of  \emph{emergency Head On} (eq. \ref{eq:computationFeasibilityCriterion}), which means that \emph{every path} was calculated on time.


\begin{table}[H]
    \centering
    \begin{tabular}{r||r|r|r|r}
    
    \multirow{2}{*}{Scenario} & \multicolumn{4}{c}{Computation load} \\ \cline{2-5} 
    & min. & max. & avg. & med. \\ \hline\hline
    Building avoidance (fig. \ref{fig:buildingAvoidanceComputationTime})		
                            &   2.20\% &  27.40\% &  12.11\%  & 13.20\% \\\hline
    Slalom (fig. \ref{fig:slalomComputationTime})				    
                            &  12.20\% &  30.50\% &  21.42\%  & 21.50\% \\\hline
    Maze (fig. \ref{fig:mazeComputationTime})					
                            &  24.90\% &  46.10\% &  31.51\%  & 30.80\% \\\hline
    Storm (fig. \ref{fig:stormComputationTime})					
                            &   2.60\% &  26.90\% &  11.57\%  & 13.90\% \\\hline\hline
    
    Emergency Converging (fig. \ref{fig:emergencyConvergingComputationTime})    
                            &   2.75\% &  16.50\% &   5.84\%  &  4.95\% \\\hline
    Emergency Head On (fig. \ref{fig:emergencyHeadOnComputationTime})	 	
                            &	3.90\% &  55.95\% &  13.19\%  &  6.90\% \\\hline
    Emergency Multiple (fig. \ref{fig:emergencyHeadOnMultipleComputationTime})	 	
                            &	5.90\% &  52.35\% &  12.77\%  &  8.56\% \\\hline\hline
    
    Rule-based Converging (fig. \ref{fig:ruleBasedCConvergingComputationTime})	
                            &   3.60\% &  13.50\% &   7.32\%  &  5.97\% \\\hline
    Rule-based Head on (fig. \ref{fig:ruleBasedHeadOnComputationTime})		
                            &   4.65\% &  41.60\% &  13.64\%  &  9.30\% \\\hline
    Rule-based Multiple	(fig. \ref{fig:ruleBasedMultipleComputationTime})	
                            &   4.37\% &  23.30\% &  11.96\%  & 10.93\% \\\hline
    Rule-based Overtake	(fig. \ref{fig:ruleBasedOvertakeComputationTime})	
                            &   3.85\% &  13.40\% &   7.62\%  &  6.70\% 
    
    \end{tabular}
    \caption{\emph{Computation load statistics} for all test cases.}
    \label{tab:computationLoadStatistics}
\end{table}

\noindent \emph{Following observations can be made:}

\begin{enumerate}
    \item \emph{Building avoidance}, \emph{Slalom}, and \emph{Maze} scenarios - the computation load is increasing with the \emph{amount of static obstacles}. The \emph{average load} for \emph{Emergency avoidance mode} in \emph{clustered environment} is $31.51\%$ (Maze).
    
    \item \emph{Storm scenario} - the overall \emph{computation load} is very low due the \emph{moving constraint implementation} (sec. \ref{s:MovingVirtualConstraints}).
    
    \item\emph{Emergency Converging/Head On/Multiple} scenarios - the \emph{overall computation load} is quite high due the ineffective \emph{body volume intersection} (sec. \ref{s:bodyvolumeIntersection}) implementation.
    
    \item \emph{Rule-based Converging/Head On/Multiple} scenarios - the \emph{median computational} load is low, because of the linear \emph{rule implementation} (sec. \ref{sec:ruleImplementation})
    
    \item \emph{Rule-based Overtake} - the \emph{average computation load} is very low, because only \emph{divergence/convergence} (rule. \ref{tab:ruleOvertakeDefinition}) waypoints are calculated and UAS stays in \emph{navigation mode}.
\end{enumerate}

%% This adds a line for the Bibliography in the Table of Contents.
\addcontentsline{toc}{chapter}{Bibliography}
%% *** Set the bibliography style. ***
%% (change according to your preference/requirements)
%\bibliographystyle{plain}
%% *** Set the bibliography file. ***
%% ("thesis.bib" by default; change as needed)
\bibliography{thesis}

%% *** NOTE ***
%% If you don't use bibliography files, comment out the previous line
%% and use \begin{thebibliography}...\end{thebibliography}.  (In that
%% case, you should probably put the bibliography in a separate file and
%% `\include' or `\input' it here).

\end{document}
