%% fcup-thesis.tex -- document template for PhD theses at FCUP
%%
%% Copyright (c) 2015 João Faria <joao.faria@astro.up.pt>
%%
%% This work may be distributed and/or modified under the conditions of
%% the LaTeX Project Public License, either version 1.3c of this license
%% or (at your option) any later version.
%% The latest version of this license is in
%%     http://www.latex-project.org/lppl.txt
%% and version 1.3c or later is part of all distributions of LaTeX
%% version 2005/12/01 or later.
%%
%% This work has the LPPL maintenance status "maintained".
%%
%% The Current Maintainer of this work is
%% João Faria <joao.faria@astro.up.pt>.
%%
%% This work consists of the files listed in the accompanying README.

%% SUMMARY OF FEATURES:
%%
%% All environments, commands, and options provided by the `ut-thesis'
%% class will be described below, at the point where they should appear
%% in the document.  See the file `ut-thesis.cls' for more details.
%%
%% To explicitly set the pagestyle of any blank page inserted with
%% \cleardoublepage, use one of \clearemptydoublepage,
%% \clearplaindoublepage, \clearthesisdoublepage, or
%% \clearstandarddoublepage (to use the style currently in effect).
%%
%% For single-spaced quotes or quotations, use the `longquote' and
%% `longquotation' environments.


%%%%%%%%%%%%         PREAMBLE         %%%%%%%%%%%%

%%  - Default settings format a final copy (single-sided, normal
%%    margins, one-and-a-half-spaced with single-spaced notes).
%%  - For a rough copy (double-sided, normal margins, double-spaced,
%%    with the word "DRAFT" printed at each corner of every page), use
%%    the `draft' option.
%%  - The default global line spacing can be changed with one of the
%%    options `singlespaced', `onehalfspaced', or `doublespaced'.
%%  - Footnotes and marginal notes are all single-spaced by default, but
%%    can be made to have the same spacing as the rest of the document
%%    by using the option `standardspacednotes'.
%%  - The size of the margins can be changed with one of the options:
%%     . `narrowmargins' (1 1/4" left, 3/4" others),
%%     . `normalmargins' (1 1/4" left, 1" others),
%%     . `widemargins' (1 1/4" all),
%%     . `extrawidemargins' (1 1/2" all).
%%  - The pagestyle of "cleared" pages (empty pages inserted in
%%    two-sided documents to put the next page on the right-hand side)
%%    can be set with one of the options `cleardoublepagestyleempty',
%%    `cleardoublepagestyleplain', or `cleardoublepagestylestandard'.
%%  - Any other standard option for the `report' document arclass can be
%%    used to override the default or draft settings (such as `10pt',
%%    `11pt', `12pt'), and standard LaTeX packages can be used to
%%    further customize the layout and/or formatting of the document.

%% *** Add any desired options. ***
%PDF
%\documentclass[a4paper,narrowmargins,12pt,oneside,draft,onehalfspaced,singlespacednotes]{fcup-thesis}
%\documentclass[a4paper,narrowmargins,12pt,oneside,onehalfspaced,singlespacednotes]{fcup-thesis}
%Print
%\documentclass[draft,a4paper,narrowmargins,12pt,twoside,openright,onehalfspaced,singlespacednotes]{fcup-thesis}
\documentclass[a4paper,narrowmargins,12pt,twoside,openright,onehalfspaced,singlespacednotes]{fcup-thesis}

%% *** Add \usepackage declarations here. ***
%% The standard packages `geometry' and `setspace' are already loaded by
%% `ut-thesis' -- see their documentation for details of the features
%% they provide.  In particular, you may use the \geometry command here
%% to adjust the margins if none of the ut-thesis options are suitable
%% (see the `geometry' package for details).  You may also use the
%% \setstretch command to set the line spacing to a value other than
%% single, one-and-a-half, or double spaced (see the `setspace' package
%% for details).
% Overfull statements
\pretolerance=150
\setlength{\emergencystretch}{3em}
% Overfull end
\usepackage[english]{babel}
\usepackage{lipsum}
\usepackage[utf8]{inputenc}


%%% Additional useful packages
%%% ----------------------------------------------------------------
\usepackage{array}
\usepackage{amsmath}  
\usepackage{amssymb}
\usepackage{amsfonts}
\DeclareFontFamily{OT1}{pzc}{}
\DeclareFontShape{OT1}{pzc}{m}{it}{<-> s * [0.900] pzcmi7t}{}
\DeclareMathAlphabet{\mathpzc}{OT1}{pzc}{m}{it}
\usepackage{amsthm}      
\usepackage[ruled,algochapter]{algorithm2e}
\usepackage{algorithmic}
\usepackage{bm}
\usepackage[mathscr]{euscript}
\usepackage{graphicx}       
\usepackage{psfrag}         
\usepackage{fancyvrb}    
\usepackage{float}
\usepackage{ltablex}
\usepackage[square,sort,comma,numbers]{natbib}        
\usepackage{bbding}         
\usepackage{dcolumn}        
\usepackage{booktabs} 
\usepackage{multirow}
\usepackage{paralist}     
\usepackage{ifdraft}  
\usepackage{indentfirst}    
\usepackage[nottoc,notlof,notlot]{tocbibind}
\usepackage{url}
\usepackage{tabularx}
\usepackage{subcaption}
\usepackage[unicode]{hyperref}
\usepackage{xcolor}

\hypersetup{pdftitle=LiDAR obstacle detection and avoidance, 
            pdfauthor=Alojz Gomola,
            colorlinks=false,
            urlcolor=blue,
            pdfstartview=FitH,
            pdfpagemode=UseOutlines,
            pdfnewwindow,
            breaklinks
          }
\usepackage{array}
\newcolumntype{L}[1]{>{\raggedright\let\newline\\\arraybackslash\hspace{0pt}}m{#1}}
\newcolumntype{C}[1]{>{\centering\let\newline\\\arraybackslash\hspace{0pt}}m{#1}}
\newcolumntype{R}[1]{>{\raggedleft\let\newline\\\arraybackslash\hspace{0pt}}m{#1}}         
\newcolumntype{B}{X}
\newcolumntype{S}[1]{>{\hsize=#1\textwidth}X}

\newcommand{\FIGDIR}{./Pics}    %%% directory containing figures
\newcommand{\twolinecellr}[2][r]{%
  \begin{tabular}[#1]{@{}r@{}}#2\end{tabular}}
\newcommand{\secState}[1]{
	\ifdraft{(#1) }{}
}
\theoremstyle{plain}
\newtheorem{theorem}{Theorem}
\newtheorem{lemma}[theorem]{Lemma}
\newtheorem{proposition}[theorem]{Proposition}

\theoremstyle{plain}
\newtheorem{definition}{Definition}
\newtheorem{problem}{Problem}
\newtheorem{example}{Example}
\newtheorem{assumption}{Assumption}

\theoremstyle{remark}
\newtheorem*{corollary}{Corollary}
\newtheorem*{note}{Note}




\newenvironment{dokaz}{
  \par\medskip\noindent
  \textit{Proof}.
}{
\newline
\rightline{\SquareCastShadowBottomRight}
}

\newenvironment{constraints}[1]{
  \par\medskip\noindent
  \textit{Constraints #1} \\
}{
\newline
\rightline{\SquareCastShadowBottomRight}
}


%\bibliographystyle{plainnat}     %% Author (year) style
\bibliographystyle{unsrt}        %% [number] style
\setcitestyle{square}

% Section  3.7 Challenge list
\newif\ifproblemchallenge   %# Build block for problem challenges
\problemchallengetrue       %# Show comments

\newcommand{\R}{\mathbb{R}}
\newcommand{\N}{\mathbb{N}}

\DeclareMathOperator{\pr}{\textsf{P}}
\DeclareMathOperator{\E}{\textsf{E}\,}
\DeclareMathOperator{\var}{\textrm{var}}
\DeclareMathOperator{\sd}{\textrm{sd}}


\newcommand{\T}[1]{#1^\top}        

\newcommand{\goto}{\rightarrow}
\newcommand{\gotop}{\stackrel{P}{\longrightarrow}}
\newcommand{\maon}[1]{o(n^{#1})}
\newcommand{\abs}[1]{\left|{#1}\right|}
\newcommand{\dint}{\int_0^\tau\!\!\int_0^\tau}
\newcommand{\isqr}[1]{\frac{1}{\sqrt{#1}}}
\newcommand{\norm}[1]{\left\lVert#1\right\rVert}


\newcommand{\pulrad}[1]{\raisebox{1.5ex}[0pt]{#1}}
\newcommand{\mc}[1]{\multicolumn{1}{c}{#1}}
\newcommand{\TBD}[1]{\color{red}\emph{--TBD:}#1\color{black}}

%%%%%%%%%%%%%%%%%%%%%%%%%%%%%%%%%%%%%%%%%%%%%%%%%%%%%%%%%%%%%%%%%%%%%%%%
%%                                                                    %%
%%                   ***   I M P O R T A N T   ***                    %%
%%                                                                    %%
%%  Fill in the following fields with the required information:       %%
%%   - \degree{...}       name of the degree obtained                 %%
%%   - \department{...}   name of the graduate department             %%
%%   - \gradyear{...}     year of graduation                          %%
%%   - \author{...}       name of the author                          %%
%%   - \title{...}        title of the thesis                         %%
%%%%%%%%%%%%%%%%%%%%%%%%%%%%%%%%%%%%%%%%%%%%%%%%%%%%%%%%%%%%%%%%%%%%%%%%

%% *** Change this example to appropriate values. ***
\degree{Doctor of Philosophy}
\department{Departamento de Matem\'{a}tica}
\gradyear{2019}
\author{Alojz Gomola}
\title{Obstacle Avoidance Framework based on Reach Sets}

%% *** NOTE ***
%% Put here all other formatting commands that belong in the preamble.
%% In particular, you should put all of your \newcommand's,
%% \newenvironment's, \newtheorem's, etc. (in other words, all the
%% global definitions that you will need throughout your thesis) in a
%% separate file and use "\input{filename}" to input it here.


%% *** Adjust the following settings as desired. ***

%% List only down to subsections in the table of contents;
%% 0=chapter, 1=section, 2=subsection, 3=subsubsection, etc.
\setcounter{tocdepth}{3}

%% Make each page fill up the entire page.
\flushbottom


%%%%%%%%%%%%      MAIN  DOCUMENT      %%%%%%%%%%%%

\begin{document}



%%%%%%%%%%%%%%%%%%%%%%%%%%%%%%%%%%%%%%%%%%%%%%%%%%%%%%%%%%%%%%%%%%%%%%%%
%%  Put your Chapters here; the easiest way to do this is to keep     %%
%%  each chapter in a separate file and `\include' all the files.     %%
%%  Each chapter file should start with "\chapter{ChapterName}".      %%
%%  Note that using `\include' instead of `\input' will make each     %%
%%  chapter start on a new page, and allow you to format only parts   %%
%%  of your thesis at a time by using `\includeonly'.                 %%
%%%%%%%%%%%%%%%%%%%%%%%%%%%%%%%%%%%%%%%%%%%%%%%%%%%%%%%%%%%%%%%%%%%%%%%%

%% *** Include chapter files here. ***
\setcounter{chapter}{3}

%04-Problem Statement
    %%\chapter{Problem Statement}


\section{Basic Definitions}\label{s:basicDefinitions}
    \begin{definition}{\emph{Obstacle} $o$} is given as any inaccessible set of points $o\subset\R^3$ it can be represented as polygon, space boundary, thick point-cloud, separation plane.
    \end{definition}
    
    \begin{definition}{Information source $s$}\label{def:informationSource} is source of obstacle information to relative vehicle position and orientation $\vec{p}\in\R^6$ with time $t$ mapping capability. The information source can be sensory reading, obstacle map,weather forecast zones, and, \emph{Air Traffic Management} restrictions.
    
    Information sources are aggregated in \emph{information source set} $\mathscr{S}$.
    
    \emph{Information source} $s\in\mathscr{S}$ provides \emph{obstacle feed} $\mathscr{O}_s$ depending on vehicle position and orientation $\vec{p}$ and time $t$. Overall obstacle set is aggregation of obstacle feeds from \emph{information source set}
    \begin{equation}\label{eq:obstacleSet}
        \mathscr{O}(\vec{p},t)=\bigcup_{s\in\mathscr{S}} \mathscr{O}_s(\vec{p},t), \quad \mathscr{O}_s(\vec{p},t)=s(\vec{p},t)
    \end{equation}
    \end{definition}
    
    \begin{definition}{\emph{Field of vision}}\label{def:} (FOV) of vehicle or sensor is given as subset $\mathscr{F}\subset\R^3$ with position and orientation $\vec{p}\in\R^6$, for time $t$. Field of the vision is denoted $\mathscr{F}(\vec{p},t,s)$, where $s\in\mathscr{S}$ is sensor from sensory field, or $\mathscr{F}(t)$, in case $\vec{p}$ is implicit from vehicle position $\vec{p}=\vec{x}(t)\to\R^6$ and sensor is implicit.
    \end{definition}
    
    \begin{note} Field of the vision for vehicle is combination of all field of the visions from information sources $\mathscr{S}$:
    \begin{equation}\label{eq:fieldOfVisonFixedTime}
        \mathscr{F}(t)=\bigcap_{s\in\mathscr{S}} \mathscr{F}(\vec{p},t,s)
    \end{equation}
    \end{note}
    
    \begin{definition}{Space segmentation}\label{def:spaceSegmentation} of any space $\mathscr{A}=\R^k, k\in\N^+$ is given as separation of space to exclusive sets $c_\mathbb{I}\subset\mathscr{A}$ called cells, with index vector $\mathbb{I}\subset\mathbb{Z}^l$ serving as unique identifier.The segmentation holds following properties:
    \begin{enumerate}
        \item Cells are covering segmented space $\bigcup_{\forall i\in \mathbb{I}\subset\mathbb{Z}^l} c_i = \mathscr{A}$
        \item Each cell $c_\mathbb{I}\in\mathscr{A}$ is nonempty set.
        \item Each point $\vec{p}\in\R^k$ in one cell $\vec{p}\in c_i$ shares same property value $P(\vec{p},\dots)$ with any point $\vec{r}\neq\vec{p}$ 
    \end{enumerate}
    
    \end{definition}
    
    \begin{definition}{Obstacle space $\mathscr{O}$} is segmentation (def. \ref{def:spaceSegmentation}) of obstacle set $\mathscr{O}(\vec{p},t)$ (\ref{eq:obstacleSet}).
    \end{definition}
    
    \begin{definition}{Free space $\mathscr{F}$} is space where exists direct visibility between observation point $\vec{p}\in\R^m$ and points in segmented space (def. \ref{def:spaceSegmentation}).
    \end{definition}
    
    \begin{definition}{Uncertain space $\mathscr{U}$} contains space segments (def. \ref{def:spaceSegmentation}) which are not in obstacle or free space.
    \begin{equation}
        \mathscr{U}=(\mathscr{A}-\mathscr{F})-\mathscr{O}
    \end{equation}
    \end{definition}
    
    \begin{definition}{\emph{Intruder}(moving obstacle) $i$} is identified via information source $s\in\mathscr{S}$ with minimal set of properties bounded to time of detection $t$:
    \begin{enumerate}
        \item position $\vec{p}\in\R^k$,
        \item velocity $\vec{v}\in\R^k$,
        \item intruder body radius (vehicle class) $r\in\R^+$,
        \item uncertainty spread $\Theta\in\R^{k-1}$.
    \end{enumerate}
    
    \end{definition}

\section{Avoidance set}\label{s:AvoidanceSet}
    \begin{definition}{Avoidance set $\mathscr{A}(t,\vec{x}_0,\vec{u}_0,\mathscr{S})$} is defined for vehicle system:
    \begin{equation}
        \dot{\vec{x}}(t)=\vec{f}(t,\vec{x},\vec{u})
    \end{equation}
    With system state $\vec{x}(t)\in\R^n$, control signal $\vec{u}(t)\in\R^k$. Regardless of system class the system entry point is given as $\vec{e}$:
    \begin{equation}
        \vec{e}= [\vec{x}_0,\vec{u}_0,t_0,d_0]\in\R^{n+k+1+1}
    \end{equation}
    Initial state $x_0$ at time $t_0$ is accompanied with initial input signal $u(t)$, initial time $t_0$, and initial decision $d_0\in\N$. Note that structure can store multiple system trajectories due the discrete decision dimension $d\in\N$.
    
    For time period $\tau\in[t_0,t]$ vehicle fly along the trajectory $\vec{x}(\tau)$ based on taken decisions $D=\{d_0,d_1,\dots,d_i\},i\in\N$ in decision times $T_D=\{t_{d1},t_{d2},\dots,t_{di}\}, t_{di}\le t$. Field of the vision $\mathscr{F}(\tau)$ along trajectory $\vec{x}(\tau)$ is given as:
    \begin{equation}
        \mathscr{F}(\tau) = \bigcup_{t_d\in T_D} \mathscr{F}(\vec{x}(t_d)\to\vec{p},t_d,\forall s\in\mathscr{S}) \quad (\ref{eq:fieldOfVisonFixedTime})
    \end{equation}
    \end{definition}
    Aggregated field of the vision $\mathscr{F}(\tau)$ is separated into obstacle space $\mathscr{O}(\tau)$ uncertain space $\mathscr{U}(\tau)$ and free space $\mathscr{F}(\tau)$. Then for any fixed time $t_{fix}\in[t_0,\infty)$ all system trajectories $x(t)$ starting within entry point $\vec{e}$ holds following conditions:
    \begin{enumerate}
        \item Each projected point $\vec{p}=\vec{x}(t_{fix})\to\mathscr{F}(\tau)$ belongs to free space $\mathscr{F}(\tau)$.
        \item Each projected point $\vec{p}$ has minimal distance from obstacle space $\mathscr{O}(t_{fix})$ or $\mathscr{U}(t_{fix})$ greater than safety margin $s_m$.
    \end{enumerate}
    
    \begin{note} Note following attributes of avoidance set ($\mathscr{A}(t,\vec{x}_0,\vec{u}_0,\mathscr{S})$):
    \begin{enumerate}
        \item \emph{Avoidance set respects vehicle dynamics} - avoidance set contains trajectories which are feasible for vehicle dynamics and control.
        \item \emph{Avoidance set stores multiple system trajectories} due the added decision dimension.
        \item \emph{Avoidance set respects all sources of obstacles} due the incorporation of information source set $\mathscr{S}$ which is depending on vehicle position, orientation and time.
        \item \emph{Avoidance set definition} is not giving away construction method it only defines relationship to free obstacle and uncertain spaces within aggregated field of vision.
        \item \emph{Avoidance set contains all previously executed trajectory segments and future possible to execute trajectories}
    \end{enumerate}
    \end{note}
    
    \begin{definition}{Safety of Avoidance set $\mathscr{A}(t,\vec{x}_0,\vec{u}_0,\mathscr{S})$} is guaranteed by its property where each projected point $\vec{p}$ has minimal distance from obstacle space $\mathscr{O}(t_{fix})$ or $\mathscr{U}(t_{fix})$ greater than safety margin $s_m$.
    \end{definition}
    
    \begin{definition}{Reachibility of Avoidance set $\mathscr{A}(t,\vec{x}_0,\vec{u}_0,\mathscr{S})$} is given by respect to the vehicle dynamics. The reach set for time period $\tau\in[t_0,t]$ and control strategy $u(\tau)\in U(\tau)$ is given as:
    \begin{equation}\label{eq:ReachibilityofAvoidanceSet}
        \mathscr{R}(\vec{x_0}:t_0,\tau)=\left\{\vec{x}(\tau):\vec{x}(\tau)=\vec{f}(\vec{x}(\tau),\vec{u}(\tau)),u(\tau)\in U(\tau)\right\}
    \end{equation}
    Each trajectory $\vec{x}(t)$ extracted from avoidance set $\mathscr{A}(t,\vec{x}_0,\vec{u}_0,\mathscr{S})$ based on decision chain $D$ is then included in reach set $\mathscr{R}$ (\ref{eq:ReachibilityofAvoidanceSet}).
    \end{definition}
    
    \begin{definition}{Weak invariance of Avoidance set $\mathscr{A}(t,\vec{x}_0,\vec{u}_0,\mathscr{S})$} is invariant for Each trajectory $\vec{x}(t)$ based on decision chain $D$.
    \end{definition}
    
    \begin{definition}{Obstacle collision time} is depending on vehicle position and vehicle velocity. Obstacle collision time can be infinite in case the vehicle is avoiding an obstacle at current path. Only obstacles where collision time is finite are possessing threat of collision with vehicle. 
    \end{definition}
    
    \begin{definition}{Avoidance execution time period} is given by subsequent decision times from avoidance set $\mathscr{A}(t,\vec{x}_0,\vec{u}_0,\mathscr{S})$ as:
    \begin{equation}
        \tau = (t_{d,k},t{d,k+1}]
    \end{equation}
    The duration of avoidance execution time period is bounded by field of the vision range $r(\mathscr{F}(t))$ and mean vehicle velocity until it reach Field of vision boundary. 
    \end{definition}
    
    \begin{definition}{Decision time $t_d \in T_D$} is given by avoidance execution time boundary at maximum and by control signal granularity $u$(t) at minimum, decision time can be forced by event related to information source, rule application or other type of event. 
    \end{definition}
    
\section{Efficient Avoidance}\label{s:efficienctAvoidance}
    \emph{To be done in addition of textual description from meeting:}
    \begin{itemize}
        \item Definition - pragmatic and formally corectt definition of the weather:
	    \item wind
	    \item disability
	    \item placeholder definition
	    \item rewrite  definition
	    \item Rules will be encoded in hybrid automaton
    \end{itemize}
    

\section{Task Oriented Avoidance}\label{s:taskOrientedAvoidance}
    \noindent text from text file

\section{Problem Statement}
    \paragraph{Situation:} A vehicle equipped with LiDAR and ADS-B receiver is flying close to the surface, with the mission given as ordered set of reachable waypoints. Obstacle map, containing prior knowledge of the vehicle is precise to some extent. 

    \paragraph{Sensing}: The vehicle is sensing static obstacles in Field of Vision (FOV) with bounded distance, horizontal range, and vertical range. Intruders are discovered and accounted in detection space bounded by ADS-B receiver range. 

    \paragraph{Space representation}
        Avoidance grid is partitioning combined sensing spaces into cells with defined boundaries. The cell can achieve following states depending on contained features:
        \begin{itemize}
            \item{Obstacle} – contains the feature of terrain or intruder intersection which intersects projected vehicle trajectory.
            \item{Occupied} – contains the feature of terrain or intruder intersection not leading to the collision.
            \item{Free} – does not contain any feature of terrain or intersection
        \end{itemize}
    \paragraph{Trajectories:}
        The set emergency maneuvers are represented as the set of trajectories originating in actual vehicle position, the trajectories can be in relation to Avoidance Grid:
        \begin{itemize}
            \item Unbounded – assume that some trajectories will leave FOV, there are no guarantees that emergency maneuver will lead to safe escape from obstacle. 
            \item Contained – assume that all trajectories are contained within FOV and there exist decision point when the vehicle can decide return to the safe area, with the implementation of conservative avoidance strategy it is possible to guarantee safe movement of the vehicle. 
        \end{itemize}

    \paragraph{Problems to be addressed}
    If emergency maneuvers are contained avoidance strategy can be defined to solve following problems:
    \begin{itemize}
        \item Safe exploration of uncharted area – guarantee vehicle safety during uncharted area exploration. 
        \item Terrain following – low altitude flight safely following terrain  to hide vehicles presence
    \end{itemize}
    
    \emph{To be done here:}
    \begin{itemize}
        \item Formal requirements
        \item Definitions
        \item Sub-problems
    \end{itemize}
    


 - NEVER USE
    \cleardoublepage
\chapter{Problem Statement}\label{c:problemStatement}

\noindent A \emph{UAS} equipped with several types of sensors is tasked to fly several types of \emph{Missions} in a 3-dimensional space. There is a \emph{Terrain} map, an \emph{Object map}, and a \emph{Weather} forecast for target region that are  known a priori. The map of the \emph{Terrain} may not be up to date, and \emph{Uncharted obstacles} may affect flight safety. The \emph{UAS} has to comply with a set of \emph{Flight Rules} specifying \emph{Flight Constraints}. The performance of the \emph{UAS}, including sensor performance, is affected by the \emph{Weather}.


Several difficulties must be faced.  First, the design space of is large and complex. Second, \emph{Trajectory calculation} Third, \emph{Navigation}.
    \section{Basic Definitions}\label{s:basicDefinitions}
\noindent A few definitions are needed.

\subsection{World}\label{s:World} 
    \noindent The world of interest consists of:

    \emph{Space} has a Global Reference Frame with three axes $X^+,Y^+,Z^+$. 
    \begin{equation}\label{eq:SpaceDefinition}
        Space \subseteq \R^3:\text{main axes:} X^+,Y^+,Z^+\text{, center} [x_0,y_0,z_0] 
    \end{equation}

    \emph{Object} (\ref{eq:ObjectDefinition}) is a generic subset of \emph{Space} which has \emph{Boundary}.
    \begin{equation}\label{eq:ObjectDefinition}
        Object \subset Space: \forall point \in Object, point \text{ is solid}
    \end{equation}

    A nonlinear first order state-space model gives the dynamic model of the UAS:
    \begin{equation}\label{eq:vehicleModelAbstract}
        \dot{x} = f(t,x,u)
    \end{equation}

    \noindent Where $x\in\R^{6+n}, n\in\N^{0+}$ is \emph{system state} containing minimal information about UAS position $[x,y,z]$ and orientation $[\theta,\varphi,\rho]$ (roll, pitch yaw), and $u(t)$ is \emph{control signal} belonging to $R^k,k\in\N^+ $, bounded by control set $u(t)\in U$.
    
    The map of the terrain is given by \emph{TerrainMap} mapping $(x,y)$ to the terrain elevation
    \begin{equation}\label{eq:TerrainMap}
        TerrainMap: (x,y) \rightarrow z 
    \end{equation}
   
    \noindent\emph{Weather} (\ref{eq:weatherProjection}) can impact the performance of the \emph{UAS}. \emph{Wind} can decrease flying capabilities and maneuverability, \emph{visibility} can impair the performance of sensors such as LiDAR and cameras, \emph{humidity} can be a serious danger to electronic equipment, and in combination low \emph{temperature} it can cause icing on the UAS body, and \emph{air pressure} can impact ranging sensor and flight performance. 
    \begin{equation}\label{eq:weatherProjection}
        Weather:\left(position\in Space\right) \times time \to
        \left[
        \begin{aligned}
            \text{wind } & \begin{aligned}&(w_x,w_y,w_z)\\&[m/s,m/s,m/s]\end{aligned}\\
            \text{visibility }& v\,[m]\\
            \text{humidity }& h\,[0-100 \%] \\
            \text{air pressure }& a\,[hPa]\\
            \text{temperature: }& \tau\,[C]\\
        \end{aligned}
        \right]
    \end{equation}

\subsection{Mission}\label{s:mission}
    \noindent\emph{Mission} (\ref{eq:missionAbstractSet}) which UAS should fly is given as a set of \emph{ordered, feasible regarding UAS dynamic (\ref{eq:vehicleModelAbstract}), waypoints} in a subspace of $Space$ (\ref{eq:SpaceDefinition}).
    
    \begin{equation}\label{eq:missionAbstractSet}
        Mission = \left\{
        \begin{aligned}
            &waypoint_1, waypoint_2, \dots,waypoint_m:\\
            &\forall_{i=1\dots m} waypoint_i \in  Space
        \end{aligned}
        \right\}, \quad m\in\N^{+},m\ge2
    \end{equation}

    \emph{Waypoint Passing} (\ref{eq:waypointPassingFunction}) function maps system (\ref{eq:vehicleModelAbstract}) trajectory \emph{projection in Space} and \emph{Mission} waypoints (\ref{eq:missionAbstractSet}) to a vector of \emph{passing times}.
    
    \begin{equation}\label{eq:waypointPassingFunction}
        WaypointPassing:TrajectoryProjection \times Mission \to Time^m
    \end{equation}

    \begin{note}
        The \emph{Mission} (\ref{eq:missionAbstractSet}) is considered as completed if and only if $\forall$ waypoints are reached and in the given order (check the output of \ref{eq:waypointPassingFunction}).
    \end{note}



\subsection{Flight Constraints}\label{s:FlightConstraints}

    \noindent Flight constraints arise from aviation rules and ATM/UTM. 

    There are $H$ sets of hard flight constraints expressed as subsets of \emph{Space}:

    \begin{equation}\label{eq:HFlightConstraints}
        HFlightConstraints=\left\{HFlightConstraint_1, \dots, HFlightConstraint_H \right\}
    \end{equation}

    The union of all \emph{$HFlightConstraint_i$} is the set \emph{$HFlightConstraint$}. \emph{Example:} turning radius of UAS, climb rate, etc. 
    
    There are $S$ sets of soft flight constraints expressed as subsets of \emph{Space}:

    \begin{equation}\label{eq:SFlightConstraints}
        SFlightConstraints=\left\{SFlightConstraint_1, \dots, SFlightConstraint_S \right\}
    \end{equation}

    The union of all \emph{$SFlightConstraint_i$} is the set \emph{SFlightConstraint}. \emph{Example:} sharp maneuvers, restricted flight areas etc.

    The \emph{Weather} may impose either soft or hard flight constraints.



\subsection{Partition of Space}\label{s:partitionOfSpace}

    \noindent In what follows it is convenient at each time $t$ to partition the \emph{Space} into four disjoint subsets \emph{Free(t)}, \emph{Restricted(t)}, \emph{Occupied(t)} and \emph{Uncertain(t)}.

    There is a \emph{SpaceClassification}  function (\ref{eq:spaceCassificationFunction})
    
    \begin{equation}\label{eq:spaceCassificationFunction}
        SpaceClassification: y \in \emph{Space} \mapsto s \in \{Free, Restricted, Occupied, Uncertain \}
    \end{equation}

    \begin{note}
        \emph{SpaceClassification} (\ref{eq:spaceCassificationFunction}) is a \emph{total} function mapping each element of \emph{Space}. Other followup \emph{classifications} are considered as \emph{total} functions. 
    \end{note}

    There is set of \emph{Space} Hard constraints, where one \emph{HardConstraint} (\ref{eq:hardSpaceConstraints}) is given as:

    \begin{equation}\label{eq:hardSpaceConstraints}
        \emph{HardConstraint} = \{point \in \emph{Space}: \emph{SpaceClassification}(point) = Occupied \}
    \end{equation}

    There is set of \emph{Space} Soft constraints, where one \emph{SoftConstraint} (\ref{eq:softSpaceConstraints}) is given as:

    \begin{equation}\label{eq:softSpaceConstraints}
        \emph{SoftConstraint} = \{point \in \emph{Space}: \emph{SpaceClassification}(point) = Restricted \}
    \end{equation}

\subsection{Information Source}\label{s:informationSources}
    \begin{definition}{Information source} \label{def:InformationSurceDefinition}         
        (\ref{eq:InformationSurceDefinition}) represents a priori information for each \emph{point} in \emph{Space} at time prior mission execution mapping it into one of distinctive categories \emph{Free, Occupied, Restriced}, and \emph{Uncertain}.
        
        \begin{equation}\label{eq:InformationSurceDefinition}
            InformationSource = SpaceClassification(PriorTime)
        \end{equation}
    \end{definition}

    \begin{definition}{Landmark}\label{def:Landmark} 
        is a partition of \emph{Space} (\ref{eq:SpaceDefinition}) which is notable in the context of \emph{society} or \emph{law} given properties. Landmark can be notable buildings, notable natural structures or crucial infrastructure. Landmark is part of terrain map, but it`s special status can induce additional properties.
    \end{definition}

    \noindent For obstacle avoidance problem following \emph{Information sources} are available:
    
    \begin{enumerate}
        \item\emph{Terrain map} - a map of terrain with notable landmarks.
        
        \item\emph{Object map} - a map of notable structures with \emph{protection zones} considered as \emph{occupied} or \emph{restricted} space.
        
        \item\emph{Fly zones restriction map} - a map of restricted flight areas considered as \emph{restricted} constraints.
    \end{enumerate}

\subsection{Single Observation by Single Sensor Classification} \label{singleObservationSingle sensor}
    \noindent The\emph{UAS} is equipped with single \emph{Sensor}, which returns \emph{Space} classification for given \emph{observation position} and observation time.


    Observation (\ref{eq:observationClassification}) at given UAS \emph{position}, \emph{time}, and for given \emph{sensor} sorts points from \emph{sensor reading} into following distinguish sets (\ref{eq:observationClassification}):
    
    \begin{enumerate}
        \item $Free_O(position,sensor,time)$ - observable space by sensor and considered as $Free$ by sensor reading,
        \item $Occupied_O(position,sensor,time)$ -  observable space by sensor and considered as $Occupied$ by sensor reading,
        \item $Uncertain_O(position,sensor,time)$ - all other points.
    \end{enumerate}
    
    \begin{equation}\label{eq:observationClassification}
        Observation(position,sensor,time)\to
        \begin{cases}
            Free_O(position,sensor,time)\\
            Occupied_O(position,sensor,time)\\
            Uncertain_O(position,sensor,time)\\
        \end{cases}
    \end{equation}


\subsection{Multiple Observations by Single Sensor Classification}\label{multipleObservationsBySingleSensor}
    \noindent Let Let add pairs of \emph{observation position} and \emph{observation time} in manner $\{(position_1,t_1),$ $(position_2,t_2),$ $\dots(position_k,t_k)\}, k\in\N^+$ that point position is independent and time is ordered in fashion $t_1 < t_2 < \dots < t_k$.
 
 
    \emph{Free space} from multiple sensor \emph{reading} over multiple \emph{positions} is inclusive space because we are obtaining additional information regarding space reachability by sensor reading independent on sensor space and orientation limitation. Therefore the union of single instances of observations is used to represent \emph{Combined free space} $Free_O(sensor)$ (\ref{eq:freeObservableSpaceForOneSensor}).

    \begin{equation}\label{eq:freeObservableSpaceForOneSensor}
        Free_O(sensor)= \bigcup_{i=\{1,\dots,k\}}Free_O(sensor,position_i,time_i)
    \end{equation}
 
    \emph{Occupied space} (\ref{eq:occupiedObservableSpaceForOneSensor}) from multiple sensor \emph{reading} over multiple \emph{positions} is inclusive space, because of increased information; therefore it has similar handling to \emph{Free space}.
    \begin{equation}\label{eq:occupiedObservableSpaceForOneSensor}
        Occupied_O(sensor)= \bigcup_{i=\{1,\dots,k\}}Occupied_O(sensor,position_i,time_i)
    \end{equation}
 
    \emph{Uncertain space} (\ref{eq:uncertainObservableSpaceForOneSensor}) from multiple sensor \emph{reading} over multiple \emph{positions} is exclusive space, because of decrease in uncertainty. 
 
    \begin{equation}\label{eq:uncertainObservableSpaceForOneSensor}
         Uncertain_O(sensor)= \bigcap_{i=\{1,\dots,k\}}Uncertain_O(sensor,position_i,time_i)
    \end{equation}

\subsection{Sensor Fusion}\label{s:SensorFusionDefinition}
    \noindent The observation at fixed time $t_{fix}$ can be made by multiple sensors $sensor_1,$ $sensor_2,$ $\dots,$ $sensor_k,$ $k\in\N^k$, for $k$ sensors execute observations $Observation_1$ $(sensor_1,$ $position_1,$ $t_{fix}),$ $Observation_2$ $(sensor_2,$ $position_2,$ $t_{fix}),$ $\dots,$ $Observation_k($ $sensor_k,$ $position_k,$ $t_{fix})$. 

    \emph{Sensor Fusion} (\ref{eq:SensorFusionFunction}) function with \emph{data fusion parameters} is introduced which combines \emph{Observations} for each \emph{sensor} and \emph{Weather}, then  uniquely maps each point into four distinguish sets: \emph{$Free(t_{fix})$, $Occupied(t_{fix})$, $Restricted(t_{fix})$, and $Uncertain(t_{fix})$} (special case of (\ref{eq:spaceCassificationFunction})).

    \begin{equation}\label{eq:SensorFusionFunction}
        SensorFusion:
        \left[
        \begin{aligned}
            &Observation_1 (sensor_1, position_1, t_{fix})\times\\
            &Observation_2 (sensor_2, position_2, t_{fix})\times\\
            &\times\dots\times\\
            &Observation_k (sensor_k, position_k, t_{fix})\times\\
            &Weather(\dots)\\
            &\textit{fixed time }t_{fix}\times\\
            &sensorFusionParameters\\
        \end{aligned}
        \right]
        \to
        \begin{cases}
            Free(t_{fix})\\
            Occupied(t_{fix})\\
            Restricted(t_{fix})\\
            Uncertain(t_{fix})
        \end{cases}
    \end{equation}

\subsection{Data Fusion}\label{s:dataFusionDefinition}
    \noindent Multiple \emph{Information Sources}: \emph{InformationSource}$_1$, \emph{InformationSource}$_2$, $\dots$, \emph{InformationSource}$_l$, $l\in\N$, where $l$ is the count of information sources needs to be fused with \emph{Sensor Fusion} function (\ref{eq:SensorFusionFunction}) outcome at \emph{fixed time} $t_{fix}$.

    \emph{Data fusion} function (\ref{eq:DataFusionFunction}) combines the classification from various \emph{Information Sources} with classification from \emph{Sensor Fusion} function (\ref{eq:SensorFusionFunction}) for UAS position at \emph{fixed time} $t_{fix}$, under given \emph{Data Fusion Parameters}. 

    \begin{equation}\label{eq:DataFusionFunction}
        DataFusion:
        \left[
        \begin{aligned}
            &InformationSource_1 \times\\
            &InformationSource_2 \times\\
            &\times\dots\times\\
            &InformationSource_l \times\\
            &SensorFusion(\dots)\times\\
            &Weather(\dots)\\
            &position\times\\
            &\textit{fixed time }t_{fix}\times\\
            &dataFusionParameters
        \end{aligned}
        \right]
        \to 
        \begin{cases}
            Free(t_{fix})\\
            Occupied(t_{fix})\\
            Restricted(t_{fix})\\
            Uncertain(t_{fix})
        \end{cases}
    \end{equation}

    Each $point$ in $Space$ is uniquely classified into one of sets\emph{Free$(t_{fix})$, Occupied$(t_{fix})$, Restricted$(t_{fix})$, Uncertain($t_{fix}$)} (special case (\ref{eq:spaceCassificationFunction})).
    
    
    \begin{note}
    Moreover \emph{Data Fusion} function is covering the case of \emph{Multiple information sources, combined with multiple sensor readings over multiple times}, including \emph{Weather} as \emph{sensor impact factor} and \emph{information source}. 
    \end{note}

\subsection{Known World}\label{s:KnownWorld}
   \noindent \emph{Known world} (\ref{eq:knownWorldSet}) for some \emph{fixed time}$t_{fix}$ is given as the joint set of points which belongs to one of \emph{Data Fusion} (\ref{eq:DataFusionFunction}) output sets $Free(t_{fix})$ or $Occupied(t_{fix})$ or $Restricted(t_{fix})$. The \emph{Known World} is compact set with existing boundary.
    
    \begin{equation}\label{eq:knownWorldSet}
        KnownWorld(t_{fix})= Free(t_{fix}) \cup Occupied(t_{fix}) \cup Restricted(t_{fix})
    \end{equation}



\subsection{Safety Margin}\label{s:SafetyMarginDefinition}
    \noindent Let say that \emph{mission} was executed in \emph{time} interval $t\in [missionStart,missionEnd]$ in \emph{Known world} (\ref{eq:knownWorldSet}). For every $position$, extracted from \emph{UAS model state} $x(t)$, keeps distance from any point in $Occupied(t)$  with greater or equal to $safetyMargin$ $(s_m)$ (\ref{eq:safetyMarginAbstract}).  
    
    \begin{multline}\label{eq:safetyMarginAbstract}
        \forall t\in [missionStart,missionEnd]:\\distance(x(t),Occupied(t),t) \ge safetyMargin
    \end{multline}


    \section{Basic Obstacle Avoidance Problem}\label{s:BasicProblemDefinition}
    \noindent Given:

    \begin{enumerate}
        \item Initial system state $state_0$, for UAS model (eq. \ref{eq:vehicleModelAbstract}).
        
        \item Mission (eq. \ref{eq:missionAbstractSet}) to be executed.
        
        \item Space (eq. \ref{eq:SpaceDefinition}), with existing objects (eq. \ref{eq:ObjectDefinition}).
        
        \item Sensor system $\{sensor_1, sensor_2,\dots,sensor_k\}$ with \emph{sensor fusion} function (eq. \ref{eq:SensorFusionFunction}).
        
        \item Weather information (eq. \ref{eq:weatherProjection}) during the flight is available.
        
        \item Information sources $informationSource_1$, $informationSource_2$, $\dots$, $informationSource_l$ containing hard (eq. \ref{eq:hardSpaceConstraints}) and soft space constraints (eq. \ref{eq:softSpaceConstraints}),with existing data fusion function (eq. \ref{eq:DataFusionFunction}).
        
        \item \emph{Hard} (eq. \ref{eq:HFlightConstraints}) and \emph{Soft} (eq. \ref{eq:SFlightConstraints}) flight constraints given by ATM and rules of the air. 
    \end{enumerate}

    \noindent Generate \emph{Control command chain to} complete a mission (eq. \ref{eq:missionAbstractSet}) with the satisfaction of following conditions:
    
    \begin{enumerate}
        \item Waypoint passing function condition (eq. \ref{eq:waypointPassingFunction}).
        
        \item Set safety margin $s_m$ to $Occupied$  space in $KnownWorld(time)$ (eq. \ref{eq:knownWorldSet}) is not breached at any time (eq. \ref{eq:safetyMarginAbstract}).
    \end{enumerate}

    \newpage
\section{Initial Assumptions} \label{s:initialAssumptions}
    \noindent\emph{Initial assumptions} are the following:

    \begin{assumption}
        {Filtered sensor readings are available}\label{ass:filteredSensors}.\\
        \emph{SensorObservation} (\ref{eq:observationClassification}) for a given \emph{position}, \emph{time} returns classification of \emph{Space} which is corresponding with the real situation.
    \end{assumption}
    
    \begin{assumption}
        {There are no moving obstacles}\label{ass:noMovingObstacles}.\\
        The initial \emph{Space Classification Function} (\ref{eq:spaceCassificationFunction}) is static for all observation times $t \in (-\infty,\infty)$. Moreover, there are no \emph{intruders} or \emph{adversaries} present. 
    \end{assumption}

    \begin{assumption}
        {The movement takes place in the unrestricted airspace.}\label{ass:openAir}
    \end{assumption}

    \begin{assumption}
        {The mission consists of a set of reachable waypoints}\label{ass:reachableWaypoints}.\\
        For specific \emph{UAS system} (\ref{eq:vehicleModelAbstract}) and  \emph{Mission} (\ref{eq:missionAbstractSet}), there exists a control which satisfies \emph{Waypoint passing} (\ref{eq:waypointPassingFunction}) criterion and \emph{SafetyMargin} (\ref{eq:safetyMarginAbstract}) condition.
    \end{assumption}
    
    \begin{assumption}
        {The UAS is moving with constant velocity}\label{ass:constantVelocity}.\\
        For given \emph{UAS system} (\ref{eq:vehicleModelAbstract}) there is a subset of state $velocity(t)\subset x(t)$ which contains velocity parameters. Then there exist transformation function $LinearVelocity(\circ)$ which maps $velocity(t)$  to \emph{linear velocity} $\in\R^1$. For time $t$ in \emph{missionStart} and \emph{missionEnd} in \emph{Mission} (\ref{eq:missionAbstractSet}) constraint (\ref{eq:constantVelocityAssmunption}) with some $constantVelocity \in \R^+$ holds.

        \begin{equation}\label{eq:constantVelocityAssmunption}
            \forall t \in \left[\begin{aligned}&missionStart,\\&missionEnd\end{aligned}\right]: LinearVelocity(velocity(t))=constantVelocity
        \end{equation}    
    \end{assumption}

    \begin{note}
        \emph{Initial assumptions} \ref{ass:filteredSensors}., \ref{ass:noMovingObstacles}., \ref{ass:openAir}., \ref{ass:reachableWaypoints}, and \ref{ass:constantVelocity}. will be relaxed in \emph{Incremental problem definition}.
    \end{note}
	\section{\secState{R}Incremental Problem Definition}\label{s:IncrementalProblemDefinition}

\noindent This section contains \emph{incremental problem definition} as increments of (sec. \ref{s:BasicProblemDefinition}). Each problem contains definition and references to addressed issues.

\begin{problem}{Basic Avoidance}\label{pro:knownWorldEvolution} (sec. \ref{s:BasicProblemDefinition}) is to navigate through \emph{KnownWorld} under the assumption that every \emph{waypoint} in \emph{Mission} is reachable. The \emph{KnownWorld} is fed through \emph{SensorFusion} function which is joining \emph{LiDAR} scanning into \emph{Free(t), Occupied(t),} and, \emph{Unknown(t)} sets in \emph{discrete scan times} $t$. 

    \begin{equation}\label{eq:basicProblemDefinition}
        \begin{aligned}
            KnownWorld:&= SensorFusion(t)\forall point\in KnownWorld(t)\\
                       &=Free(t) \cup Occupied(t) \cup Unknown(t)\\
            Mission:&= \forall waypoint\in Mission \text{ are reachable}\\
            Sensors:&= \{LiDAR\}\\
            SensorFusion:&= \{\text{Clasificaiton function}\}\\
            HFlightConstraints:&=\{\text{vehicle dynamic}\}\\
        \end{aligned}
    \end{equation}
    
    
    \noindent \emph{Challenges for problem  \ref{pro:knownWorldEvolution}. :}
    \begin{enumerate}
        \item \emph{Navigation Loop Implementation} (sec. \ref{s:missionControlRun}).
        
        \item \emph{Avoidance Loop Implementation}  (sec. \ref{s:aviudabceGridRun}).
    \end{enumerate}   
\end{problem}


\begin{problem}{Intruder Problem}\label{pro:intruderDetection}
    in addition to \emph{Known world evolution} (pr.\ref{pro:knownWorldEvolution}) the \emph{ADS-B} sensor is introduced into \emph{Sensors} array, this imposes \emph{HardConstraint} of \emph{Flight corridor} for detected intruders impacting the evolution of \emph{Free}(t), and \emph{Occupied}(t) sets significantly.
    
    \begin{equation}\label{eq:intruderDetectionProblemdefinition}
        \begin{aligned}
            KnownWorld:&= SensorFusion(t)\forall point\in KnownWorld(t)\\
                       &=Free(t) \cup Occupied(t) \cup Unknown(t)\\
            Mission:&= \forall waypoint\in Mission \text{ are reachable}\\
            Sensors:&= \{LiDAR,ADS-B\}\\
            SensorFusion:&= \{\text{Advanced joint sets}\}\\
            HFlightConstraints:&=\{\text{vehicle dynamic}\}\\
            HardConstraints:&=\{\text{intruder corridors}\}\\
        \end{aligned}
    \end{equation}
    

    \noindent \emph{Challenges for problem  \ref{pro:intruderDetection}. :}
    \begin{enumerate}
        \item \emph{Intruder Intersection Models} (minimal operation requirements achieved):
        \begin{enumerate}[a.]
            \item \emph{Linear Intersection Model} (sec. \ref{s:linearIntersectionModel}).
            \item \emph{Body-volume intersection} (sec. \ref{s:bodyvolumeIntersection}).
            \item \emph{Maneuverability uncertainty intersection} (sec. \ref{s:uncertaintyIntersection}).
        \end{enumerate}
        
        \item \emph{Flight Corridors} (sec. \ref{s:virtualConstraints}).
    \end{enumerate}
    
    \noindent \emph{Relaxed Assumption: } \ref{ass:noMovingObstacles}., the \emph{UAS} encountering cooperative and non-cooperative intruders.
\end{problem}




\begin{problem}{Static restrictions}\label{pro:staticRestrictions},
    in addition to the \emph{Intruder} problem (pr. \ref{pro:intruderDetection}) the \emph{InformationSources} are expanded by \emph{static restriction} sources: 
    \begin{enumerate}
        \item \emph{ObstacleMap} - a database containing notable landmarks, buildings, structures, with well-defined \emph{protection zones}.
        \item \emph{FlightRestrictions} - a database containing ATM flight restrictions in non-segregated airspace for UAS relevant airspace categories. 
    \end{enumerate}
    \noindent This change impacts \emph{DataFusion} by splitting \emph{Free(t)} set into \emph{Free(t)} and \emph{Restricted(t)} disjoint sets. Also \emph{SoftConstraints} are introduced which contain restricted areas from relevant information sources.  
    
    \begin{equation}\label{eq:staticRestrictionsProblemDefinition}
        \begin{aligned}
            KnownWorld:&= DataFusion(t)\forall point\in KnownWorld(t)\\
                       &=Free(t) \cup Occupied(t) \cup Unknown(t)\cup Restricted(t)\\
            Mission:&= \forall waypoint\in Mission \text{ are reachable}\\
            Sensors:&= \{LiDAR,ADS-B\}\\
            SensorFusion:&= \{\text{Advanced joint sets}\}\\
            InformationSources:&=\{Terrain Map,Obstacle Map,Flight Restriction\}\\
            DataFusion:&= \{\text{Advanced data fusion}\}\\
            HFlightConstraints:&=\{\text{vehicle dynamic}\}\\
            HardConstraints:&=\{\text{intruder corridors,terrain,obstacles}\}\\
            Softconstraints:&=\{\text{protection zones}\}
        \end{aligned}
    \end{equation}
    

    \noindent \emph{Challenges for problem  \ref{pro:staticRestrictions}. :}
    \begin{enumerate}
        \item \emph{Obstacle Map} (sec. \ref{s:mapObstacles}).
        \item \emph{Visibility Rating Concept} (fig. \ref{fig:hindranceImpactOnVisibility}).
        \item \emph{Static Constraints} (sec. \ref{s:virtualConstraints}).
    \end{enumerate}

    \noindent \emph{Relaxed Assumption: } \ref{ass:openAir}., the \emph{UAS} is moving in \emph{restricted space} now.
\end{problem}

\begin{problem}{Dynamic restrictions}\label{pro:dynamicRestrictions}
    in addition to \emph{Static restrictions} (pr. \ref{pro:staticRestrictions}), the \emph{Weather} as information source is introduced. \emph{Soft constraints} are extended by medium level dangerous zones from weather map. \emph{Hard constraints} are expanded by protection zones where the \emph{weather} conditions are harsh. Overall \emph{Weather} constraints are dynamic and changing position and shape over mission time. Modern weather systems can provide streamline overview of weather situation. 
    
    \begin{equation}\label{eq:dynamicRestrictionsProblemDefinition}
        \begin{aligned}
            KnownWorld:&= DataFusion(t)\forall point\in KnownWorld(t)\\
                       &=Free(t) \cup Occupied(t) \cup Unknown(t) \cup Restricted(t)\\
            Mission:&= \forall waypoint\in Mission \text{ are reachable}\\
            Sensors:&= \{LiDAR,ADS-B\}\\
            SensorFusion:&= \{\text{Advanced joint sets}\}\\
            InformationSources:&=\{Terrain Map,Obstacle Database,\\
                               &\quad Flight Restriction,Weather\}\\
            DataFusion:&= \{Advanced data fusion\}\\
            HFlightConstraints:&=\{\text{vehicle dynamic}\}\\
            HardConstraints:&=\{\text{intruder corridors,terrain,obstacles, protection zones}\}\\
            Softconstraints:&=\{\text{protection zones}\}
        \end{aligned}
    \end{equation}
    
    \noindent \emph{Challenges for problem  \ref{pro:dynamicRestrictions}. :}
    \begin{enumerate}
        \item\emph{Moving Constraints including Weather} (sec. \ref{s:MovingVirtualConstraints}).
        \item\emph{Weather Avoidance Case} (sec. \ref{sec:weatherCase}).
    \end{enumerate}
\end{problem}


\begin{problem}{Rules of the air}\label{pro:rulesOfTheAir}, 
    in addition to \emph{Dynamic restrictions} (pr. \ref{pro:dynamicRestrictions}), \emph{Rules of the air} framework introduction, inducing new \emph{SFlightConstraints} including air-spaces and rules of air impact on control mechanism.
    \begin{equation}\label{eq:rulesOfTheAir}
        \begin{aligned}
            KnownWorld:&= DataFusion(t)\forall point\in KnownWorld(t)\\
                       &=Free(t) \cup Occupied(t) \cup Unknown(t) \cup Restricted(t)\\
            Mission:&= \forall waypoint\in Mission \text{ are reachable}\\
            Sensors:&= \{LiDAR,ADS-B\}\\
            SensorFusion:&= \{\text{Advanced joint sets}\}\\
            InformationSources:&=\{Terrain Map,Obstacle Database,\\
                               &\quad Flight Restriction,Weather\}\\
            DataFusion:&= \{Advanced data fusion\}\\
            HFlightConstraints:&=\{\text{vehicle dynamic}\}\\
            SFlightConstratins:&=\{\text{airspaces, rules of the air}\}\\
            HardConstraints:&=\{\text{intruder corridors,terrain,obstacles, protection zones}\}\\
            Softconstraints:&=\{\text{protection zones}\}
        \end{aligned}
    \end{equation}
    
    \noindent \emph{Challenges for problem  \ref{pro:rulesOfTheAir}. :}
    \begin{enumerate}
        \item UTM Implementation (sec. \ref{sec:UASTrafficManagement}).
        \item Rule Engine for UAS (sec. \ref{s:RuleEngineArchitecture}).
        \item Rule Implementation (sec. \ref{sec:ruleImplementation}).
    \end{enumerate}
    
    \noindent \emph{Relaxed Assumption: } \ref{ass:constantVelocity}., the \emph{UAS} is required to move with different velocity during \emph{Overtake maneuver}.
\end{problem}

\begin{note}
    The assumptions \ref{ass:filteredSensors}. for \emph{filtered sensor output} and \ref{ass:reachableWaypoints}. \emph{Reachable waypoints} hold for all \emph{problem increments.}
\end{note}
    \section{\secState{D}Avoidance Requirements}\label{s:AvoidanceRequirements}

\paragraph{SAA systems} have the following conflicting performance criteria:
\begin{enumerate}
    \item \emph{Energy efficiency} - minimize energy consumption and flight time.
    \item \emph{Trajectory tracking} - stick to the proclaimed trajectory in a mission plan.
    \item \emph{Safety} - avoid harm sources during the mission execution.
\end{enumerate}

\begin{note}
    \emph{Energy efficiency} and \emph{Trajectory Tracking} an optional criteria, while \emph{the safety} is mandatory.
\end{note}

\subsection{\secState{D}Energy Efficiency}\label{s:EnergyEfficiency}
\paragraph{Energy efficiency} can be measured by \emph{cost function} (eq. \ref{eq:consFunctionMeta}), consisting from the \emph{cost of flown trajectory} (eq. \ref{eq:costFunctionExecuted}) and the \emph{expected reach cost} (eq. \ref{eq:costFunctionReachable}) portions. There are optimalizaiton techniques based on \emph{Reach sets} \cite{kurzhanski2001dynamic}. The inputs for the \emph{cost function} are:
\begin{enumerate}
    \item \emph{Time} - current mission time.
    \item \emph{Initial state} - UAS state at the beginning of a \emph{mission}.
    \item \emph{Applied movements} - list of already executed movements.
    \item  \emph{Future movements} - list of movements to be applied in future.
    \item \emph{Current state} - UAS state at \emph{Time}, with current position and orientation included.
    \item \emph{Waypoint} - current goal waypoint.
\end{enumerate}


\begin{equation}\label{eq:consFunctionMeta}
    Cost (t,\dots)= costTrajectoryFlown(t,\dots) + expectedReachCost(t,waypoint,\dots)
\end{equation}


\paragraph{Cost of flown trajectory} (eq. \ref{eq:costFunctionExecuted}) from the \emph{initial state} to the \emph{current state} is calculated as a sum of energy consumed for each movement with the following components:
\begin{enumerate}
    \item \emph{Direct cost} - a cost of consumed energy to execute the movement.
    
    \item \emph{Horizontal cost} - a portion of the direct cost which was used for horizontal steering multiplied by \emph{horizontal penalization}.
    
    \item \emph{Vertical cost} - a portion of the direct cost which was used for ascending/descending multiplied by \emph{vertical penalization}.
\end{enumerate}


\begin{equation}\label{eq:costFunctionExecuted}
   \scriptsize \begin{gathered}costTrajectoryFlown \end{gathered} \left (\begin{gathered}time,\\ initialState,\\ appliedMovements \end{gathered}\right) = \sum_{\begin{gathered}  movement\in\\ appliedMovements\end{gathered}\normalsize} \left(\begin{gathered}
        duration.directCost+\\
        horizontal(cost,penalization)+\\
        vertical(cost,penalization)
   \end{gathered}
   \right) 
   \normalsize
\end{equation}
 
\paragraph{Expected reach cost} (eq. \ref{eq:costFunctionExecuted}) is calculated for a \emph{planned trajectory} portion and a \emph{direct waypoint distance} to the \emph{latest future UAS position}. \emph{Cost of the planned trajectory} is calculated by the same formula as a \emph{cost of flown trajectory} (eq. \ref{eq:costFunctionExecuted}), the initial state is replaced with a  \emph{current state}, and \emph{executed movements} are replaced with \emph{planned movements}.
    
\begin{equation}\label{eq:costFunctionReachable}
    \scriptsize
    \begin{gathered} expectedReachCost\end{gathered} \left(\begin{gathered}time,\\currentState,\\futureMovements,\\waypoint\end{gathered}\right)
    =
    \left (\begin{gathered}distance(futureState,waypoint)+\\ costTrajectoryFlown\left(\begin{gathered}currectState,\\futureMovements\end{gathered}\right) \end{gathered}\right)
    \normalsize
\end{equation}

\begin{note}
    The tuning parameters of cost function are \emph{Horizontal penalization} $\in [0,\infty]$ and \emph{Vertical penalization} $\in [0,\infty]$. Which are used to enhance the outcome of the \emph{cost function}.
    
    Following setup of tuning parameters are used in our simulations:
    \begin{enumerate}
        \item $horizontalPenalization \le verticalPenalization < \infty$ - in  \emph{uncontrolled airspace}, all kind of maneuvers are allowed. Horizontal maneuvers are cheaper for a plane UAS.
        \item $horizontalPenalizaiton < vetricalPenalization = \infty$ - in  \emph{controlled airspace} any kind of horizontal maneuvering must be allowed by UTM.
    \end{enumerate}
    
    The tuning parameters are set up $verticalPenalizaiton \le horizontalPenalizaiton < \infty$ for \emph{copter} UAS in an \emph{uncontrolled airspace}
\end{note}

\subsection{\secState{D}Trajectory Tracking}\label{s:trajectoryTracking}
\paragraph{Trajectory Tracking} is a crucial parameter for \emph{controlled airspace} and is expected to be important in \emph{upcoming UTM systems}. There is a \emph{mission plan} which is compared with \emph{real-time airspace situation} obtained from UAS \emph{Position notifications}. The optimalization based on \emph{Reach Set} is given in \cite{varaiya2000reach}.

\paragraph{Motivation:} \emph{Situation awareness} for modern DAA systems depends on planned trajectory tracking. The main conflict is between \emph{navigation precision} and \emph{situation evaluation}. If the planned trajectory is defined for the \emph{continuous domain}, it takes much effort to calculate collision points. 

The discrete domain of \emph{Movement Automaton} (def. \ref{def:movementAutomaton}) can be used as a \emph{tool for situation awareness}. The main idea is to use \emph{Movement automaton as a predictor for trajectory intersection} \cite{frazzoli2000trajectory,frazzoli2001robust}.

\paragraph{Movement Automaton trajectory tracking:} There is a \emph{reference trajectory} which is used as comparison by \emph{aviation authority} (ATM,UTM) given as:

\begin{multline}\label{eq:MissionReferencetrajectory}
    Reference Trajectory = \{(point_1,time_1), (point_2,time_2),\dots,\\,\dots(point_n,time_n)\} \quad n\in\N^+, point_k\in \R^3
\end{multline}

\emph{Reference Trajectory} (eq. \ref{eq:MissionReferencetrajectory}) is given as set of \emph{points} in \emph{Global Coordinate System} for given \emph{UAS}, \emph{operational time-frame} and other authority depending properties.

The \emph{movement} automaton is executing \emph{Trajectory(initialState,buffer)} where buffer is set of \emph{Movements}. The buffer is changing according to following pattern during \emph{mission} time frame:
\begin{equation}\label{eq:BufferEvolutionMission}
    \begin{aligned}
    \texttt{M}&\texttt{ission Start:}\\
    &buffer=\{planned Movements\},planned= m\\
    \texttt{D}&\texttt{uring Mission:}\\
    &buffer=\{executedMovements,plannedMovements\}, \\
    &\quad executed = n, planned =o\\
    \texttt{M}&\texttt{ission End:}\\
    &buffer=\{executedMovements\},executed = p;\\
    \texttt{M}&\texttt{ovement count constraints:}\\
    &m,n,o,p \in \N^+, n+o =m, m\le p
    \end{aligned}
\end{equation}

At the beginning of the mission (eq. \ref{eq:BufferEvolutionMission}) the buffer is filled with $m$ movements, the \emph{Trajectory} generated from this buffer and \emph{initial state} is \emph{predicted}.

During the \emph{mission execution phase}, the buffer contains \emph{executed movements} and \emph{planned movements}. Trajectory created from the \emph{initial state} and this buffer can be split into:

\begin{enumerate}
    \item\emph{Executed part} - trajectory portion generated and executed from \emph{executed movements}
    
    \item\emph{Predicted part} - trajectory portion generated as a future reference from \emph{planned movements} 
\end{enumerate}

After \emph{mission} execution, there is only \emph{executed movements} The trajectory generated from the \emph{initial state} and buffer is \emph{Executed Trajectory}

\begin{note}
    The part of the trajectory bounded to the past, the part of the trajectory lies in the future. The strong point of \emph{Movement Automaton} is its ability to work as \emph{predictor} and \emph{trajectory memory} at the same time.
    
    By selecting proper time series $t_1\dots t_n$ one can compare future or past segments of trajectory (eq. \ref{eq:BufferEvolutionMission}) with reference (eq. \ref{eq:MissionReferencetrajectory})
\end{note}

\paragraph{Reference Trajectory Deviation} for reference trajectory given by (eq. \ref{eq:MissionReferencetrajectory}) and \emph{Trajectory segment} (Executed/Predicted) (\ref{eq:BufferEvolutionMission}) with existing \emph{State projection function} (eq. \ref{eq:stateprojection}) for \emph{time series} is given as:

\begin{equation}\label{eq:ReferenceTrajectoryDeviation}
    Deviation\left(\begin{aligned}&timeSeries,\\ &Trajectory,\\ &Reference\end{aligned}\right) = \sum_{\scriptsize\begin{gathered}time_i \in\\ timeSeries\end{gathered}\normalsize} \left(\begin{gathered}StateProjection(Trajectory,time_i)\\-\\Reference(time_i)\end{gathered}\right)^2
\end{equation}

\emph{Reference Trajectory Deviation} (eq. \ref{eq:ReferenceTrajectoryDeviation}) is designed as discrete \emph{Mean Square Error} function, where the \emph{timeSeries} is set of \emph{times} from \emph{reference trajectory} $(point_i,time_i)$ pair. The \emph{state projection}.

\paragraph{Trajectory tracking} is defined as \emph{dual minimization problem} where the \emph{primary objective} is depending on the \emph{airspace type}:

\begin{enumerate}
    \item \emph{Reference trajectory deviation} (eq. \ref{eq:ReferenceTrajectoryDeviation}) in \emph{Controlled Airspace}.
    
    \item \emph{Cost of Flown Trajectory} (eq. \ref{eq:consFunctionMeta}) in \emph{Non-controlled Airspace}.
\end{enumerate}

\newpage
\noindent\emph{Trajectory tracking} can be defined as an optimization problem (eq. \ref{eq:trajectoryTrackingOptimalizaitonProblem}). 

\begin{equation}\label{eq:trajectoryTrackingOptimalizaitonProblem}
    \begin{aligned}
         &\texttt{Minimize:} && costOfTrajectoryFlown &(\ref{eq:consFunctionMeta})\\
         &\texttt{Minimize:} && reference Trajectory Deviation
         &(\ref{eq:ReferenceTrajectoryDeviation})\\
         &\texttt{Subject to:} &&&\\
         &&&UAS\text{ }Dynamics&(\ref{eq:madSystemdefinition})\\
         &&&MovementAutomatonControl&\begin{gathered}(\ref{eq:madInitialState})\\\vdots\\(\ref{eq:madStateProjection})\end{gathered}\\
         &&&Mission&(\ref{eq:missionAbstractSet})\\
         &&&KnownWorld(t)&(\ref{eq:knownWorldSet})\\
         &&&SafetyMargin(t)&(\ref{eq:safetyMarginAbstract})\\
         &&&HardFlightConstraints&(\ref{eq:HFlightConstraints})\\
         &&&SoftFlightConstraints&(\ref{eq:SFlightConstraints})\\
         &&&HardSpaceConstraints&(\ref{eq:hardSpaceConstraints})\\
         &&&SoftSpaceConstraints&(\ref{eq:softSpaceConstraints})\\
    \end{aligned}
\end{equation}

The \emph{reference trajectory} is given by \emph{mission} set of \emph{waypoints}. The \emph{UAS} dynamics with specific \emph{Movement Automaton} goal is to fly in \emph{Known World} to keep \emph{Safety Margin} form  \emph{Obstacle Space}. The \emph{Obstacle space} is a result of \emph{Data fusion} procedure (sec. \ref{s:dataFusionDefinition}) combining the \emph{sensor reading}, information sources, and constraints.

\paragraph{Feasible Trajectory} for \emph{tracking problem} (eq. \ref{eq:trajectoryTrackingOptimalizaitonProblem}) is a trajectory which in addition to \emph{basic obstacle problem} (sec. \ref{s:BasicProblemDefinition}) keeps deviation from the \emph{reference trajectory} under certain threshold:
\begin{equation}\label{feasibleTrajectoryCondition}
    Deviation(timeSeries, Trajectory, Reference) \le performanceMargin \in \R^+
\end{equation}

Feasible trajectory condition (eq. \ref{feasibleTrajectoryCondition}) is used as \emph{margin} for airworthiness, and \emph{Deviation} is used as a performance indicator further in this work.


\subsection{\secState{D}Safety}\label{s:Safety}
\noindent \emph{Safety} is very broad term there are following incidents which can occur and will be discussed in (app. \ref{s:safetyMarginCalculation}) \emph{Safety margin} is a broad term describing minimal distance to the center of intruder/adversary, a surface of the obstacle, a boundary of the protected area.

\paragraph{Controlled airspace safety}
\noindent  Safety for \emph{controlled airspace} in given \emph{flight level} is given a list of incidents:
\begin{enumerate}
    \item \emph{Soft constrained zone breach} - UAS fly to \emph{soft constraint body} or \emph{hard constraint protection zone}, these incidents can happen, and have least avoidance priority. 
    \item \emph{Hard constrained zone breach} - UAS fly to \emph{hard constraint protection zone}, typical geo-fencing, restricted airspace breaches. 
    \item \emph{Well-clear breach} - UAS fly to \emph{well-clear barrel} without impacting other aircraft, via the wake turbulence or other induced physical phenomenon and vice-versa. This type of breaches are allowed in case of inevitable \emph{near miss situations} or \emph{Clash incidents} 
    \item \emph{Near miss situation}- UAS fly to \emph{near miss} cone/barrel, inducing wake turbulence, or other kinds of flight disturbance. These incidents are allowed at very low rate (near $1:10^6$).
    \item \emph{Clash incident} - UAS body impacts another aircraft hull/propulsion/steering systems and components. This kind of incidents are very severe, and they should never happen.
\end{enumerate}
\begin{note}
    It is assumed that flight level in controlled airspace is free of terrain, static ground obstacles, the climb/descent maneuvers are not covered in this work, and they are topic for multiple dissertation theses. For more information refer to ICAO document 4444.
\end{note}

The \emph{relation} for breach of \emph{safety margin} and \emph{body margin} for each object is given in  (tab. \ref{tab:controlledAirspaceViolations}):

\begin{tabularx}{\textwidth}{S{0.20}||S{0.35}|S{0.35}}
    Violation of: & Safety Margin & Body Margin\\\hline\hline
    Soft constraint & none & Soft constraint zone breach \\\hline 
    Hard constraint & Soft constrained zone breach & Hard constrained zone breach\\\hline 
    Intruder  & \begin{minipage}{0.30\textwidth}Well clear breach,\\ Near Miss situation\end{minipage} & Clash incident\\
    \caption{Controlled airspace margins violations incidents.}
    \label{tab:controlledAirspaceViolations}
\end{tabularx}

\paragraph{Uncontrolled airspace safety}
\noindent Safety for \emph{uncontrolled airspace} is applied in $F/G$ class of airspace, which is given as airspace between the \emph{ground} level and \emph{other airspace prevalence}.
\begin{note}{Clarification of controlled/uncontrolled airspace:}
    \begin{itemize}
        \item[1.] \emph{Class F} airspace is given as space between the ground level or water surface, and it is constrained up to the 500 feet above ground level.
        \item[2.] \emph{Class C} airspace or \emph{Controlled airspace} is considered starting at first flight level, which is given by Air traffic control zone starting at least at 300 feet from highest ATC zone ground point. It is measured based on Above Sea Level altitude. \emph{This is not a problem in Portugal, because of terrain diversity, but its a huge problem in the Netherlands}.
        \item[3.] \emph{Class A} airspace starts at ground level and covers the majority of airport infrastructure - this is not a problem, because it`s modeled as hard constraint, which is unbreakable in non controlled airspace.
    \end{itemize}
\end{note}

\noindent Safety for \emph{uncontrolled airspace} is given a list of incidents:
\begin{enumerate}
    \item \emph{Soft constrained zone breach} - UAS fly into the \emph{soft constrained zone} or \emph{hard constraint protection} zone, it is allowed to happen on very a low rate.
    
    \item \emph{Hard constrained zone breach} - UAS fly into the \emph{hard constrained zone}, only airports and critical infrastructure are considered as hard constraints; it is not allowed to happen.  
    
    \item \emph{Intruder near miss} - UAS fly into \emph{other aircraft near miss zone}, it is allowed to happen on a very low rate in case of other intermediate threats with higher priority.
    
    \item \emph{Intruder clash} - UAS has contact with other man-made aircraft, it is not allowed to happen.
    
    \item \emph{Adversary clash} - UAS has contact with another flying object which did not intentionally avoided UAS. (\emph{Bird strike, Differential games,etc..} is out of the scope of this thesis).
    
    \item \emph{Structure harm} - UAS fly close to structure, and its propulsion can damage/harm structure.
    
    \item \emph{Structure crash} - UAS fly into a natural/man-made ground structure (building, tree, human).
    
    \item \emph{Ground harm} - UAS fly close to the ground, and its propulsion reflection can impact Ground or UAS.
    
    \item \emph{Ground collision} - UAS collides with the ground.
\end{enumerate} 

The \emph{relation} for breach of  \emph{safety margin} and \emph{body margin} for each object is given in (tab. \ref{tab:uncontrolledAirspaceViolations}):

\begin{tabularx}{\textwidth}{S{0.20}||S{0.35}|S{0.35}}
    Violation of: & Safety Margin & Body Margin\\\hline\hline
    Soft constraint & none & Soft constraint zone breach \\\hline 
    Hard constraint & Soft constrained zone breach & Hard constrained zone breach\\\hline 
    Intruder  & Intruder near miss & Intruder clash \\\hline
    Adversary & none & Adversary clash \\\hline
    Structure & Structure harm & Structure crash \\\hline
    Ground    & Ground harm    & Ground crash \\
    \caption{Non-controlled airspace margins violations incidents.}
    \label{tab:uncontrolledAirspaceViolations}
\end{tabularx}
    \section{Navigation Requirements}\label{s:navigationRequirements}
\noindent \emph{Navigation requirements} are not the  main part of this work; they are used to show the variability of the approach for \emph{DAA} requirements:
\begin{enumerate}
    \item \emph{Contextual behavior} - change navigation and decision behavior based on context:
    \begin{enumerate}[a.]
        \item \emph{Airspace type} - Controlled/Uncontrolled, 
        \item \emph{Navigation mode} - Cooperative/Emergency. 
    \end{enumerate}    
    
    \item \emph{Determinism} - same result for same dataset in finite time.
    
    \item  \emph{Threat prioritization} - threats prioritization based on \emph{context}, (tab. \ref{tab:controlledAirspaceViolations}) and (tab. \ref{tab:uncontrolledAirspaceViolations}).
    
    \item \emph{Rule compliance} - compliance with a given set of rules based on context (focus on rules of the air).
\end{enumerate}

\begin{tabularx}{\textwidth}{S{0.25}||S{0.65}}
    Requirement & Evaluation metrics \\ \hline\hline
    Constrained space navigation & Mission scenario does not have a direct path between waypoints, additional borderline cases.\\\hline
    Contextual behavior & Avoidance system changes behavior based on the mission and vehicle context.\\\hline
    Determinism & Multiple runs of same non-borderline scenario returns same avoidance paths.\\\hline
    Rule compliance & Rules applied comply with aviation standardization.\\
    \caption{Navigation requirements evaluation metrics.}
    \label{tab:navigationRequirementsEvaluationMetrics}
\end{tabularx}


	
%% This adds a line for the Bibliography in the Table of Contents.
\addcontentsline{toc}{chapter}{Bibliography}
%% *** Set the bibliography style. ***
%% (change according to your preference/requirements)
%\bibliographystyle{plain}
%% *** Set the bibliography file. ***
%% ("thesis.bib" by default; change as needed)
\bibliography{thesis}

%% *** NOTE ***
%% If you don't use bibliography files, comment out the previous line
%% and use \begin{thebibliography}...\end{thebibliography}.  (In that
%% case, you should probably put the bibliography in a separate file and
%% `\include' or `\input' it here).

\end{document}
