\cleardoublepage
\chapter{\secState{R/W}Conclusion and Future Work}\label{ch:Conclusion}
\noindent This section summarizes \emph{obstacle avoidance framework} functionality (sec. \ref{s:conclusionSummary}), provides \emph{comparison to other methods} (sec. \ref{s:OtherMethodsComparison}), outlines \emph{approach reusability} (sec. \ref{s:approachReusability}), summarizes \emph{lessons learned} (sec. \ref{s:lessonsLearned}) and introduces possible \emph{future research heading} (sec. \ref{s:futureWork}).

\section{\secState{W}Summary}\label{s:conclusionSummary}
\noindent The \emph{obstacle avoidance} is a complex topic the work has adressed:
\begin{itemize}
    \item Reactive obstacle avoidance (practically for large scale of threats) - obstacles, intruders, constraints
    
    \item Event Based obstacle avoidance (partially for smal subset of events +UTM) - there are issues which has not been adressed (and they will be later)
    
    \item Pre-emptive obstacle avoidance - this has not been adressed at all still holding the reachibility of waypoint assumption (link assumption) 
    \item link the avoidance levels from chapter 6
\end{itemize}

    
\subsection{\secState{W}Reach Set Role in Avoidance}\label{s:conclusionReachSet}

\noindent The reach set represents a set of avoidance/movement (manuevering) strategies which can be used in different context (situations)
\begin{itemize}
    \item Chaotic - intermediate emergency avoidance
    \item Harmonic - navigation
    \item Combined - non controlled airspace navigation + avoidance
    \item ACAS like  controlled airspace navigation + avoidance
\end{itemize}


\subsection{\secState{W}Avoidance/Navigation Run}\label{s:conclusionAvoidanceNavigationRun}

\noindent The principle of hierarchical decision making
\begin{itemize}
    \item Avoidance Run - one decision, one context
    \item Navigation Run - multiple avoidance runs with different context (hierarchical applicaiton), joining multiple decisions over time
    \item short summary with references
\end{itemize}


\section{\secState{W}Comparison to Other Methods}\label{s:OtherMethodsComparison}

\begin{enumerate}
	\item Vector field avoidance \cite{borenstein1991vector}
	\item Potential field \cite{koren1991potential}
\end{enumerate}

\subsection{\secState{W}Scalability}\label{s:conclusionScalability}

\begin{itemize}
    \item the scalability is a key for everchanging rules/regulations accomodations
    
    \item many concepts have margins hardwired
    
    \item our concept is limited by turning ratio + body radius < margin < max range of sensor field - avoidance grid
    
    \item our approach is scalable trough concept of multiple  margins:
    \begin{itemize}
        \item body margin
        \item safety margin
        \item soft constraitns - warning margin
        \item hard constraints - body margin
    \end{itemize}
    
\end{itemize}

\subsection{\secState{W}Conservative Method Comparison}\label{s:conservativeComparison}
\begin{itemize}
    \item Take notes from martin hrdlik work - compare the method 
    \item Key concept/Idea: keep awayat leas double of truning radius
    \item Show calculation/comparison
\end{itemize}

\subsection{\secState{W}Potential Field Method Comparison}\label{s:potentialComparison}
\begin{itemize}
    \item Take ntoes from martin hrdlik work - compare the methods
    \item key concept/Idea: Every obstacle have charge proportional to expected mass. our UAS is repelled by this charge, the charges can aslo have static/dinamic energy emulating obstacles/intruders
    \item method has good performance but do not guarantees the safety like ours
\end{itemize}

\section{\secState{W}Approach Reusability}\label{s:approachReusability}
\begin{itemize}
    \item write introduction, check the content is something is not missing
\end{itemize}
\paragraph{UTM Services:} The constrained \emph{UTM functionality} is outlined in (sec. \ref{sec:UASTrafficManagement}) including:
\begin{enumerate}
    \item \emph{Future UTM Communication Architecture} (fig. \ref{fig:UTMArchitectureOverview}) as the authority over \emph{airspace segment} (fig. \ref{fig:DAMExample})\cite{gerdes2016dynamic}.
    
    \item \emph{Cooperative Conflict Resolution Under UTM Supervision} (fig. \ref{fig:CooperativeConflictResolutionUTM}) designed as mild/feasible directives (commands) with \emph{constant supervision}.
    
    \item \emph{Rules of the Air Enforcement} (sec. \ref{sec:handlingHeadOnApproach}, \ref{sec:handlingConvergingManuever}, \ref{sec:handlingOvertakeManuever}) including designs of \emph{Position Notification} (sec. \ref{sec:positionNotification}) and \emph{Collision Case Structure/Calculation} (sec. \ref{sec:collisionCase}).
    
    \item \emph{Divergence/Convergence Waypoints} concept is showcased in \emph{Overtake Rule} (rule \ref{tab:ruleOvertakeDefinition}). 
    
    \item \emph{Weather Avoidance} (sec. \ref{sec:weatherCase}) is using similar concept to \emph{Collision Case}: \emph{Weather Case}. The information are provided by \emph{Local Airspace Authority}.
    
\end{enumerate}

\paragraph{Emergency Avoidance Functionality:}  The standard framework implementation (fig. \ref{fig:missionControlRunActivityDiagram}) can handle the situations given in non-cooperative test cases (sec. \ref{s:noncooperativeTestCases}). The list of threats is given by (tab. \ref{tab:uncontrolledAirspaceViolations}). 

\paragraph{Event Based Avoidance Functionality:} The standard framework implementation (fig. \ref{fig:missionControlRunActivityDiagram}) with active $C2$ link and rules setup (fig. \ref{fig:RuleEngineInstanceLevels}) can handle the situations given in cooperative test cases (\ref{s:cooperativeTestCases}). The list of threats is given by (tab. \ref{tab:controlledAirspaceViolations}). The \emph{Avoidance Mode Concept} enables to switch between \emph{Event Based Avoidance} (Navigation) and \emph{Emergency Avoidance.}
\begin{note}
    The emergency Avoidance Functionality is included in \emph{Event Based Avoidance} (Navigation) mode.
    The prioritization of \emph{threats} may differ (tab. \ref{tab:controlledAirspaceViolations}).
\end{note}

\paragraph{Reusability for More Complex Systems:} The framework (fig. \ref{fig:avoidanceConcept}) with implemented rule engine (fig. \ref{fig:RuleEngineBasicArchitecture}) can be used on \emph{any system}, with appropriate \emph{Movement automaton} (sec. \ref{s:segmentedMovementAutomaton}) enabling \emph{wave-front} propagation (alg. \ref{alg:Wavefront Propagation}) for reach set estimation. Following artifacts needs to be delivered for concept reuse:

\begin{enumerate}
    \item The \emph{Movement Automaton} is used to generate \emph{thick series of waypoints} which guarantees desired degree of safety.
    
    \item The \emph{complex UAS system} is following the \emph{reference trajectory} (sec. \ref{s:referenceTrajectoryGenerator}).
    
    \item The \emph{Sensor Fusion} (sec. \ref{s:SensorFusionDefinition}) implementation including classification to \emph{Free}, \emph{Occupied}, \emph{Restricted} space type.
    
    \item The \emph{sensor field} supporting detection of threats. There should be at least one sensor with capability of feeding \emph{Avoidance Grid}. Our implementation was based on LiDAR/ADS-B feeds.
    
    \item The \emph{Information Sources} supporting the online/offline threat processing. This one is completely optional.
\end{enumerate}

\begin{note}
    \emph{On UTM integration:} The future UTM system will not giving the extreme commands, the directives are more like constraints, therefore our system can provide the guidance and constraint evaluation
\end{note}

\begin{note}
    \emph{On Safety Margin:} The disparity between real flown trajectory (nonlinear dynamics) and planned trajectory (Movement Automaton) needs to be accounted into \emph{Safety Margin}.
\end{note}

\paragraph{Reach Set Approximations:} The \emph{wave-front} approach (alg. \ref{alg:Wavefront Propagation}) can be used with \emph{Constrained expansion function} (sec. \ref{s:constrainedTrajectoryExpansion}) to create own \emph{Reach set Approximation Method}. Existing  reach set approximation methods are always following a different goal, they can be reused for other tasks (perf. \ref{sec:reducedReachSetPerformance}):

\begin{enumerate}
    \item \emph{Chaotic} (def. \ref{s:chaoticReachSet}) - high space coverage, ideal for unpredictable and complex avoidance maneuvers.
    
    \item \emph{Harmonic} (def. \ref{s:harmonicReachSet}) - smooth trajectories, medium space coverage, ideal for navigation maneuvers.
    \item \emph{Combined} (def. \ref{s:combinedReachSet}) - combination of the \emph{harmonic} and 
    
    \emph{chaotic} approximations, the cost function defines preferred trajectories. The procedure is reusable for any reach set approximation types ($2^+$) combination.
    
    \item \emph{ACAS-X Like} (def. \ref{s:acasReachSet}) - following \emph{TCAS/ACAS separation modes}, can be used as alternative for \emph{controlled avoidance} and \emph{navigation}.
\end{enumerate}


\section{\secState{W}Lessons Learned}\label{s:lessonsLearned}
What can be done differently
\begin{itemize}
	\item planar/euc grid: cube cells vs planar cells:
	\begin{itemize}
	    \item there is good rate - increasing distance from uas, increasing cell volume , decreasing importance
	    \item the linear cell count growth with increasing distance, covering more space vs quadratic cell count in euclidiean
	\end{itemize}
	
	\item Intruder uncertainity model ideas:
	\begin{itemize}
	    \item intruder as line/curve ontersection
	    \item intruder as movement automaton simulated trajectory
	    \item intruder well clear implementation as rule - potential fields
	\end{itemize}
	
	\item Probabilistic vs rating approach:
	\begin{itemize}
	    \item Probablistic approach - too formal
	    \item Deterministic approach - true/false - too constrained
	    \item rating +  post deterministic: 
	    \begin{itemize}
	        \item freedom of probablilstic
	        \item regulation and business accomodation/introduction
	    \end{itemize}
	    \item Ratings reflects probabilities, they share basic mechanisms
	    \item man made rules incorporation - some things are automatic no go, even if probabilistic approach says its "safe"
	    \item Situation assessment trough tresholding
	\end{itemize}
\end{itemize}

\section{\secState{W}Future Work}\label{s:futureWork}
\begin{enumerate}
	\item Adversarial avoidance
	\begin{itemize}
	    \item further implementation of game theory
	    \item pursuer reach set u(t)-v(t) as input
	    \item solving problems from 7.6.2
	\end{itemize}
	
	\item Real system implementation:
	\begin{itemize}
	    \item future implementation in Ground station
	    \item future implementation onboard
	    \item what should be put into OS/Cloud
	    \item use LSTS architecture
	\end{itemize}
\end{enumerate}
