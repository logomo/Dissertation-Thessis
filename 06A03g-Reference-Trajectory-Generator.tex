\section{\secState{R}Reference Trajectory Generator}\label{s:referenceTrajectoryGenerator}

\paragraph{Reference Trajectory Generator:} Segmented Movement Automaton (def.  \ref{def:segmentedMovementAutomaton}) with \emph{trajectory function} (eq. \ref{eq:ourTrajectoryImplementationSegmented}) is used as a \emph{reference trajectory generator} for \emph{complex systems}. 

There is an assumption that precise \emph{path tracking} implementation exist for such system which with \emph{thick reference trajectory} gives similar results to \emph{plain movement automaton control}.

The \emph{Reference trajectory} (eq. \ref{eq:generatedReferenceTrajectory}) for \emph{Planned} movement set is given as projection  of \emph{Trajectory} time series to position time series $[x,y,z,t]$:

\begin{equation}\label{eq:generatedReferenceTrajectory}
    Reference Trajectory:Trajectory\left(\begin{gathered}state(now),\\Planned\end{gathered}\right) 
    \to 
    \begin{bmatrix}
        x_{ref} \in \R^{|Planned|}\\
        y_{ref} \in \R^{|Planned|}\\
        z_{ref} \in \R^{|Planned|}\\
        t_{ref} \in \R^{|Planned|}
    \end{bmatrix}
\end{equation}

\paragraph{Predictor:} The \emph{Reference Trajectory Generator} (eq. \ref{eq:generatedReferenceTrajectory}) can also be used as a predictor. 

\begin{note}
    The \emph{Segmented Movement Automaton} (def. \ref{def:segmentedMovementAutomaton}) is used in this work with one Segment equal to State space with input function given by (\ref{tab:movements1}, \ref{tab:movements2}). The predictor used in \emph{Reach set computation} is given by (eq. \ref{eq:generatedReferenceTrajectory}).
\end{note}

\paragraph{State Projection} (eq. \ref{eq:ourStateProjection}) for the \emph{Trajectory} (eq. \ref{eq:ourTrajectoryImplementation}) is given as follow:
\begin{equation}\label{eq:ourStateProjection}
    StateProjection(Trajectory,time) = Trajectory.getMemberByIndex(time+1)
\end{equation}

\begin{note}
    \emph{Movement Automaton} for system (eq. \ref{eq:UASNonlinearModelSimple}) with given (as. \ref{ass:transitionTime}) is established with all related properties (sec. \ref{def:movementAutomaton}).
\end{note}
