\subsection{(R) NASA UTM}\label{sec:NASAUtm}

\noindent The \emph{NASA UTM}\footnote{Related research and articles: \url{https://utm.arc.nasa.gov/documents.shtml}} is UAS Traffic Concept developed by \emph{National Aeronautics and Space Administration} (NASA) in cooperation \emph{Federal Aviation Administration} (FAA). The \emph{concept} is very similar to \emph{EASA U-SPACE}.

\begin{note}
    This work is focused on \emph{European Airspace}, the \emph{details} will be omitted. 
\end{note}

\paragraph{Useful concepts:} The \emph{NASA UTM} concept has greater maturity level in terms of \emph{Detect and Avoid} concept than European \emph{U-Space}. There is wast amount of publications which can be used in \emph{U-Space} from this publications the following useful studies containing DAA concepts were taken into account:

\begin{enumerate}    
    \item The non-cooperative intruder avoidance concept \cite{cone2017uas} provides general idea about \emph{topic}. The \emph{vertical separation} and \emph{vertical encounter model} is presented.
    
    \item The \emph{Detect and Avoid} performance evaluation is crucial for system performance assessment. The assessment framework \cite{lee2016wide} provides us with methodological guidelines. The the used concepts are abstracting the multidimensional performance criteria into simple metrics:
    
    \begin{enumerate}[a.]
        \item \emph{Crash Distance} - the distance to the obstacle/intruder margin.
        
        \item \emph{Safety Margin} - the virtual margin around obstacle/intruder.
    \end{enumerate}
    
    \item To \emph{Ensure} the compatibility between \emph{UAS Detect And Avoid System} and \emph{Manned Aviation Collision Avoidance} (ACAS/TCAS) systems the following approach were proposed \cite{thipphavong2017ensuring}.
\end{enumerate}
