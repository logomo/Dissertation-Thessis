\cleardoublepage
\section{Reach Set Approximation}\label{s:reachSet}

    \noindent\paragraph{Motivation:} \emph{Reach set} is strong tool for \emph{Obstacle Avoidance} because it contains all possible \emph{avoidance maneuvers}. The current implementations (sec. \ref{sec:ReachSetComputationMethodsASIS}) have following flaws:
    
    \begin{enumerate}
        \item \emph{Realistic approximation} - \emph{nonlinear systems} or \emph{heavily constrained systems} cannot be approximated well by \emph{linear continuous-time Reach Sets}.
        
        \item \emph{Finite count of possibilities} - continuous-time \emph{Reach Set} contains  infinite possibilities for \emph{avoidance maneuvers}, the SAA system demands conflict resolution in finite time.
        
        \item \emph{Computationally feasible data structures} - binding related properties seems problematic, because \emph{continuous- time reach sets} does not have unique identifier of maneuver, trajectory nor segment. 
    \end{enumerate}
    
    \paragraph{Proposed Solution Features:} Our Reach set Estimation method will provide following features:
    
    \begin{enumerate}
        \item \emph{System Control Interface} - implemented via \emph{Movement Automaton}, requiring only \emph{discrete command chain} to approximate system behaviour.
        
        \item \emph{Finite count of possibilities} - finite number of elements in \emph{Reach set} will enable \emph{scalable} calculation.
        
        \item \emph{Computationally feasible data structures} - approximation of Reach set as a set of trajectories, each trajectory can be split into finite number of segments. Each element will have unique identifier enabling both-side  property binding.
        
        \item \emph{Computationally feasible data-structures} - some specific behavior, like horizontal/vertical separation, or maneuver shape can be encoded into a different types of the reach set approximation algorithms.
    \end{enumerate}