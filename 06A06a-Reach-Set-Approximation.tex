\cleardoublepage
\section{Reach Set Approximation}\label{s:reachSet}
\paragraph{Summary:} There is a need to have a tool with a finite count of trajectories, which has enough variability to support avoidance task. The reach set covers all possible trajectories, but it is not countable. Trajectories represented as tree originating in the same initial state can be considered as a skeleton of the reach set. There is a need to compare trajectories regarding avoidance capability. To achieve this, it is necessary to distinguish them based on measurable criteria. Different approximation based on the measurable criteria is introduced to cover different avoidance behaviors. 

    \noindent\paragraph{Motivation:} \emph{Reach set} is a strong tool for \emph{Obstacle Avoidance} because it contains all possible \emph{avoidance maneuvers}. The current implementations (sec. \ref{sec:ReachSetComputationMethodsASIS}) have the following flaws:
    
    \begin{enumerate}
        \item \emph{Realistic approximation} - \emph{nonlinear systems} or \emph{heavily constrained systems} cannot be approximated well by \emph{linear continuous-time Reach Sets}.
        
        \item \emph{Finite count of possibilities} - continuous-time \emph{Reach Set} contains  infinite possibilities for \emph{avoidance maneuvers}; the DAA system demands conflict resolution in finite-time.
        
        \item \emph{Computationally feasible data structures} - binding related properties seem problematic because \emph{continuous- time reach sets} do not a have unique identifier of maneuver, trajectory nor segment. 
    \end{enumerate}
    
    \paragraph{Proposed Solution Features:} Our Reach set Estimation method will provide the following features:
    
    \begin{enumerate}
        \item \emph{System Control Interface} - implemented via \emph{Movement Automaton}, requiring only a \emph{discrete command chain} to approximate system behavior.
        
        \item \emph{Finite count of possibilities} - finite number of elements in \emph{Reach set} will enable \emph{scalable} calculation.
        
        \item \emph{Computationally feasible data structures} - approximation of Reach set as a set of trajectories, each trajectory can be split into a finite number of segments. Each element will have a unique identifier enabling both-side  property binding.
        
        \item \emph{Computationally feasible data-structures} - some specific behavior, like horizontal/vertical separation, or maneuver shape can be encoded into  different types of reach set approximation algorithms.
    \end{enumerate}