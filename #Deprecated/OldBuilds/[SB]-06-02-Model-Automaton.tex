%% fcup-thesis.tex -- document template for PhD theses at FCUP
%%
%% Copyright (c) 2015 João Faria <joao.faria@astro.up.pt>
%%
%% This work may be distributed and/or modified under the conditions of
%% the LaTeX Project Public License, either version 1.3c of this license
%% or (at your option) any later version.
%% The latest version of this license is in
%%     http://www.latex-project.org/lppl.txt
%% and version 1.3c or later is part of all distributions of LaTeX
%% version 2005/12/01 or later.
%%
%% This work has the LPPL maintenance status "maintained".
%%
%% The Current Maintainer of this work is
%% João Faria <joao.faria@astro.up.pt>.
%%
%% This work consists of the files listed in the accompanying README.

%% SUMMARY OF FEATURES:
%%
%% All environments, commands, and options provided by the `ut-thesis'
%% class will be described below, at the point where they should appear
%% in the document.  See the file `ut-thesis.cls' for more details.
%%
%% To explicitly set the pagestyle of any blank page inserted with
%% \cleardoublepage, use one of \clearemptydoublepage,
%% \clearplaindoublepage, \clearthesisdoublepage, or
%% \clearstandarddoublepage (to use the style currently in effect).
%%
%% For single-spaced quotes or quotations, use the `longquote' and
%% `longquotation' environments.


%%%%%%%%%%%%         PREAMBLE         %%%%%%%%%%%%

%%  - Default settings format a final copy (single-sided, normal
%%    margins, one-and-a-half-spaced with single-spaced notes).
%%  - For a rough copy (double-sided, normal margins, double-spaced,
%%    with the word "DRAFT" printed at each corner of every page), use
%%    the `draft' option.
%%  - The default global line spacing can be changed with one of the
%%    options `singlespaced', `onehalfspaced', or `doublespaced'.
%%  - Footnotes and marginal notes are all single-spaced by default, but
%%    can be made to have the same spacing as the rest of the document
%%    by using the option `standardspacednotes'.
%%  - The size of the margins can be changed with one of the options:
%%     . `narrowmargins' (1 1/4" left, 3/4" others),
%%     . `normalmargins' (1 1/4" left, 1" others),
%%     . `widemargins' (1 1/4" all),
%%     . `extrawidemargins' (1 1/2" all).
%%  - The pagestyle of "cleared" pages (empty pages inserted in
%%    two-sided documents to put the next page on the right-hand side)
%%    can be set with one of the options `cleardoublepagestyleempty',
%%    `cleardoublepagestyleplain', or `cleardoublepagestylestandard'.
%%  - Any other standard option for the `report' document arclass can be
%%    used to override the default or draft settings (such as `10pt',
%%    `11pt', `12pt'), and standard LaTeX packages can be used to
%%    further customize the layout and/or formatting of the document.

%% *** Add any desired options. ***
%PDF
%\documentclass[a4paper,narrowmargins,12pt,oneside,draft,onehalfspaced,singlespacednotes]{fcup-thesis}
%\documentclass[a4paper,narrowmargins,12pt,oneside,onehalfspaced,singlespacednotes]{fcup-thesis}
%Print
%\documentclass[draft,a4paper,narrowmargins,12pt,twoside,openright,onehalfspaced,singlespacednotes]{fcup-thesis}
\documentclass[a4paper,narrowmargins,12pt,twoside,openright,onehalfspaced,singlespacednotes]{fcup-thesis}

%% *** Add \usepackage declarations here. ***
%% The standard packages `geometry' and `setspace' are already loaded by
%% `ut-thesis' -- see their documentation for details of the features
%% they provide.  In particular, you may use the \geometry command here
%% to adjust the margins if none of the ut-thesis options are suitable
%% (see the `geometry' package for details).  You may also use the
%% \setstretch command to set the line spacing to a value other than
%% single, one-and-a-half, or double spaced (see the `setspace' package
%% for details).
% Overfull statements
\pretolerance=150
\setlength{\emergencystretch}{3em}
% Overfull end
\usepackage[english]{babel}
\usepackage{lipsum}
\usepackage[utf8]{inputenc}


%%% Additional useful packages
%%% ----------------------------------------------------------------
\usepackage{array}
\usepackage{amsmath}  
\usepackage{amssymb}
\usepackage{amsfonts}
\DeclareFontFamily{OT1}{pzc}{}
\DeclareFontShape{OT1}{pzc}{m}{it}{<-> s * [0.900] pzcmi7t}{}
\DeclareMathAlphabet{\mathpzc}{OT1}{pzc}{m}{it}
\usepackage{amsthm}      
\usepackage[ruled,algochapter]{algorithm2e}
\usepackage{algorithmic}
\usepackage{bm}
\usepackage[mathscr]{euscript}
\usepackage{graphicx}       
\usepackage{psfrag}         
\usepackage{fancyvrb}    
\usepackage{float}
\usepackage{ltablex}
\usepackage[square,sort,comma,numbers]{natbib}        
\usepackage{bbding}         
\usepackage{dcolumn}        
\usepackage{booktabs} 
\usepackage{multirow}
\usepackage{paralist}     
\usepackage{ifdraft}  
\usepackage{indentfirst}    
\usepackage[nottoc,notlof,notlot]{tocbibind}
\usepackage{url}
\usepackage{tabularx}
\usepackage{subcaption}
\usepackage[unicode]{hyperref}
\usepackage{xcolor}

\hypersetup{pdftitle=LiDAR obstacle detection and avoidance, 
            pdfauthor=Alojz Gomola,
            colorlinks=false,
            urlcolor=blue,
            pdfstartview=FitH,
            pdfpagemode=UseOutlines,
            pdfnewwindow,
            breaklinks
          }
\usepackage{array}
\newcolumntype{L}[1]{>{\raggedright\let\newline\\\arraybackslash\hspace{0pt}}m{#1}}
\newcolumntype{C}[1]{>{\centering\let\newline\\\arraybackslash\hspace{0pt}}m{#1}}
\newcolumntype{R}[1]{>{\raggedleft\let\newline\\\arraybackslash\hspace{0pt}}m{#1}}         
\newcolumntype{B}{X}
\newcolumntype{S}[1]{>{\hsize=#1\textwidth}X}

\newcommand{\FIGDIR}{./Pics}    %%% directory containing figures
\newcommand{\twolinecellr}[2][r]{%
  \begin{tabular}[#1]{@{}r@{}}#2\end{tabular}}
\newcommand{\secState}[1]{
	\ifdraft{(#1) }{}
}
\theoremstyle{plain}
\newtheorem{theorem}{Theorem}
\newtheorem{lemma}[theorem]{Lemma}
\newtheorem{proposition}[theorem]{Proposition}

\theoremstyle{plain}
\newtheorem{definition}{Definition}
\newtheorem{problem}{Problem}
\newtheorem{example}{Example}
\newtheorem{assumption}{Assumption}

\theoremstyle{remark}
\newtheorem*{corollary}{Corollary}
\newtheorem*{note}{Note}




\newenvironment{dokaz}{
  \par\medskip\noindent
  \textit{Proof}.
}{
\newline
\rightline{\SquareCastShadowBottomRight}
}

\newenvironment{constraints}[1]{
  \par\medskip\noindent
  \textit{Constraints #1} \\
}{
\newline
\rightline{\SquareCastShadowBottomRight}
}


%\bibliographystyle{plainnat}     %% Author (year) style
\bibliographystyle{unsrt}        %% [number] style
\setcitestyle{square}

% Section  3.7 Challenge list
\newif\ifproblemchallenge   %# Build block for problem challenges
\problemchallengetrue       %# Show comments

\newcommand{\R}{\mathbb{R}}
\newcommand{\N}{\mathbb{N}}

\DeclareMathOperator{\pr}{\textsf{P}}
\DeclareMathOperator{\E}{\textsf{E}\,}
\DeclareMathOperator{\var}{\textrm{var}}
\DeclareMathOperator{\sd}{\textrm{sd}}


\newcommand{\T}[1]{#1^\top}        

\newcommand{\goto}{\rightarrow}
\newcommand{\gotop}{\stackrel{P}{\longrightarrow}}
\newcommand{\maon}[1]{o(n^{#1})}
\newcommand{\abs}[1]{\left|{#1}\right|}
\newcommand{\dint}{\int_0^\tau\!\!\int_0^\tau}
\newcommand{\isqr}[1]{\frac{1}{\sqrt{#1}}}
\newcommand{\norm}[1]{\left\lVert#1\right\rVert}


\newcommand{\pulrad}[1]{\raisebox{1.5ex}[0pt]{#1}}
\newcommand{\mc}[1]{\multicolumn{1}{c}{#1}}
\newcommand{\TBD}[1]{\color{red}\emph{--TBD:}#1\color{black}}

%%%%%%%%%%%%%%%%%%%%%%%%%%%%%%%%%%%%%%%%%%%%%%%%%%%%%%%%%%%%%%%%%%%%%%%%
%%                                                                    %%
%%                   ***   I M P O R T A N T   ***                    %%
%%                                                                    %%
%%  Fill in the following fields with the required information:       %%
%%   - \degree{...}       name of the degree obtained                 %%
%%   - \department{...}   name of the graduate department             %%
%%   - \gradyear{...}     year of graduation                          %%
%%   - \author{...}       name of the author                          %%
%%   - \title{...}        title of the thesis                         %%
%%%%%%%%%%%%%%%%%%%%%%%%%%%%%%%%%%%%%%%%%%%%%%%%%%%%%%%%%%%%%%%%%%%%%%%%

%% *** Change this example to appropriate values. ***
\degree{Doctor of Philosophy}
\department{Departamento de Matem\'{a}tica}
\gradyear{2019}
\author{Alojz Gomola}
\title{Obstacle Avoidance Framework based on Reach Sets}

%% *** NOTE ***
%% Put here all other formatting commands that belong in the preamble.
%% In particular, you should put all of your \newcommand's,
%% \newenvironment's, \newtheorem's, etc. (in other words, all the
%% global definitions that you will need throughout your thesis) in a
%% separate file and use "\input{filename}" to input it here.


%% *** Adjust the following settings as desired. ***

%% List only down to subsections in the table of contents;
%% 0=chapter, 1=section, 2=subsection, 3=subsubsection, etc.
\setcounter{tocdepth}{3}

%% Make each page fill up the entire page.
\flushbottom


%%%%%%%%%%%%      MAIN  DOCUMENT      %%%%%%%%%%%%

\begin{document}


%%%%%%%%%%%%%%%%%%%%%%%%%%%%%%%%%%%%%%%%%%%%%%%%%%%%%%%%%%%%%%%%%%%%%%%%
%%  Put your Chapters here; the easiest way to do this is to keep     %%
%%  each chapter in a separate file and `\include' all the files.     %%
%%  Each chapter file should start with "\chapter{ChapterName}".      %%
%%  Note that using `\include' instead of `\input' will make each     %%
%%  chapter start on a new page, and allow you to format only parts   %%
%%  of your thesis at a time by using `\includeonly'.                 %%
%%%%%%%%%%%%%%%%%%%%%%%%%%%%%%%%%%%%%%%%%%%%%%%%%%%%%%%%%%%%%%%%%%%%%%%%

%% *** Include chapter files here. ***

\setcounter{chapter}{6}
\setcounter{section}{1}
%06-Approach

    %06-03 Movement Automaton
    \cleardoublepage
\section{\secState{R}UAS Model and Control}\label{s:modelMAImplementation}

\noindent The key feature of \emph{Movement Automaton} is to interface \emph{continuous-control signal} as the \emph{discrete command chain}. Following topics are introduced in this section:

\begin{enumerate}
    \item \emph{UAS Nonlinear Model} (sec. \ref{s:UASNonlinearModel}) - simple plane model used in this work as \emph{controlled plant}.
    
    \item \emph{Movement Automaton} (sec. \ref{s:movementAutomatonDefinition}) - movement automaton for \emph{UAS Nonlinear Model} constructed from scratch.
    
    \item \emph{Segmented Movement Automaton} (sec. \ref{s:segmentedMovementAutomaton}) - for more complex systems the \emph{State Space} can be \emph{separated into Segments} and \emph{segment movement automaton} is used to generate \emph{thick reference trajectory}.
    
    \item \emph{Reference Trajectory Generator} (sec. \ref{s:referenceTrajectoryGenerator}) - other use of \emph{Movement Automaton} as predictor for \emph{reference trajectory calculation}.
\end{enumerate}


\subsection{\secState{R}UAS Nonlinear Model}\label{s:UASNonlinearModel}
\paragraph{Motivation:} Simplified rigid body kinematic model will be used. This model have decoupled roll, yaw and pitch angles. The focus is on \emph{reach set approximation methods}, therefore \emph{UAS model} is simplified.

\paragraph{State Vector} (eq. \ref{eq:simple3dStatevector}) defined as positional state in euclidean position in right-hand euclidean space, where \emph{x, y, z} can be abstracted as latitude, longitude, altitude.
\begin{equation}\label{eq:simple3dStatevector}
    state = \left [ x,y,z, roll, pitch, yaw \right]^T
\end{equation}


\paragraph{Input Vector} (eq. \ref{eq:simple3dInputVector}) is defined as linear velocity of UAS $v$ and angular speed of rigid body $\omega_{roll}, \omega_{pitch},\omega_{yaw}$.

\begin{equation}\label{eq:simple3dInputVector}
    input = \left [ v, \omega_{roll}, \omega_{pitch},\omega_{yaw}\right ]^T
\end{equation}


\noindent Velocity distribution function (eq. \ref{eq:simple3dvelocityDistribution})  is is defined trough standard rotation matrix  and linear velocity $v$, oriented velocity [$v_x$, $v_y$, $v_z$] given by (eq. \ref{eq:UASNonlinearModelSimple}).

\begin{equation}\label{eq:simple3dvelocityDistribution}
    \begin{bmatrix}
    v_x\\
    v_y\\
    v_z\
    \end{bmatrix}
    =
    \begin{bmatrix}
         v\cos(pitch)\cos(yaw)\\
         v\cos(pitch)\sin(yaw)\\
         -v\sin(pitch)\\
    \end{bmatrix}
\end{equation}

\paragraph{UAS Nonlinear Model} (eq. \ref{eq:UASNonlinearModelSimple}) is given by \emph{first order equations:}

\begin{equation}\label{eq:UASNonlinearModelSimple}
    \begin{aligned}
        \frac{\partial x}{\partial time} &= v\cos(pitch)\cos(yaw);\\
        \frac{\partial y}{\partial time} &= v\cos(pitch)\sin(yaw);\\
        \frac{\partial z}{\partial time} &= -v\sin(pitch);\\
    \end{aligned}\\\quad\quad
    \begin{aligned}
        \frac{\partial roll}{\partial time} &= \omega_{roll};\\
        \frac{\partial pitch}{\partial time} &= \omega_{pitch};\\
        \frac{\partial yaw}{\partial time} &= \omega_{yaw};\\
    \end{aligned}
\end{equation}

\subsection{\secState{R}Movement Automaton for UAS Model}\label{s:movementAutomatonDefinition}

\paragraph{Motivation:} An \emph{UAS Nonlinear Model} (eq. \ref{eq:UASNonlinearModelSimple}) can be modeled by \emph{Movement Automaton} (def. \ref{def:movementAutomaton}). 

\paragraph{Movement Primitives} by (def. \ref{def:MovementPrimitive})  are given as (eq. \ref{eq:movementPrimitive}). To define primitives the \emph{minimal time} is $1 s$. The \emph{maximal duration} is also $1s$. 

\begin{assumption}\label{ass:transitionTime}
    Let assume that \emph{transition time} of \emph{roll, pitch, yaw, linear velocity} is $0 s$.
\end{assumption}

Under the assumption (as. \ref{ass:transitionTime}) the \emph{movement transitions} (def. \ref{def:movementTransition}) have $0$ duration.

\begin{note}
    The assumption (as. \ref{ass:transitionTime}) can be relaxed under condition that \emph{path tracking controller exists}.
\end{note}

\paragraph{Movements} (def. \ref{def:Movement}) for \emph{fixed step} $k$ we start with discretization of the input variables.

\noindent The \emph{linear velocity} in text step is given:
\begin{equation}\label{eq:applyMovement}
    v(k+1) = v(k) +\delta v(k)
\end{equation}

\noindent The \emph{roll, pitch, yaw} for next step are given 

\begin{equation}\label{eq:applyMovement1}
    \begin{aligned}
        roll(k+1)  &= roll(k) + \delta roll(k)\\
        pitch(k+1) & = pitch(k) + \delta pitch(k)\\
        yaw(k+1) & = yaw(k) + \delta yaw(k)\\
    \end{aligned}    
\end{equation}

\noindent The $\delta v(k)$ is \emph{velocity change}, $\delta roll(k)$, $\delta pitch(k)$, $\delta yaw(k)$, are \emph{orientation changes} for current discrete step $k$. If the duration of \emph{transition} is $0 s$ (as. \ref{ass:transitionTime}) then 3D trajectory evolution in discrete time is given as: 

\begin{equation}\label{eq:applyMovement2}
    \begin{aligned}
        x(k+1)&= x(k) + v(k+1) \cos(pitch(k+1)) \cos(yaw(k+1)) & = \delta x(k)\\
        y(k+1)&= y(k) + v(k+1) \cos(pitch(k+1)) \sin(yaw(k+1)) & = \delta y(k)\\
        z(k+1)&= z(k) - v(k+1) \sin(pitch(k+1))                & = \delta z(k)\\
        time(k+1)& = time(k)+1                                & = \delta time(k)
    \end{aligned}    
\end{equation}

\noindent The $\delta x(k)$, $\delta y(k)$, $\delta z(k)$ are positional differences depending on \emph{input vector} for given discrete time $k$:
\begin{equation}\label{eq:ourImput}
    input(k) = \left[
                    \begin{gathered}
                    \delta x(k), \delta y(k), \delta z(k), \delta v (k),\\
                    \delta roll (k), \delta pitch(k), \delta yaw(k),\delta time (k)
                    \end{gathered} 
                \right]^T
\end{equation}

\noindent The \emph{state vector} for discrete time is given:
\begin{equation}\label{eq:ourState}
    state(k) = \left[
                    \begin{gathered}
                     x(k),  y(k),  z(k),  v (k),\\
                     roll (k),  pitch(k),  yaw(k), time (k)
                    \end{gathered} 
                \right]^T
\end{equation}

\noindent The nonlinear model (eq. \ref{eq:UASNonlinearModelSimple}) is then reduced to \emph{linear discrete model} (eq. \ref{eq:uasLinearDiscreteModel}) given by \emph{apply movements} function (eq. \ref{eq:applyMovement}, \ref{eq:applyMovement1}, \ref{eq:applyMovement2}).

\begin{equation}\label{eq:uasLinearDiscreteModel}
    state(k+1) = applyMovement(state(k), input(k)) 
\end{equation}

\paragraph{Movement Set} for linear discrete model (eq. \ref{eq:uasLinearDiscreteModel}) is defined as set of extreme unitary movements on main axes (tab. \ref{tab:movements1}) and diagonal axes (tab. \ref{tab:movements2}).

\begin{table}[H]
    \centering
    \begin{tabular}{r||r|r|r|r|r}
    
        $input(movement)$           &    Straight  & Down & Up & Left  & Right   \\\hline\hline
        $\delta     x(k)[m]$           &    1.00	  & 0.98  & 0.98  & 0.98 & 0.98  \\\hline
        $\delta     y(k)[m]$           &    0	      & 0	  & 0	  & 0.13 & -0.13 \\\hline
        $\delta     z(k)[m]$           &    0	      & -0.13 & 0.13  &	0	 & 0     \\\hline
        $\delta  roll(k) [^\circ]$	   &    0	      & 0	  & 0	  & 0    & 0     \\\hline
        $\delta pitch(k) [^\circ]$     &    0	      & $15^\circ$  & -$15^\circ$ & 0	 & 0     \\\hline
        $\delta   yaw(k) [^\circ]$     &    0	      & 0	  & 0	  & $15^\circ$ & -$15^\circ$ \\
    \end{tabular}
    \caption{Input values for main axes movements.}
    \label{tab:movements1}
\end{table}
\begin{table}[H]
    \centering
    \begin{tabular}{r||r|r|r|r}
        $input(movement)$             & Down-Left & Down-Right & Up-Left  & Up-Right   \\\hline\hline
        $\delta     x(k)[m]$           & 0.76  & 0.76  & 0.76 & 0.76  \\\hline
        $\delta     y(k)[m]$           & -0.13	& 0.13	& 0.13 & -0.13 \\\hline
        $\delta     z(k)[m]$           & -0.13 & -0.13 & 0.13 & 0.13  \\\hline
        $\delta  roll(k) [^\circ]$	& 0	    & 0	    & 0    & 0     \\\hline
        $\delta pitch(k) [^\circ]$     & -$15^\circ$ & -$15^\circ$ & $15^\circ$ & $15^\circ$     \\\hline
        $\delta   yaw(k) [^\circ]$    & $15^\circ$	& -$15^\circ$	& $15^\circ$ & -$15^\circ$ \\
    \end{tabular}
    \caption{Input values for diagonal axes movements.}
    \label{tab:movements2}
\end{table}

\begin{note}
    \emph{Movement set} in shorten form is given as
    \begin{equation}\label{eq:OurMovementSet}
        Movement Set= \left\{
        \begin{gathered}
            Straight, Left,Right, Up, Down,\\
            Down Left, Down Right,  Up Left,   Up Right
        \end{gathered}
        \right\}
    \end{equation}
\end{note}

\paragraph{Trajectory} by (def. \ref{def:MovementAutomatonTrajectory}) for initial time $time = 0$ , initial state $state(0)$ and \emph{Movement Buffer} (from def. \ref{def:MovementBuffer}):
\begin{equation}\label{eq:ourBuffer}
    Buffer \in Movement Set^* (eq. \ref{eq:OurMovementSet}), \quad  |Buffer| \in \N
\end{equation}

\noindent Trajectory (eq. \ref{eq:ourTrajectoryImplementation}) is then given as the time-series of discrete states:
\begin{equation}\label{eq:ourTrajectoryImplementation}
    Trajectory(state(0),Buffer)= \left\{\begin{gathered}state(0)+\sum_{j=0}^{i-1} input(movement(j)):\\i \in\left\{1\dots |Buffer|+1\right\}, \\movement(\cdot) \in Buffer\end{gathered}\right\}
\end{equation}

\noindent Trajectory (eq. \ref{eq:ourTrajectoryImplementation}) is ordered set of states bounded to discrete time $0\dots n$ , where $n$ is member count of \emph{Buffer}. Trajectory set has $n+1$ members:

\begin{equation}
    \begin{aligned}
    T&rajectory(state(0),Buffer)=\\
        &\left\{
        \begin{aligned}
            state(0) &= state(0) + \{\}\\
            state(1) &= state(0) + input(movement(1))\\
            state(2) &= state(0) + input(movement(1)) +input(movement(2))\\
             \vdots  &= \vdots\\
            state(n) &= state(0) + input(movement(1))+\dots+input(movement(n))\\
        \end{aligned}
        \right\}
    \end{aligned}
\end{equation}

\paragraph{State Projection} (eq. \ref{eq:ourStateProjection}) for the \emph{Trajectory} (eq. \ref{eq:ourTrajectoryImplementation}) is given as follow:
\begin{equation}\label{eq:ourStateProjection}
    StateProjection(Trajectory,time) = Trajectory.getMemberByIndex(time+1)
\end{equation}

\begin{note}
    \emph{Movement Automaton} for system (eq. \ref{eq:UASNonlinearModelSimple}) with given (as. \ref{ass:transitionTime}) is established with all related properties (sec. \ref{def:movementAutomaton}).
\end{note}

\newpage
\subsection{\secState{R}Segmented Movement Automaton}\label{s:segmentedMovementAutomaton}
\paragraph{Motivation:} Constructing \emph{Movement Automaton} for more complex system can be tedious. Used \emph{Movement Automaton} for \emph{UAS system} (\ref{eq:UASNonlinearModelSimple}) has decoupled control which is not true for most of the copters/planes \cite{fossen2011mathematical}.

\paragraph{Partitioning UAS State Space:} Proposed movement automaton is defined by its Movement set (tab. \ref{tab:movements1},\ref{tab:movements2}). Those can be scaled depending on maneuverability in the  \emph{Initial state} $state(0)$:
\begin{enumerate}
    \item \emph{Climb/Descent Rate} $\delta pitch_{max}(k)$ - the maximal climb or descent rate for Up/Down movements.
    \item \emph{Turn Rate} $\delta yaw_{max}(k)$ - the maximal turn rate for Left/Right movement.
    \item \emph{Acceleration} $\delta v_{max}(k)$ - the maximal acceleration in cruising speed range.
\end{enumerate}

\begin{definition}{State Space partition}\label{def:stateSpacePartition}
    \emph{Maneuverability} is depending on \emph{Initial State}. There can not be the infinite count of \emph{Movement Automatons}.
    
    The state space $State Space \in \R^n$ can be separated into two exclusive subsets:
    \begin{equation}
        StateSpace = [ImpactStates, NonImpactingStates]
    \end{equation}
    
    The \emph{Impacting states} are states which bounds the \emph{Maneuverability}: $\delta pitch_{max}(k)$, $\delta yaw_{max}(k)$, $\delta v_{max}(k)$. For each \emph{impact state} is possible to define upper and lower boundary:
    \begin{multline}
        \forall impactState\in ImpactStates, \exists:\\ lower(impactState) \le value(impactState) \le upper(impactState) 
    \end{multline}
    
	\noindent    The bounded interval of impact state can be separated into distinctive \emph{impact state segments} like follow:
    \begin{multline}
        impactState\in [lower,upper]:\\ \{[lower,separator_1[\dots\cup\dots[separator_i,separator_{i+1}[\dots\cup\dots\\\dots\cup\dots[separator_n,upper]\}=\\
        = impactStateIntervals(impactState)
    \end{multline}
    \begin{note}
        The interval length depends on model dynamics. The rule of thumb is to keep maximal climb/descend/turn/acceleration rates near constant value. 
    \end{note}
        
\noindent When partitioning of \emph{all impact States} finishes, the count of partitions is given as product of \emph{count of partitions} for each member of \emph{Impact States}:
    
    \begin{equation}
        partition Count = \prod_{impactState \in}^{ImpactStates} |impactStateIntervals(impactState)| 
    \end{equation}
    
    \begin{note}
        Try to keep the count of partitions to minimum, each new interval increases the count of partitions geometrically. 
    \end{note}
    
    There is finite number $n$ of \emph{Impacting States}, these are separated into $impactState-$ $Intervals_i$ with respective index $i \in 1\dots n$. The \emph{segment} with index defining position used \emph{impacting state} intervals is given as \emph{constrained space}:
    
    \begin{equation}
        Segment(index) = \left[
            \begin{gathered}
                impactState_1 \in impactStateIntervals_1[index_1],\\
                \vdots\\
                impactState_n \in impactStateIntervals_n[index_n],\\
                \vdots\\
                NonImpactingStates    
                \end{gathered}\right]
    \end{equation}
    
    Each \emph{Segment} covers one of impacting state intervals combination, because the original intervals are exclusive, also \emph{Segments} are exclusive. The \emph{union} of all segments covers \emph{State Space}:
    
    \begin{equation}\label{eq:segmentedStateSpace}
        StateSpace = \bigcup_{\forall\quad index \in |impactStateIntervals|^n} Segment(index)
    \end{equation}
\end{definition}

\paragraph{Segmented Movement Automaton:} The segmentation of \emph{state space} is done  in (def. \ref{def:stateSpacePartition}) any \emph{state} belongs exactly to \emph{Segment} of \emph{State Space}. For each \emph{Segment} in \emph{State Space}it is possible to assess: \emph{Climb/Descent Rate} $\delta pitch_{max}(k)$, \emph{Turn Rate} $\delta yaw_{max}(k)$, and, \emph{Acceleration} $\delta v_{max}(k)$.


\begin{definition}{Movement Automaton for Segment(index)}\label{def:segmentMovementAutomaton}
    

\noindent For for Model(eq. \ref{eq:uasLinearDiscreteModel}) with State (eq. \ref{eq:ourState}) the input vector (eq. \ref{eq:ourImput}) is for position $[x,y,z]$ and velocity defined like: 

\begin{equation}
    \begin{aligned}
        \delta x(k)& = \left(v(k)+\delta v(k)\right) \cos(\delta pitch(k)) \cos(\delta yaw(k))\\
        \delta y(k)& = \left(v(k)+\delta v(k)\right) \cos(\delta pitch(k)) \sin(\delta yaw(k))\\
        \delta z(k)& = -\left(v(k)+\delta v(k)\right)\cos(\delta pitch(k))\\
        \delta v(k)&\in [-\delta v(k)_{max},\delta v(k)_{max}]
    \end{aligned}
\end{equation}
\end{definition}

\noindent The acceleration $\delta v(k)$ is in interval $[-\delta v(k)_{max},\delta v(k)_{max}]$, usually set to 0 $ms^{-1}$. The change of the orientation angles for \emph{Movement Set} (eq. \ref{eq:OurMovementSet}) is given in (tab. \ref{tab:movements3},\ref{tab:movements4}).

\begin{table}[H]
    \centering
    \begin{tabular}{r||r|r|r|r|r}
    
        $input(movement)$           &    Straight  & Down & Up & Left  & Right   \\\hline\hline
        $\delta  roll(k) [^\circ]$	   &    0	      & 0	  & 0	  & 0    & 0     \\\hline
        $\delta pitch(k) [^\circ]$     &    0	      & $\delta pitch_{max}$  & -$\delta pitch_{max}$ & 0	 & 0     \\\hline
        $\delta   yaw(k) [^\circ]$     &    0	      & 0	  & 0	  & $\delta yaw_{max}$ & -$\delta yaw_{max}$ \\
    \end{tabular}
    \caption{Orientation input values for main axes movements.}
    \label{tab:movements3}
\end{table}


\begin{table}[H]
    \centering
    \begin{tabular}{r||r|r|r|r}
        $input(movement)$             & Down-Left & Down-Right & Up-Left  & Up-Right   \\\hline\hline

        $\delta  roll(k) [^\circ]$	& 0	    & 0	    & 0    & 0     \\\hline
        $\delta pitch(k) [^\circ]$     & -$\delta pitch_{max}$ & -$\delta pitch_{max}$ & $\delta pitch_{max}$ & $\delta pitch_{max}$     \\\hline
        $\delta   yaw(k) [^\circ]$    & $\delta yaw_{max}$	& -$\delta yaw_{max}$	& $\delta yaw_{max}$ & -$\delta yaw_{max}$ \\
    \end{tabular}
    \caption{Orientation input values for diagonal axes movements.}
    \label{tab:movements4}
\end{table}

\begin{note}
    The \emph{Trajectory} is calculated same as in (eq. \ref{eq:ourTrajectoryImplementation}). The \emph{State Projection} is given as in (eq. \ref{eq:ourStateProjection}).
\end{note}

\noindent Then the \emph{Movement Automaton} for \emph{Segment} $\in$ \emph{State Space} is defined.

\begin{definition}{Segmented Movement Automaton}\label{def:segmentedMovementAutomaton}
    For system with segmented state space (eq. \ref{eq:segmentedStateSpace}) there is for each $state(k)$ in $StateSpace$ injection function:
    \begin{equation} \label{eq:activeMovementAutomaton}
        ActiveMovementAutomaton:StateSpace\to MovementAutomaton
    \end{equation}
    
\noindent Selecting appropriate \emph{movement automaton} implementation (def. \ref{def:segmentMovementAutomaton}) for \emph{state(k)} $\in$ \emph{Segment} $\subset$ \emph{State Space}. The mapping function (eq. \ref{eq:activeMovementAutomaton}) is injection mapping every state(k) to Segment then \emph{Movement Automaton Implementation}. The trajectory generated is then given:
    
    \begin{equation}\label{eq:ourTrajectoryImplementationSegmented}
        Trajectory\left(\begin{gathered}state(0),\\Buffer\end{gathered}\right)= 
        \left\{
            \begin{gathered}
                state(0)+\dots\\\sum_{j=0}^{i-1} 
                    \begin{aligned} 
                        &ActiveMovementAutomaton(state(j-1)).\\
                        &\quad.input(movement(j))
                    \end{aligned}:\\
                i \in\left\{1\dots |Buffer|+1\right\}, \\
                movement(\cdot) \in Buffer
            \end{gathered}
        \right\}
    \end{equation}
    
\end{definition}

\newpage
\subsection{\secState{R}Reference Trajectory Generator}\label{s:referenceTrajectoryGenerator}

\paragraph{Reference Trajectory Generator:} Segmented Movement Automaton (def.  \ref{def:segmentedMovementAutomaton}) with \emph{trajectory function} (eq. \ref{eq:ourTrajectoryImplementationSegmented}) is used as \emph{reference trajectory generator} for \emph{complex systems}. 

There is assumption that precise \emph{path tracking} implementation exist for such system which with \emph{thick reference trajectory} gives similar results to \emph{plain movement automaton control}.

The \emph{Reference trajectory} (eq. \ref{eq:generatedReferenceTrajectory}) for \emph{Planned} movement set is given as projection  of \emph{Trajectory} time series to position time series $[x,y,z,t]$:

\begin{equation}\label{eq:generatedReferenceTrajectory}
    Reference Trajectory:Trajectory\left(\begin{gathered}state(now),\\Planned\end{gathered}\right) 
    \to 
    \begin{bmatrix}
        x_{ref} \in \R^{|Planned|}\\
        y_{ref} \in \R^{|Planned|}\\
        z_{ref} \in \R^{|Planned|}\\
        t_{ref} \in \R^{|Planned|}
    \end{bmatrix}
\end{equation}

\paragraph{Predictor:} The \emph{Reference Trajectory Generator} (eq. \ref{eq:generatedReferenceTrajectory}) can be also used as predictor. 

\begin{note}
    The \emph{Segmented Movement Automaton} (def. \ref{def:segmentedMovementAutomaton}) is used in this work with one Segment equal to State space with input function given by (\ref{tab:movements1}, \ref{tab:movements2}). The predictor used in \emph{Reach set computation} is given by (eq. \ref{eq:generatedReferenceTrajectory}).
\end{note}

    	\subsection{\secState{R}Movement Automaton Applications}\label{sec:MovementAutomatonBackground}

    \noindent\emph{Movement Automaton} is basic interface approach for discretization of \emph{trajectory evolution}  or \emph{control input} for any \emph{continuous or discrete system model}.
    
    \emph{Main function} of \emph{Movement Automaton is} for system given by equation $\dot{state}=f(time,state,input)$ with initial state $state_0$ to generate \emph{reference trajectory} $\hat{state}(t)$ or \emph{control signal} $input(t)$.
    
    Using \emph{Movement Automaton} as \emph{Control Proxy} will provide us with \emph{discrete command chain} interface. This will reduce the \emph{non deterministic} element from \emph{Evasive trajectory} generation, by reducing infinite maneuver set to finite \emph{movement set}.
    
    \emph{Non determinism} of \emph{Avoidance Maneuver} have been discussed as an issue in following works:
    \begin{enumerate}
        \item Newton gradient method for evasive car maneuvers \cite{vsantin2011combined}.
        \item Non-holistic methods for trajectory generation \cite{pin1990autonomous}.
        \item Stochastic approach to elliptic trajectories generation \cite{andrzejak2001epileptic}.
    \end{enumerate}
    
\noindent\emph{Examples} of \emph{Movement Automaton Implementation} as \emph{Control Element} can be mentioned as follows:
    \begin{enumerate}
        \item Control of traffic flow \cite{kuwata2009real}.
        \item Complex air traffic collision situation resolution system  \cite{frazzoli2001robust,frazzoli2000trajectory}.
        \item SAA/DAA capable avoidance system \cite{gomola2017obstacle}.
    \end{enumerate}



		\subsection{\secState{R}UAS Model}\label{s:UASNonlinearModel}
\paragraph{Motivation:} Simplified rigid body kinematic model will be used. This model have decoupled roll, yaw and pitch angles. The focus is on \emph{reach set approximation methods}, therefore \emph{UAS model} is simplified.

\paragraph{State Vector} (eq. \ref{eq:simple3dStatevector}) defined as positional state in euclidean position in right-hand euclidean space, where \emph{x, y, z} can be abstracted as latitude, longitude, altitude.
\begin{equation}\label{eq:simple3dStatevector}
    state = \left [ x,y,z, roll, pitch, yaw \right]^T
\end{equation}


\paragraph{Input Vector} (eq. \ref{eq:simple3dInputVector}) is defined as linear velocity of UAS $v$ and angular speed of rigid body $\omega_{roll}, \omega_{pitch},\omega_{yaw}$.

\begin{equation}\label{eq:simple3dInputVector}
    input = \left [ v, \omega_{roll}, \omega_{pitch},\omega_{yaw}\right ]^T
\end{equation}


\noindent Velocity vector function (eq. \ref{eq:simple3dvelocityDistribution})  is is defined through standard rotation matrix  and linear velocity $v$, oriented velocity [$v_x$, $v_y$, $v_z$] given by (eq. \ref{eq:UASNonlinearModelSimple}).

\begin{equation}\label{eq:simple3dvelocityDistribution}
    \begin{bmatrix}
    v_x\\
    v_y\\
    v_z\
    \end{bmatrix}
    =
    \begin{bmatrix}
         v\cos(pitch)\cos(yaw)\\
         v\cos(pitch)\sin(yaw)\\
         -v\sin(pitch)\\
    \end{bmatrix}
\end{equation}

\paragraph{UAS Nonlinear Model} (eq. \ref{eq:UASNonlinearModelSimple}) is given by \emph{first order equations:}

\begin{equation}\label{eq:UASNonlinearModelSimple}
    \begin{aligned}
        \frac{\text{d} x}{\text{d} time} &= v\cos(pitch)\cos(yaw);\\
        \frac{\text{d} y}{\text{d} time} &= v\cos(pitch)\sin(yaw);\\
        \frac{\text{d} z}{\text{d} time} &= -v\sin(pitch);\\
    \end{aligned}\\\quad\quad
    \begin{aligned}
        \frac{\text{d} roll}{\text{d} time} &= \omega_{roll};\\
        \frac{\text{d} pitch}{\text{d} time} &= \omega_{pitch};\\
        \frac{\text{d} yaw}{\text{d} time} &= \omega_{yaw};\\
    \end{aligned}
\end{equation}

\paragraph{Discretization} for \emph{fixed step} $k$ we start with discretization of the input variables:

\noindent The \emph{linear velocity} in text step is given:
\begin{equation}\label{eq:applyMovement}
    v(k+1) = v(k) +\delta v(k)
\end{equation}

\noindent The \emph{roll, pitch, yaw} for next step are given 

\begin{equation}\label{eq:applyMovement1}
    \begin{aligned}
        roll(k+1)  &= roll(k) + \delta roll(k)\\
        pitch(k+1) & = pitch(k) + \delta pitch(k)\\
        yaw(k+1) & = yaw(k) + \delta yaw(k)\\
    \end{aligned}    
\end{equation}

\noindent The $\delta v(k)$ is \emph{velocity change}, $\delta roll(k)$, $\delta pitch(k)$, $\delta yaw(k)$, are \emph{orientation changes} for current discrete step $k$. If the duration of \emph{transition} is $0 s$ (as. \ref{ass:transitionTime}) then 3D trajectory evolution in discrete time is given as: 

\begin{equation}\label{eq:applyMovement2}
    \begin{aligned}
        x(k+1)&= x(k) + v(k+1) \cos(pitch(k+1)) \cos(yaw(k+1)) & = \delta x(k)\\
        y(k+1)&= y(k) + v(k+1) \cos(pitch(k+1)) \sin(yaw(k+1)) & = \delta y(k)\\
        z(k+1)&= z(k) - v(k+1) \sin(pitch(k+1))                & = \delta z(k)\\
        time(k+1)& = time(k)+1                                & = \delta time(k)
    \end{aligned}    
\end{equation}

\noindent The $\delta x(k)$, $\delta y(k)$, $\delta z(k)$ are positional differences depending on \emph{input vector} for given discrete time $k$:
\begin{equation}\label{eq:ourImput}
    input(k) = \left[
                    \begin{gathered}
                    \delta x(k), \delta y(k), \delta z(k), \delta v (k),\\
                    \delta roll (k), \delta pitch(k), \delta yaw(k),\delta time (k)
                    \end{gathered} 
                \right]^T
\end{equation}

\noindent The \emph{state vector} for discrete time is given:
\begin{equation}\label{eq:ourState}
    state(k) = \left[
                    \begin{gathered}
                     x(k),  y(k),  z(k),  v (k),\\
                     roll (k),  pitch(k),  yaw(k), time (k)
                    \end{gathered} 
                \right]^T
\end{equation}


		\subsection{\secState{R}UAS Movement Automaton}\label{s:movementAutomatonDefinition}

\paragraph{Motivation:} An \emph{UAS Nonlinear Model} (eq. \ref{eq:UASNonlinearModelSimple}) can be modeled by \emph{Movement Automaton} (def. \ref{def:movementAutomaton}). 

\paragraph{Movement Primitives} by (def. \ref{def:MovementPrimitive})  are given as (eq. \ref{eq:movementPrimitive}). To define primitives the \emph{minimal time} is $1 s$. The \emph{maximal duration} is also $1s$.

\begin{note}
	Each movement primitive will last for fixed duration $1s$.
\end{note} 

\begin{assumption}\label{ass:transitionTime}
    Let assume that \emph{transition time} of \emph{roll, pitch, yaw, linear velocity} is $0 s$.
\end{assumption}

Under the assumption (as. \ref{ass:transitionTime}) the \emph{movement transitions} (def. \ref{def:movementTransition}) have zero duration. Therefore movement primitives can be considered as movements.

\begin{note}
    The assumption (as. \ref{ass:transitionTime}) can be relaxed under condition that \emph{path tracking controller exists}.
\end{note}

\paragraph{Movements} satisfying (def. \ref{def:Movement}), for the nonlinear model (eq. \ref{eq:UASNonlinearModelSimple}) reduced to \emph{discrete model} (eq. \ref{eq:uasLinearDiscreteModel}), are given by \emph{apply movements} function (eq. \ref{eq:applyMovement}, \ref{eq:applyMovement1}, \ref{eq:applyMovement2}).

\begin{equation}\label{eq:uasLinearDiscreteModel}
    state(k+1) = applyMovement(state(k), input(k)) 
\end{equation}

\paragraph{Movement Set} for discrete model (eq. \ref{eq:uasLinearDiscreteModel}) is defined as set of unitary movements on main axes (tab. \ref{tab:movements1}) and diagonal axes (tab. \ref{tab:movements2}). 

The maneuvering capability of several commercial small fixed wing UAS were merged together. The turning rate on horizontal/vertical is defined as $15^\circ$.

The deltas are posed in \emph{UAS body-fixed coordinate frame} (ap. \ref{sec:complementsOfAlgebra}) for discrete time $k$. 

\begin{table}[H]
    \centering
    \begin{tabular}{r||r|r|r|r|r}
    	\multirow{2}{*}{Parameter} & \multicolumn{5}{c}{Movement} \\\cline{2-6} 
                &    Straight  & Down & Up & Left  & Right   \\\hline\hline
        $\delta     x(k)[m]$           &    1.00	  & 0.98  & 0.98  & 0.98 & 0.98  \\\hline
        $\delta     y(k)[m]$           &    0	      & 0	  & 0	  & 0.13 & -0.13 \\\hline
        $\delta     z(k)[m]$           &    0	      & -0.13 & 0.13  &	0	 & 0     \\\hline
        $\delta  roll(k) [^\circ]$	   &    0	      & 0	  & 0	  & 0    & 0     \\\hline
        $\delta pitch(k) [^\circ]$     &    0	      & $15^\circ$  & -$15^\circ$ & 0	 & 0     \\\hline
        $\delta   yaw(k) [^\circ]$     &    0	      & 0	  & 0	  & $15^\circ$ & -$15^\circ$ \\
    \end{tabular}
    \caption{Input values for main axes movements.}
    \label{tab:movements1}
\end{table}
\begin{table}[H]
    \centering
    \begin{tabular}{r||r|r|r|r}
    	\multirow{2}{*}{Parameter} & \multicolumn{4}{c}{Movement} \\\cline{2-5} 
                    & Down-Left & Down-Right & Up-Left  & Up-Right   \\\hline\hline
        $\delta     x(k)[m]$           & 0.76  & 0.76  & 0.76 & 0.76  \\\hline
        $\delta     y(k)[m]$           & -0.13	& 0.13	& 0.13 & -0.13 \\\hline
        $\delta     z(k)[m]$           & -0.13 & -0.13 & 0.13 & 0.13  \\\hline
        $\delta  roll(k) [^\circ]$	& 0	    & 0	    & 0    & 0     \\\hline
        $\delta pitch(k) [^\circ]$     & -$15^\circ$ & -$15^\circ$ & $15^\circ$ & $15^\circ$     \\\hline
        $\delta   yaw(k) [^\circ]$    & $15^\circ$	& -$15^\circ$	& $15^\circ$ & -$15^\circ$ \\
    \end{tabular}
    \caption{Input values for diagonal axes movements.}
    \label{tab:movements2}
\end{table}

\begin{note}
    \emph{Movement set} in shorten form is given as
    \begin{equation}\label{eq:OurMovementSet}
        Movement Set= \left\{
        \begin{gathered}
            Straight, Left,Right, Up, Down,\\
            Down Left, Down Right,  Up Left,   Up Right
        \end{gathered}
        \right\}
    \end{equation}
\end{note}

\paragraph{Trajectory} by (def. \ref{def:MovementAutomatonTrajectory}) for initial time $time = 0$ , initial state $state(0)$ and \emph{Movement Buffer} (from def. \ref{def:MovementBuffer}):
\begin{equation}\label{eq:ourBuffer}
    Buffer = \left\{
                movement(j):
                \begin{aligned}
                    &movement(j)\in Movement Set (eq. \ref{eq:OurMovementSet}),\\
                    & j \in 1\dots n, n \in N^+
                \end{aligned}
            \right\}
\end{equation}

\begin{assumption}
    The buffer is always non-empty, ordered, finite list of \emph{movements}.
\end{assumption}

\begin{note}
  The buffer have finite count $n$ of movements stored. The buffer is the planning instrument used by higher level navigation/avoidance algorithm to control UAS (Control/Command interface) (fig. \ref{fig:AvoidanceFrameworkConceptNew}).
\end{note}

%\noindent Trajectory (eq. \ref{eq:ourTrajectoryImplementation}) is then %given as the time-series of discrete states:
%\begin{equation}\label{eq:ourTrajectoryImplementation}
%    Trajectory(state(0),Buffer)= \left\{\begin{gathered}state(0)+\sum_{j=0}%^{i-1} input(movement(j)):\\i \in\left\{1\dots |Buffer|+1\right\}, \%\movement(\cdot) \in Buffer\end{gathered}\right\}
%\end{equation}

The discrete trajectory (eq. \ref{eq:ourTrajectoryImplementation}) is ordered set of states bounded to discrete time $0\dots n$, where $n$ is movement count of \emph{Buffer}. Trajectory set has $n+1$ members defined like the following:

\begin{equation}
    \begin{aligned}
    T&rajectory(state(0),Buffer)=\\
        &\left\{
        \begin{aligned}
            state(0) &= state(0),\\
            state(1) &= apply Movement\left(state(0), movement(1)\right),  \\
            state(2) &= apply Movement\left(state(1), movement(2)\right),  \\
             \vdots  &= \vdots\\
            state(n-1) &= apply Movement\left(state(n-2), movement(n-1)\right),  \\
            state(n)   &= apply Movement\left(state(n-1), movement(n)\right)  \\
        \end{aligned}
        \right\}
    \end{aligned}
\end{equation}

\noindent The $apply Movement$ (eq. \ref{eq:uasLinearDiscreteModel}) is concatenation function for discrete UAS model.
	
%% This adds a line for the Bibliography in the Table of Contents.
%\addcontentsline{toc}{chapter}{Bibliography}
%% *** Set the bibliography style. ***
%% (change according to your preference/requirements)
%\bibliographystyle{plain}
%% *** Set the bibliography file. ***
%% ("thesis.bib" by default; change as needed)
\bibliography{thesis}

%% *** NOTE ***
%% If you don't use bibliography files, comment out the previous line
%% and use \begin{thebibliography}...\end{thebibliography}.  (In that
%% case, you should probably put the bibliography in a separate file and
%% `\include' or `\input' it here).

\end{document}
