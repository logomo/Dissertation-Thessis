%%%%% Single page layout:
%%%%% ----------------------------------------------------
\documentclass[12pt, a4paper]{report}
\setlength\textwidth{160mm}
\setlength\textheight{247mm}
\setlength\oddsidemargin{0mm}
\setlength\evensidemargin{0mm}
\setlength\topmargin{0mm}
\setlength\headsep{0mm}
\setlength\headheight{0mm}
\let\openright=\clearpage
\usepackage[utf8]{inputenc}


%%% Additional useful packages
%%% ----------------------------------------------------------------
\usepackage{array}
\usepackage{amsmath}  
\usepackage{amssymb}
\usepackage{amsfonts}
\DeclareFontFamily{OT1}{pzc}{}
\DeclareFontShape{OT1}{pzc}{m}{it}{<-> s * [0.900] pzcmi7t}{}
\DeclareMathAlphabet{\mathpzc}{OT1}{pzc}{m}{it}
\usepackage{amsthm}      
\usepackage[ruled,algochapter]{algorithm2e}
\usepackage{algorithmic}
\usepackage{filecontents}
\usepackage{bibentry}
\usepackage{bm}
\usepackage[mathscr]{euscript}
\usepackage{graphicx}       
\usepackage{psfrag}         
\usepackage{fancyvrb}    
\usepackage{ifdraft}
\usepackage{float}
\usepackage{ltablex}
\usepackage[square,sort,comma,numbers]{natbib}        
\usepackage{bbding}         
\usepackage{dcolumn}        
\usepackage{booktabs} 
\usepackage{multirow}
\usepackage{paralist}       
\usepackage{indentfirst}    
\usepackage[nottoc,notlof,notlot]{tocbibind}
\usepackage{url}
\usepackage{tabularx}
\usepackage{subcaption}
\usepackage[unicode]{hyperref}
\usepackage{xcolor}
\hypersetup{pdftitle=LiDAR obstacle detection and avoidance, 
            pdfauthor=Alojz Gomola,
            colorlinks=false,
            urlcolor=blue,
            pdfstartview=FitH,
            pdfpagemode=UseOutlines,
            pdfnewwindow,
            breaklinks
          }
\usepackage{array}
\newcolumntype{L}[1]{>{\raggedright\let\newline\\\arraybackslash\hspace{0pt}}m{#1}}
\newcolumntype{C}[1]{>{\centering\let\newline\\\arraybackslash\hspace{0pt}}m{#1}}
\newcolumntype{R}[1]{>{\raggedleft\let\newline\\\arraybackslash\hspace{0pt}}m{#1}}         
\newcolumntype{B}{X}
\newcolumntype{S}[1]{>{\hsize=#1\textwidth}X}
\newcommand{\FIGDIR}{./Pics}    %%% directory containing figures
\newcommand{\twolinecellr}[2][r]{%
  \begin{tabular}[#1]{@{}r@{}}#2\end{tabular}}
\newcommand{\TBD}[1]{\color{red}\emph{--TBD:}#1\color{black}}
\theoremstyle{plain}
\newtheorem{theorem}{Theorem}
\newtheorem{lemma}[theorem]{Lemma}
\newtheorem{proposition}[theorem]{Proposition}

\theoremstyle{plain}
\newtheorem{definition}{Definition}
\newtheorem{problem}{Problem}
\newtheorem{example}{Example}
\newtheorem{assumption}{Assumption}

\theoremstyle{remark}
\newtheorem*{corollary}{Corollary}
\newtheorem*{note}{Note}




\newenvironment{dokaz}{
  \par\medskip\noindent
  \textit{Proof}.
}{
\newline
\rightline{\SquareCastShadowBottomRight}
}


%\bibliographystyle{plainnat}     %% Author (year) style
\bibliographystyle{unsrt}        %% [number] style
\setcitestyle{square}


\title{Dissertation thesis}
\author{Alojz Gomola}
\date{February 2019}

%%%%% ------------------------------------------------------------
\DefineVerbatimEnvironment{PCinout}{Verbatim}{fontsize=\small, frame=single}



\newcommand{\R}{\mathbb{R}}
\newcommand{\N}{\mathbb{N}}

\DeclareMathOperator{\pr}{\textsf{P}}
\DeclareMathOperator{\E}{\textsf{E}\,}
\DeclareMathOperator{\var}{\textrm{var}}
\DeclareMathOperator{\sd}{\textrm{sd}}


\newcommand{\T}[1]{#1^\top}        

\newcommand{\goto}{\rightarrow}
\newcommand{\gotop}{\stackrel{P}{\longrightarrow}}
\newcommand{\maon}[1]{o(n^{#1})}
\newcommand{\abs}[1]{\left|{#1}\right|}
\newcommand{\dint}{\int_0^\tau\!\!\int_0^\tau}
\newcommand{\isqr}[1]{\frac{1}{\sqrt{#1}}}
\newcommand{\norm}[1]{\left\lVert#1\right\rVert}


\newcommand{\pulrad}[1]{\raisebox{1.5ex}[0pt]{#1}}
\newcommand{\mc}[1]{\multicolumn{1}{c}{#1}}
\newcommand{\secState}[1]{
	\ifdraft{(#1) }{}
}
\begin{document}
\paragraph{Self Separation Property:} There is two UAS flying in open air. The 

\begin{equation*}
    collision Distance (UAS_1,UAS_2) \le well Clear Margin
\end{equation*} 

for some 
\begin{equation*}
    time Point \in Future(currect Time)    
\end{equation*}


If both UAS are equipped with position notification and future collision is detected within 

\begin{equation*}
    well Clear Margin < decision Margin(UAS_1,UAS_2) < detection Margin    
\end{equation*}

The both UAS will avoid \emph{well clear breach}.
\begin{note}
    The \emph{decision Margin} is minimal mutual distance when both UAS have manuevering capability to avoid each other.
    
    \begin{equation*}
        \exists trajectory \in \left(reach set Approximation(UAS_i) \cup constraint(UAS_j) \right)
    \end{equation*}
\end{note}

\noindent Reasoning and proof outline:
\begin{enumerate}
    \item The \emph{self-separation} is most important in controlled/non-controlled airspace.
    
    \item It is mandatory for an aircraft to have self separation and required awareness (sensors, IFR equipment) to be considered airworthy.
    
    \item The proof is simple, refer to \emph{Head on approach test case}, for each decision frame there exists a feasible trajectory to avoid an intruder with sufficient \emph{well clear margin}.
    
    \item The condition of \emph{awareness} is mandatory.
\end{enumerate}

\paragraph{Trajectory Optimally:}  The trajectory generated by movement automaton is optimal for given time frame and sub-optimal for mission

\emph{Proof:}For any mission given by feasible waypoints:

\begin{equation*}
    waypoint_1, \dots, waypoint_n
\end{equation*}

The \emph{Every decision frame} one trajectory from feasible \emph{trajectories in reach set approximation is selected}. The \emph{optimal trajectory} is flown \emph{until next decision point}. The final trajectory is joint of partial trajectories between decision points:

\begin{equation*}
    Mission Trajectory(state_0,B) = \bigcup_{i=1\dots|Decision Points|} Trajectory(state_i,buffer_i) 
\end{equation*}

The \emph{trajectory segment}

\begin{equation*}
    Trajectory(state_i,buffer_i) 
\end{equation*}

is optimal  for \emph{navigation problem} where \emph{goal} is given for \emph{decision time}. The joint trajectory is therefore sub-optimal as: 

\begin{equation*}
    \bigcup_{i=1\dots|Decision Points|}goal(decision Time _i)  = waypoint_1, \dots, waypoint_n
\end{equation*}

\noindent \emph{Reasoning:} The \emph{open-end navigation} with \emph{finite prediction horizon} (end of the Avoidance/Navigation Grid) can not generate the optimal trajectory for any cost function (because the decision does not account future event or occurrences). 

This property is important, because nobody will use a method which cost a lot of energy to maintain, cost \& benefit analysis will dismantle such method.

\paragraph{Low calculation Complexity:} The computation footprint is low, there is already theoretical proof and practical testing in place.

\paragraph{Scalability:} The approach can be used for any type of aircraft with \emph{Minimal Operational Performance specification} (MOPS): for other aircraft models the movement automaton generates reference trajectory, the approach is scalable in terms of:
\begin{itemize}
    \item Avoidance grid scaling - active sensor/operational space
    \item state ranges - the difference for each initial state group - sensitive model parameters
    \item maximal turning horizontal/vertical radius
    \item required climb/descent/turning rates
    \item velocity range
\end{itemize}

\emph{proof:} This is shown trough dissertation thesis, formal proof of \emph{Reach set linear growth} with \emph{avoidance grid scaling} is required.

\paragraph{Avoidance Priority Customization:} The current implementation of implements following priority listing:
\begin{itemize}
    \item Static obstacles
    \item Intruders
    \item Hard constraints
    \item Soft constraints
\end{itemize}

\noindent\emph{Reasoning:} The \emph{threat prioritization} is sought after business feature. The current prioritization is based on real air-traffic assumptions:
\begin{itemize}
    \item The \emph{static obstacles} does not have any self-separation capability, they can not avoid our UAS.
    
    \item The \emph{intruder} is not equal to \emph{static obstacles}. Each intruder have some \emph{operational performance} which is limited by the \emph{Minimal Operational Performance Standards} (MOPS). Sadly, there is no \emph{Maximal allowed operational performance}.
    
    Therefore aircraft's outclass each other. This is why the \emph{overtake maneuver} is tested using the \emph{speed difference ratio}. There is also expectation that higher maneuverability intruder will avoid others. The active avoidance is expected behaviour. If the \emph{UAS} is outclassed by intruder it should keep the original heading to ease avoidance done by intruder.
    
    \item The \emph{Hard constraints} namely:
    
    \begin{itemize}
        \item Geo-fencing areas 
        \item Severe weather conditions
        \item Private airspace
        \item Restricted airspace portions, by \emph{National/Local Aviation authority}
    \end{itemize}
    
    are considered as equal in terms of avoidance and there is a possibility of intersection. The good \emph{flight plan} avoids geo-fencing areas and restricted airspace portions preemptively. The dynamic occurrences of \emph{private airspace portions} (military training, special reservation) and \emph{several weather cases} occurs over time and can not be preemptively incorporated into flight plan.
    
    The \emph{severe weather conditions} are disputable, because the flight plan is approved only when the weather is sustainable for flight in mission adjacent areas. The \emph{weather impact} is more severe on UAS of all categories, because of the low take of weight and low propelling output. 
    
    \item The \emph{Soft constraints} are breakable, but it should be respected, namely:
    \begin{itemize}
        \item flyable weather conditions,
        \item various protection zones,
    \end{itemize}
    These items are to long to prioritize i will omit them
\end{itemize}

\noindent \emph{Notes:} These prioritization is stated and established in 6.6, there is nothing to prove

\paragraph{Other Properties:}
\begin{itemize}
    \item \emph{Static obstacle avoidance} - under specific conditions the UAS will not crash into static obstacle. Static obstacle avoidance is just another variation of self separation, instead intruders the static obstacles are present. 
    
    \item \emph{Maze solving capabilities} - see navigation issues,
    
    \item \emph{Dynamic waypoints} - see navigation issues,
\end{itemize}

\emph{General Navigation Issues:} I believe the key focus of this work was a \emph{Reach sets}, the properties of the reach sets are self containing, therefore no formal proof is required. The existence of avoidance path within one time frame is organic and easy to prove as I mentioned earlier.

The problem of real avoidance capability lies within \emph{navigation algorithm} which is not great in my work (but sufficient to cover required functionality). The properties over multiple decision frames are hard to prove. The proof based on infinite avoidance grid is possible, but is not giving any additional value.

Therefore the navigation related properties of the approach should be taken into account with a "grain of salt".
%\bibliography{thesis}
\end{document}
