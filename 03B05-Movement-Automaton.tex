\section{\secState{D}Hybrid Automaton}\label{s:HybridAutomaton}
    \noindent First the notion of  \emph{hybrid} automaton  \cite{lazar2006model,borrelli2006mpc,daws1996tool} needs to be introduced:

    \begin{definition}{Hybrid automaton} (\ref{eq:hybridAutomaton}) is given as structure:
        \begin{equation}\label{eq:hybridAutomaton}
        \begin{aligned}
            HybridAutomaton(&States,SystemState,VectorField,\\
                            &DiscreteTransition,ResetMap)
        \end{aligned}
        \end{equation}
    
        \emph{States} ($Q$) is given as set of discrete states, for every time $t\in Domain$ hybrid automaton stays in exactly one of \emph{states}.
    
        \emph{SystemState} ($x$), is given in domain $x\in\R^n,n\in\N^+$, representing the trajectory evolution.
        
        \emph{VectorField} ($f$) (\ref{eq:vectorField}) is bounded to single $State\in States$ and represents local SystemState evolution, when given automaton State is Active.
        
        \begin{equation}\label{eq:vectorField}
            VectorField: State\times SystemState \to SystemState
        \end{equation}
        
        \emph{DiscreteTransition} ($\varphi$) (eq. \ref{eq:discreteTransition}) indicates changes of states in automaton, the changes are triggered by satisfying specific condition given by State and SystemState. 
        
        \begin{equation}\label{eq:discreteTransition}
            DiscreteTransition:State\times SystemState \to State
        \end{equation}
        
        \emph{ResetMap} ($\rho$) (eq. \ref{eq:resetMap}) defines changes of State to some default value, this change is triggered by specific automaton State and SystemState.
        
        \begin{equation}\label{eq:resetMap}
            ResetMap:State\times SystemState \to SystemState
        \end{equation}
    
    \end{definition}
