%% fcup-thesis.tex -- document template for PhD theses at FCUP
%%
%% Copyright (c) 2015 João Faria <joao.faria@astro.up.pt>
%%
%% This work may be distributed and/or modified under the conditions of
%% the LaTeX Project Public License, either version 1.3c of this license
%% or (at your option) any later version.
%% The latest version of this license is in
%%     http://www.latex-project.org/lppl.txt
%% and version 1.3c or later is part of all distributions of LaTeX
%% version 2005/12/01 or later.
%%
%% This work has the LPPL maintenance status "maintained".
%%
%% The Current Maintainer of this work is
%% João Faria <joao.faria@astro.up.pt>.
%%
%% This work consists of the files listed in the accompanying README.

%% SUMMARY OF FEATURES:
%%
%% All environments, commands, and options provided by the `ut-thesis'
%% class will be described below, at the point where they should appear
%% in the document.  See the file `ut-thesis.cls' for more details.
%%
%% To explicitly set the pagestyle of any blank page inserted with
%% \cleardoublepage, use one of \clearemptydoublepage,
%% \clearplaindoublepage, \clearthesisdoublepage, or
%% \clearstandarddoublepage (to use the style currently in effect).
%%
%% For single-spaced quotes or quotations, use the `longquote' and
%% `longquotation' environments.


%%%%%%%%%%%%         PREAMBLE         %%%%%%%%%%%%

%%  - Default settings format a final copy (single-sided, normal
%%    margins, one-and-a-half-spaced with single-spaced notes).
%%  - For a rough copy (double-sided, normal margins, double-spaced,
%%    with the word "DRAFT" printed at each corner of every page), use
%%    the `draft' option.
%%  - The default global line spacing can be changed with one of the
%%    options `singlespaced', `onehalfspaced', or `doublespaced'.
%%  - Footnotes and marginal notes are all single-spaced by default, but
%%    can be made to have the same spacing as the rest of the document
%%    by using the option `standardspacednotes'.
%%  - The size of the margins can be changed with one of the options:
%%     . `narrowmargins' (1 1/4" left, 3/4" others),
%%     . `normalmargins' (1 1/4" left, 1" others),
%%     . `widemargins' (1 1/4" all),
%%     . `extrawidemargins' (1 1/2" all).
%%  - The pagestyle of "cleared" pages (empty pages inserted in
%%    two-sided documents to put the next page on the right-hand side)
%%    can be set with one of the options `cleardoublepagestyleempty',
%%    `cleardoublepagestyleplain', or `cleardoublepagestylestandard'.
%%  - Any other standard option for the `report' document arclass can be
%%    used to override the default or draft settings (such as `10pt',
%%    `11pt', `12pt'), and standard LaTeX packages can be used to
%%    further customize the layout and/or formatting of the document.

%% *** Add any desired options. ***
%PDF
%\documentclass[a4paper,narrowmargins,12pt,oneside,draft,onehalfspaced,singlespacednotes]{fcup-thesis}
%\documentclass[a4paper,narrowmargins,12pt,oneside,onehalfspaced,singlespacednotes]{fcup-thesis}
%Print
%\documentclass[draft,a4paper,narrowmargins,12pt,twoside,openright,onehalfspaced,singlespacednotes]{fcup-thesis}
\documentclass[a4paper,narrowmargins,12pt,twoside,openright,onehalfspaced,singlespacednotes]{fcup-thesis}

%% *** Add \usepackage declarations here. ***
%% The standard packages `geometry' and `setspace' are already loaded by
%% `ut-thesis' -- see their documentation for details of the features
%% they provide.  In particular, you may use the \geometry command here
%% to adjust the margins if none of the ut-thesis options are suitable
%% (see the `geometry' package for details).  You may also use the
%% \setstretch command to set the line spacing to a value other than
%% single, one-and-a-half, or double spaced (see the `setspace' package
%% for details).
% Overfull statements
\pretolerance=150
\setlength{\emergencystretch}{3em}
% Overfull end
\usepackage[english]{babel}
\usepackage{lipsum}
\usepackage[utf8]{inputenc}


%%% Additional useful packages
%%% ----------------------------------------------------------------
\usepackage{array}
\usepackage{amsmath}  
\usepackage{amssymb}
\usepackage{amsfonts}
\DeclareFontFamily{OT1}{pzc}{}
\DeclareFontShape{OT1}{pzc}{m}{it}{<-> s * [0.900] pzcmi7t}{}
\DeclareMathAlphabet{\mathpzc}{OT1}{pzc}{m}{it}
\usepackage{amsthm}      
\usepackage[ruled,algochapter]{algorithm2e}
\usepackage{algorithmic}
\usepackage{bm}
\usepackage[mathscr]{euscript}
\usepackage{graphicx}       
\usepackage{psfrag}         
\usepackage{fancyvrb}    
\usepackage{float}
\usepackage{ltablex}
\usepackage[square,sort,comma,numbers]{natbib}        
\usepackage{bbding}         
\usepackage{dcolumn}        
\usepackage{booktabs} 
\usepackage{multirow}
\usepackage{paralist}     
\usepackage{ifdraft}  
\usepackage{indentfirst}    
\usepackage[nottoc,notlof,notlot]{tocbibind}
\usepackage{url}
\usepackage{tabularx}
\usepackage{subcaption}
\usepackage[unicode]{hyperref}
\usepackage{xcolor}

\hypersetup{pdftitle=LiDAR obstacle detection and avoidance, 
            pdfauthor=Alojz Gomola,
            colorlinks=false,
            urlcolor=blue,
            pdfstartview=FitH,
            pdfpagemode=UseOutlines,
            pdfnewwindow,
            breaklinks
          }
\usepackage{array}
\newcolumntype{L}[1]{>{\raggedright\let\newline\\\arraybackslash\hspace{0pt}}m{#1}}
\newcolumntype{C}[1]{>{\centering\let\newline\\\arraybackslash\hspace{0pt}}m{#1}}
\newcolumntype{R}[1]{>{\raggedleft\let\newline\\\arraybackslash\hspace{0pt}}m{#1}}         
\newcolumntype{B}{X}
\newcolumntype{S}[1]{>{\hsize=#1\textwidth}X}

\newcommand{\FIGDIR}{./Pics}    %%% directory containing figures
\newcommand{\twolinecellr}[2][r]{%
  \begin{tabular}[#1]{@{}r@{}}#2\end{tabular}}
\newcommand{\secState}[1]{
	\ifdraft{(#1) }{}
}
\theoremstyle{plain}
\newtheorem{theorem}{Theorem}
\newtheorem{lemma}[theorem]{Lemma}
\newtheorem{proposition}[theorem]{Proposition}

\theoremstyle{plain}
\newtheorem{definition}{Definition}
\newtheorem{problem}{Problem}
\newtheorem{example}{Example}
\newtheorem{assumption}{Assumption}

\theoremstyle{remark}
\newtheorem*{corollary}{Corollary}
\newtheorem*{note}{Note}




\newenvironment{dokaz}{
  \par\medskip\noindent
  \textit{Proof}.
}{
\newline
\rightline{\SquareCastShadowBottomRight}
}

\newenvironment{constraints}[1]{
  \par\medskip\noindent
  \textit{Constraints #1} \\
}{
\newline
\rightline{\SquareCastShadowBottomRight}
}


%\bibliographystyle{plainnat}     %% Author (year) style
\bibliographystyle{unsrt}        %% [number] style
\setcitestyle{square}

% Section  3.7 Challenge list
\newif\ifproblemchallenge   %# Build block for problem challenges
\problemchallengetrue       %# Show comments

\newcommand{\R}{\mathbb{R}}
\newcommand{\N}{\mathbb{N}}

\DeclareMathOperator{\pr}{\textsf{P}}
\DeclareMathOperator{\E}{\textsf{E}\,}
\DeclareMathOperator{\var}{\textrm{var}}
\DeclareMathOperator{\sd}{\textrm{sd}}


\newcommand{\T}[1]{#1^\top}        

\newcommand{\goto}{\rightarrow}
\newcommand{\gotop}{\stackrel{P}{\longrightarrow}}
\newcommand{\maon}[1]{o(n^{#1})}
\newcommand{\abs}[1]{\left|{#1}\right|}
\newcommand{\dint}{\int_0^\tau\!\!\int_0^\tau}
\newcommand{\isqr}[1]{\frac{1}{\sqrt{#1}}}
\newcommand{\norm}[1]{\left\lVert#1\right\rVert}


\newcommand{\pulrad}[1]{\raisebox{1.5ex}[0pt]{#1}}
\newcommand{\mc}[1]{\multicolumn{1}{c}{#1}}
\newcommand{\TBD}[1]{\color{red}\emph{--TBD:}#1\color{black}}

%%%%%%%%%%%%%%%%%%%%%%%%%%%%%%%%%%%%%%%%%%%%%%%%%%%%%%%%%%%%%%%%%%%%%%%%
%%                                                                    %%
%%                   ***   I M P O R T A N T   ***                    %%
%%                                                                    %%
%%  Fill in the following fields with the required information:       %%
%%   - \degree{...}       name of the degree obtained                 %%
%%   - \department{...}   name of the graduate department             %%
%%   - \gradyear{...}     year of graduation                          %%
%%   - \author{...}       name of the author                          %%
%%   - \title{...}        title of the thesis                         %%
%%%%%%%%%%%%%%%%%%%%%%%%%%%%%%%%%%%%%%%%%%%%%%%%%%%%%%%%%%%%%%%%%%%%%%%%

%% *** Change this example to appropriate values. ***
\degree{Doctor of Philosophy}
\department{Departamento de Matem\'{a}tica}
\gradyear{2019}
\author{Alojz Gomola}
\title{Obstacle Avoidance Framework based on Reach Sets}

%% *** NOTE ***
%% Put here all other formatting commands that belong in the preamble.
%% In particular, you should put all of your \newcommand's,
%% \newenvironment's, \newtheorem's, etc. (in other words, all the
%% global definitions that you will need throughout your thesis) in a
%% separate file and use "\input{filename}" to input it here.


%% *** Adjust the following settings as desired. ***

%% List only down to subsections in the table of contents;
%% 0=chapter, 1=section, 2=subsection, 3=subsubsection, etc.
\setcounter{tocdepth}{3}

%% Make each page fill up the entire page.
\flushbottom


%%%%%%%%%%%%      MAIN  DOCUMENT      %%%%%%%%%%%%

\begin{document}


%%%%%%%%%%%%%%%%%%%%%%%%%%%%%%%%%%%%%%%%%%%%%%%%%%%%%%%%%%%%%%%%%%%%%%%%
%%  Put your Chapters here; the easiest way to do this is to keep     %%
%%  each chapter in a separate file and `\include' all the files.     %%
%%  Each chapter file should start with "\chapter{ChapterName}".      %%
%%  Note that using `\include' instead of `\input' will make each     %%
%%  chapter start on a new page, and allow you to format only parts   %%
%%  of your thesis at a time by using `\includeonly'.                 %%
%%%%%%%%%%%%%%%%%%%%%%%%%%%%%%%%%%%%%%%%%%%%%%%%%%%%%%%%%%%%%%%%%%%%%%%%

%% *** Include chapter files here. ***

\setcounter{chapter}{6}
\setcounter{section}{4}

    %06-05 Situation assessment
    \subsection{Constraints}\label{s:virtualConstraints}
\paragraph{Static Constraints:} The \emph{constraints} (ex. weather, airspace) usually covers a large portion of the \emph{operation airspace}. 

Converting constraints into valued \emph{point-cloud} is not feasible, due to the \emph{huge amount of created points} and low \emph{intersection rate}. The \emph{polygon intersection} or \emph{circular boundary of a 2D polygon} is a simple and effective solution \cite{ritter1990efficient,welzl1991smallest}. 

The key idea is to create \emph{constraint barrels} around dangerous areas. Each \emph{constraint barrel} is defined by a circle on the \emph{horizontal plane} and the \emph{vertical limit range}.

\paragraph{Representation:} The \emph{minimal representation} is based on (sec. \ref{sec:WellClear}, \ref{sec:WeatherImpact}) and geo-fencing principle. The \emph{horizontal-vertical separation} is ensured by \emph{projecting boundary} as 2D polygon oh horizontal plane and \emph{vertical boundary} (barrel height) as \emph{altitude limit}. 

The \emph{static constraint} (eq. \ref{eq:staticConstraint}) is defined as a structure vector including:
\begin{enumerate}
    \item \emph{Position} - the center position in the global coordinates\emph{2D horizontal plane}.
    
    \item \emph{Boundary} - the ordered set of boundary points forming edges in the global coordinates\emph{2D horizontal plane}.
    
    \item \emph{Altitude Range} - the \emph{barometric altitude} range $[altitude_{start},$ $altitude_{end}]$.
    
    \item \emph{Safety Margin} - the \emph{protection zone} (soft constraint) around constraint body (hard constraints) in meters.
\end{enumerate}

\begin{equation}\label{eq:staticConstraint}
    constraint = \{position,boundary, altitude_{start},altitude_{end}, safety Margin\}
\end{equation}

\paragraph{Active constrain selection:} The \emph{active constraints} are constraints which are impacting \emph{UAS active avoidance range}. 

The \emph{active constraints set} (eq. \ref{eq:activeConstraintSet}) is defined as a set of \emph{constraints} from all \emph{reliable Information Sources} where the \emph{distance} between UAS and constraint body (including safety margin) is lesser than the avoidance grid range. The \emph{horizontal altitude range} of avoidance grid must also intersect with \emph{constraint altitude range}.

\begin{multline}\label{eq:activeConstraintSet}
    Active Constraints = \dots\\\dots =
    \left\{\begin{aligned}constraint& \in Information Source:\\ 
    &distance(constraint,UAS) \le Avoidance Grid. distance,\\
    &constraint.altitude Range \cap UAS.altitude Range \neq \varnothing 
    \end{aligned}\right\}
\end{multline}

\paragraph{Cell Intersection:} The \emph{importance of constraints} is on their impact on \emph{avoidance grid} $cells$. The \emph{most of the constraints} (weather, ATC) are represented as 2D convex polygons. Even the \emph{irregularly shaped constraints} are usually split into smaller convex 2D polygons.

The idea is to represent convex polygon boundary as a sufficiently large circle to cover polygon. The Welzl algorithm to find \emph{minimal polygon cover circle} \cite{welzl1991smallest} is used.

First the \emph{set of contraint edges} (eq. \ref{eq:constraintEdgeSet}) is a enclosed set of 2D edges between neighboring points defined as follow:

\begin{equation}\label{eq:constraintEdgeSet}
    edges(constraint) =
    \left\{
    \begin{bmatrix}
        point_{i},point_{j}
    \end{bmatrix}:
    \begin{aligned}
    &point\in boundary,\\
    &i \in \{1,\dots,|boundary|\},\\
    &j \in \{2,\dots, |boundary|,1\}\\
    \end{aligned}
    \right\}
\end{equation}

\noindent The \emph{constraint circle boundary} with calculated center on the  2D horizontal plane and radius (representing body margin) is defined in (eq. \ref{eq:constraintCircleBoundary}).

\begin{equation}\label{eq:constraintCircleBoundary}
    circle(constraint)=
    \left[
        \begin{aligned}
            & center = \frac{\sum boundary.point}{|boundary.point|} + correction\\
            & radius = smallest Circle(edges(constraints)) 
        \end{aligned}
    \right]
\end{equation}

\noindent The $(cell_{i,j,k}$ and \emph{constraint} intersection (eq. \ref{eq:contraintToCellIntersection}) is classification function. The \emph{classification} is necessary, because one \emph{constraint} induce: 
\begin{enumerate}
    \item \emph{Body Constraint} (hard constraint) - the distance between $cell_{i,j,k}$ closest border and \emph{circular boundary} center is in interval $[0,radius]$.
    
    \item \emph{Protection Zone Constraint} (soft constraint) - the distance between $cell_{i,j,k}$ closest border and \emph{circular boundary} center is in interval $]radius,radius+safety Margin]$.
\end{enumerate}


\begin{multline}\label{eq:contraintToCellIntersection}
    intersection,constraint)=\dots\\\dots = 
    \begin{cases}
        hard &:\left[
            \begin{aligned}
                &distance(cell_{i,j,k},circle(constraint)) \le\dots\\ 
                &\quad\dots\le circle(constraint).radius,\\
                & constraint.altitude Range \cap cell_{i,j,k}.altitude Range \neq \varnothing,
            \end{aligned}\right]\\
             &\\
        soft &:\left[
            \begin{aligned}
                &distance(cell_{i,j,k},circle(constraint)) >\dots\\ 
                &\quad\dots > circle(constraint).radius,\\
                &distance(cell_{i,j,k},circle(constraint)) \le\dots\\ 
                &\quad\dots\le circle(constraint).radius + safety Margin,\\
                & constraint.altitude Range \cap cell_{i,j,k}.altitude Range \neq \varnothing,
            \end{aligned}\right]\\
             &\\
        none &:otherwise
    \end{cases}
\end{multline}

\noindent The \emph{intersection impact} of constraint is handled separately for \emph{soft} and  \emph{hard} constraints. The \emph{avoidance} of hard constraints is \emph{mandatory}, the \emph{avoidance} of soft constraints is \emph{voluntary}.

The constraints which have a \emph{soft intersection with the cell} are added to cells impacting constraints set: 
\begin{equation}\label{eq:softConstraintsCellIntersections}
    cell_{i,j,k}. soft Constraints = 
    \left\{
        \begin{aligned}
            &constraint \in Active Constraints:\\ 
            &\quad intersection(cell_{i,j,k},constraint) = soft
        \end{aligned}
    \right\}
\end{equation}

\noindent The constraints which have a \emph{hard intersection with the cell} are added to cells impacting constraints set:

\begin{equation}\label{eq:hardConstraintsCellIntersections}
    cell_{i,j,k}. hard Constraints = 
    \left\{
        \begin{aligned}
            &constraint \in Active Constraints:\\ 
            &\quad intersection(cell_{i,j,k},constraint) = hard
        \end{aligned}
    \right\}
\end{equation}

\begin{note}
    The final \emph{constraint rate value} (eq. \ref{eq:constraintRatingForCell}) is determined based on \emph{mission control run} feed to \emph{avoidance grid} (fig. \ref{fig:missionControlRunActivityDiagram}) defined in  7\textsuperscript{th} to the 10\textsuperscript{th} step.
\end{note}
    

    \subsection{(W) Moving Constraints}\label{s:MovingVirtualConstraints}
\paragraph{Idea:} The basic ideas is the same as in case \emph{static constraints} (sec. \ref{s:virtualConstraints}). There is horizontal constraint and altitude constraint outlining the constrained space. The only additional concept is moving of \emph{constraint} on horizontal plane in global coordinate system. 

The constraint intersection  with \emph{avoidance grid} is done in \emph{fixed decision Time}, for cell in \emph{fixed cell leave time} (eq. \ref{eq:cellLeaveTime}), which means concept from static obstacles can be fully reused.

\paragraph{Definition:} The \emph{moving constraint definition} (eq. \ref{eq:movingConstraintDefinition}) covers minimal data scope for  moving constraint, assuming linear constraint movement. 

The original definition (eq. \ref{eq:staticConstraint}) is enhanced with additional parameters to support constraint moving:

\begin{enumerate}
    \item \emph{Velocity} - velocity vector on 2D horizontal plane.
    
    \item \emph{Detection time} - the time when \emph{constraint} was created/detected, this is the time when \emph center and boundary points position were valid.
\end{enumerate}

\begin{multline}\label{eq:movingConstraintDefinition}
    constraint = \{position,boundary,\dots\\\dots, velocity, detection Time, \dots \\\dots altitude_{start},altitude_{end}, safety Margin\}
\end{multline}

\paragraph{Cell Intersection:} The \emph{intersection algorithm} follows (eq. \ref{eq:contraintToCellIntersection}), only shift of the \emph{center and boundary points} is required. 

First let us introduce $\Delta time$ (eq. \ref{eq:deltatimeMovingconst}), which represents difference between the constraint detection time and expected cell leave time (eq. \ref{eq:cellLeaveTime}).

\begin{equation}\label{eq:deltatimeMovingconst}
    \Delta time = UAS_{leave}(cell_{i,j,k}) - detection Time
\end{equation}

\noindent The constraint boundary is shifted to:

\begin{multline}
    shifted Boundary(constraint) = \{new Point = point + velocity \times \Delta time:\dots\\\dots \forall point \in constraint.boundary \}
\end{multline}

\noindent The constraint center is shifted to:

\begin{equation}
    shifted Center(constraint) = constraint.center + velocity
\end{equation}

\begin{note}
    The $\Delta time$ is calculated separately for each $cell_{i,j,k}$, because \emph{UAS} is also  moving and reaching cells in different times. The \emph{cell leave time} can be calculated in advance after reach set approximation.
\end{note}

\paragraph{Alternative Intersection Implementation:} The alternative used for intersection selected based on polygon intersection algorithms review \citep{bentley1979algorithms}, the selected algorithm  is \emph{Shamos-Hoey} \cite{shamos1976geometric}.

The implementation was tested on \emph{Storm scenario} (sec. \ref{s:testStorm}) and it yelds same results.

 

%% This adds a line for the Bibliography in the Table of Contents.
\addcontentsline{toc}{chapter}{Bibliography}
%% *** Set the bibliography style. ***
%% (change according to your preference/requirements)
%\bibliographystyle{plain}
%% *** Set the bibliography file. ***
%% ("thesis.bib" by default; change as needed)
\bibliography{thesis}

%% *** NOTE ***
%% If you don't use bibliography files, comment out the previous line
%% and use \begin{thebibliography}...\end{thebibliography}.  (In that
%% case, you should probably put the bibliography in a separate file and
%% `\include' or `\input' it here).

\end{document}
