%%%%% Single page layout:
%%%%% ----------------------------------------------------
%\documentclass[12pt, a4paper,draft]{report}
\documentclass[12pt, a4paper]{report}
\setlength\textwidth{160mm}
\setlength\textheight{247mm}
\setlength\oddsidemargin{0mm}
\setlength\evensidemargin{0mm}
\setlength\topmargin{0mm}
\setlength\headsep{0mm}
\setlength\headheight{0mm}
\let\openright=\clearpage
% Overfull statements
\pretolerance=150
\setlength{\emergencystretch}{3em}
% Overfull end

\usepackage[utf8]{inputenc}


%%% Additional useful packages
%%% ----------------------------------------------------------------
\usepackage{array}
\usepackage{amsmath}  
\usepackage{amssymb}
\usepackage{amsfonts}
\DeclareFontFamily{OT1}{pzc}{}
\DeclareFontShape{OT1}{pzc}{m}{it}{<-> s * [0.900] pzcmi7t}{}
\DeclareMathAlphabet{\mathpzc}{OT1}{pzc}{m}{it}
\usepackage{amsthm}      
\usepackage[ruled,algochapter]{algorithm2e}
\usepackage{algorithmic}
\usepackage{bm}
\usepackage[mathscr]{euscript}
\usepackage{graphicx}       
\usepackage{psfrag}         
\usepackage{fancyvrb}    
\usepackage{float}
\usepackage{ltablex}
\usepackage[square,sort,comma,numbers]{natbib}        
\usepackage{bbding}         
\usepackage{dcolumn}        
\usepackage{booktabs} 
\usepackage{multirow}
\usepackage{paralist}       
\usepackage{indentfirst}    
\usepackage[nottoc,notlof,notlot]{tocbibind}
\usepackage{url}
\usepackage{tabularx}
\usepackage{subcaption}
\usepackage[unicode]{hyperref}
\usepackage{xcolor}

\hypersetup{pdftitle=LiDAR obstacle detection and avoidance, 
            pdfauthor=Alojz Gomola,
            colorlinks=false,
            urlcolor=blue,
            pdfstartview=FitH,
            pdfpagemode=UseOutlines,
            pdfnewwindow,
            breaklinks
          }
\usepackage{array}
\newcolumntype{L}[1]{>{\raggedright\let\newline\\\arraybackslash\hspace{0pt}}m{#1}}
\newcolumntype{C}[1]{>{\centering\let\newline\\\arraybackslash\hspace{0pt}}m{#1}}
\newcolumntype{R}[1]{>{\raggedleft\let\newline\\\arraybackslash\hspace{0pt}}m{#1}}         
\newcolumntype{B}{X}
\newcolumntype{S}[1]{>{\hsize=#1\textwidth}X}

\newcommand{\FIGDIR}{./Pics}    %%% directory containing figures
\newcommand{\twolinecellr}[2][r]{%
  \begin{tabular}[#1]{@{}r@{}}#2\end{tabular}}
\theoremstyle{plain}
\newtheorem{theorem}{Theorem}
\newtheorem{lemma}[theorem]{Lemma}
\newtheorem{proposition}[theorem]{Proposition}

\theoremstyle{plain}
\newtheorem{definition}{Definition}
\newtheorem{problem}{Problem}
\newtheorem{example}{Example}
\newtheorem{assumption}{Assumption}

\theoremstyle{remark}
\newtheorem*{corollary}{Corollary}
\newtheorem*{note}{Note}




\newenvironment{dokaz}{
  \par\medskip\noindent
  \textit{Proof}.
}{
\newline
\rightline{\SquareCastShadowBottomRight}
}

\newenvironment{constraints}[1]{
  \par\medskip\noindent
  \textit{Constraints #1} \\
}{
\newline
\rightline{\SquareCastShadowBottomRight}
}


%\bibliographystyle{plainnat}     %% Author (year) style
\bibliographystyle{unsrt}        %% [number] style
\setcitestyle{square}

% Section  3.7 Challenge list
\newif\ifproblemchallenge   %# Build block for problem challenges
\problemchallengetrue       %# Show comments


\title{Dissertation thesis}
\author{Alojz Gomola}
\date{February 2019}

%%%%% ------------------------------------------------------------
\DefineVerbatimEnvironment{PCinout}{Verbatim}{fontsize=\small, frame=single}



\newcommand{\R}{\mathbb{R}}
\newcommand{\N}{\mathbb{N}}

\DeclareMathOperator{\pr}{\textsf{P}}
\DeclareMathOperator{\E}{\textsf{E}\,}
\DeclareMathOperator{\var}{\textrm{var}}
\DeclareMathOperator{\sd}{\textrm{sd}}


\newcommand{\T}[1]{#1^\top}        

\newcommand{\goto}{\rightarrow}
\newcommand{\gotop}{\stackrel{P}{\longrightarrow}}
\newcommand{\maon}[1]{o(n^{#1})}
\newcommand{\abs}[1]{\left|{#1}\right|}
\newcommand{\dint}{\int_0^\tau\!\!\int_0^\tau}
\newcommand{\isqr}[1]{\frac{1}{\sqrt{#1}}}
\newcommand{\norm}[1]{\left\lVert#1\right\rVert}


\newcommand{\pulrad}[1]{\raisebox{1.5ex}[0pt]{#1}}
\newcommand{\mc}[1]{\multicolumn{1}{c}{#1}}
\newcommand{\TBD}[1]{\color{red}\emph{--TBD:}#1\color{black}}

\begin{document}

%01-Introduction
    \cleardoublepage
\chapter{\secState{R/W}Introduction}\label{ch:introduction}

\noindent This works present an approach based on \emph{reach set approximation} to \emph{detect \& avoid} various sort of threats in \emph{controlled/non-controlled} airspace environment. 

The \emph{motivation} is summarized in (sec. \ref{s:motivation}). The work \emph{goals} are given in (sec. \ref{s:goals}). A \emph{thesis organization} with notes is summarized in (sec. \ref{s:Overview}). A notable contributions of work are listed in (sec. \ref{s:Contributions}). The listing of student publications/technical reports/open source contributions are given in (sec. \ref{sec:listOfPublications}).
    \section{(W) Related Work}\label{s:relatedWork}
    \emph{To be done here:}
    \begin{itemize}
        \item Ramasy work, Sabatiny work on LiDAR and obstacle avoidance, introduce movement automaton etc...
        \item Lattice search related work, the problem of lattice search above 4th dimension
        \item Reach set approximation related work
    \end{itemize}
    \section{Goals}\label{s:goals}
\paragraph{Situation:} The \emph{UAS} equipped with cooperative/non-cooperative surveillance sensors, with prior knowledge of operation space has to fly a mission represented ordered set of waypoints. The set of sensors can change depending on UAS construction. The minimal airworthiness for a given operation is assumed.

\paragraph{Problem:} Given environment and artifact definitions (sec. \ref{s:basicDefinitions}) with \emph{initial assumptions} (sec. \ref{s:initialAssumptions}) and \emph{incremental problem definition} (sec. \ref{s:IncrementalProblemDefinition}) develop \emph{obstacle avoidance framework} which will satisfy \emph{avoidance} (sec. \ref{s:AvoidanceRequirements}) and \emph{navigation} (sec. \ref{s:navigationRequirements}).

\paragraph{Expected Solution:} Define an approach based on \emph{reach sets} which are capable of:

\begin{enumerate}
    \item \emph{Static obstacle avoidance} - to avoid the ground, man-made structures in open terrain. 
    
    \item \emph{Intruders avoidance} - to avoid flying objects which does not have the intention to harm our UAS, detected in sufficient distance. 

    \item \emph{Geo-fencing support} - to avoid known zones/airspace portions, which have forbidden entry.

    \item \emph{Weather avoidance} - to avoid known zones of harmful weather conditions.

    \item \emph{Cooperative conflict resolution} - to communicate own position to authority and to follow authority orders.
    
    \item \emph{Treat prioritization} - to assess avoidance according to natural or man-made priorities. (Rather break geo-fence, than crashing into the ground).
\end{enumerate}

\paragraph{Validation:} Develop test-framework to showcase approach properties. Define \emph{test scenarios} (sec. \ref{s:testingApproach}) to validate \emph{Expected Solution Performance} (sec.  tab. \ref{tab:testCasesSummary}) concerning \emph{avoidance capability} (sec. \ref{s:performanceEvaluationTable}) and \emph{computational feasibility} (sec. \ref{s:ComputaitonFootprint}).

\paragraph{Application Requirements:} There are  following application requirements, based on similar applications for \emph{manned aviation} and \emph{industry expectations}:

\begin{enumerate}
    \item \emph{Low-performance requirements} - the computational footprint of the approach should be polynomial. The most of actual UAS systems have \emph{embedded computer} with low computation power.
    
    \item \emph{Deterministic} - the \emph{avoidance strategy} should be achieved in finite time frame. The mandatory requirement for \emph{airborne operation support application}, the advice needs to be reproducible under the same conditions.
    
    \item \emph{Scalability} - the \emph{avoidance framework} should be portable to the different platforms, and it should work with different sensor array. The interface requirement for \emph{control} and \emph{data fusion} coming from other \emph{collision avoidance systems}.
    
    \item \emph{Adaptability} - the \emph{avoidance process} should have tuning points where is possible to change behavior according to UAS context. The regulations are changing with location, time and circumstances, the part of calculation/control process needs to be implemented dynamically.
\end{enumerate}
    \section{(W) Assumptions}\label{s:assumptions}
    \emph{To be done here:}
    \begin{itemize}
        \item  Guarantee feasible safe trajectory in open world space at low altitude of the flight. Manage information resources about real obstacles, weather obstacles, ATM restrictions.
        \item create previously mentioned points as assumptions, xxx is accessible to feasible extent, etc ...
    \end{itemize}
    \section{(W) Overview}\label{s:Overview}
    \emph{To be done here:}
    \begin{itemize}
        \item  Chapter overview 
        \item Notation of key concepts
    \end{itemize}
    \section{(W) Contributions}\label{s:Contributions}
    \emph{To be done here:}
    \begin{itemize}
        \item  Notable contributions
        \item List of concepts and articles with references
    \end{itemize}


%09-Bibliography
    \bibliography{thesis}
\end{document}
