\subsection{\secState{R}Used UAS Nonlinear Model}\label{s:UASNonlinearModel}
\paragraph{Motivation:} Simplified rigid body kinematic model will be used. This model have decoupled roll, yaw and pitch angles. The focus is on \emph{reach set approximation methods}, therefore \emph{UAS model} is simplified.

\paragraph{State Vector} (eq. \ref{eq:simple3dStatevector}) defined as positional state in euclidean position in right-hand euclidean space, where \emph{x, y, z} can be abstracted as latitude, longitude, altitude.
\begin{equation}\label{eq:simple3dStatevector}
    state = \left [ x,y,z, roll, pitch, yaw \right]^T
\end{equation}


\newpage 
\paragraph{Input Vector} (eq. \ref{eq:simple3dInputVector}) is defined as linear velocity of UAS $v$ and angular speed of rigid body $\omega_{roll}, \omega_{pitch},\omega_{yaw}$.

\begin{equation}\label{eq:simple3dInputVector}
    input = \left [ v, \omega_{roll}, \omega_{pitch},\omega_{yaw}\right ]^T
\end{equation}


\noindent Velocity distribution function (eq. \ref{eq:simple3dvelocityDistribution})  is is defined trough standard rotation matrix  and linear velocity $v$, oriented velocity [$v_x$, $v_y$, $v_z$] given by (eq. \ref{eq:UASNonlinearModelSimple}).

\begin{equation}\label{eq:simple3dvelocityDistribution}
    \begin{bmatrix}
    v_x\\
    v_y\\
    v_z\
    \end{bmatrix}
    =
    \begin{bmatrix}
         v\cos(pitch)\cos(yaw)\\
         v\cos(pitch)\sin(yaw)\\
         -v\sin(pitch)\\
    \end{bmatrix}
\end{equation}

\paragraph{UAS Nonlinear Model} (eq. \ref{eq:UASNonlinearModelSimple}) is given by \emph{first order equations:}

\begin{equation}\label{eq:UASNonlinearModelSimple}
    \begin{aligned}
        \frac{\text{d} x}{\text{d} time} &= v\cos(pitch)\cos(yaw);\\
        \frac{\text{d} y}{\text{d} time} &= v\cos(pitch)\sin(yaw);\\
        \frac{\text{d} z}{\text{d} time} &= -v\sin(pitch);\\
    \end{aligned}\\\quad\quad
    \begin{aligned}
        \frac{\text{d} roll}{\text{d} time} &= \omega_{roll};\\
        \frac{\text{d} pitch}{\text{d} time} &= \omega_{pitch};\\
        \frac{\text{d} yaw}{\text{d} time} &= \omega_{yaw};\\
    \end{aligned}
\end{equation}

