\subsection{\secState{R}UAS Model}\label{s:UASNonlinearModel}
\paragraph{Motivation:} Simplified rigid body kinematic model will be used. This model has decoupled roll, yaw and pitch angles. The focus is on \emph{reach set approximation methods}; therefore the \emph{UAS model} is simplified.

\paragraph{State Vector} (eq. \ref{eq:simple3dStatevector}) defined as a positional state in euclidean position in right-hand euclidean space, where \emph{x, y, z} can be abstracted as latitude, longitude, altitude.
\begin{equation}\label{eq:simple3dStatevector}
    state = \left [ x,y,z, roll, pitch, yaw \right]^T
\end{equation}


\paragraph{Input Vector} (eq. \ref{eq:simple3dInputVector}) is defined as the linear velocity of UAS $v$ and angular speed of rigid body $\omega_{roll}, \omega_{pitch},\omega_{yaw}$.

\begin{equation}\label{eq:simple3dInputVector}
    input = \left [ v, \omega_{roll}, \omega_{pitch},\omega_{yaw}\right ]^T
\end{equation}


\noindent Velocity vector function (eq. \ref{eq:simple3dvelocityDistribution}) is defined through the standard rotation matrix  and linear velocity $v$, oriented velocity [$v_x$, $v_y$, $v_z$] given by (eq. \ref{eq:UASNonlinearModelSimple}).

\begin{equation}\label{eq:simple3dvelocityDistribution}
    \begin{bmatrix}
    v_x\\
    v_y\\
    v_z\
    \end{bmatrix}
    =
    \begin{bmatrix}
         v\cos(pitch)\cos(yaw)\\
         v\cos(pitch)\sin(yaw)\\
         -v\sin(pitch)\\
    \end{bmatrix}
\end{equation}

\paragraph{UAS Nonlinear Model} (eq. \ref{eq:UASNonlinearModelSimple}) is given by \emph{first order equations:}

\begin{equation}\label{eq:UASNonlinearModelSimple}
    \begin{aligned}
        \frac{\text{d} x}{\text{d} time} &= v\cos(pitch)\cos(yaw);\\
        \frac{\text{d} y}{\text{d} time} &= v\cos(pitch)\sin(yaw);\\
        \frac{\text{d} z}{\text{d} time} &= -v\sin(pitch);\\
    \end{aligned}\\\quad\quad
    \begin{aligned}
        \frac{\text{d} roll}{\text{d} time} &= \omega_{roll};\\
        \frac{\text{d} pitch}{\text{d} time} &= \omega_{pitch};\\
        \frac{\text{d} yaw}{\text{d} time} &= \omega_{yaw};\\
    \end{aligned}
\end{equation}

\paragraph{Discretization} for \emph{fixed step} $k$ we start with discretization of the model:

\noindent The \emph{linear velocity} in text step is given:
\begin{equation}\label{eq:applyMovement}
    v(k+1) = v(k) +\delta v(k)
\end{equation}

\noindent The \emph{roll, pitch, yaw} for next step are given 

\begin{equation}\label{eq:applyMovement1}
    \begin{aligned}
        roll(k+1)  &= roll(k) + \delta roll(k)\\
        pitch(k+1) & = pitch(k) + \delta pitch(k)\\
        yaw(k+1) & = yaw(k) + \delta yaw(k)\\
    \end{aligned}    
\end{equation}

\noindent The $\delta v(k)$ is \emph{velocity change}, $\delta roll(k)$, $\delta pitch(k)$, $\delta yaw(k)$, are \emph{orientation changes} for current discrete step $k$. If the duration of \emph{transition} is $0 s$ (as. \ref{ass:transitionTime}) then 3D trajectory evolution in discrete time is given as: 

\begin{equation}\label{eq:applyMovement2}
    \begin{aligned}
        x(k+1)&= x(k) + v(k+1) \cos(pitch(k+1)) \cos(yaw(k+1)) & = \delta x(k)\\
        y(k+1)&= y(k) + v(k+1) \cos(pitch(k+1)) \sin(yaw(k+1)) & = \delta y(k)\\
        z(k+1)&= z(k) - v(k+1) \sin(pitch(k+1))                & = \delta z(k)\\
        time(k+1)& = time(k)+1                                & = \delta time(k)
    \end{aligned}    
\end{equation}

\noindent The $\delta x(k)$, $\delta y(k)$, $\delta z(k)$ are positional differences depending on \emph{input vector} for given discrete time $k$:
\begin{equation}\label{eq:ourImput}
    input(k) = \left[
                    \begin{gathered}
                    \delta x(k), \delta y(k), \delta z(k), \delta v (k),\\
                    \delta roll (k), \delta pitch(k), \delta yaw(k),\delta time (k)
                    \end{gathered} 
                \right]^T
\end{equation}

\noindent The \emph{state vector} for discrete time is given:
\begin{equation}\label{eq:ourState}
    state(k) = \left[
                    \begin{gathered}
                     x(k),  y(k),  z(k),  v (k),\\
                     roll (k),  pitch(k),  yaw(k), time (k)
                    \end{gathered} 
                \right]^T
\end{equation}

