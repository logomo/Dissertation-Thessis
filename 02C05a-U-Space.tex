\subsection{\secState{R}U-Space}\label{sec:USpace}

\noindent The \emph{Concept Of OpeRations of U-Space} (CORUS) \cite{corus2018} have been released recently. This concept describes the difference between the standard ATM and proposed European UTM solution. This section will get through the interesting part of this pivotal document.

\noindent The \emph{U-space} is separated into following functionality based phases:
\begin{itemize}
    \item[\texttt{U1}] (year $<$ 2020) - sets the scene with registration and geo-fencing.
    
    \item[\texttt{U2}] (year$<$ 2025) - introduces tracking, flight planning and messages sent to the remote pilot during flight.
    
    \item[\texttt{U3}] (year$<$ 2030) introduces collaborative detect and avoid and tactical conflict resolution.
    
    \item[\texttt{U4}] (year $<$ 2035) brings safe interoperation with manned aviation.
\end{itemize}

The \emph{Aspects} important for \emph{Obstacle avoidance} will be oultined and discussed over this section. Our work focuses on \emph{European Airspace} (EASA), therefore more focus will be on \emph{U-space}

\paragraph{Small UAS Classification:} Manned aviation is covered by existing rules, for example \cite{icaoAnnex2,ec201208ref5}. Excluding some specific situations, manned aviation does not fly below VFR airspace, do not enter \emph{very low level} (VLL) altitudes \cite{ec200802ref4}. 

The certified airworthiness is mandatory for \emph{airspace attendants} with \emph{Maximum Take-Off Mass} (MTOM) over $150 kg$. The other \emph{airspace attendants} needs to fulfill only \emph{Minimal Operation Performance Specification} (MOPS). 

In \cite{easa201801op} EASA proposed several classes for UAS below $150kg$ MTOM; see Appendix 1 of the annex to
Opinion 1-2018 entitled …on making available on the market of unmanned aircraft intended for use in
the ‘open’ category and on third-country UAS operators. In that text, the next smaller mass mentioned
below 150kg is 25kg MTOM. A similar break is proposed in some national legislation, for example in
the UK at 20kg. As a working definition this little chart shows a possible breakdown by MTOM. Note
that EASA classes depend on many factors, not only MTOM.

\begin{table}[H]
    \centering
    \begin{tabular}{c||c|l}
    \begin{tabular}[c]{@{}c@{}}EASA\\ class\end{tabular} & \begin{tabular}[c]{@{}c@{}}Maximum \\ take-off mass\end{tabular} & Remarks \\\hline\hline
     C0 & $\le 250 g$   & "Child`s Toy" with very limited capabilities \\\hline
     C1 & $\le 900 g$   & "Adult`s Toy", small flying camera \\\hline
     C2 & $\le 4 kg$    & "Small UAS" with package \\\hline
     C3 & $\le 25 kg$   & "Standard UAS", attainable AGL altitude $\le 120m$ \\\hline
     C4 & $\le 25 kg$   & "Standard UAS", no automatic control mode  altitude $> 120m$\\\hline
     -  & $\le 150 kg$  & "Heavy UAS", not defined in \cite{easa201801op} \\\hline
     -  & $> 150 kg kg$ & Regulated by EASA in same manner than manned aircraft \cite{ec200802ref4} 
    \end{tabular}
    \caption{Small UAS Classes according to EASA. \cite{corus2018}}
    \label{tab:smallDroneClessesAccordingtoEASA}
\end{table}

\begin{note}
    The class C3 and C4 are different in operational restrictions. This work focuses mainly on UAS classes C2/C3/C4, because they have enabled \emph{automatic control mode} 
\end{note}

\paragraph{Separation Minima:} The \emph{Separation Minima} defines \emph{minimal distances} between airspace attendants to ensure secure \emph{operation}.  The separation minima is taken from Corus \cite{corus2018}. 

All ideas for safe concurrent operation of UAS are based on the idea of keeping the UAS systems apart or physically distant from some risk source. 

A geo-fence for example is simply a method of providing separation. There will need to be separation minima for UAS just as there are for manned aircraft and these will be needed by services such as Monitoring which seeks to warn about loss of separation and Tactical Conflict Resolution which may act to avoid loss of separation.

Separation minima will be different from those for manned aircraft as small UAS systems are generally much smaller and often slower moving than manned aircraft. CORUS proposes the following as separation minima (tab. \ref{tab:proposedseparationMinimaforUAS})

\begin{table}[H]
    \centering
    \begin{tabular}{l||l|l|l}
        \multicolumn{1}{c||}{\begin{tabular}[c]{@{}c@{}}Flight Type\\ Interaction\end{tabular}} & \multicolumn{1}{c|}{Horizontal} & \multicolumn{1}{c|}{Vertical|} & \multicolumn{1}{c}{Remark} \\\hline\hline
        %TableLine 1
        \begin{tabular}[c]{@{}l@{}}
            Any UAS - Manned or\\
            person carrying
        \end{tabular} & 
        \begin{tabular}[c]{@{}l@{}}
            2.5 NM
        \end{tabular} & 
        \begin{tabular}[c]{@{}l@{}}
            500 ft
        \end{tabular} & 
        \begin{tabular}[c]{@{}l@{}}
            Half the current manned\\
            aircraft separation
        \end{tabular}\\\hline
        
        %TableLine 2
        \begin{tabular}[c]{@{}l@{}}
            VLOS - VLOS
        \end{tabular} & 
        \begin{tabular}[c]{@{}l@{}}
            Remain\\
            "Well Clear"
        \end{tabular} & 
        \begin{tabular}[c]{@{}l@{}}
            Remain\\
            "Well Clear"
        \end{tabular} & 
        \begin{tabular}[c]{@{}l@{}}
            The remote pilot is not\\
            expected to judge distance by\\
            sight from the remote piloting\\
            position.
        \end{tabular}\\\hline
        
        %TableLine 3
        \begin{tabular}[c]{@{}l@{}}
            VLOS - BVLOS
        \end{tabular} & 
        \begin{tabular}[c]{@{}l@{}}
            Remain\\
            "Well Clear"\\
            + 200 ft
        \end{tabular} & 
        \begin{tabular}[c]{@{}l@{}}
            Remain\\
            "Well Clear"\\
            + 200 ft
        \end{tabular} & 
        \begin{tabular}[c]{@{}l@{}}
            The remote pilot is not\\
            expected to judge distance by\\
            sight from the remote piloting\\
            position.
        \end{tabular}\\\hline
        %TableLine
        \begin{tabular}[c]{@{}l@{}}
            BVLOS – BVLOS
        \end{tabular} & 
        \begin{tabular}[c]{@{}l@{}}
            200 ft
        \end{tabular} & 
        \begin{tabular}[c]{@{}l@{}}
            150 ft
        \end{tabular} & 
        \begin{tabular}[c]{@{}l@{}}
            Figures come from a \\
            pessimistic estimate of\\
            satellite navigation\\
            performance.
        \end{tabular}
\end{tabular}
    \caption{Proposed separation minima for UAS. \cite{corus2018}}
    \label{tab:proposedseparationMinimaforUAS}
\end{table}

\begin{note}
    The \emph{BVLOS – BVLOS} separation minima are interesting us, because the \emph{autonomous mode} is considered as BVLOS to  BVLOS avoidance in case of autonomous UAS.
    
    The \emph{separation minima} for \emph{UAS - Manned aviation} is unreasonably huge (2 nautical miles) and in current it should be considered as moving constraint.
\end{note}


\paragraph{Flight Rules:} The \emph{aspects} of UAS flight rules for U-SPACE concept is summarized in table:
\begin{table}[H]
    \centering
    \begin{tabular}{l||l}
        \multicolumn{1}{c||}{
        \begin{tabular}[c]{@{}c@{}}
            Aspect
        \end{tabular}} & 
        %\multicolumn{1}{c}{Hobby Flight Rules} &
        \multicolumn{1}{c}{UAS Flight Rules} \\\hline\hline
        %riadok 1
        \begin{tabular}[c]{@{}l@{}}
            Flight plan required
        \end{tabular} & 
        %\begin{tabular}[c]{@{}l@{}}
        %    No, but allowed
        %\end{tabular} & 
        \begin{tabular}[c]{@{}l@{}}
            Yes
        \end{tabular}\\\hline
        %riadok 2
        \begin{tabular}[c]{@{}l@{}}
            Allowed flight type
        \end{tabular} & 
        %\begin{tabular}[c]{@{}l@{}}
        %    VLOS, EVLOS
        %\end{tabular} & 
        \begin{tabular}[c]{@{}l@{}}
            VLOS, EVLOS, BVLOS
        \end{tabular}\\\hline
        %riadok 3
        \begin{tabular}[c]{@{}l@{}}
            Provision of separation in U1,\\
            VLOS \& EVLOS
        \end{tabular} & 
        %\begin{tabular}[c]{@{}l@{}}
        %    Pilot
        %\end{tabular} & 
        \begin{tabular}[c]{@{}l@{}}
            Pilot
        \end{tabular}\\\hline
        %riadok 4
        \begin{tabular}[c]{@{}l@{}}
            Provision of separation in U1,\\
            BVLOS
        \end{tabular} & 
        %\begin{tabular}[c]{@{}l@{}}
        %    N/A
        %\end{tabular} & 
        \begin{tabular}[c]{@{}l@{}}
            Geo-fence / Geo-cage
        \end{tabular}\\\hline
        %riadok 5
        \begin{tabular}[c]{@{}l@{}}
            Provision of separation in U2
        \end{tabular} & 
        %\begin{tabular}[c]{@{}r@{}l@{}}
        %    1. & Pilot (visual)\\
        %    2. & Limited support may be\\
        %       & obtained by using a Traffic\\
        %       & Information Service\\
        %\end{tabular} & 
        \begin{tabular}[c]{@{}r@{}l@{}}
            1. &  Strategic Conflict Resolution\\
               &  enabled by flight planning\\
            2. &  Traffic Information Service\\
               &  enabled by position reporting\\
            3. &  Pilot (visual)\\
        \end{tabular}\\\hline
        %riadok 6
        \begin{tabular}[c]{@{}l@{}}
            Provision of separation in U3 \&\\
            U4
        \end{tabular} & 
        \begin{tabular}[c]{@{}r@{}l@{}}
            1. &   Strategic Conflict Resolution\\
               &   enabled by flight planning\\
            2. &   Traffic Information Service\\
               &   enabled by position reporting\\
            3. &   Cooperative Tactical\\
               &   Conflict Resolution\\
            4. &   Detect and Avoid\\
            5. &   Pilot (visual)
        \end{tabular}\\\hline
        %riadok 7
        \begin{tabular}[c]{@{}l@{}}
            Separation from manned\\
            aviation in U2, VLOS or EVLOS\\
            UAS flight
        \end{tabular} & 
        \begin{tabular}[c]{@{}l@{}}
            The pilot is responsible\\
            to get the UAS out of the way\\
            of the manned aircraft.\\
        \end{tabular}\\\hline
        %riadok 8
        \begin{tabular}[c]{@{}l@{}}
            Separation from manned\\
            aviation in U2, BVLOS UAS\\
            flight\\
        \end{tabular} & 
        \begin{tabular}[c]{@{}l@{}}
            Flight plan required from both.\\
            Separation by planning.\\
            BVLOS pilot should use traffic\\
            information to avoid the\\
            manned aircraft (which is\\
            tracked).
        \end{tabular}\\
    \end{tabular}
    \caption{Aspects for UAS flight rules.\cite{corus2018}}
    \label{tab:aspectsOfUasFlightRules}
\end{table}

\noindent Following \emph{detect and avoid} requirements can be outlined based on (tab. \ref{tab:aspectsOfUasFlightRules}).
\begin{enumerate}
    \item \emph{Separation in U1} - only \emph{identification services} are provided in this phase. The \emph{Detect And Avoid} support can be provided only to UAS pilot in form of visual or sound advisories. 
    
    \item \emph{Separation in U2} - the \emph{position notifications} are added, enabling, preemptive collision avoidance by flight planning (mission control). The \emph{traffic information} can be added to pilot software for better situation awareness.  


    \item \emph{Separation in U3 \& U4} - the \emph{advanced} avoidance concepts, from our perspective following aspects are interesting: 
    \begin{enumerate}[a]
        \item \emph{Cooperative Tactical Conflict Resolution} - The UTM infrastructure and hierarchy for \emph{cooperative conflict} resolution must be established. in form of UTM \emph{directives} and UAS \emph{fulfillment}.
        
        \item \emph{Detect and Avoid} - reactive obstacle/intruder avoidance and situation awareness on very high level.
    \end{enumerate}
    
    \item \emph{Separation from Manned Aviation} - the \emph{well clear threshold} (fig. \ref{fig:WellClearTreshold}) for manned aviation  (tab.\ref{tab:proposedseparationMinimaforUAS}) are too big. The \emph{effective} application of \emph{reactive obstacle avoidance} is not reasonable, because the manned aviation will be out of range for most sensors (except ADS-B).
\end{enumerate}

\begin{note}
    Our work covers \emph{cooperative conflict resolution} and \emph{detect and avoid}.
\end{note}

\paragraph{Geo-fencing Modes:} A Geo-fencing appears in U1, U2 and U3 and is successively refined. It is supported by aeronautical information for UAS systems. This table summarizes the different features by level:

\begin{table}[H]
    \centering
    \begin{tabular}{l||l|l}
        \multicolumn{1}{c||}{Capability} & \multicolumn{1}{c|}{Level} & \multicolumn{1}{c}{Features} \\\hline\hline
        %line 1
        \begin{tabular}[c]{@{}l@{}}
            Pre-Tactical \\Geo-Fencing
        \end{tabular} & 
        \begin{tabular}[c]{@{}l@{}}
            U1
        \end{tabular} & 
        \begin{tabular}[c]{@{}l@{}}
            Information provided before flight. The user should have\\
            access to AIP and NOTAM defined geo-fences in a form that\\
            can be used when planning and that can be loaded onto the\\
            UAS if it has geo-fence fence features in its navigation\\
            system
        \end{tabular}\\\hline
        %line 2
        \begin{tabular}[c]{@{}l@{}}
            On-board \\Geo-Fencing
        \end{tabular} & 
        \begin{tabular}[c]{@{}l@{}}
            U1
        \end{tabular} & 
        \begin{tabular}[c]{@{}l@{}}
            The ability of the UAS to keep itself on the correct side of a\\
            geo-fence by having geo-fence definitions (location, time,\\
            height) within its navigation system\\
        \end{tabular}\\\hline
        %line
        \begin{tabular}[c]{@{}l@{}}
            Tactical \\Geo-Fencing
        \end{tabular} & 
        \begin{tabular}[c]{@{}l@{}}
            U2
        \end{tabular} & 
        \begin{tabular}[c]{@{}l@{}}
            This service delivers to the pilot and /or UAS operator\\
            updates to and new definitions of Geo-Fences occurring at\\
            any time, including during flight.\\
            The creation of geo-fences with immediate effect require\\
            that they are defined outside the AIP. 
        \end{tabular}\\\hline
        %line
        \begin{tabular}[c]{@{}l@{}}
            UAS \\Aeronautical\\
            Information\\ Management
        \end{tabular} & 
        \begin{tabular}[c]{@{}l@{}}
            U2
        \end{tabular} & 
        \begin{tabular}[c]{@{}l@{}}
            U2 include a non-AIP repository of Geo-Fences. The\\
            UAS Aeronautical Information Management service\\
            includes all information coming from such a source,\\
            combined with information from the AIP and NOTAMS\\
            together with any other UAS relevant sources.
        \end{tabular}\\\hline
        %line
        \begin{tabular}[c]{@{}l@{}}
            Dynamic\\Geo-Fencing
        \end{tabular} & 
        \begin{tabular}[c]{@{}l@{}}
            U3
        \end{tabular} & 
        \begin{tabular}[c]{@{}l@{}}
            This service delivers updates definitions of geo-fences\\
            directly into the UAS, even in flight. This service\\
            relies on capabilities of the UAS in U3 to receive\\
            communications from U-space and to deal with geo-fence\\
            updates.
        \end{tabular}
    \end{tabular}
    \caption{Geo-fencing in U-space. \cite{corus2018}}
    \label{tab:geofencingInUspace}
\end{table}

\noindent The \emph{impact of geo-fencing} on \emph{Detect and Avoid} system is following:
\begin{enumerate}
    \item \emph{Pre tactical} - The \emph{flight plan} (mission) is prepared to avoid all \emph{known forbidden areas}. In this phase the geo-fence covers static space constraints.
    
    \item \emph{On board} - The \emph{flight plan} (mission) specification does not contain all static space constraints. These constraints are known prior the flight. If UAS approaches such constraints it needs to avoid them. The concept of soft constraints - restricted, but breakable space constraints emerges. 
    
    \item \emph{Tactical} - The \emph{space constraints} ar updated during the flight. This can be used also for notifying the weather situations, restricted airspace and all sort of \emph{static or moving constraints}.
    
    \item \emph{Dynamic} - The \emph{space constraints} the updates are real time.
\end{enumerate}

\begin{note}
    The work covers on dynamic and tactical \emph{Geo-fencing}.
\end{note}

\paragraph{Actively maintaining separation during flight:} The standard ATM functionality \cite{icao4444} and \emph{Rules Of the Air} \cite{icaoAnnex2,icaoAnnex11} are covered and to be implemented for \emph{U-Space}.