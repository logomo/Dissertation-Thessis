\subsection{(W) Performance Evaluation}\label{s:performanceEvaluation}
\paragraph{Evaluation method:} \emph{Test cases} were evaluated according to performance requirements defined in (sec. \ref{s:AvoidanceRequirements}). The method was tracking critical parameter for \emph{Safety} (sec . \ref{s:Safety}) (primary) and \emph{Trajectory Tracking} (sec. \ref{s:trajectoryTracking}) (secondary) including \emph{Energy Efficiency} (sec. \ref{s:EnergyEfficiency}).

\paragraph{Safety Margin Performance Evaluation:} The \emph{safety of UAS} is main concern of \emph{DAA system}. The common concept of \emph{safety margin} is evaluated. 

The \emph{threat} is multidimensional, there are often multiple \emph{static obstacles, intruders} or \emph{weather constraints}. To reduce the multidimensional threats to one dimensional value \emph{crash distance} concept is used:

\begin{multline}\label{eq:crashDistance}
    crashDistance(t) =  distance(UAScenter(t),threat) \\\text{  where \emph{selection criterion} is:  }\\ \min \left\{\begin{gathered}\left( \begin{gathered} distance(UAScenter(t),threat)-\dots\\\dots-threat.SafetyMargin\end{gathered}\right)\\:\forall threat \in KnownWorld (t)\end{gathered}\right\}
\end{multline}

The \emph{crash distance} (eq. \ref{eq:crashDistance}) for given time is evaluated as shortest distance between UAS center and threat. The threat origins from known world (sec. \ref{s:KnownWorld}). The \emph{threat} have safety margin. The distance to safety margin is used as prioritization criterion in our test cases (tab. \ref{tab:testCasesSummary}).


The \emph{safety margin} evolution over time (eq. \ref{eq:safetyMarginOverTimeEvolution}) is calculated similar to \emph{crash distance}. The most dangerous threat is selected based on \emph{distance to safety margin} criterion. The value of \emph{safety margin} property is then used.

\begin{multline}\label{eq:safetyMarginOverTimeEvolution}
    safety Margin(t) =  threat.SafetyMargin\\\text{  where \emph{selection criterion} is:  }\\ \min \left\{\begin{gathered}\left( \begin{gathered} distance(UAScenter(t),threat)-\dots\\\dots-threat.SafetyMargin\end{gathered}\right)\\:\forall threat \in KnownWorld (t)\end{gathered}\right\}
\end{multline}

The \emph{distance to safety margin} (eq. \ref{eq:distanceToSafetyMargin}) is calculated as a difference between \emph{crash distance} (eq. \ref{eq:crashDistance}) and \emph{safety margin} (eq. \ref{eq:safetyMarginOverTimeEvolution}). The \emph{acceptance criteria} for safety is \emph{distance to safety margin} $\ge$ 0.

\begin{equation}\label{eq:distanceToSafetyMargin}
    distanceToSafetyMargin(t) =  crashDistance(t) - safetyMargin(t) \ge 0
\end{equation}

\paragraph{Distance to Safety Margin:}
\begin{enumerate}
    \item \emph{Minimal}
    \item \emph{Maximal}
\end{enumerate}

\paragraph{Breach Indicator:} 
\begin{enumerate}
    \item \emph{Yes/No}
    \item \emph{Distance to Safety Margin Reference}
\end{enumerate}

\paragraph{Trajectory Tracking Evaluation:}

\begin{enumerate}
    \item \emph{Waypoint reach:}
    \begin{enumerate}[a.]
        \item \emph{Yes/No}
        \item \emph{UAS ID}
        \item \emph{Trajectory tracking figure reference}
    \end{enumerate}
    \item \emph{Reference deviation:}
    \begin{enumerate}[a.]
        \item \emph{Waypoint ID}
        \item \emph{Peak deviation value}
    \end{enumerate}
    \item \emph{Acceptable deviation:}
    \begin{enumerate}[a.]
        \item \emph{Yes/No}
        \item \emph{Trajectory tracking performance table reference}
    \end{enumerate}
\end{enumerate}