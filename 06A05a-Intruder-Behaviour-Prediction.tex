\newpage
\subsection{\secState{R}Intruder Behaviour Prediction}\label{s:intruderBehaviourPrediction}
\paragraph{Idea:} \emph{Intruder Intersection Models} is about space-time intersection of \emph{intruder body} with \emph{avoidance Grid} and \emph{Reach Set}:
\begin{enumerate}
    \item The \emph{UAS} reach set defines \emph{time boundaries} to \emph{enter/leave} cell in avoidance grid.
    \item The \emph{Intruder} behavioral pattern defines \emph{rate} of \emph{space intersection} with cell bounded space in avoidance grid.
\end{enumerate}

The multiplication of \emph{space intersection rate} and \emph{time intersection rate} will give us \emph{intruder intersection} rate for our \emph{UAS} and intruder.


\paragraph{Intruder Dynamic Model:} The  definition of avoidance grid enforces the  most of these methods to be numeric. Let us introduce intruder dynamic model:

\begin{equation}\label{eq:intruderBasicLinearModel}
    \begin{aligned}
        \partial position /\partial time = velocity 
    \end{aligned}
    \quad | \quad
    \begin{aligned}
        position_x(t) = position_x(0) + velocity_x \times t\\
        position_y(t) = position_y(0) + velocity_y \times t\\
        position_z(t) = position_z(0) + velocity_z \times t
    \end{aligned}
\end{equation}

\noindent Position vector in euclidean coordinates $[x,y,z]$   is transformed into \emph{Avoidance Grid} coordinate frame. Velocity vector for $[x,y,z]$  is \emph{estimated and not changing}. The time  is in interval $[entry,leave]$, where $entry$ is intruder entry time into avoidance grid and $leave$ is intruder leave time from avoidance grid. 

\begin{note}
    If \emph{intruder} is considered, time of entry is marked as $intruder_{entry,k}$ where k is intruder identification, time of leave is marked as $intruder_{leave,k}$ where k is intruder identification. 
\end{note}

\paragraph{Cell Entry and Leave Times} $UAS_{entry}(cell_{i,j,k})$ and $UAS_{leave}(cell_{i,j,k})$ are depending on intersecting  \emph{Trajectories} and \emph{bounded cell space} (eq. \ref{eq:boundedSpaceCell}). There is \emph{Trajectory Intersection} function from (def. \ref{def:ContainedReducedReachSet}) which evaluates \emph{Trajectory segment} entry and leave time. 

The UAS \emph{Cell Entry} time is given as minimum of all \emph{passing trajectory segments} entry times (eq. \ref{eq:cellEntryTime}), if there is no \emph{passing trajectories} the UAS \emph{entry time} is set to 0.

\begin{equation}\label{eq:cellEntryTime}
    UAS_{entry}(cell_{i,j,k}) =  \min 
    \left\{\begin{aligned}
    0,en&try(Trajectory,cell_{i,j,k}):\\ &Trajectory\in Passing Trajectories
    \end{aligned}\right\}
\end{equation}

The UAS \emph{Cell Leave} time is given as maximum of all \emph{passing trajectory segments} entry times (eq. \ref{eq:cellLeaveTime}), if there is no \emph{passing trajectories} the UAS \emph{leave time} is set to 0.

\begin{equation}\label{eq:cellLeaveTime}
    UAS_{leave}(cell_{i,j,k}) =  \max 
    \left\{\begin{aligned}
    0,lea&ve(Trajectory,cell_{i,j,k}):\\ &Trajectory\in Passing Trajectories
    \end{aligned}\right\}
\end{equation}

\paragraph{Time Intersection Rate:} The key idea is to calculate how long the \emph{UAS} and \emph{Intruder} spends together in same space portion ($cell_{i,j,k}$). 
The \emph{Intruder} can spent some time in $cell_{i,j,k}$ bounded by interval of \emph{intruder} entry/leave time. 

\noindent The \emph{UAS} can spent some time, depending on \emph{selected trajectory} from \emph{Reach Set}. The time spent by UAS is bounded by entry (eq. \ref{eq:cellEntryTime}) and leave (eq. \ref{eq:cellLeaveTime}). 

The intersection duration of these two intervals creates \emph{time intersection rate} numerator, the \emph{maximal duration} of \emph{UAS} stay gives us \emph{denominator}. The \emph{time intersection rate} is formally defined in (eq. \ref{eq:timeIntersectionRate}). 

\begin{equation}\label{eq:timeIntersectionRate}
    time\left(\begin{gathered}UAS,\\Intruder,\\cell_{i,j,k}=\circ\end{gathered}\right)=  
    \frac{
        \left|
        \begin{gathered}
            \ [intruder_{entry}(\circ),intruder_{leave}(\circ)] \\
            \cap\\
            [UAS_{entry}(\circ),UAS_{leave}(\circ)]
        \end{gathered}\right|
        }
        {
        \left|\left[UAS_{entry}(\circ),UAS_{leave}(cell_{\circ})\right]\right|
        }
\end{equation}


\paragraph{Intruder Intersection Rate:} The \emph{Intruder Intersection Rate} (eq. \ref{eq:intruderIntersectionProbability}) is calculated as \emph{multiplication} of \emph{space intersection rate} (defined later) and \emph{time intersection rate} (eq. \ref{eq:timeIntersectionRate}).

\begin{equation}\label{eq:intruderIntersectionProbability}
    intruder\left(\begin{gathered}UAS,\\Intruder,\\cell_{i,j,k}\end{gathered}\right) = time \left(\begin{gathered}UAS,\\Intruder,\\cell_{i,j,k}\end{gathered}\right) \times space\left(\begin{gathered}UAS,\\Intruder,\\cell_{i,j,k}\end{gathered}\right)
\end{equation}

\begin{note}
    If there is no information to derive \emph{Intruder} entry/leave time for cells the \emph{time intersection rate} is considered 1.
\end{note}

The \emph{Intruder cell reach} time (eq. \ref{eq:intruderIntersectionTimeonPoint}) is bounded to discrete point in intersection model \cite{shamos1975closest,bentley1980optimal}. The intruder \emph{entry/leave time} is calculated similar to \emph{UAS cell entry (eq. \ref{eq:cellEntryTime})/leave (eq. \ref{eq:cellLeaveTime}) time}.

\begin{equation}\label{eq:intruderIntersectionTimeonPoint}
    point Reach Time(Intruder,point) = \frac{distance(Intruder.initial Position, point)}{|Intruder.velocity}
\end{equation}


\paragraph{Space Intersection Rate:} The \emph{Space Intersection Rate} reflects probability of \emph{Intruder} intersection with portion of space bounded by $cell_{i,j,k}$, to be precise with intruder trajectory or vehicle body shifted along the trajectory. The principles for \emph{space intersection rate} calculation are following:




\begin{enumerate}
    \item \textit{Line trajectory} - \emph{intruder} trajectory is given by linear approximation (eq. \ref{eq:intruderBasicLinearModel}), depending on \emph{intruder size} the intersection with avoidance grid can be:
    
    \begin{enumerate}[a.]
        \item \emph{Simple line} - intersection is going along the trajectory line line defined by intruder model (eq.\ref{eq:intruderBasicLinearModel}).
    
        \item \emph{Volume line} - intersection is going along the trajectory line defined by intruder model (eq. \ref{eq:intruderBasicLinearModel}) and intruder`s \emph{body radius} is considered in intersection.
    \end{enumerate}
    
    \item \emph{Elliptic cone} - initial position is considered as the top of a cone, the main cone axis is defined by intruder linear trajectory (eq. \ref{eq:intruderBasicLinearModel}) $time \in [0,\infty]$. The cone width is set by horizontal and vertical spread.
\end{enumerate}
