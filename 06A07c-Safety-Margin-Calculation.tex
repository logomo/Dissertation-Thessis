\newpage
\section{Guideline - Safety Margin Calculation}\label{s:safetyMarginCalculation}
\paragraph{Safety Margin Determination:} To determine \emph{safety Margin} the \emph{Rule of Thumb} is used:

\begin{equation}
    maximal Body Radius \le safety Margin \le 2 \times turning Radius
\end{equation}

\noindent The \emph{lower boundary} is given by \emph{UAS} construction because the \emph{UAS} body is considered as a \emph{unit ball} with the radius given as \emph{maximal body radius}. 

The \emph{upper boundary} is optional, The \emph{double of \emph{turning radius}} is used by the \emph{conservative approach} \cite{borenstein1991vector}.


\paragraph{Safety Margin Bloating:}  The \emph{discretization} of \emph{Reach Set}, \emph{Operation Space} and \emph{Decisions} imposes standard \emph{mixed integer} problem considering \emph{safety}. This section covers a \emph{non-exhaustive} list of possible \emph{Safety Margin Bloats} in our approach.

\paragraph{Own Position Uncertainty Bloat:} The \emph{sensor fusion} is precise, but not \emph{exact} in own UAS position determination. The usual maximal disparity needs to be accounted into \emph{Safety Margin}.

\paragraph{Intruder Position Uncertainty Bloat:} The \emph{sensor fusion} of Intruder is precise, but not \emph{exact} in own UAS position determination. The usual maximal disparity needs to be accounted into \emph{Safety Margin}.

\paragraph{Weather bloat:} The \emph{Weather} impact type may result in increased \emph{safety margin}. Example: UAS is not humidity resistant, the clouds will be avoided from a greater distance.

\paragraph{Airspace bloat:} The \emph{Airspace} depending on cluster or \emph{country} may require greater separation distances, depending on circumstances. The example can be UAS directive to keep minimal separation from obstacles. The \emph{Safety Margin} is usually overridden by UTM directive value.

\paragraph{UTM Synchronization Bloat:} \emph{Both UAS} decision times were \emph{synchronized}. The \emph{intruder} can be offset for the \emph{full decision frame}. This is not an assumption, but it shows critical performance. Usually, safety margin is bloated for (worst case offset):
\begin{equation}\label{safetyMarginBloat}
    safetyMarginBloat = \left( \begin{aligned}
    &intruderVelocity \times\dots \\ &intruderDecisionFrame \end{aligned}\right)[m,ms^{-1},s]
\end{equation}